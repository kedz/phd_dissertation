\subsection{The Locally-Optimal Learning-to-Search Algorithm}
\label{sec:algorithm}

In a perfect world, we would have a training set of state-action 
pairs, $(\state_t,\bsal_t)$,
and their associated costs $\cost(\state_t, \bsal_t)$, drawn from a distribution
of states similar to the one our final learned policy would produce.
With these states, actions, and costs in hand we could fit two linear regressions, one for estimating the cost of extract actions and one for skip actions
from a givem state's feature representation;
we would immediately have a suitable stream summarization
policy. Unfortunately, we do not 
have a reasonable distribution of state-action pairs let alone the 
associated costs. Instead we turn to the locally-optimal learning-to-search (LOLS) algorithm \citep{chang2015}, presented in \autoref{fig:lols},
to iteratively refine our learned policy as it attempts to follow an
oracle policy. 

At a high level, we begin with an initally random learned policy.  We leverage
a heuristic oracle stream summarizer and our learned policy to collect
state-action pairs and their costs from the training set query-streams, with
our learned policy gradually learning to immitate the heuristic oracle. Using
both the oracle and learned policy to sample state-action-cost triples is
beneficial as it exposes the learned policy to a mix of states it is likely to
encounter when following its own actions, but also state-actions of the better
performing reference summarizer. Exploring with the learned policy alone may
be less optimal because it may overestimate the costs of good decisions since
its roll-outs at the beginning of training are likely to be quite bad.
Exploring with only the oracle would also be harmful as the distribution of
states it produces will reflect an optimistic level of performance from
learned policy that it will not be able to match in practice.


\subsubsection{Oracle Policy}

For a given training query $\query$ and stream $\sentstream^{(\query)}$ we
construct a greedy heuristic oracle policy $\oracle_\query$.  Let
$\updateSummary_t$ be the update summary at state $\state_t$. Additionally,
let
\[
\strmNuggets^{(\query)}_{\denotes{\updateSummary_t}} = 
\left\{\strmnugget_i \Bigg\vert
        \left(\strmnugget_i,\nugtimestamp_i\right) \in \strmNuggets^{(\query)},
         \exists \left(\strmupdate_j, \updatetimestamp_j\right) \in \updateSummary_t : \denotes{\strmnugget_i} \in
        \denotes{\strmupdate_j}
       \right\}
\]
be the set of nuggets covered by the updates in the update summary at
$\state_t$, and let
\[
\strmNuggets^{(\query)}_{\denotes{\strmSent_t}} = 
\left\{\strmnugget_i \Bigg\vert
        \left(\strmnugget_i,\nugtimestamp_i\right) \in \strmNuggets^{(\query)},
        \denotes{\strmnugget_i} \in
        \denotes{\strmSent_t}
       \right\}
\]
be the set of nuggets covered by $\strmSent_t$.  The oracle $\oracle_\query$
has clairvoyant knowledge of these sets given a state $\state_t$ has the
following policy,
\[
    \oracle_\query(\state_t) = \begin{cases} 
        1 & \textrm{if }  
        \exists \strmnugget : \strmnugget \in \strmNuggets^{(\query)}_{\denotes{\strmSent_t}} \wedge 
   \strmnugget \not\in \strmNuggets^{(\query)}_{\denotes{\updateSummary_t} }
     \\
     0 & \textrm{otherwise.} \end{cases}
\]
In words, the oracle policy will extract $\strmSent_t$ at state $\state_t$
if $\strmSent_t$ covers any nuggets not yet covered by updates in the
update summary $\updateSummary_t$. We describe how we obtain comprehensive
nugget/sentence covers, i.e. the sets $\strmNuggets^{(\query)}_{\denotes{\strmSent_t}}$ in \autoref{sec:reljudge}.

\subsubsection{Loss Function}
We design our loss function to penalize policies that  over- or
under-generate. Let $\state_1, \ldots, \state_T$ be the sequence
of states associated with the action sequence $\bsals$.
%Given 
%Given two sets of decisions, usually one from the oracle and
%another from the candidate model, 
%We define the loss as the complement of
%the Dice coefficient between the decisions,
We define the loss of a sequence as 
\begin{align*}
  \lossfunc(\bsals) = 1 - 2 \times 
    \frac{ \bsals^\T \bsals^* }{ \sum^T_{i=1} \bsal_i + \sum^T_{j=1}\bsal_j^* }
\end{align*}
where $\bsals^* = \left[\bsal_1^*,\ldots, \bsal_T^* \right]$
is a reference sequence of consisting of the one-step optimal deviations
under the oracle policy, i.e. $\bsal_t^* = \oracle_\query(\state_t)$.

Note that $\lossfunc$ is the complement of the Dice coefficient.
This encourages not only local agreement between policies (the numerator of
the second term) but that the learned and oracle policy should generate
roughly the same number of updates (the denominator in the second term).

\begin{algorithm}[t]
  \RestyleAlgo{boxruled}
  \LinesNumbered
  \KwIn{training dataset of query-relevant streams and oracle policies $\{\sentstream_\query, \oracle_{\query} \}_{\query\in\queries}$, number of 
        training epochs $N$, learning rate $\lambda$, and a mixture parameter $\beta \in (0,1)$ for 
        selecting a roll-out policy.}
  \KwOut{learned summarization policy $\policy_\params$}
  Initialize $\params$ \label{alg:lolsinit}\\
  \For{$n \gets 1,2,\ldots, N$}{
    \For{$\query \in \queries$}{
      $\Train \gets \{\}$\\    
      \For{$t \in \{1,\ldots,T \}$}{
        Roll-in by executing $\policy_\params$ on $\sentstream_\query$ for $t-1$ rounds and reach state $s_t$. \label{alg:lolsrollin} \\
        \For{$y \in \{0,1\}$}{
            Pick roll-out policy $\rolloutpolicy \gets \begin{cases} \oracle_{\query} & \textrm{with probability $\beta$}\\ \policy_\params & \textrm{with probability $1 -\beta$.}\end{cases}$. \label{alg:lolsrollout}\\
            Compute roll-out action sequence $\bsals^{(t,\bsal,\rolloutpolicy)}$. \label{alg:lolsaction}\\
        }      
        \For{$y \in \{0,1\}$}{
            $\cost(\state_t, \bsal) \gets \ell\left(\bsals^{\left(t,\bsal,\rolloutpolicy\right)}\right) - \min_{\bsal^\prime \in \{0,1\}} \ell\left(\bsals^{\left(t,\bsal^\prime,\rolloutpolicy\right)}\right) $ \label{alg:lolscoststart}  \\
            $\Train \gets \Train \cup \left\{\left( \Feat(\state_t,\bsal),
            c(\state_t, \bsal), \bsal\right)\right\}$ \label{alg:lolscoststop}  }}
    \For{$\left(\Feat(\state,\bsal),
            c(\state, \bsal), \bsal\right) \in \Train$ \label{alg:lolsmsestart} }{
        $\params_\bsal \gets \params_\bsal - \lambda \nabla_{\params_\bsal}(c(\state, \bsal) - \Feat(\state,\bsal)\cdot \params_\bsal  )^2$ \label{alg:lolsmsestop}
}
}}
  \KwResult{$\policy_\params$}
  \caption{Locally Optimal Learning-to-Search for Stream Summarization} 
  \label{fig:lols}
\end{algorithm}




\subsubsection{Learning with LOLS}

We now step through the LOLS algorithm in detail.
Our initial policy parameters $\params_0$ and $\params_1$ are randomly initialized (\hyperref[fig:lols]{Algorithm~\ref{fig:lols} line \ref{alg:lolsinit}}). 
The LOLS algorithm works by iteratively using the current learned policy
$\policy_\params$ to explore state sequences from a training stream $\sentstream^{(\query)}$
(\hyperref[fig:lols]{Algorithm~\ref{fig:lols} line \ref{alg:lolsrollin}}).
At each state $\state_t$, a roll-out policy $\rolloutpolicy$ is selected
at random from $\policy_\params$ and the oracle $\oracle_\query$
(\hyperref[fig:lols]{Algorithm~\ref{fig:lols} line \ref{alg:lolsrollout}}).
Losses
are then computed for each continuation sequence $\roaseq{t}{0}$ and 
$\roaseq{t}{1}$
(\hyperref[fig:lols]{Algorithm~\ref{fig:lols} line \ref{alg:lolsaction}}).
Costs for each action are then computed and state features and costs are 
cached in $\Train$ 
(\hyperref[fig:lols]{Algorithm~\ref{fig:lols} lines \ref{alg:lolscoststart}--\ref{alg:lolscoststop}}).
After the roll-in has explored all $T$ states, we update the policy
parameters $\params$ by performing gradient descent on the squared
error of the linear regressor on the cached feature-cost pairs
in $\Train$ 
(\hyperref[fig:lols]{Algorithm~\ref{fig:lols} lines \ref{alg:lolsmsestart}--\ref{alg:lolsmsestop}}).
