\subsection{Policy-based Stream Summarization}

In the induced search problem, each search state $\state_t \in \states$
corresponds to observing the first $t$ sentences in the stream $\left[
\strmSent_1,\ldots, \strmSent_t \right] \subset \sentstream$ and a sequence of
$t-1$ actions $\bsal_1, \ldots, \bsal_{t-1}$.  For each state $\state_t \in
\states$, the set of actions is $\bsal_t \in \{0,1\}$ with $\bsal_t =1$
indicating we extract the $t$-th sentence and add it to our update summary,
and $\bsal_t = 0$ indicating we skip it. For simplicity of exposition, we
assume a fixed length stream of size $T$ but this is not strictly necessary. 

\begin{algorithm}[t]
  \RestyleAlgo{boxruled}
  \LinesNumbered
  \KwIn{
    query string $\query$, 
    query category $\category$, 
    stream $\sentstream^{(\query)}$, 
    period of interest ($\timestamp_\starttok$, $\timestamp_\stoptok$), 
    summarization policy $\policy$}
  \KwOut{update summary $\updateSummary$}
  $\updateSummary \gets \{\}$

    \For{$\strmSent_t \in \sentstream^{(\query)}_{\timestamp_\starttok:\timestamp_\stoptok}$}{
       $\state_t \gets (\strmSent_1,\ldots,\strmSent_t, \bsal_1,\ldots,\bsal_{t-1},\query,\category)$\\
       $\bsal_t \gets \policy(\state_t)$\\
      \If{$\bsal_t = 1$}{
        $\updateSummary \gets \updateSummary \cup \left\{\left(\strmSent_t, \senttimestamp_t \right) \right\} $
        }
    }
  \caption{Policy-based Stream Summarization} 
  \label{alg:policysum}
\end{algorithm}




A policy, $\policy : \states \rightarrow \{0,1\}$, is a mapping from states to
an extraction decision. Given a policy, the policy-based stream summarization
algorithm (\autoref{alg:policysum}) is trivial, iterating sequentially over
sentences in the stream, and adding each sentence to the update summary if it
is the current action determined by $\policy$.  In practice, we use a linear
cost-sensitive classifier to implement $\policy_\params$, with \[ \bsal_t =
\policy_\params(\state_t) = \argmin_{\bsal \in \{0,1\}} \Feat(\state_t, \bsal)\cdot
\params_\bsal\] where we encode each state-potential action pair
$(\state_t,\bsal)$ as a $d$-dimensional feature vector  $\Feat(\state_t,
\bsal) \in \mathbb{R}^d$ and $\params_0, \params_1 \in \reals^{d}$ are learned
parameters.  Note that $\Feat(\state_t,\bsal)\cdot \params_\bsal$ is a linear
regression to predict the cost, which we define shortly, associated with
taking action $\bsal$ in state $\state_t$. Given estimates of our two
available actions, extracting a sentence or ignoring it, our policy $\policy_\policy$
selects the action with minimum cost.

Before we can define cost more concretely, we must first introduce some additional notation and concepts.
For a given sequence of states $\state_1,\ldots,\state_T$ explored by a policy $\policy$, let $\bsals= \left[\bsal_1,\ldots,\bsal_T\right]$ be the associated 
sequence of actions taken by $\policy$, i.e. $\bsal_t = \policy(\state_t)$.
A loss function, $\lossfunc : \{0,1\}^T \rightarrow \reals$, measures the quality of an action 
sequence $\bsals$. In our present case this might be the negative \textsc{Rouge} 
score of the update summary that results from $\bsals$ or 
some other relevant measure of performance. 

Let $\policy$ be a policy, $\state_1,\ldots,\state_T$ a sequence of states
explored by $\policy$, and the corresponding action sequence $\bsals$. We also
define a second policy, $\rolloutpolicy$, which we call the roll-out policy
and which may or may not be distinct from $\policy$.  For any state $\state_t$
we can then define two additional action sequences,
\[
    \roaseqt = \left[
        \bsal_1, 
        \ldots, 
        \bsal_{t-1},
        \hat{\bsal}_t,
        \roat_{t+1}, 
        \ldots, 
        \roat_{T}, 
    \right] \quad \forall \predbsal_t \in \{0,1\} 
\] 
where $\roaseqt$ is the sequence of actions that result from following
$\policy$ on the first $t-1$ states, taking action $\predbsal_t$ at state
$\state_t$ and then following $\rolloutpolicy$ on the remaining $T - t$
states. That is, for $k > t$, $\roa_{k} = \rolloutpolicy(\state^\prime_k)$,
where $\state^\prime_k$ is the state that results from taking action
$\predbsal_t$ in $\state_t$ and following $\rolloutpolicy$ on a sequence of
states $\state^\prime_{t+1}, \ldots, \state^\prime_k$. The loss
$\lossfunc\left( \roaseq{t}{0} \right)$ then reflects the evaluation measure
for an update summary where sentence $\strmSent_t$ was not extracted, and
$\rolloutpolicy$ completed the summary of the stream. Similarly,
$\lossfunc\left( \roaseq{t}{1}  \right)$ reflects the evaluation measure for
an update summary where sentence $\strmSent_t$ was extracted, and
$\rolloutpolicy$ completed the summary of the stream. 
 
We can now define the cost of an action $\bsal$ in state $\state_t$ as 
\[ 
    \cost(\state_t, \bsal) = 
        \lossfunc\left(\roaseq{t}{\bsal}   \right) 
            - \min_{\bsal^\prime \in \{0,1\} }
                \lossfunc\left( \roaseq{t}{\bsal^\prime} \right).
\]
Note that the cost is also a function of $\rolloutpolicy$, which determines
how the action sequence is completed after $\state_t$. 
The costs connect the overall summary loss $\lossfunc\left(\roaseqt\right)$
to a particular action in $\state_t$ that builds the summary.
We depict an example of the cost computation in \autoref{fig:rollouts}.
 We discuss how to collect costs and learn $\params_0$
and $\params_1$ such that they are good estimators of cost in the next section.
 

\begin{figure}[p]
\resizebox{\textwidth}{!}{\begin{tikzpicture}[
    state/.style={fill,draw,circle,minimum height=1mm,inner sep=0.0mm},
    policy/.style={line width=0.5mm,circle,minimum height=2.25mm,
                   inner sep=0.0mm},
    rollin/.style={policy,draw=green},
    rolloutA/.style={policy,draw=purple},
    rolloutB/.style={policy,draw=orange},
    action/.style={line width=0.75mm},
    rolloutAline/.style={action,draw=purple},
    rolloutBline/.style={action,draw=orange},
    rollinline/.style={action,draw=green},
]

\def\WIDTH{0.32};
\def\TH{6}

\node at (5,\TH + 1.5) {\large \textbf{Search Space}};
\node at (13,\TH + 1.5) {\large \textbf{Roll-out Action Sequences}};
\node at (-0.1,\TH + 1) 
    {$\sentstream^{(\query)}_{\timestamp_\starttok:\timestamp_\stoptok}$};
\node at (-0.1,\TH + 0.9) 
    {\uline{$\phantom{\sentstream^{(\query)}_{\timestamp_\starttok:\timestamp_\stoptok}}$}};

\foreach \n [
    % compute the number of nodes \numnodes from the line \n
    evaluate=\n as \numnodes using int(2^(\TH-1)/(2^(\n-1)), 
    % compute the (\n-1)th line.
    evaluate=\n as \nmo using int(\n-1)
] in {1,...,\TH} {

    
    \ifthenelse{\n=6}{}{
        \node at (-0.1,\TH - \n + 1) {$\strmSent_\n$};
    }
    
    \foreach \x [
        evaluate=\x as \xcoord using \x*\WIDTH*2^(\n-1)-\WIDTH*2^(\n-1)/2,
        evaluate=\x as \lc using int(2*\x - 1),
        evaluate=\x as \rc using int(2*\x)
    ] in {1,...,\numnodes} {

        \node[state] (a\n\x) at (\xcoord,\n) {};
        \ifthenelse{\nmo=0}{}{
            \ifthenelse{\n=4 \AND \x=3}{
                \draw[draw=blue,dashed,line width=0.5mm] 
                    (a\n\x.south west) -- (a\nmo\lc.north);
                \draw[draw=red,dashed,line width=0.5mm] 
                    (a\n\x.south east) -- (a\nmo\rc.north);
            }{
                \draw (a\n\x.south west) -- (a\nmo\lc.north);
                \draw (a\n\x.south east) -- (a\nmo\rc.north);
            }
        }
    }

}

\node[rollin] (ri1) at (a61) {};
\node[rollin] (ri2) at (a52) {};
\node[rollin] (ri3) at (a43) {};
\draw[rollinline] (ri1.east) -- (ri2.north);
\draw[rollinline] (ri2.west) -- (ri3.north);
\node[rotate=-20] at ($(ri1)!0.5!(ri2)+(0,0.25)$) {\tiny extract};
\node[rotate=32] at ($(ri2)!0.5!(ri3)+(0,0.25)$) {\tiny skip};
\node[rolloutA] (roa4) at (a35) {};
\node[rolloutA] (roa5) at (a210) {};
\node[rolloutA] (roa6) at (a119) {};
\node[rolloutB] (rob4) at (a36) {};
\node[rolloutB] (rob5) at (a211) {};
\node[rolloutB] (rob6) at (a121) {};
\node at ($(ri1)+(0.25,0.25)$) {$\state_1$}; 
\node at ($(ri2)+(0.25,0.25)$) {$\state_2$}; 
\node at ($(ri3)+(-0.25,0.25)$) {$\state_3$}; 
\node[rotate=-73.5] at ($(roa4)!0.5!(roa5)+(0.25,0)$) {\tiny extract};
\node[rotate=84] at ($(roa5)!0.5!(roa6)+(-0.25,0)$) {\tiny skip};
\node[rotate=73.5] at ($(rob4)!0.5!(rob5)+(-.25,0)$) {\tiny skip};
\node[rotate=84] at ($(rob5)!0.5!(rob6)+(-0.25,0)$) {\tiny skip};
\draw[rolloutAline] (roa4.south east) -- (roa5.north);
\draw[rolloutAline] (roa5.south west) -- (roa6.north);
\draw[rolloutBline] (rob4.south west) -- (rob5.north);
\draw[rolloutBline] (rob5.south west) -- (rob6.north);

\def\COffset{36*\WIDTH}
\node at (\COffset, \TH+0.5) {$\uline{\roaseq{3}{0}\phantom{= 1}}$};
\node at (\COffset + 3, \TH+0.5) {$\uline{\roaseq{3}{1}\phantom{= 1}}$};

\def\SPACE{\phantom{\left(t,\hat{\bsal},\rolloutpolicy\right)}}
\foreach \ya/\yb [count=\t from 1] in {1/1, 0/0, 0/1, 1/0, 0/0} {
    \ifthenelse{
        \t<3
    }{ 
        \node (ya\t) at (\COffset, \TH-\t + 0.5) 
            {$\bsal^{\SPACE}_\t=\ya$};
        \node (yb\t) at (\COffset+3, \TH-\t + 0.5) 
            {$\bsal^{\SPACE}_\t=\yb$};
    }{
        \ifthenelse{
            \t=3                
        }{
            \node (ya\t) at (\COffset, \TH-\t + 0.5) 
                {$\predbsal^{\SPACE}_\t=\ya$};
            \node (yb\t) at (\COffset+3, \TH-\t + 0.5) 
                {$\predbsal^{\SPACE}_\t=\yb$};
        }{
            \node (ya\t) at (\COffset, \TH-\t + 0.5) 
                {$\roa{3}{0}_\t=\ya$};
            \node (yb\t) at (\COffset+3, \TH-\t + 0.5) 
                {$\roa{3}{1}_\t=\yb$};

        }
    }
}

\draw[draw=green,fill=green!20,line width=0.25mm] 
    (ya1.north west) rectangle (ya2.south east);
\draw[draw=green,fill=green!20,line width=0.25mm] 
    (yb1.north west) rectangle (yb2.south east);
\draw[draw=blue,fill=blue!20,line width=0.25mm] 
    (ya3.north west) rectangle (ya3.south east);
\draw[draw=red,fill=red!20,line width=0.25mm] 
    (yb3.north west) rectangle (yb3.south east);
\draw[draw=purple,fill=purple!20,line width=0.25mm] 
    (ya4.north west) rectangle (ya5.south east);
\draw[draw=orange,fill=orange!20,line width=0.25mm] 
    (yb4.north west) rectangle (yb5.south east);


%\node at (by1) {$\bsal_1^{\phantom{\left(3,0,\rolloutpolicy\right)}}=1$};
%\node at (by2) {$\bsal_2^{\phantom{\left(3,0,\rolloutpolicy\right)}}=0$};
%\node at (by3) {$\hat{\bsal}_3^{\phantom{\left(3,0,\rolloutpolicy\right)}}=1$};
%\node at (by4) {$\bsal^{\left(3,1,\rolloutpolicy\right)}_4=0$};
%\node at (by5) {$\bsal^{\left(3,1,\rolloutpolicy\right)}_5=0$};
%
%\node (ay1) at (36*\WIDTH, \TH-1+0.5) {$\bsal_1^{\phantom{\left(3,0,\rolloutpolicy\right)}}=1$};
%\node (ay2) at (36*\WIDTH, \TH-2+0.5) {$\bsal_2^{\phantom{\left(3,0,\rolloutpolicy\right)}}=0$};
%\node (ay3) at (36*\WIDTH, \TH-3+0.5) {$\hat{\bsal}_3^{\phantom{\left(3,0,\rolloutpolicy\right)}}=0$};
%\node (ay4) at (36*\WIDTH, \TH-4+0.5) {$\bsal^{\left(3,0,\rolloutpolicy\right)}_4=1$};
%\node (ay5) at (36*\WIDTH, \TH-5+0.5) {$\bsal^{\left(3,0,\rolloutpolicy\right)}_5=0$};

\foreach \ya/\yb [count=\t from 1] in {1/1, 0/0, 0/1, 1/0, 0/0} {
    \ifthenelse{
        \t<3
    }{ 
        \node (ya\t) at (\COffset, \TH-\t + 0.5) 
            {$\bsal^{\SPACE}_\t=\ya$};
        \node (yb\t) at (\COffset+3, \TH-\t + 0.5) 
            {$\bsal^{\SPACE}_\t=\yb$};
    }{
        \ifthenelse{
            \t=3                
        }{
            \node (ya\t) at (\COffset, \TH-\t + 0.5) 
                {$\predbsal^{\SPACE}_\t=\ya$};
            \node (yb\t) at (\COffset+3, \TH-\t + 0.5) 
                {$\predbsal^{\SPACE}_\t=\yb$};
        }{
            \node (ya\t) at (\COffset, \TH-\t + 0.5) 
                {$\roa{3}{0}_\t=\ya$};
            \node (yb\t) at (\COffset+3, \TH-\t + 0.5) 
                {$\roa{3}{1}_\t=\yb$};

        }
    }
}

\node[draw,anchor=north west,align=left,text width=15.4cm] at (0,0) {
\textbf{Computing Costs  $\cost(\state_3,0)$ and $\cost(\state_3,1)$}\\

~\\

\textbf{Step 1. Roll-in to $\state_3$ with policy $\policy$.}\\
Use policy $\policy$ to explore to state $\state_3$, shown in {\color{green}green} in the search space above.\\~\\  
\textbf{Step 2. Roll-out with $\rolloutpolicy$ to create action sequences
$\roaseq{3}{0}$ and $\roaseq{3}{1}$.}\\
For each action $\predbsal_3 \in \{0,1\},$ use $\rolloutpolicy$ to 
complete the roll-outs after having made action $\predbsal_3$ in state $\state_3$ (shown in {\color{purple}purple} and {\color{orange}orange} respectively) and create
alternative action sequences $\roaseq{3}{0}$ and $\roaseq{3}{1}$ (shown on 
the right).\\~\\
\textbf{Step 3. Compute losses.}\\
After completing the roll-outs, compute losses 
$\lossfunc\left(\roaseq{3}{0}\right)$ and $\lossfunc\left(\roaseq{3}{1}\right)$.\\~\\
\textbf{Step 4. Compute costs.}\\
Compute 
\[\cost(\state_3,0) = \lossfunc\left( \roaseq{3}{0} \right) - \min_{\bsal^\prime \in \{0,1\}} \lossfunc\left( \roaseq{3}{\bsal^\prime} \right)\]
and  
\[\cost(\state_3,1) = \lossfunc\left( \roaseq{3}{1} \right) - \min_{\bsal^\prime \in \{0,1\}} \lossfunc\left( \roaseq{3}{\bsal^\prime} \right).\]

};

\end{tikzpicture}}
\caption{Example of computing costs of actions at $\state_3$ using roll-out
policy $\rolloutpolicy$.}
\label{fig:rollouts}
\end{figure}



