\section{Task Definition}
\label{sec:strmsumProbDef}

We now describe the query focused, sentence extractive, streaming
news summarization problem in detail. 
We start with the query, $\query$, which is a brief string
describing an event of interest to be summarized. See \autoref{tab:queries} for
some example queries. All relevance judgements about a sentence are made with
respect to the query string. Additionally, the query is used to construct the 
news stream which we now turn to. The news stream is an ordered
sequence of text approximately relevant to the query (i.e. the results of 
an information retrieval system). It is useful to be able to talk  
about this stream from 
two perspectives, as either a stream of documents, $\docstream^{(\query)}$, or 
a flat stream of sentences, $\sentstream^{(\query)}$.

\begin{table}[p]
\resizebox{\linewidth}{!}{
\begin{tabular}{c ccc cc}
\toprule
  & & \multicolumn{2}{c}{Period of Interest} \\
\cmidrule(lr){3-4}
TREC Year & Wikipedia Page Title & Start Time ($\timestamp_\starttok$) & Stop Time ($\timestamp_\stoptok$) & Query String ($\query)$ & Event Category ($c$)\\
\midrule
2013 & 2012 Buenos Aires Rail Disaster & 02/22/2012 11:33am & 03/03/2012 11:33am & \textit{buenos aires train crash} & accident\\
2013 & 2012 Pakistan garment factory fires & 09/11/2012 ~1:00pm & 09/21/2012 ~1:00pm & \textit{pakistan factory fire} & accident\\
2013 & 2012 Aurora shooting & 07/20/2012 ~6:38am & 07/30/2012 ~6:38am & \textit{colorado shooting} & shooting\\
2013 & Wisconsin Sikh temple shooting & 08/05/2012 ~3:25pm & 08/15/2012 ~3:25pm & \textit{sikh temple shooting} & shooting\\
2013 & Hurricane Isaac (2012) & 08/28/2012 ~4:20pm & 09/07/2012 ~4:20pm & \textit{hurricane isaac} & storm\\
2013 &Hurricane Sandy & 10/24/2012 ~3:00pm & 11/03/2012 ~3:00pm & \textit{hurricane sandy} & storm\\
2013 & June 2012 North American derecho & 06/29/2012 ~3:00pm & 07/09/2012 ~3:00pm & \textit{midwest derecho} & storm\\
2013 & Typhoon Bopha & 11/30/2012 ~2:45pm & 12/10/2012 ~2:45pm & \textit{typhoon bopha} & storm \\
2013 & 2012 Guatemala earthquake & 11/07/2012 ~4:35pm & 11/17/2012 ~4:35pm & \textit{guatemala earthquake} & earthquake \\
2013 & 2012 Tel Aviv bus bombing & 11/21/2012 10:00am & 12/01/2012 10:00am & \textit{tel aviv bus bombing} & bombing \\
\midrule
2014 & Costa Concordia disaster and recovery & 01/13/2012 ~9:45pm & 02/01/2012 12:00am & \textit{costa concordia} & accident \\
2014 & Early 2012 European cold wave & 01/22/2012 12:00am & 02/18/2012 12:00am & \textit{european cold wave} & storm \\
2014 & 2013 Eastern Australia floods & 01/17/2013 12:00am & 01/30/2013 12:00am & \textit{queensland floods} & storm \\
2014 & Boston Marathon bombings & 04/15/2013 ~6:49pm & 04/20/2013 11:59pm & \textit{boston marathon bombing} & bombing \\
2014 & Port Said Stadium riot & 02/01/2012 ~1:30pm & 02/11/2012 ~1:30pm  & \textit{egyptian riots} & riot \\
2014 & 2012 Afghanistan Quran burning protests & 02/20/2012 ~5:30pm & 02/28/2012 12:00am & \textit{quran burning protests} & protest \\
2014 & In Amenas hostage crisis & 01/16/2013 12:00am & 01/20/2013 12:00am & \textit{in amenas hostage crisis} & hostage \\
2014 & 2011-13 Russian protests & 12/04/2011 12:00am & 12/25/2011 12:00am & \textit{russian protests} & protest \\
2014 & 2012 Romanian protests & 01/12/2012 12:00am & 01/26/2012 12:00am & \textit{romanian protests} & protest \\
2014 & 2012-13 Egyptian protests & 11/18/2012 12:00am & 12/01/2012 12:00am  & \textit{egyptian protests} & protest \\
2014 & Chelyabinsk meteor & 02/15/2013 ~3:20am & 02/25/2013 ~3:20am & \textit{russia meteor} & impact event \\
2014 & 2013 Bulgarian protests against the Borisov cabinet & 02/10/2013 12:00am & 02/20/2013 11:59pm & \textit{bulgarian protests} & protest \\
2014 & 2013 Shahbag protests & 02/05/2013 12:00am & 02/22/2013 11:59pm & \textit{shahbag protests} & protest \\
2014 & February 2013 nor'easter & 02/07/2013 12:00am & 02/18/2013 11:59pm & \textit{nor'easter} & storm \\
2014 & Christopher Dorner shootings and manhunt & 02/03/2013 12:00am & 02/13/2013 ~7:59am & \textit{Southern California shooting} & shooting \\
\midrule 
2015 & Vauxhall helicopter crash & 01/16/2013 ~7:59am & 01/31/2013 ~7:59am  & \textit{vauxhall helicopter crash} & accident \\
2015 & Cyclone Nilam & 10/27/2012 12:00am & 11/02/2012 12:00am & \textit{cyclone nilam} & storm \\
2015 & 2013 Dhaka garment factory collapse & 04/24/2013 ~2:45am & 05/04/2013 ~2:45am & \textit{savar building collapse} & accident \\
2015 &2013 Hyderabad blasts & 02/21/2013 ~1:58pm & 03/03/2013 ~1:58pm & \textit{hyderabad explosion} & bombing \\
2015 & Brazzaville arms dump blasts & 03/04/2012 ~7:00am & 03/14/2012 ~7:00am & \textit{brazzaville explosion} & accident \\
2015 & 2012 India blackouts & 07/29/2012 ~9:18pm & 08/03/2012 ~9:18pm & \textit{india power blackouts} & accident \\
2015 & Reactions to Innocence of Muslims & 09/11/2012 12:00am & 09/30/2012 12:00am & \textit{innocence of muslims protests} & protest \\
2015 & Battle of Konna  & 01/10/2013 12:00am & 01/19/2013 12:00am & \textit{konna battle} & conflict \\
2015 & February 2013 Quetta bombing & 02/16/2013 12:00am & 02/20/2013 12:00am & \textit{quetta bombing} & bombing \\
2015 &15 April 2013 Iraq attacks & 04/15/2013 12:00am & 04/20/2013 12:00am & \textit{iraq bombing} & bombing \\
2015 & 19 March 2013 Iraq attacks & 03/19/2013 12:00am & 03/24/2013 12:00am & \textit{iraq bombing} & bombing \\
2015 & 2011-12 Los Angeles arson attacks & 12/29/2011 ~9:00am & 01/05/2012 ~9:00am & \textit{los angeles arson} & bombing \\
2015 & 2013 Thane building collapse  & 04/04/2013 12:00am & 04/13/2013 12:00am & \textit{thane building collapsed} & accident \\
2015 & 2013 United States embassy bombing in Ankara & 02/01/2013 12:00am & 02/05/2013 12:00am & \textit{suicide bomber ankara} & bombing \\ 
2015 & 22 December 2011 Baghdad bombings & 12/21/2011 ~9:00pm & 12/26/2011 ~9:00pm & \textit{baghdad bomb} & bombing \\
2015 &  Aleppo University bombings & 01/15/2013 12:00am & 01/25/2013 12:00am & \textit{aleppo university explosion} & bombing \\
2015 & Carnival Triumph 2013 Engine Room Fire & 02/10/2013 12:00am & 02/15/2013 12:00am & \textit{carnival triumph fire} & accident \\
2015 & USS Guardian (MCM-5) January 2013 Grounding  & 01/17/2013 12:00am & 01/22/2013 12:00am & \textit{uss guardian grounding} & accident \\
2015 & 2012 Indian Ocean earthquakes & 04/11/2012 12:00am & 04/16/2012 12:00am & \textit{aceh earthquake} & earthquake \\
2015 &  2012 Haida Gwaii earthquake & 10/28/2012 ~3:00am & 11/07/2012 ~3:00am & \textit{haida gwaii earthquake} & earthquake \\
2015 & 2012 Catalan independence demonstration & 09/11/2012 12:00am & 09/16/2012 12:00am & \textit{catalan protest} & protest \\
\bottomrule
\end{tabular}}
\caption{TREC Temporal Summarization shared-task query events for the years 2013-2015.  All times are UTC.}
\label{tab:queries}
\end{table}


From the document perspective, the stream is an ordered sequence of
$\strmDocSize_\query$ documents  \[ \docstream^{(\query)} = \left[ \strmDoc_1,
\strmDoc_2, \ldots, \strmDoc_{\strmDocSize_\query} \right] \] where each
document $\strmDoc_i$ is itself an ordered sequence of $\strmSentSize_i$
sentences, \[ \strmDoc_i = \left[ \strmSent_{i,1}, \strmSent_{i,2}, \ldots,
\strmSent_{i,\strmSentSize_i}  \right].  \] Each document $\strmDoc_i$ also
has a timestamp $\timestamp^{(d)}_i$ and which is also shared by all of its
sentences $\strmSent_{i,j}  \in \strmDoc_i$.  The stream is ordered by
timestamp, so we have $\doctimestamp_i < \doctimestamp_{i+1}$ for all $i \in
\{1,\ldots, \strmDocSize_\query\}$.  From the sentence perspective, the news
stream is an ordered sequence of $\strmSentSize_\query$ sentences, \[
\sentstream^{(\query)} = \left[ \strmSent_1, \strmSent_2, \ldots,
\strmSent_{\strmSentSize_\query}\right], \] where each sentence $\strmSent_i$
has a timestamp $\senttimestamp_i$ and $\senttimestamp_i \le
\senttimestamp_{i+1}$ for all $i \in \{1,\ldots,\strmSentSize_\query - 1\}$.
The two points of view are equivalent in the sense that the concatenation of
$\docstream^{(\query)}$ equals $\sentstream^{(\query)}$, \[  \strmDoc_1 \oplus
\strmDoc_2 \oplus \cdots \oplus \strmDoc_{\strmDocSize_\query} =
\sentstream^{(\query)} \] and $\doctimestamp_i = \senttimestamp_j$ for all
$\strmSent_j \in \strmDoc_i$.


A query focused, sentence extractive stream summarization model must process
the stream sequentially in time and determine for each sentence
$\strmSent_j\in \sentstream^{(\query)}$ whether to extract it
or to skip it. That is, the system must decide to add the sentence to a summary of the stream, or to ignore
the sentence. To indicate this decision, we associate with each sentence, $\strmSent_j$, a binary extraction
variable $\strmExtr_j \in \{0,1\}$, with $\strmExtr_j=1$ indicating the
sentence is to be extracted and included in the summary.  

Crucially, the summarization model has a system time $\systemtime_t$
which indicates what information from the stream has been read and can
be used to make extraction predictions. When the system time is $\systemtime_t$,
the summarization model can in theory use any information collected from all
documents $\strmDoc_i \in \docstream^{(\query)}$ such that $\doctimestamp_i \le \systemtime_t$ (for more significant query-streams, it may not be practical for a summarization model to store all previous documents). For any sentences not yet extracted, it can similary decide to extract any sentence $\strmSent_j \in \sentstream^{(\query)}$  
such that $\senttimestamp_j \le \systemtime_t$ (previous extraction decisions, however, cannot be undone). The system time can be incremented by arbitrary
positive amounts to $\systemtime_{t+1}$ (i.e. $\systemtime_t < \systemtime_{t+1}$) to allow the summarization model to observe and extract more sentences.
Given two timestamps $\timestamp_1,\timestamp_2$ with $\timestamp_1 < \timestamp_2$, we denote the sub-sequence of documents in the stream that fall
between them as $\docstream_{\timestamp_1:\timestamp_2}$. Similarly, 
$\sentstream_{\timestamp_1:\timestamp_2}$ indicates the subsequence of 
sentences with timestamps that fall between $\timestamp_1$ and $\timestamp_2$.

We refer to a sentence that has been extracted as an update. Let
$\strmupdate_k$ be the $k$-th update extracted by the system.  $\strmupdate_k$
has a corresponding timestamp, $\updatetimestamp_k$ that is equal to the
system time that it was extracted by the summarizer.  That is, if
$\strmupdate_k$ was extracted at $\systemtime_t$, then $\updatetimestamp_k =
\systemtime_t$. We refer to a collection of timestamped updates as an update
summary, $\updateSummary = \left[ \left(\strmupdate_1,
\updatetimestamp_1\right) ,\ldots,\left(\strmupdate_K, \updatetimestamp_K
\right)\right]$.

For evaluation purposes we compare the update summary to a human reference
abstract summary of the query event. The reference abstract summary,
$\strmNuggets^{(\query)} = \left[ \left(\strmnugget_1, \nugtimestamp_1\right),
\ldots, \left(\strmnugget_\nuggetSize, \nugtimestamp_{\nuggetSize}\right)
\right]$, contains a series of $\nuggetSize$ sentences describing important
facts about  the event. We refer to these facts as \textit{nuggets}. Each
nugget $\strmnugget_i$ has a timestamp, $\nugtimestamp_i$, indicating the
reference time that information became known.

We say an update $\strmupdate$ covers a nugget $\strmnugget$, which we write $\denotes{\strmnugget} \in \denotes{\strmupdate}$, if the information
in the nugget $\strmnugget$ is described in the update $\strmupdate$. Note
that it is possible for an update to cover multiple nuggets.
For example if we have,
\begin{align*}
    \strmnugget_1 & =  \textit{Hurricane Sandy was a category 5 hurricane.}\\
\strmnugget_2 & = \textit{Hurricane Sandy made landfall on Saturday.}\\
\strmupdate & = \textit{Hurricane Sandy, which made landfall on Saturday,
                was upgraded to a category 5 storm.}
\end{align*}
then $\denotes{\strmnugget_1},\denotes{\strmnugget_2} \in
\denotes{\strmupdate}$. Given an update summary $\updateSummary$ and a nugget
$\strmnugget$, we define the earliest cover, \[\earlymatch(\strmnugget,
\updateSummary) = \begin{cases} \argmin_{\left(\strmupdate_k,\updatetimestamp_k\right) \in \updateSummary :
\denotes{\strmnugget} \in \denotes{\strmupdate_k}} \updatetimestamp_k &
\textrm{if there exists a $\left(\strmupdate_k,\updatetimestamp_k\right) \in \updateSummary$ such that
$\denotes{\strmnugget} \in \denotes{\strmupdate}$} \\ \emptyset &
\textrm{otherwise} \end{cases}\] as the earliest update that covers the nugget
$\strmnugget$, or the empty set if no update in the update summary covers
$\strmnugget$. We also define the inverse mapping,
\[\invearlymatch(\strmupdate, \updateSummary, \strmNuggets) =
\left\{\left(\strmnugget_l, \nugtimestamp_l\right)\in \strmNuggets \Big\vert \strmupdate = \earlymatch(\strmnugget_l,
\updateSummary) \right\},\] which is the set of nuggets that have $\strmupdate$
as its earliest cover.



The objective of the summarization model is to produce an update summary
$\updateSummary$ such that the set of updates covers the information expressed
by the nuggets without containing much repeated information. Additionally,
the summarization system should try to minimize the latency between 
the time information in a nugget is available in the stream and the system
time that information is extracted. 

We now formalize these metrics.  Given an update and reference summary,
$\updateSummary$ and $\strmNuggets$ respectively, we define the gain of an
update as \[  \gain(\strmupdate_k,\updatetimestamp_k, \updateSummary, \strmNuggets) =  \sum_{
 \left(\strmnugget_j, \nugtimestamp_j\right) \in  \invearlymatch(\strmupdate_k,  \updateSummary, \strmNuggets)} 1 \] which is
essentially the number of nuggets covered by an update $\strmupdate$.  We also
define a latency penalized version of this function, \[
\gain_\latency\left(\strmupdate_i, \updateSummary,\strmNuggets \right) =
\sum_{
 \left(\strmnugget_j, \nugtimestamp_j\right) \in  \invearlymatch(\strmupdate_k,  \updateSummary, \strmNuggets)} \latency\left(\nugtimestamp_j, \updatetimestamp_i\right)  \]
where $\latency(\timestamp^*, \timestamp) = 1 - \frac{2}{\pi}\arctan\left(
\frac{\timestamp - \timestamp^*}{3600 * 6} \right) $ where
$\latency(\timestamp^*,\timestamp) = 1$ when $\timestamp=\timestamp^*$ and
gradually approaches 0 as $\timestamp - \timestamp^*$ increases. In real time,
the latency weighting
$\latency\left(\nugtimestamp_j,\updatetimestamp_i\right)$ aproaches 2 if
$\strmupdate_i$ covers $\strmnugget_j$ over 50 hours before $\nugtimestamp_j$;
$\latency\left(\nugtimestamp_j,\updatetimestamp_i\right)$ approaches 0 if
$\strmupdate_i$ covers $\strmnugget_j$ over 50 hours after $\nugtimestamp_j$.
 

The first metric we use to evaluate a summary is the normalized expected gain,
\[
n\mathbb{E}[\gain]\left(\updateSummary\right) = \frac{1}{Z\setsize{\updateSummary}} 
\sum_{\strmupdate_i \in \updateSummary} 
    \gain(\strmupdate_i, \updateSummary, \strmNuggets)\]
where $Z$ is maximum achievable expected gain (computed per query). 
This metric can be thought of as roughly analogous to precision.
We report a recall focused metric, called comprehensiveness, which
is defined as 
\[
\comp\left(\updateSummary\right) = \frac{1}{\setsize{\strmNuggets}} 
\sum_{\strmupdate_i \in \updateSummary} 
    \gain(\strmupdate_i, \updateSummary, \strmNuggets)\]
%
%\[ \comp(\updateSummary) = \frac{1}{\setsize{\strmNuggets}} \sum_{\left(\strmnugget_j, \nugtimestamp_j\right) \in \strmNuggets} \mathds{1}\left\{\earlymatch(\strmnugget_j, \updateSummary) \ne \emptyset \right\}\] 
which measures the percentage of
nuggets covered by the update summary. We also report the harmonic mean of normalized expected
gain and  comprehensiveness,
 \[ \mathcal{H}(\updateSummary) = 
2 \frac{n\mathbb{E}[\gain]\left(\updateSummary\right) \times \comp(\updateSummary)}
{  n\mathbb{E}[\gain]\left(\updateSummary\right) + \comp(\updateSummary)   }.
\]
A latency-penalized version of
normalized expected gain, comprehensiveness, and their harmonic mean 
can be obtained by replacing
$\gain$ with $\gain_\latency$ in the above calculations.











%?In our datasets, each query corresponds to a real world event that 
%?was  significant enough to have its own Wikipedia page. For evaluation
%?purposes we treat its lead paragraphs as a reference abstract summary.
%?What is helpful for this task is that a Wikipedia page evolves over time.
%?In the case of the query events in our datasets, these summaries evolved in
%?real time, reflecting changing conditions or statistics. Because all of the
%?Wikipedia 
%?
%?A human reference abstract summary of the  query
%?event is used  
%?
%?
