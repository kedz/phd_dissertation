\subsection{TREC 2014 Experiments and Results}

\begin{table}[t]
\centering
\begin{tabular}{cccccc}
\toprule
TeamID & RunID &$n\mathbb{E}[\gain](\updateSummary)$ & $\comp(\updateSummary)$ & $\comp_L(\updateSummary)$ & $\mathcal{H}_L(\updateSummary)$ \\
\midrule
\textbf{cunlp} & \textbf{2APSal} & 0.0631 & 0.3220 & 1.2068 & \textbf{0.1162} \\
BJUT & Q1 & \textbf{0.0657} & 0.4088 & 1.1491 & 0.1110\\
BJUT & Q2 & 0.0632 & 0.3979 & 1.1669 & 0.1091\\
BJUT & Q0 & 0.0632 & 0.3979 & 1.1669 & 0.1091\\
uogTr & uogTr2A & 0.0467 & 0.4453 & 1.2322 & 0.0986\\
uogTr & uogTr4AC & 0.0347 & 0.4539 & 1.2751 & 0.0793\\
uogTr & uogTr4ARas & 0.0387 & 0.3691 & 1.2328 & 0.0772\\
IRIT & KW30H5NW3600 & 0.0383 & 0.3521 & 1.2221 & 0.0723\\
IRIT & KW30H5NW300 & 0.0378 & 0.3538 & 1.2208 & 0.0714\\
uogTr & uogTr4A & 0.0281 & \textbf{0.4733} & 1.2522 & 0.0677\\
\midrule
\multicolumn{2}{c}{\textit{average}} & 0.0327 & 0.3615 & 1.2943 & 0.0620\\
\midrule
IRIT & KW80H5NW3600 & 0.0289 & 0.3764 & 1.2191 & 0.0604\\
IRIT & KW30H10NW300 & 0.0298 & 0.3780 & 1.2617 & 0.0602\\
\textbf{cunlp} & \textbf{1APSalRed} & 0.0325 & 0.3058 & 1.1507 & 0.0602\\
IRIT & KW80H5NW300 & 0.0285 & 0.3806 & 1.2164 & 0.0596\\
ICTNET & run3 & 0.0531 & 0.1081 & 0.7004 & 0.0530\\
BUPT\_PRIS & Cluster4 & 0.0155 & 0.2692 & 1.9140 & 0.0508\\
IRIT & KW80H10NW300 & 0.0225 & 0.4012 & 1.2621 & 0.0503\\
BUPT\_PRIS & Cluster3 & 0.0115 & 0.3380 & 1.9165 & 0.0407\\
\textbf{cunlp} & \textbf{3AP} & 0.0174 & 0.4265 & 1.3689 & 0.0403\\
ICTNET & run2 & 0.0418 & 0.0934 & 0.6266 & 0.0311\\
BUPT\_PRIS & Cluster2 & 0.0059 & 0.3728 & 1.9170 & 0.0222\\
ICTNET & run4 & 0.0079 & 0.4070 & 1.2364 & 0.0178\\
ICTNET & run1 & 0.0070 & 0.4090 & 1.2314 & 0.0160\\
BUPT\_PRIS & Cluster1 & 0.0033 & 0.4369 & \textbf{1.9175} & 0.0127\\
\bottomrule
\end{tabular}
\caption{Official TREC 2014 Temporal Summarization shared-task results using
manual update/nugget matches.}
\label{tab:trec2014}
\end{table}


We submitted three different model variations to the 2014 TREC Temporal
Summarization shared-task \citep{aslam2015}. The models were submitted under
the Team ID \textit{cunlp} and given
the following Run IDs:
\begin{enumerate}
\item 1APSalRed --- the SAP summarizer where salience predictions are penalized by similarity to prior updates. Let $\predbsal_k = f\left(\phi\left(\strmupdate_k,\query,\category\right)\right)$ for each update in $\updateSummary$. The 
salience estimate for a new sentence $\strmSent_i$ is 
$f\left(\phi\left(\strmSent_i,\query,\category\right)\right) - \sum_{\strmupdate_k, \updatetimestamp_k \in \updateSummary} \predbsal_k \operatorname{similarity}\left(\strmSent_i,\strmupdate_k \right)$. 
\item 2APSal --- the default SAP summarizer described in this section.
\item 3AP --- the summarizer with no salience estimation, all non-singleton
 exemplars are passed on to the redundancy filter and update selection
stage.
\end{enumerate}

We use the 10 TREC 2013 events to train/tune our submitted models.
3 of the events were selected at random as
a development set. All system salience and similarity threshold parameters are
tuned on the development set to maximize \textsc{Rouge}-2 F1 scores with
respect to the reference summaries $\strmNuggets^{(\query)}$.
We fit separate salience models using 1,000 sentences randomly sampled from
each of the 7 remaining TREC 2013 query events using its associated document stream.
When predicting the salience for a new sentence at test time, we use the
average prediction of all 7 models. 


Participant systems were run on 15 query events and associated document streams
 curated for the evaluation.
Updates from the participant systems were pooled and manually matched 
to reference nuggets by NIST assessors. These matches were used to compute
the  shared-task evaluation metrics (\autoref{sec:methods}):
normalized expected gain, $n\mathbb{E}[\gain](\updateSummary)$, comprehensiveness $\comp(\updateSummary)$, latency-penalized comprehensiveness $\comp_L(\updateSummary)$,
and the overall track evaluation measure, the harmonic mean $\mathcal{H}_L(\updateSummary)$ of the latency-penalized
$n\mathbb{E}[\gain](\updateSummary)$ and $\comp(\updateSummary)$ variants.

\input{strmsum/figures/outputs.tex}

To give a qualitative flavor of our produced summaries,
we show excerpts of two summaries produced by the 2APSal, i.e. SAP, model,
in \autoref{fig:sapsummaries}.
Results from the official TREC 2014 Temporal Summarization shared-task 
are shown in \autoref{tab:trec2014}. We see that our 2APSal
model had the highest overall $\mathcal{H}_L(\updateSummary)$. It was also one of the 
most precise models along with BJUT runs Q1, Q2, and Q0. The 2APSal and 
the BJUT runs return far fewer updates on average while still managing to 
include updates that matched to nuggets. The 2APSal run returned 
381.4 updates per query on average while the overall track average was 8528.55
updates per query on average. 

When we look at 3AP, which had no salience
componet, it had higer recall (i.e. comprehensiveness) than 2APSal 
but was much less 
precise, returning 5,967 updates on average. This suggests that the salience
estimation was important for guiding the summarizer. 

The performance of 1APSalRed was between 3AP and 2APSal. The redundancy
penalty meant that the average self-similarities were flatter than 
those of 2APSal
yielding more diverse updates, which were not as likely to be matched
to ground-truth nuggets, and therefore hurting the expected gain metrics.


All three of our proposed models had a latency-penalized comprehensiveness, $\comp_L(\updateSummary)$,
above $1$ ($1.1507 - 1.3689$). Values above $1$ indicate the average update that matched a nugget
was selected before that information made it into Wikipedia (and confusingly
getting a latency reward for early returns). In this case 
our proposed model was finding nugget information around 3-5 hours ahead of
its publication in Wikipedia.  

