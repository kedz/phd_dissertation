\section{Introduction}

In many cases, estimating salience is not the entirety of the summarization
system's task. Accounting for redundancy is also an important factor in many
summarization systems (especially multi-document summarization) since the same
information can often be restated multiple times. Additionally, in many
situations, salience is dynamic, changing over time with the information need
of the summary receiver.  In this chapter, we explore ways of encorporating
salience predictions into more holistic algorithms for constructing extract
summaries using information about text unit redundancy or prior extraction
decisions.  

Since this approach to summarization is not totally necessary for single
document summarization, we motivate the models in this chapter wth a more
difficult summarization challenge: query focused, sentence extractive,
streaming news summarization.  In this problem, the summarization system must
monitor a stream of news articles and extract sentences, which we call
updates, that are relevant to the user query. Collectively, these updates
constitute an update summary.  As in the last chapter, we rely a data-driven
assignment of update salience, where an update is salient if it contains
information that was found in a human authored summary of the query/document
stream. 

A notable aspect of the stream summarization task is the notion of system time
-- the summarization system can consider all sentences that have entered the
stream before the current system time. Advancing the system time allows the
summarizer to observe more sentences from the stream. However, the salience of
relevant information decreases monotonically from the earliest system time
that information was revealed to the final system time it was actually
extracted for the update summary.  Because of this, we must extract sentences
in an online fashion, attempting to minimize the latency between the time
important information is first revealed and the time the summarization system
extracts that information.

Since there is little supervised data for this task, we rely on a
feature-based regression model to provide our salience estimates.   The time
constraint makes things particularly challenging as the typical features for
summarization make use of static term frequency.  In the streaming case, these
features are now constantly evolving with time, and at the start of the
stream, estimates of term frequency may not be very reliable. 

A second but important issue is that salience estimates do not occur in
isolation. As we add updates to the summary, the salience of our remaining
inputs is likely to change based on redundancy and other factors.
Unfortunately, adding summary/sentence interaction features introduces an
element of exploration to training a salience estimation model for now various
summary configuration and candidate sentence pairs must be considered.

Our two proposed feature-based summarization models deal with these issues in
slightly different ways.  The first model, salience-biased affinity
propagation (SAP) \citep{kedzie2015}, combines independent, sentence-level
salience estimates with the affinity propagation clustering algorithm
\citep{frey2007clustering}. Affinity propagation forms clusters by identifying
a set of ``exemplar'' inputs and mapping the remaing inputs to one of the
exemplars. Under our modification of the clustering algorithm, we jointly
select exemplars that are individually highly salient but also representative
of the inputs. The resulting exemplars constitute updates in our update
summary.

Our second model, learning-to-summarize (L2S) \citep{kedzie2016}, allows us to
freely incorporate summary/sentence interaction features, as we train the
salience model using the learning-to-search regime \citep{daume2005,chang2015}
where learning takes place using different exploration policies. Using this
method we can learn a summarization policy that makes greedy sentence
extraction decisions that also correlate with a good final summary.  In the
next sections, we will introduce the query focused, sentence
extractive, streaming news summarization task and dataset, before discussing our proposed
SAP and L2S models.
