
\begin{figure}[p]
    \fbox{\resizebox{\textwidth}{!}{\begin{tikzpicture}[
    state/.style={fill,draw,circle,minimum height=1mm,inner sep=0.0mm},
    policy/.style={line width=0.5mm,circle,minimum height=2.25mm,
                   inner sep=0.0mm},
    rollin/.style={policy,draw=green},
    rolloutA/.style={policy,draw=purple},
    rolloutB/.style={policy,draw=orange},
    action/.style={line width=0.75mm},
    rolloutAline/.style={action,draw=purple},
    rolloutBline/.style={action,draw=orange},
    rollinline/.style={action,draw=green},
]

\def\WIDTH{0.32};
\def\TH{6}

\node at (0.1,\TH + 1.6) {\large \textbf{Sentence Stream}};
\node at (5,\TH + 1.6) {\large \textbf{Search Space}};
\node at (13,\TH + 1.6) {\large \textbf{Roll-out Action Sequences}};
\node at (-0.1,\TH + 1) 
    {$\sentstream_{\poestart:\poestop}$};
\node at (-0.1 ,\TH + 0.9) 
    {\uline{$\phantom{\sentstream^{(\query)}_{\timestamp_\starttok:\timestamp_\stoptok}}$}};

\foreach \n [
    % compute the number of nodes \numnodes from the line \n
    evaluate=\n as \numnodes using int(2^(\TH-1)/(2^(\n-1)), 
    % compute the (\n-1)th line.
    evaluate=\n as \nmo using int(\n-1)
] in {1,...,\TH} {

    
    \ifthenelse{\n=6}{}{
        \node at (-0.1,\TH - \n + 1) {$\ssent_\n$};
    }
    
    \foreach \x [
        evaluate=\x as \xcoord using \x*\WIDTH*2^(\n-1)-\WIDTH*2^(\n-1)/2,
        evaluate=\x as \lc using int(2*\x - 1),
        evaluate=\x as \rc using int(2*\x)
    ] in {1,...,\numnodes} {

        \node[state] (a\n\x) at (\xcoord,\n) {};
        \ifthenelse{\nmo=0}{}{
            \ifthenelse{\n=4 \AND \x=3}{
                \draw[draw=blue,dashed,line width=0.5mm] 
                    (a\n\x.south west) -- (a\nmo\lc.north);
                \draw[draw=red,dashed,line width=0.5mm] 
                    (a\n\x.south east) -- (a\nmo\rc.north);
            }{
                \draw (a\n\x.south west) -- (a\nmo\lc.north);
                \draw (a\n\x.south east) -- (a\nmo\rc.north);
            }
        }
    }

}

\node[rollin] (ri1) at (a61) {};
\node[rollin] (ri2) at (a52) {};
\node[rollin] (ri3) at (a43) {};
\draw[rollinline] (ri1.east) -- (ri2.north);
\draw[rollinline] (ri2.west) -- (ri3.north);
\node[rotate=-20] at ($(ri1)!0.5!(ri2)+(0,0.25)$) {\tiny extract};
\node[rotate=32] at ($(ri2)!0.5!(ri3)+(0,0.25)$) {\tiny skip};
\node[rolloutA] (roa4) at (a35) {};
\node[rolloutA] (roa5) at (a210) {};
\node[rolloutA] (roa6) at (a119) {};
\node[rolloutB] (rob4) at (a36) {};
\node[rolloutB] (rob5) at (a211) {};
\node[rolloutB] (rob6) at (a121) {};
\node at ($(ri1)+(0.25,0.25)$) {$\state_1$}; 
\node at ($(ri2)+(0.25,0.25)$) {$\state_2$}; 
\node at ($(ri3)+(-0.25,0.25)$) {$\state_3$}; 
\node[rotate=-73.5] at ($(roa4)!0.5!(roa5)+(0.25,0)$) {\tiny extract};
\node[rotate=84] at ($(roa5)!0.5!(roa6)+(-0.25,0)$) {\tiny skip};
\node[rotate=73.5] at ($(rob4)!0.5!(rob5)+(-.25,0)$) {\tiny skip};
\node[rotate=84] at ($(rob5)!0.5!(rob6)+(-0.25,0)$) {\tiny skip};
\draw[rolloutAline] (roa4.south east) -- (roa5.north);
\draw[rolloutAline] (roa5.south west) -- (roa6.north);
\draw[rolloutBline] (rob4.south west) -- (rob5.north);
\draw[rolloutBline] (rob5.south west) -- (rob6.north);

\def\COffset{36*\WIDTH}
\node at (\COffset, \TH+0.5) {$\uline{\roaseq{3}{0}\phantom{= 1}}$};
\node at (\COffset + 3, \TH+0.5) {$\uline{\roaseq{3}{1}\phantom{= 1}}$};

\def\SPACE{\phantom{\left(t,\hat{\bsal},\rolloutpolicy\right)}}
\foreach \ya/\yb [count=\t from 1] in {1/1, 0/0, 0/1, 1/0, 0/0} {
    \ifthenelse{
        \t<3
    }{ 
        \node (ya\t) at (\COffset, \TH-\t + 0.5) 
            {$\bsal^{\SPACE}_\t=\ya$};
        \node (yb\t) at (\COffset+3, \TH-\t + 0.5) 
            {$\bsal^{\SPACE}_\t=\yb$};
    }{
        \ifthenelse{
            \t=3                
        }{
            \node (ya\t) at (\COffset, \TH-\t + 0.5) 
                {$\predbsal^{\SPACE}_\t=\ya$};
            \node (yb\t) at (\COffset+3, \TH-\t + 0.5) 
                {$\predbsal^{\SPACE}_\t=\yb$};
        }{
            \node (ya\t) at (\COffset, \TH-\t + 0.5) 
                {$\roa{3}{0}_\t=\ya$};
            \node (yb\t) at (\COffset+3, \TH-\t + 0.5) 
                {$\roa{3}{1}_\t=\yb$};

        }
    }
}

\draw[draw=green,fill=green!20,line width=0.25mm] 
    (ya1.north west) rectangle (ya2.south east);
\draw[draw=green,fill=green!20,line width=0.25mm] 
    (yb1.north west) rectangle (yb2.south east);
\draw[draw=blue,fill=blue!20,line width=0.25mm] 
    (ya3.north west) rectangle (ya3.south east);
\draw[draw=red,fill=red!20,line width=0.25mm] 
    (yb3.north west) rectangle (yb3.south east);
\draw[draw=purple,fill=purple!20,line width=0.25mm] 
    (ya4.north west) rectangle (ya5.south east);
\draw[draw=orange,fill=orange!20,line width=0.25mm] 
    (yb4.north west) rectangle (yb5.south east);


%\node at (by1) {$\bsal_1^{\phantom{\left(3,0,\rolloutpolicy\right)}}=1$};
%\node at (by2) {$\bsal_2^{\phantom{\left(3,0,\rolloutpolicy\right)}}=0$};
%\node at (by3) {$\hat{\bsal}_3^{\phantom{\left(3,0,\rolloutpolicy\right)}}=1$};
%\node at (by4) {$\bsal^{\left(3,1,\rolloutpolicy\right)}_4=0$};
%\node at (by5) {$\bsal^{\left(3,1,\rolloutpolicy\right)}_5=0$};
%
%\node (ay1) at (36*\WIDTH, \TH-1+0.5) {$\bsal_1^{\phantom{\left(3,0,\rolloutpolicy\right)}}=1$};
%\node (ay2) at (36*\WIDTH, \TH-2+0.5) {$\bsal_2^{\phantom{\left(3,0,\rolloutpolicy\right)}}=0$};
%\node (ay3) at (36*\WIDTH, \TH-3+0.5) {$\hat{\bsal}_3^{\phantom{\left(3,0,\rolloutpolicy\right)}}=0$};
%\node (ay4) at (36*\WIDTH, \TH-4+0.5) {$\bsal^{\left(3,0,\rolloutpolicy\right)}_4=1$};
%\node (ay5) at (36*\WIDTH, \TH-5+0.5) {$\bsal^{\left(3,0,\rolloutpolicy\right)}_5=0$};

\foreach \ya/\yb [count=\t from 1] in {1/1, 0/0, 0/1, 1/0, 0/0} {
    \ifthenelse{
        \t<3
    }{ 
        \node (ya\t) at (\COffset, \TH-\t + 0.5) 
            {$\bsal^{\SPACE}_\t=\ya$};
        \node (yb\t) at (\COffset+3, \TH-\t + 0.5) 
            {$\bsal^{\SPACE}_\t=\yb$};
    }{
        \ifthenelse{
            \t=3                
        }{
            \node (ya\t) at (\COffset, \TH-\t + 0.5) 
                {$\predbsal^{\SPACE}_\t=\ya$};
            \node (yb\t) at (\COffset+3, \TH-\t + 0.5) 
                {$\predbsal^{\SPACE}_\t=\yb$};
        }{
            \node (ya\t) at (\COffset, \TH-\t + 0.5) 
                {$\roa{3}{0}_\t=\ya$};
            \node (yb\t) at (\COffset+3, \TH-\t + 0.5) 
                {$\roa{3}{1}_\t=\yb$};

        }
    }
}

\node[anchor=north west,align=left,text width=17.0cm] at (-1.5,0) {
\textbf{Computing Costs  $\cost(\state_3,0)$ and $\cost(\state_3,1)$}\\

~\\

\textbf{Step 1. Roll-in to $\state_3$ with policy $\policy$.}\\
Use policy $\policy$ to explore to state $\state_3$, shown in {\color{green}green} in the search space above.\\~\\  
\textbf{Step 2. Roll-out with $\rolloutpolicy$ to create action sequences
$\roaseq{3}{0}$ and $\roaseq{3}{1}$.}\\
For each action $\predbsal_3 \in \{0,1\},$ use $\rolloutpolicy$ to 
complete the roll-outs after having made action $\predbsal_3$ in state $\state_3$ (shown in {\color{purple}purple} and {\color{orange}orange} respectively) and create
alternative action sequences $\roaseq{3}{0}$ and $\roaseq{3}{1}$ (shown on 
the right).\\~\\
\textbf{Step 3. Compute losses.}\\
After completing the roll-outs, compute losses 
$\lossfunc\left(\roaseq{3}{0}\right)$ and $\lossfunc\left(\roaseq{3}{1}\right)$.\\~\\
\textbf{Step 4. Compute costs.}\\
Compute 
\[\cost(\state_3,0) = \lossfunc\left( \roaseq{3}{0} \right) - \min_{\bsal^\prime \in \{0,1\}} \lossfunc\left( \roaseq{3}{\bsal^\prime} \right)\]
and  
\[\cost(\state_3,1) = \lossfunc\left( \roaseq{3}{1} \right) - \min_{\bsal^\prime \in \{0,1\}} \lossfunc\left( \roaseq{3}{\bsal^\prime} \right).\]

};

\end{tikzpicture}}}
\caption{Example of computing costs of actions at $\state_3$ using roll-out
policy $\rolloutpolicy$.}
\label{fig:rollouts}
\end{figure}
