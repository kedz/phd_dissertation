\section{Learning-to-Search (L2S) Summarizer}

While our previous summarization model proved reasonably capable of summarizing
events over time, by processing the stream in hourly batches it was limited in
its ability to respond quickly to unfolding events. One reason for this
limitation is an implicit assumption in that model, and most multi-document
summarization models, that frequency of a unit of text is a proxy for its
salience. In retrospective summarization of static document collections, this
is a reasonable assumption, and has been exploited successfully in various
ways: TF-IDF term weightings, document language models derived from frequency,
and random-walks on sentence-graphs  whose edges are determined by frequently
co-occurrring terms
\citep{lin2000,radev2000,erkan2004,mihalcea2004,nenkova2005,daume2006}. 

In the streaming setting these proxy features are constantly evolving. There
may be fallow periods in an event where nothing happens and then sudden bursts
of activity. The behavior of SAP is unsuited for this, in that it is run every
hour regardless.  In the SAP model, by collecting an hour's worth of documents
and performing a salience-biased clustering, we try to walk the line between
using clustering, where the frequency of like text units are effectively votes
for the most salient unit, and predictions about salience from our regression
model, which makes more use of the semantics of the query event and text unit
under analysis.  However, as we showed in the feature ablation, incorporating
time-based frequency features made the model worse. While we are able to
incorporate salience estimate successfully into the summarization model with
SAP, we have yet to successfully provide a learning-based of model the entire
summarization process.

In this section, we attempt to overcome these limitations, removing the
clustering component from the update selection, and develop a fully online
streaming summarization model, one that learns to make extraction decisions
immediately upon seeing the next sentence from $\sentstream$, using features
derived from the entire observed document stream, the state of the current
update summary, and the model's prior extraction decisions.  To that end, we
present a novel streaming-document summarization system based on sequential
decision making.  Specifically, we adopt the ``learning to search'' approach, a
technique which adapts methods from reinforcement learning for structured
prediction problems \citep{daume2009,ross2010}.  In this framework, we cast
streaming summarization as a form of greedy search and train our system to
imitate the behavior of an oracle summarization system.  Effectively, we train
a linear classifier to predict when to extract a sentence $\ssent \in
\sentstream$ using features of the sentence, query, previous summary updates,
and other aggregate statistics collected from the stream up to the current time
$\systs_t$. 

As in the previous section,  we report both the TREC 2015 Temporal
Summarization shared-task manual evaluation as well as our own independent
automatic evaluation.  In our evaluation we demonstrate a 28.3\% improvement in
summary $\harm_\latency(\updateSummary)$ and a 43.8\% improvement in
time-sensitive $\harm_\latency(\updateSummary)$ metrics against several
state-of-the-art baselines.  In shared-task evaluation of our system at TREC
2015, where we were the second-best performing team in the main evaluation, and
top system for a secondary evaluation run on a pre-filtered, highly relevant
document stream.

\subsection{Stream Summarization as Sequential Decision Making}

We could na{\"i}vely treat the stream summarization problem as classification
and predict which sentences to extract or skip. However, this would make it
difficult to take advantage of many features (e.g. sentence novelty with regard
to previous updates).  What is more concerning, however, is that the
classification objective for this task is somewhat ill-defined: successfully
predicting extract on one sentence changes the true label (from extract to
skip) for sentences that contain the same information but occur later in the
stream.
 
In this work, we pose stream summarization as a greedy search over a binary
branching tree where each level corresponds to a position in the stream (see
\autoref{fig:search}).  The height of the tree corresponds to the length of
stream.  A path through the tree is determined by the system extract and skip
decisions.
 
\begin{figure}
  \center
\fbox{\begin{tikzpicture}[
    state/.style={fill,draw,circle,minimum height=1mm,inner sep=0.0mm},
    policy/.style={line width=0.5mm,circle,minimum height=2.25mm,
                   inner sep=0.0mm},
    rollin/.style={policy,draw=green},
    rolloutA/.style={policy,draw=purple},
    rolloutB/.style={policy,draw=orange},
    action/.style={line width=0.75mm},
    rolloutAline/.style={action,draw=purple},
    rolloutBline/.style={action,draw=orange},
    rollinline/.style={action,draw=green},
]

\def\WIDTH{0.32};
\def\TH{6}

\node at (5,\TH + 1.6) {\large \textbf{Search Space}};
\node at (-0.1,\TH + 1.6) {\large \textbf{Sentence Stream}};
\node at (-0.1,\TH + 1) 
    {$\sentstream_{\poestart:\poestop}$};
\node at (-0.1,\TH + 0.9) 
    {\uline{$\phantom{\sentstream^{(\query)}_{\timestamp_\starttok:\timestamp_\stoptok}}$}};

\foreach \n [
    % compute the number of nodes \numnodes from the line \n
    evaluate=\n as \numnodes using int(2^(\TH-1)/(2^(\n-1)), 
    % compute the (\n-1)th line.
    evaluate=\n as \nmo using int(\n-1)
] in {1,...,\TH} {

    
    \ifthenelse{\n=6}{}{
        \node at (-0.1,\TH - \n + 1) {$\ssent_\n$};
    }
    
    \foreach \x [
        evaluate=\x as \xcoord using \x*\WIDTH*2^(\n-1)-\WIDTH*2^(\n-1)/2,
        evaluate=\x as \lc using int(2*\x - 1),
        evaluate=\x as \rc using int(2*\x)
    ] in {1,...,\numnodes} {

        \node[state] (a\n\x) at (\xcoord,\n) {};
        \ifthenelse{\nmo=0}{}{
                \draw (a\n\x.south west) -- (a\nmo\lc.north);
                \draw (a\n\x.south east) -- (a\nmo\rc.north);
        }
    }

}
\node[rollin] (ri1) at (a61) {};
\node[rollin] (ri2) at (a52) {};
\node[rollin] (ri3) at (a43) {};
\node[rollin] (ri4) at (a35) {};
\node[rollin] (ri5) at (a29) {};
\node[rollin] (ri6) at (a118) {};

\foreach \n [
    % compute the (\n-1)th line.
    evaluate=\n as \nmo using int(\n-1)
] in {2,...,6} {
    \draw[rollinline] (ri\nmo.south) -- (ri\n.north);
};

\node[rotate=-20] at ($(ri1)!0.5!(ri2)+(0,0.25)$) {\tiny extract};
\node[rotate=32] at ($(ri2)!0.5!(ri3)+(0,0.25)$) {\tiny skip};
\node[rotate=45] at ($(ri3)!0.5!(ri4)+(-0.25,0.25)$) {\tiny skip};
\node[rotate=58] at ($(ri4)!0.5!(ri5)+(-0.25,0.25)$) {\tiny skip};
\node[rotate=-76] at ($(ri5)!0.5!(ri6)+(+0.25,0.0)$) {\tiny extract};

\node at ($(ri1)+(0.25,0.25)$) {$\state_1$}; 
\node at ($(ri2)+(0.25,0.25)$) {$\state_2$}; 
\node at ($(ri3)+(-0.25,0.25)$) {$\state_3$}; 
\node at ($(ri4)+(-0.25,0.25)$) {$\state_4$}; 
\node at ($(ri5)+(+0.25,0.25)$) {$\state_5$}; 
\node at ($(ri6)+(-0.25,-.25)$) {$\state_6$}; 
%\draw[rollinline] (ri2.west) -- (ri3.north);
\end{tikzpicture}}


\caption{Search space for a stream of 5 sentences.  Left branches skip the
current sentence.  Right branches indicate extracting the current sentence as
an update.  The path in green corresponds to one trajectory through this space
consisting of extracting sentence $\ssent_1$, then skipping sentences
$\ssent_2,\ldots,\ssent_4$ and extracting sentence $\ssent_5$.}
  \label{fig:search}
\end{figure}



When treated as a sequential decision making problem, our task reduces to
defining a policy for selecting a sentence based on its features as well as the
features of its ancestors (i.e., all of the observed sentences and previous
extraction decisions).  The union of these features represents the current
state in the decision making process. 

The feature representation provides state abstraction both within a given
query's search tree as well as to states in other queries' search trees, and
also allows for complex interactions between the current update summary,
candidate sentences, and stream dynamics unlike the classification approach.

In order to learn an effective policy for a query $\query$, we can take one of
several approaches.  We could use a simulator to provide feedback to a
reinforcement learning algorithm.  Alternatively, if provided access to an
evaluation algorithm at training time, we can simulate (approximately) optimal
decisions. That is, using the training data, we can define an \emph{oracle
policy} that is able to omnisciently determine which sentences to extract and
which to skip.  Moreover, it can make these determinations by starting at the
root or at an arbitrary node in the tree, allowing us to observe optimal
performance in states unlikely to be reached by the oracle.  We adopt
\textit{locally optimal learning to search} to learn our model from the oracle
policy \citep{chang2015}.  

\subsection{Policy-based Stream Summarization}

In the induced search problem, each search state $\state_t \in \states$
corresponds to observing the first $t$ sentences in the stream $\left[
\ssent_1,\ldots, \ssent_t \right] \subset \sentstream$ and a sequence of $t-1$
actions $\bsal_1, \ldots, \bsal_{t-1}$.  For each state $\state_t \in \states$,
the set of actions is $\bsal_t \in \{0,1\}$ with $\bsal_t =1$ indicating we
extract the $t$-th sentence and add it to our update summary, and $\bsal_t = 0$
indicating we skip it. For simplicity of exposition, we assume a fixed length
stream of size $T$ but this is not strictly necessary. 

\begin{algorithm}[t]
  \RestyleAlgo{boxruled}
  \LinesNumbered
  \KwIn{
    query string $\query$, 
    query category $\category$, 
    stream $\sentstream$, 
    period of interest ($\poestart$, $\poestop$), 
    summarization policy $\policy$}
  \KwOut{update summary $\updateSummary$}
  $\updateSummary \gets \{\}$

\For{$\ssent_t \in \sentstream_{\poestart:\poestop}$}{
    $\state_t \gets (\ssent_1,\ldots,\ssent_t, \bsal_1,\ldots,\bsal_{t-1},
                     \query,\category)$\\
    $\bsal_t \gets \policy(\state_t)$\\
    \If{$\bsal_t = 1$}{
        $\updateSummary \gets \updateSummary \cup \left\{\left(\ssent_t, \sentts_t \right) \right\} $
        }
    }
  \caption{Policy-based Stream Summarization} 
  \label{alg:policysum}
\end{algorithm}




A policy, $\policy : \states \rightarrow \{0,1\}$, is a mapping from states to
an extraction decision. Given a policy, the policy-based stream summarization
algorithm (\autoref{alg:policysum}) is trivial, iterating sequentially over
sentences in the stream, and adding each sentence to the update summary if it
is the current action determined by $\policy$.  In practice, we use a linear
cost-sensitive classifier to implement $\policy_\params$, with \[ \bsal_t =
\policy_\params(\state_t) = \argmin_{\bsal \in \{0,1\}} \Feat(\state_t,
\bsal)\cdot \params_\bsal\] where we encode each state-potential action pair
$(\state_t,\bsal)$ as a $d$-dimensional feature vector  $\Feat(\state_t, \bsal)
\in \mathbb{R}^d$ and $\params_0, \params_1 \in \reals^{d}$ are learned
parameters.  Note that $\Feat(\state_t,\bsal)\cdot \params_\bsal$ is a linear
regression to predict the cost, which we define shortly, associated with taking
action $\bsal$ in state $\state_t$. Given estimates of our two available
actions, extracting a sentence or ignoring it, our policy $\policy_\policy$
selects the action with minimum cost.

Before we can define cost more concretely, we must first introduce some
additional notation and concepts.  For a given sequence of states
$\state_1,\ldots,\state_T$ explored by a policy $\policy$, let $\bsals=
\left[\bsal_1,\ldots,\bsal_T\right]$ be the associated sequence of actions
taken by $\policy$, i.e. $\bsal_t = \policy(\state_t)$.  A loss function,
$\lossfunc : \{0,1\}^T \rightarrow \reals$, measures the quality of an action
sequence $\bsals$. In our present case this might be the negative
\textsc{Rouge} score of the update summary that results from $\bsals$ or some
other relevant measure of performance. 

Let $\policy$ be a policy, $\state_1,\ldots,\state_T$ a sequence of states
explored by $\policy$, and the corresponding action sequence $\bsals$. We also
define a second policy, $\rolloutpolicy$, which we call the roll-out policy and
which may or may not be distinct from $\policy$.  For any state $\state_t$ we
can then define two additional action sequences,
\[
    \roaseqt = \left[
        \bsal_1, 
        \ldots, 
        \bsal_{t-1},
        \hat{\bsal}_t,
        \roat_{t+1}, 
        \ldots, 
        \roat_{T}, 
    \right] \quad \forall \predbsal_t \in \{0,1\} 
\] 
where $\roaseqt$ is the sequence of actions that result from following
$\policy$ on the first $t-1$ states, taking action $\predbsal_t$ at state
$\state_t$ and then following $\rolloutpolicy$ on the remaining $T - t$ states.
That is, for $k > t$, $\roa_{k} = \rolloutpolicy(\state^\prime_k)$, where
$\state^\prime_k$ is the state that results from taking action $\predbsal_t$ in
$\state_t$ and following $\rolloutpolicy$ on a sequence of states
$\state^\prime_{t+1}, \ldots, \state^\prime_k$. The loss, $\lossfunc\left(
\roaseq{t}{0} \right)$, then reflects the evaluation measure for an update
summary where sentence $\ssent_t$ was not extracted, and $\rolloutpolicy$
completed the summary of the stream. Similarly, $\lossfunc\left( \roaseq{t}{1}
\right)$ reflects the evaluation measure for an update summary where sentence
$\ssent_t$ was extracted, and $\rolloutpolicy$ completed the summary of the
stream. 
 
We can now define the cost of an action $\bsal$ in state $\state_t$ as 
\[ 
    \cost(\state_t, \bsal) = 
        \lossfunc\left(\roaseq{t}{\bsal}   \right) 
            - \min_{\bsal^\prime \in \{0,1\} }
                \lossfunc\left( \roaseq{t}{\bsal^\prime} \right).
\]
Note that the cost is also a function of $\rolloutpolicy$, which determines how
the action sequence is completed after $\state_t$.  The costs connect the
overall summary loss $\lossfunc\left(\roaseqt\right)$ to a particular action in
$\state_t$ that builds the summary.  We depict an example of the cost
computation in \autoref{fig:rollouts}.  We discuss how to collect costs and
learn $\params_0$ and $\params_1$ such that they are good estimators of cost in
the next section.


\begin{figure}[p]
    \fbox{\resizebox{\textwidth}{!}{\begin{tikzpicture}[
    state/.style={fill,draw,circle,minimum height=1mm,inner sep=0.0mm},
    policy/.style={line width=0.5mm,circle,minimum height=2.25mm,
                   inner sep=0.0mm},
    rollin/.style={policy,draw=green},
    rolloutA/.style={policy,draw=purple},
    rolloutB/.style={policy,draw=orange},
    action/.style={line width=0.75mm},
    rolloutAline/.style={action,draw=purple},
    rolloutBline/.style={action,draw=orange},
    rollinline/.style={action,draw=green},
]

\def\WIDTH{0.32};
\def\TH{6}

\node at (0.1,\TH + 1.6) {\large \textbf{Sentence Stream}};
\node at (5,\TH + 1.6) {\large \textbf{Search Space}};
\node at (13,\TH + 1.6) {\large \textbf{Roll-out Action Sequences}};
\node at (-0.1,\TH + 1) 
    {$\sentstream_{\poestart:\poestop}$};
\node at (-0.1 ,\TH + 0.9) 
    {\uline{$\phantom{\sentstream^{(\query)}_{\timestamp_\starttok:\timestamp_\stoptok}}$}};

\foreach \n [
    % compute the number of nodes \numnodes from the line \n
    evaluate=\n as \numnodes using int(2^(\TH-1)/(2^(\n-1)), 
    % compute the (\n-1)th line.
    evaluate=\n as \nmo using int(\n-1)
] in {1,...,\TH} {

    
    \ifthenelse{\n=6}{}{
        \node at (-0.1,\TH - \n + 1) {$\ssent_\n$};
    }
    
    \foreach \x [
        evaluate=\x as \xcoord using \x*\WIDTH*2^(\n-1)-\WIDTH*2^(\n-1)/2,
        evaluate=\x as \lc using int(2*\x - 1),
        evaluate=\x as \rc using int(2*\x)
    ] in {1,...,\numnodes} {

        \node[state] (a\n\x) at (\xcoord,\n) {};
        \ifthenelse{\nmo=0}{}{
            \ifthenelse{\n=4 \AND \x=3}{
                \draw[draw=blue,dashed,line width=0.5mm] 
                    (a\n\x.south west) -- (a\nmo\lc.north);
                \draw[draw=red,dashed,line width=0.5mm] 
                    (a\n\x.south east) -- (a\nmo\rc.north);
            }{
                \draw (a\n\x.south west) -- (a\nmo\lc.north);
                \draw (a\n\x.south east) -- (a\nmo\rc.north);
            }
        }
    }

}

\node[rollin] (ri1) at (a61) {};
\node[rollin] (ri2) at (a52) {};
\node[rollin] (ri3) at (a43) {};
\draw[rollinline] (ri1.east) -- (ri2.north);
\draw[rollinline] (ri2.west) -- (ri3.north);
\node[rotate=-20] at ($(ri1)!0.5!(ri2)+(0,0.25)$) {\tiny extract};
\node[rotate=32] at ($(ri2)!0.5!(ri3)+(0,0.25)$) {\tiny skip};
\node[rolloutA] (roa4) at (a35) {};
\node[rolloutA] (roa5) at (a210) {};
\node[rolloutA] (roa6) at (a119) {};
\node[rolloutB] (rob4) at (a36) {};
\node[rolloutB] (rob5) at (a211) {};
\node[rolloutB] (rob6) at (a121) {};
\node at ($(ri1)+(0.25,0.25)$) {$\state_1$}; 
\node at ($(ri2)+(0.25,0.25)$) {$\state_2$}; 
\node at ($(ri3)+(-0.25,0.25)$) {$\state_3$}; 
\node[rotate=-73.5] at ($(roa4)!0.5!(roa5)+(0.25,0)$) {\tiny extract};
\node[rotate=84] at ($(roa5)!0.5!(roa6)+(-0.25,0)$) {\tiny skip};
\node[rotate=73.5] at ($(rob4)!0.5!(rob5)+(-.25,0)$) {\tiny skip};
\node[rotate=84] at ($(rob5)!0.5!(rob6)+(-0.25,0)$) {\tiny skip};
\draw[rolloutAline] (roa4.south east) -- (roa5.north);
\draw[rolloutAline] (roa5.south west) -- (roa6.north);
\draw[rolloutBline] (rob4.south west) -- (rob5.north);
\draw[rolloutBline] (rob5.south west) -- (rob6.north);

\def\COffset{36*\WIDTH}
\node at (\COffset, \TH+0.5) {$\uline{\roaseq{3}{0}\phantom{= 1}}$};
\node at (\COffset + 3, \TH+0.5) {$\uline{\roaseq{3}{1}\phantom{= 1}}$};

\def\SPACE{\phantom{\left(t,\hat{\bsal},\rolloutpolicy\right)}}
\foreach \ya/\yb [count=\t from 1] in {1/1, 0/0, 0/1, 1/0, 0/0} {
    \ifthenelse{
        \t<3
    }{ 
        \node (ya\t) at (\COffset, \TH-\t + 0.5) 
            {$\bsal^{\SPACE}_\t=\ya$};
        \node (yb\t) at (\COffset+3, \TH-\t + 0.5) 
            {$\bsal^{\SPACE}_\t=\yb$};
    }{
        \ifthenelse{
            \t=3                
        }{
            \node (ya\t) at (\COffset, \TH-\t + 0.5) 
                {$\predbsal^{\SPACE}_\t=\ya$};
            \node (yb\t) at (\COffset+3, \TH-\t + 0.5) 
                {$\predbsal^{\SPACE}_\t=\yb$};
        }{
            \node (ya\t) at (\COffset, \TH-\t + 0.5) 
                {$\roa{3}{0}_\t=\ya$};
            \node (yb\t) at (\COffset+3, \TH-\t + 0.5) 
                {$\roa{3}{1}_\t=\yb$};

        }
    }
}

\draw[draw=green,fill=green!20,line width=0.25mm] 
    (ya1.north west) rectangle (ya2.south east);
\draw[draw=green,fill=green!20,line width=0.25mm] 
    (yb1.north west) rectangle (yb2.south east);
\draw[draw=blue,fill=blue!20,line width=0.25mm] 
    (ya3.north west) rectangle (ya3.south east);
\draw[draw=red,fill=red!20,line width=0.25mm] 
    (yb3.north west) rectangle (yb3.south east);
\draw[draw=purple,fill=purple!20,line width=0.25mm] 
    (ya4.north west) rectangle (ya5.south east);
\draw[draw=orange,fill=orange!20,line width=0.25mm] 
    (yb4.north west) rectangle (yb5.south east);


%\node at (by1) {$\bsal_1^{\phantom{\left(3,0,\rolloutpolicy\right)}}=1$};
%\node at (by2) {$\bsal_2^{\phantom{\left(3,0,\rolloutpolicy\right)}}=0$};
%\node at (by3) {$\hat{\bsal}_3^{\phantom{\left(3,0,\rolloutpolicy\right)}}=1$};
%\node at (by4) {$\bsal^{\left(3,1,\rolloutpolicy\right)}_4=0$};
%\node at (by5) {$\bsal^{\left(3,1,\rolloutpolicy\right)}_5=0$};
%
%\node (ay1) at (36*\WIDTH, \TH-1+0.5) {$\bsal_1^{\phantom{\left(3,0,\rolloutpolicy\right)}}=1$};
%\node (ay2) at (36*\WIDTH, \TH-2+0.5) {$\bsal_2^{\phantom{\left(3,0,\rolloutpolicy\right)}}=0$};
%\node (ay3) at (36*\WIDTH, \TH-3+0.5) {$\hat{\bsal}_3^{\phantom{\left(3,0,\rolloutpolicy\right)}}=0$};
%\node (ay4) at (36*\WIDTH, \TH-4+0.5) {$\bsal^{\left(3,0,\rolloutpolicy\right)}_4=1$};
%\node (ay5) at (36*\WIDTH, \TH-5+0.5) {$\bsal^{\left(3,0,\rolloutpolicy\right)}_5=0$};

\foreach \ya/\yb [count=\t from 1] in {1/1, 0/0, 0/1, 1/0, 0/0} {
    \ifthenelse{
        \t<3
    }{ 
        \node (ya\t) at (\COffset, \TH-\t + 0.5) 
            {$\bsal^{\SPACE}_\t=\ya$};
        \node (yb\t) at (\COffset+3, \TH-\t + 0.5) 
            {$\bsal^{\SPACE}_\t=\yb$};
    }{
        \ifthenelse{
            \t=3                
        }{
            \node (ya\t) at (\COffset, \TH-\t + 0.5) 
                {$\predbsal^{\SPACE}_\t=\ya$};
            \node (yb\t) at (\COffset+3, \TH-\t + 0.5) 
                {$\predbsal^{\SPACE}_\t=\yb$};
        }{
            \node (ya\t) at (\COffset, \TH-\t + 0.5) 
                {$\roa{3}{0}_\t=\ya$};
            \node (yb\t) at (\COffset+3, \TH-\t + 0.5) 
                {$\roa{3}{1}_\t=\yb$};

        }
    }
}

\node[anchor=north west,align=left,text width=17.0cm] at (-1.5,0) {
\textbf{Computing Costs  $\cost(\state_3,0)$ and $\cost(\state_3,1)$}\\

~\\

\textbf{Step 1. Roll-in to $\state_3$ with policy $\policy$.}\\
Use policy $\policy$ to explore to state $\state_3$, shown in {\color{green}green} in the search space above.\\~\\  
\textbf{Step 2. Roll-out with $\rolloutpolicy$ to create action sequences
$\roaseq{3}{0}$ and $\roaseq{3}{1}$.}\\
For each action $\predbsal_3 \in \{0,1\},$ use $\rolloutpolicy$ to 
complete the roll-outs after having made action $\predbsal_3$ in state $\state_3$ (shown in {\color{purple}purple} and {\color{orange}orange} respectively) and create
alternative action sequences $\roaseq{3}{0}$ and $\roaseq{3}{1}$ (shown on 
the right).\\~\\
\textbf{Step 3. Compute losses.}\\
After completing the roll-outs, compute losses 
$\lossfunc\left(\roaseq{3}{0}\right)$ and $\lossfunc\left(\roaseq{3}{1}\right)$.\\~\\
\textbf{Step 4. Compute costs.}\\
Compute 
\[\cost(\state_3,0) = \lossfunc\left( \roaseq{3}{0} \right) - \min_{\bsal^\prime \in \{0,1\}} \lossfunc\left( \roaseq{3}{\bsal^\prime} \right)\]
and  
\[\cost(\state_3,1) = \lossfunc\left( \roaseq{3}{1} \right) - \min_{\bsal^\prime \in \{0,1\}} \lossfunc\left( \roaseq{3}{\bsal^\prime} \right).\]

};

\end{tikzpicture}}}
\caption{Example of computing costs of actions at $\state_3$ using roll-out
policy $\rolloutpolicy$.}
\label{fig:rollouts}
\end{figure}


\subsection{The Locally-Optimal Learning-to-Search Algorithm}
\label{sec:algorithm}

In a perfect world, we would have a training set of state-action pairs,
$(\state_t,\bsal_t)$, and their associated costs $\cost(\state_t, \bsal_t)$,
drawn from a distribution of states similar to the one our final learned policy
would produce.  With these states, actions, and costs in hand we could fit two
linear regressions, one for estimating the cost of extract actions and one for
skip actions from a given state's feature representation; we would immediately
have a suitable stream summarization policy. Unfortunately, we do not have a
reasonable distribution of state-action pairs let alone the associated costs.
Instead we turn to the locally-optimal learning-to-search (LOLS) algorithm
\citep{chang2015}, presented in \autoref{fig:lols}, to iteratively refine our
learned policy as it attempts to follow an oracle policy. 

At a high level, we begin with an initially random learned policy.  We leverage
a heuristic oracle stream summarizer and our learned policy to collect
state-action pairs and their costs from the training set query-streams, with
our learned policy gradually learning to imitate the heuristic oracle. Using
both the oracle and learned policy to sample state-action-cost triples is
beneficial as it exposes the learned policy to a mix of states it is likely to
encounter when following its own actions, but also state-actions of the better
performing reference summarizer. Exploring with the learned policy alone may be
less optimal because it may overestimate the costs of good decisions since its
roll-outs at the beginning of training are likely to be quite bad.  Exploring
with only the oracle would also be harmful as the distribution of states it
produces will reflect an optimistic level of performance from learned policy
that it will not be able to match in practice.

\subsubsection{Oracle Policy}

For a given policy $\policy$, training query $\query$, stream $\sentstream$,
and reference summary $\snuggets$, we construct a greedy heuristic oracle
policy $\oracle_\query$.  Let $\updateSummary_t$ be the update summary at state
$\state_t$ reached by $\policy$. Additionally, let
\[
\snuggets_{\denotes{\updateSummary_t}} = 
\left\{\snugget_i \Bigg\vert
        \left(\snugget_i,\nts_i\right) \in \snuggets,
         \exists \left(\supdate_j, \uts_j\right) \in \updateSummary_t : \denotes{\snugget_i} \in
        \denotes{\supdate_j}
       \right\}
\]
be the set of nuggets covered by the updates in the update summary at
$\state_t$, and let
\[
\snuggets_{\denotes{\ssent_t}} = 
\left\{\snugget_i \Bigg\vert
        \left(\snugget_i,\nts_i\right) \in \snuggets,
        \denotes{\snugget_i} \in
        \denotes{\ssent_t}
       \right\}
\]
be the set of nuggets covered by $\ssent_t$.  The oracle $\oracle_\query$,
which has clairvoyant knowledge of these sets, performs the following actions
given a state $\state_t$,
\[
    \oracle_\query(\state_t) = \begin{cases} 
        1 & \textrm{if }  
        \exists \snugget : \snugget \in \snuggets_{\denotes{\ssent_t}} \wedge 
   \snugget \not\in \snuggets_{\denotes{\updateSummary_t} }
     \\
     0 & \textrm{otherwise.} \end{cases}
\]
In other words, the oracle policy will extract $\ssent_t$ at state $\state_t$
if $\ssent_t$ covers any nuggets not yet covered by updates in the update
summary $\updateSummary_t$. We describe how we obtain comprehensive
nugget/sentence covers, i.e. the sets $\snuggets_{\denotes{\ssent_t}}$ in
\autoref{sec:reljudge}.

\subsubsection{Loss Function}
We design our loss function to penalize policies that  over- or under-generate.
Let $\state_1, \ldots, \state_T$ be the sequence of states associated with the
action sequence $\bsals$.  We define the loss of a sequence as 
\begin{align*}
  \lossfunc(\bsals) = 1 - 2 \times 
    \frac{ \bsals^\T \bsals^* }{ \sum^T_{i=1} \bsal_i + \sum^T_{j=1}\bsal_j^* }
\end{align*}
where $\bsals^* = \left[\bsal_1^*,\ldots, \bsal_T^* \right]$ is a reference
sequence of consisting of the one-step optimal deviations under the oracle
policy, i.e. $\bsal_t^* = \oracle_\query(\state_t)$.

Note that $\lossfunc$ is the complement of the Dice coefficient.  This
encourages not only local agreement between policies (the numerator of the
second term) but that the learned and oracle policy should generate roughly the
same number of updates (the denominator in the second term).

\begin{algorithm}[t]
  \RestyleAlgo{boxruled}
  \LinesNumbered
  \KwIn{training dataset of query-relevant streams and oracle policies $\{\sentstream_\query, \oracle_{\query} \}_{\query\in\queries}$, number of 
        training epochs $N$, learning rate $\lambda$, and a mixture parameter $\beta \in (0,1)$ for 
        selecting a roll-out policy.}
  \KwOut{learned summarization policy $\policy_\params$}
  Initialize $\params$ \label{alg:lolsinit}\\
  \For{$n \gets 1,2,\ldots, N$}{
    \For{$\query \in \queries$}{
      $\Train \gets \{\}$\\    
      \For{$t \in \{1,\ldots,T \}$}{
        Roll-in by executing $\policy_\params$ on $\sentstream_\query$ for $t-1$ rounds and reach state $s_t$. \label{alg:lolsrollin} \\
        \For{$y \in \{0,1\}$}{
            Pick roll-out policy $\rolloutpolicy \gets \begin{cases} \oracle_{\query} & \textrm{with probability $\beta$}\\ \policy_\params & \textrm{with probability $1 -\beta$.}\end{cases}$. \label{alg:lolsrollout}\\
            Compute roll-out action sequence $\bsals^{(t,\bsal,\rolloutpolicy)}$. \label{alg:lolsaction}\\
        }      
        \For{$y \in \{0,1\}$}{
            $\cost(\state_t, \bsal) \gets \ell\left(\bsals^{\left(t,\bsal,\rolloutpolicy\right)}\right) - \min_{\bsal^\prime \in \{0,1\}} \ell\left(\bsals^{\left(t,\bsal^\prime,\rolloutpolicy\right)}\right) $ \label{alg:lolscoststart}  \\
            $\Train \gets \Train \cup \left\{\left( \Feat(\state_t,\bsal),
            c(\state_t, \bsal), \bsal\right)\right\}$ \label{alg:lolscoststop}  }}
    \For{$\left(\Feat(\state,\bsal),
            c(\state, \bsal), \bsal\right) \in \Train$ \label{alg:lolsmsestart} }{
        $\params_\bsal \gets \params_\bsal - \lambda \nabla_{\params_\bsal}(c(\state, \bsal) - \Feat(\state,\bsal)\cdot \params_\bsal  )^2$ \label{alg:lolsmsestop}
}
}}
  \KwResult{$\policy_\params$}
  \caption{Locally Optimal Learning-to-Search for Stream Summarization} 
  \label{fig:lols}
\end{algorithm}




\subsubsection{Learning with LOLS}

We now step through the LOLS algorithm in detail.  Our initial policy
parameters $\params_0$ and $\params_1$ are randomly initialized
(\hyperref[fig:lols]{Algorithm~\ref{fig:lols} line \ref{alg:lolsinit}}).  The
LOLS algorithm works by iteratively using the current learned policy
$\policy_\params$ to explore state sequences from a training stream
$\sentstream^{(\query)}$ (\hyperref[fig:lols]{Algorithm~\ref{fig:lols} line
\ref{alg:lolsrollin}}).  At each state $\state_t$, a roll-out policy
$\rolloutpolicy$ is selected at random from $\policy_\params$ and the oracle
$\oracle_\query$ (\hyperref[fig:lols]{Algorithm~\ref{fig:lols} line
\ref{alg:lolsrollout}}).  Losses are then computed for each continuation
sequence $\roaseq{t}{0}$ and $\roaseq{t}{1}$
(\hyperref[fig:lols]{Algorithm~\ref{fig:lols} line \ref{alg:lolsaction}}).
Costs for each action are then computed and state features and costs are cached
in $\Train$ (\hyperref[fig:lols]{Algorithm~\ref{fig:lols} lines
\ref{alg:lolscoststart}--\ref{alg:lolscoststop}}).  After the roll-in has
explored all $T$ states, we update the policy parameters $\params$ by
performing gradient descent on the squared error of the linear regressor on the
cached feature-cost pairs in $\Train$
(\hyperref[fig:lols]{Algorithm~\ref{fig:lols} lines
\ref{alg:lolsmsestart}--\ref{alg:lolsmsestop}}).

\subsection{Features}

As mentioned in the previous section, we represent each state as a feature
vector.  In general, at state $\state_t$, these features are functions of the
current sentence $\ssent_t$, the  stream history $\ssent_1, \ldots,
\ssent_{t-1}$, the query string $\query$ and category $\category$, and/or the
decision history $\bsal_1,\ldots,\bsal_{t-1}$.  We refer to features only
determined by $\ssent_t$,  $\query$, and $\category$ as static features and all
others as dynamic features. 

\subsubsection{Static Features}

\textbf{Basic Features } Our most basic features look at the length in words of
a sentence, its position in the document, and the ratio of specific named
entity tags to non-named entity tokens.  We also compute the average number of
sentence tokens that match the event query words and synonyms using WordNet.

\textbf{Language Model Features } Similar to \autoref{sec:sap}, we compute the
average token log probability of the sentence on two language models: i) an
event category specific language model and ii) a general newswire language
model.  The first language model is built from Wikipedia articles relevant to
the event-type domain. The second model is built from the New York Times and
Associate Press sections of the Gigaword-5 corpus \citep{graff2003english}.

\textbf{Single Document Summarization Features } These features are computed
using the current sentence's document as a context and are also commonly used
as ranking features in other document summarization systems. Where a sentence
representation is needed, we use both TF-IDF bag-of-words representation as
well as the latent vector representation under the WMF method described in
\autoref{sec:salpred}.

We compute SumBasic features \citep{nenkova2005}: the average and sum of
unigram probabilities in a sentence.  We compute the arithmetic and geometric
means of the sentence's cosine distance to the other sentences of the document
\citep{guo2013}. We refer to this quantity as novelty and compute it with both
vector representations. We also compute the centroid rank \citep{radev2000} and
LexRank of each sentence \citep{erkan2004}, again using both vector
representations.

\textbf{Summary Content Probability} For a subset of the stream sentences, we
have manual judgements as to whether they match to reference summary nuggets or
not. We use this data (restricted to sentences from the training query
streams), to train a decision tree classifier, using the sentences' term ngrams
as the classifier features. As this data is aggregated across the training
queries, the purpose of this classifier is to capture the importance of general
ngrams predictive of summary worthy content.  Using this classifier, we obtain
the probability that the current sentence $\strmSent_t$ contains summary
content and use this as a model feature.

\subsubsection{Dynamic Features}

\textbf{Stream Language Models} We maintain several unigram language models
that are updated with each new document in the stream. Using these counts, we
compute the sum, average, and maximum token probability of the non-stop words
in the sentence. We compute similar quantities restricted to \textit{person},
\textit{location}, and \textit{organization} named entities as well.

\textbf{Update Similarity} The average and maximum cosine similarity of the
current sentence to all previous updates is computed under both the TF-IDF
bag-of-words and latent vector representation. We also include indicator
features for when the set of updates is empty (i.e. at the beginning of a run)
and when either similarity is 0.

\textbf{Document Frequency} We also compute the hour-to-hour percent change in
document frequency of the stream. This feature helps gauge breaking
developments in an unfolding event. As this feature is also heavily affected by
the daily news cycle (larger average document frequencies in the morning and
evening) we compute the 0-mean/unit-variance of this feature using the training
streams to find the mean and variance for each hour of the day.

\textbf{Feature Interactions} Many of our features are helpful for determining
the importance of a sentence with respect to its document.  However, they are
more ambiguous for determining importance to the event as a whole. For example,
it is not clear how to compare the document level PageRank of sentences from
different documents. To compensate for this, we leverage two features which we
believe to be good global indicators of update selection: the summary content
probability and the document frequency.  These two features are proxies for
detecting (1) a good summary sentences (regardless of novelty with respect to
other previous decisions) and (2) when an event is likely to be producing novel
content. We compute the conjunctions of all previously mentioned features with
the summary content probability and document frequency separately and together.

\subsection{Expanding Relevance Judgments}
 \label{sec:reljudge}

Because of the large size of the corpus, less than 1\% of the sentences
received manual review by NIST assessors during the 2013-15 shared-task
evaluations.  In order to increase the amount of data for training and
evaluation of our system, we augmented the manual sentence-nugget match
judgements with automatic matches.  Sentence-nugget matches are critical for
our experiments because they enable us to compute evaluation metrics, but also
the oracle summarization policy used in the LOLS algorithm.

In order to automatically tag sentences in the corpus with additional nugget
matches, we trained a separate decision-tree classifier for each nugget with
more than 10 manual sentence matches.  Manually matched sentences were used as
positive training data and an equal number of manually judged non-matching
sentences were used as negative examples.  Sentence n-grams (1-5), percentage
of nugget terms covered by the sentence, semantic similarity of the sentence to
nugget were used as features, along with an interaction term between the
semantic similarity and coverage.  After training the classifiers we used them
to automatically tag corpus sentences with nugget matches.  When augmenting the
relevance judgments with these nugget match labels, we only include those that
have a probability greater than 90\% under the classifier.  For evaluation, the
summarization system only has access to the query and the document stream,
without knowledge of any nugget matches (manual or automatic).

\subsection{TREC 2015 Experiments and Results}

There were three tasks at the TREC 2015 Temporal Summarization shared-task
evaluation \citep{aslam2015}:
\begin{enumerate}
\item  \textbf{Full Filtering and Summarization} Participants must summarize
a \textbf{very} high volume stream of documents where very few documents are 
likely to be similar to the query.
\item  \textbf{Partial Filtering and Summarization} Participants must summarize
a high volume stream of documents that has been pre-filtered for query
relevance
by an IR component developed by the task organizers.
\item \textbf{Summarization Only} Participants must summarize a low-volume
stream of on-topic documents.
\end{enumerate}

We participated in tasks 1 and 2, submitting, with Team ID \textit{cunlp},
the following Run IDs:

\begin{itemize}
\item \textbf{Task 1. Full Filtering and Summarization}
\begin{itemize}
    \item  2LtoSnofltr20 -- The L2S summarizer, where documents were
truncated to there first 20 sentences.
\end{itemize}
\item \textbf{Task 2. Partial Filtering and Summarization}
\begin{itemize}
\item  1LtoSfltr20 -- The L2S summarizer, where documents were
truncated to the first 20 sentences.
\item  3LtoSfltr5 -- The L2S summarizer, where documents were truncated to 
their first 5 sentences.
\item  4APSAL -- Our SAP summarizer that was the overall best system 
    at TREC 2014. We updated the salience predictions with additional training
data from the TREC 2014 Temporal Summarization query events.
\end{itemize}
\end{itemize}

To train the L2S summarizer we randomly selected 3 events to be a development
set and used the remaining 21 events from the 2013-2014 Temporal Summarization
query events as our training set. The document streams for the events are
unfortunately too large to be used directly with \autoref{fig:lols}.  In order
to make training time reasonable yet representative, we downsample each stream
to a length of 100 sentences. The downsampling is done uniformly over the
entire stream. This is repeated 10 times for each training event to create a
total of 210 training streams. In the event that a downsample contains no
nuggets (either human or automatically labeled) we resample until at least one
exists in the sample.  We select the best model iteration based on the
automatically computed $\mathcal{H}(\updateSummary)$ on the development set.
We show an example summary produced by the L2S system in
\autoref{fig:l2ssystem-summary}.

\begin{figure}[t]
  \fbox{\resizebox{\textwidth}{!}{
    \begin{tabular}{p{.15\columnwidth}p{.77\columnwidth}}
 4:19 p.m. & Two explosions shattered the euphoria of the Boston
 Marathon finish line on Monday, sending authorities out on the course to
 carry off the injured while the stragglers were rerouted away...\\[0.5em]
 4:31 p.m. &  Police in New York City and London are stepping up
 security following explosions at the Boston Marathon. \\[0.5em]
 4:31 p.m. & A senior U.S. intelligence official says two more
 explosive devices have been found near the scene of the Boston marathon where
 two bombs detonated earlier. \\[0.5em]
 5:10 p.m. & Several candidates for Massachusetts' Senate
 special election have suspended campaign activity in response to the
 explosions... \\[0.5em]
    \end{tabular}}}
  \caption{Excerpt of the L2S summary for the query \textit{boston marathon bombing}
           generated from an input document stream.}
  \label{fig:l2ssystem-summary}
\end{figure}



\begin{table}[t]
\centering
\begin{tabular}{cccccc}
\toprule
Team ID & Run ID &$n\mathbb{E}[\gain](\updateSummary)$ & $\comp(\updateSummary)$ & $\comp_L(\updateSummary)$ & $\mathcal{H}_L(\updateSummary)$ \\
\midrule
\textbf{cunlp} & \textbf{2LtoSnofltr20} & 0.1224 & 0.4691 & \textbf{0.8086}  & \textbf{0.1531} \\
CWI & IGnPrecision & 0.1894 & 0.4678 & 0.6273 & 0.1396 \\
\midrule
\multicolumn{2}{c}{\textit{average}} & 0.1533 & 0.4575 & 0.6507 & 0.1279 \\
\midrule
CWI &  IGn & 0.1620 & \textbf{0.5137} & 0.6538 &0.1248 \\
CWI & docs & 0.1242 & 0.4680 & 0.6658 &0.1222 \\
CWI & titles & \textbf{0.1915} & 0.3107 & 0.5171 & 0.1150 \\
\bottomrule
\end{tabular}
\caption{Official TREC 2015 Temporal Summarization Task 1 results
using manual update/nugget matches.}
\label{tab:trec15t1}
\end{table}


\autoref{tab:trec15t1} shows the official results for task 1, the full
filtering and summarization task. Our L2S run, 2LtoSnofltr20, was the top run
in this track, although only one other team submitted runs due to the high
computational cost of running on the large document stream.  2LtoSnofltr20 had
the lowest precision (i.e., $n\mathbb{E}[\gain](\updateSummary)$) of the
submitted runs, and only achieved average recall (i.e.,
$\comp(\updateSummary)$).  However, it had lower latency, yielding the highest
latency weighted comprehensiveness, $\comp_L(\updateSummary)$; on the task's
overall summary measure $\harm_\latency(\updateSummary)$, 2LtoSnofltr20 is
ranked first, largely due to its low latency.  Here we can see that fully
online/greedy nature of L2S summarizer pays off in terms of latency, as salient
content is identified relativey quickly once it has entered the stream.

\begin{table}[p]
\centering
\begin{tabular}{cccccc}
\toprule
Team ID & Run ID &$n\mathbb{E}[\gain](\updateSummary)$ & $\comp(\updateSummary)$ & $\comp_L(\updateSummary)$ & $\mathcal{H}_L(\updateSummary)$ \\
\midrule
WaterlooClarke & UWCTSRun1 & 0.2350 & 0.3520 & 0.6612 & \textbf{0.1762}\\
WaterlooClarke & UWCTSRun3 & 0.2252 & 0.3421 & \textbf{0.6643} & 0.1718\\
WaterlooClarke & UWCTSRun2 & \textbf{0.2872} & 0.2584 & 0.6551 & 0.1710\\
\textbf{cunlp} & \textbf{3LtoSfltr5} & 0.1371 & 0.4870 & 0.6392 & 0.1282\\
\textbf{cunlp} & \textbf{1LtoSfltr20} & 0.1203 & 0.5372 & 0.6287 & 0.1100\\
IRIT & FS1A & 0.0849 & 0.4959 & 0.6051 & 0.0719\\
\textbf{cunlp} & \textbf{4APSAL} & 0.1011 & 0.4584 & 0.5108 & 0.0674\\
\midrule
\multicolumn{2}{c}{\textit{average}} & 0.0666 & 0.4342 & 0.4697 & 0.0499\\
\midrule
IRIT & FS2A & 0.0518 & 0.5899 & 0.6285 & 0.0476\\
BJUT & DMSL1NMF2 & 0.0445 & 0.6123 & 0.4539 & 0.0354\\
BJUT & DMSL1AP1 & 0.0413 & \textbf{0.6155} & 0.4701 & 0.0338\\
l3sattrec15 & l3sattrec15run1 & 0.0408 & 0.3612 & 0.3743 & 0.0268\\
l3sattrec15 & l3sattrec15run3 & 0.0400 & 0.3669 & 0.3712 & 0.0262\\
IRIT & FS1B & 0.0422 & 0.2939 & 0.3913 & 0.0259\\
IRIT & FS2B & 0.0306 & 0.3391 & 0.4491 & 0.0239\\
USI & InL2DecrQE1ID1 & 0.0182 & 0.5713 & 0.5806 & 0.0196\\
USI & InL2DecrQE2ID2 & 0.0169 & 0.5758 & 0.5836 & 0.0184\\
udel\_fang & WikiOnlyFS2 & 0.0206 & 0.5819 & 0.4600 & 0.0176\\
udel\_fang & ProfOnlyFS3 & 0.0258 & 0.5294 & 0.4122 & 0.0174\\
USI & InL2StabQE2ID3 & 0.0171 & 0.6133 & 0.5238 & 0.0170\\
udel\_fang & WikiProfMixFS1 & 0.0189 & 0.5965 & 0.4660 & 0.0166\\
l3sattrec15 & l3sattrec15run2 & 0.0283 & 0.2276 & 0.2560 & 0.0164\\
USI & InL2IncrQE2ID4 & 0.0179 & 0.5837 & 0.2888 & 0.0108\\
\bottomrule
\end{tabular}
\caption{Official TREC 2015 Temporal Summarization Task 2 results
using manual update/nugget matches.}
\label{tab:trec15t2}
\end{table}


\autoref{tab:trec15t2} shows the official results for Task 2, the partial
filtering and summarization task. The top runs by WaterlooClarke examined only
the titles or first sentences of the documents, while our systems did not use
titles, and used the first five or twenty sentences of documents (3LtoSfltr5
and 3LtoSfltr20 respectively). The WaterlooClarke runs had significantly higher
$n\mathbb{E}[\gain](\updateSummary)$ than the other systems but lower than
average $\comp(\updateSummary)$. Our L2S runs, 3LtoSfltr5 and 3LtoSfltr20, had
higher $\comp(\updateSummary)$, although not the highest overall. The fourth
and five place overall performance (i.e., $\mathcal{H}_L(\updateSummary)$) by
3LtoSfltr5 and 3LtoSfltr20 are due both to the high $\comp(\updateSummary)$ but
also low-latency at which it added important information to the summary.

The SAP run, 4APSAL, was also above the track average for all evaluation
measures. It is, however, dominated by both L2S runs on all measures.  This
suggests that our L2S is an overall improvement over the SAP model in both
precision, recall, and latency. 

\subsection{Automatic Experiments} We evaluate our method on the publicly
available TREC Temporal Summarization Track data using the data from the 2013,
2014, and 2015 years of the shared task. This collection contains  44 unique
query events.  To evaluate our model, we randomly select five events to use as
a development set and then perform a leave-one-out style evaluation on the
remaining 39 events.

As we did in the official Temporal Summarization submission,  we  downsample
each training stream to a length of 100 sentences. The downsampling is done
uniformly over the entire stream. This is repeated 10 times for each training
event to create a total of 380 training streams. In the event that a downsample
contains no nuggets (either human or automatically labeled) we resample until
at least one exists in the sample.  We select the best model iteration for each
training fold based on the automatically computed $\mathcal{H}(\updateSummary)$
on the development set.

\subsubsection{Baselines and Model Variants}

We refer to our ``learning to search'' model in the results as \textsc{L2S}.
We compare our proposed model against several baselines and extensions. 

\textbf{Cosine Similarity Threshold} WaterlooClarke, the top performing team at
TREC 2015 used a heuristic method that only examined article first sentences,
selecting those that were below a cosine similarity threshold to any of the
previously selected updates. We implemented a variant of that approach using
the latent-vector representation used throughout this work. The development set
was used to set the threshold. We refer to this model as \textsc{Cos}.
  
\textbf{SAP} We also compare to the SAP model described in \autoref{sec:sap}.
In this evaluation.

\textbf{Learning2Search+Cosine Similarity Threshold} In this model, which we
refer to as \textsc{L2S-Cos}, we run \textsc{L2S} as before, but filter the
resulting updates using the same cosine similarity threshold method as in
\textsc{Cos}. The threshold was also tuned on the development set. 

\subsection{Results} \label{sec:results}

Results for system runs are shown in \autoref{fig:tr15autoresults}.  On
average, \textsc{L2S} and \textsc{L2S-Cos} achieve higher
$\mathcal{H}(\updateSummary)$ scores than the baseline systems in both latency
penalized and unpenalized evaluations. For \textsc{L2S-Cos}, the difference in
mean $\mathcal{H}(\updateSummary)$ score was significant compared to all other
systems (for both latency settings). Intuitively, L2S has higher
comprehensiveness than \textsc{L2S-Cos}; adding the the cosine similairt filter
to L2S reduces the comprehensiveness, but increases the average gain of the
updates by a larger amount, yielding an improved harmonic mean of the two
metrics ($\mathcal{H}(\updateSummary)$).
 
\textsc{SAP} achieved the overall highest expected gain, partially because it
was the tersest system we evaluated (at 8 updates per query on average).
However, only \textsc{Cos} was statistically significantly worse than it on
this measure. We also see that SAP suffers from the latency-weighted
evaluation, receiving latency penalties for retrieving updates after that
information had been added to Wikipedia. By comparison, all other systems are
actually rewarded in the latency-weighted evaluation, as they consistently
retrieve information before it is published in Wikipedia. While SAP had
previously beaten Wikiepdia in previous evaluations, the added events for the
2015 Temporal Summarization had less media coverage, suggesting the clustering
based approach is less suited for lower-volume news streams.

\begin{figure}[t]
    \center
\begin{tabular}{ l   l l l   l l l  l }
\toprule
   &\multicolumn{3}{c}{unpenalized}&\multicolumn{3}{c}{latency-penalized}&\\
 \cmidrule(lr){2-4} \cmidrule(lr){5-7}
   & $n\mathbb{E}[\gain](\updateSummary)$     & $\comp(\updateSummary)$ 
        & $\mathcal{H}(\updateSummary)$ & $n\mathbb{E}[\gain_L](\updateSummary)$     & $\comp_L(\updateSummary)$ & $\mathcal{H}_L(\updateSummary)$ & ~~~$\setsize{\updateSummary}$\\
    \bottomrule
%\textsc{AP}         
%    & $0.083$            & $0.09$                     & $0.078$ 
%    & $0.079$            & $0.095$                    & $0.077$ 
%    & ~~$20.846^{s}$ \\
\textsc{SAP}      
    & $\mathbf{0.119}^c$ & $0.09$                     & $0.094$ 
    & $0.105$            & $0.088$                    & $0.088$ 
    & ~~~~$8.333$ \\
\textsc{Cos}     
    & $0.075$            & $0.176^{s}$              & $0.099$ 
    & $0.095$            & $0.236^{s}$              & $0.128^{s}$ 
    & $145.615^{s,f}$ \\
\textsc{L2S}       
    & $0.097$            & $\mathbf{0.207}^{s,f}$   & $0.112$ 
    & $0.136^{c}$      & $\mathbf{0.306}^{s,c,f}$ & $0.162^{s}$
    & ~~$89.872^{s,f}$ \\
\textsc{L2S-Cos}
    & $0.115^{c,l}$        & $0.189^{s}$ & ${\bf0.127}^{s,c,l}$
    & ${\bf0.162}^{s,c,l}$ & $0.276^{s}$ & ${\bf0.184}^{s,c,l}$
 & ~~$29.231^{s,c}$ \\
\bottomrule
\end{tabular}
\caption{
 Average system performance 
 and average number of updates per event.
 Superscripts indicate significant improvements ($p < 0.05$) between the run and
 competing algorithms using the 
  paired randomization test with the Bonferroni correction for multiple 
  comparisons ($s$: \textsc{APSal}, 
 $c$: \textsc{Cos}, $l$: \textsc{LS}, $f$: \textsc{LS-Cos}). 
}
\label{fig:tr15autoresults}
\end{figure}



 
In comprehensiveness, \textsc{L2S} recalls on average a fifth of the nuggets
for each event. This is even more impressive when  compared to the average
number of updates produced by each system; while \textsc{Cos} achieves similar
comprehensiveness, it takes on average about 1.6 times more updates than
\textsc{L2S} and almost 5 times more updates than \textsc{L2S-Cos}. The output
size of \textsc{Cos} stretches the limit of the term ``summary,'' which is
typically far shorter than 145 sentences in length.  This is especially
important if the intended application is negatively affected by verbosity (e.g.
crisis monitoring).

\begin{table}[t]
  \center
  \begin{tabular}{ l   l l l  }
\toprule
    &\multicolumn{3}{c}{latency-penalized}\\
\cmidrule(lr){2-4} 
    & $n\mathbb{E}[\gain_L](\updateSummary)$     & $\comp_L(\updateSummary)$ & $\mathcal{H}_L(\updateSummary)$ \\
    \midrule
    \textsc{Cos}  & $0.095$ & $\mathbf{0.236}$ & $0.128$ \\
     \textsc{L2S-FS}  & $0.164$ & $0.220$ & $0.157$ \\
    \textsc{L2S-Cos-FS} & 
      $\mathbf{0.207}$ & $0.18~~$ & $\mathbf{0.163}$ \\
\bottomrule
  \end{tabular}
  \caption{Average system performance. \textsc{L2S-FS} and \textsc{L2S-Cos-FS} 
           runs are trained and evaluated on first sentences only (like the 
           \textsc{Cos} system). Ranking is consistent with unpenalized results.}
  \label{tab:results-trunc}
\end{table}



\subsection{Discussion} \label{sec:discussion}

Since \textsc{Cos} only considers the first sentence of each document, it may
miss relevant sentences below the article's lead. In order to confirm the
importance of modeling the oracle, we also trained and evaluated the
\textsc{L2S} based approaches on first sentence only streams.
\autoref{tab:results-trunc} shows the latency penalized results of the first
sentence only runs.  The \textsc{L2S} approaches still dominate \textsc{Cos}
and receive larger positive effects from the latency penalty despite also being
restricted to the first sentence. Clearly having a model (beyond similarity) of
what to select is helpful. Ultimately we do much better when we can look at the
whole document.

\begin{figure}[t]
  \center
  \begin{tabular}{l ccccc }
\toprule
    & Miss (Lead) &     Miss (Body) &         Empty &   Duplicates & Total \\
    \midrule
    \small \textsc{SAP}         
      & 29.6\% & 68.7\% & ~~1.6\% & 0.1\% & 15,986\\
    \small \textsc{Cos}     
      & 17.8\% & 39.4\% & 41.1\% & 1.7\% & 12,873\\
    \small \textsc{L2S-FS}      
      & 25.4\% & 71.7\% & ~~2.0\% & 0.9\% & 13,088\\
    \small \textsc{L2S-Cos-FS}
      & 27.9\% & 70.8\% & ~~1.0\% & 0.2\% & 15,756\\
    \small \textsc{L2S}           
      & 19.6\% & 55.3\% & 19.9\% & 5.1\% & 13,380\\
    \small \textsc{L2S-Cos}    
      & 24.6\% & 66.7\% & ~~7.5\% & 1.2\% & 11,613\\
\bottomrule
  \end{tabular}
  \caption{Percent of errors made and total errors on test set.}
  \label{tab:errors}
\end{figure}


 
We also performed an error analysis to further understand how each system
operates.  \autoref{tab:errors} shows the errors made by each system on the
test streams.  Errors were broken down into four categories. \emph{Miss lead}
and \emph{miss body} errors occur when a system skips a sentence containing a
novel nugget in the lead or article body respectively. An \emph{empty} error
indicates an update was selected that contained no nugget.  \emph{Duplicate}
errors occur when an update contains nuggets but none are novel. 
 
Overall, errors of the miss type are most common and suggest future development
effort should focus on summary content identification.  About a fifth to a
third of all system error comes from missing content in the lead sentence
alone.
 
After misses, empty errors (false positives) are the next largest source of
error. \textsc{Cos} was especially prone to empty errors (41\% of its total
errors). \textsc{L2S} is also vulnerable to empties (19.9\%) but after applying
the similarity filter and restriting to first sentences, these errors can be
reduced dramatically (to 1\%).
  
Surprisingly, duplicate errors are a minor issue in our evaluation. This is not
to suggest we should ignore this component, however, as efforts to increase
recall (reduce miss errors) are likely to require more robust redundancy
detection. 
