\section{Conclusion}

In this chapter we introduced two stream summarization algorithms, 
where
we focused on incorporating salience predictions into the update
selection stage. In the first model, \textsc{SAP}, we were able to
incorporate these predictions into the affinity propagation clustering 
algorithm, thereby balancing the twin objectives of selecting representative
updates while ensuring the updates contained essential information that 
was relevant to the query event. In the second method,  \textsc{L2S},
we were able to train a stream summarization policy which predicts the salience
of potential updates using a feature representation of the current update
summary and stream state. This method improves over \textsc{SAP}
in that it can make predictions using dynamically updated features based
on previous behavior and can make decisions on updates immediately
 without needing to collect a sizeable cache of sentences for clustering.

Future work could focus on a number of improvements. One of the most important
ones would be to improve the understanding of novelty. Currently, we do not
have any explicit handling of information that is refined or updated over time.
For example, the number of reported casualities in a disaster typically
changes over time as more cases are reported and nubmers are refined. 
Under our current models
these statements might all look very similar since they have a fairly 
boilerplate text, e.g. for we have the following nuggets for a train 
crash event presented in chronological order,
\begin{itemize}
\item 55 dead
\item 50 confirmed dead
\item 51 confirmed dead.
\end{itemize}
As this information is textually similar, after a system selects one of these
updates, it may not extract others.

Also as issues of trust and veracity in reporting in online sources have become more important,
it might also be necessary to model the reliability of an information 
source. In this
case some numbers are reported and some are also confirmed. Our model
currently has no distinction between official and unofficial sourcing,
which would be necessary in any real implemenation of these models.

We also clearly have room to improve salience predictions. The expected gain
of our systems hovered around 0.1, meaning it would take on average ten updates to produce a novel piece of information. As our error analysis showed,
misses were the most prevalent error. All implemented
systems have the highest error rates when trying to find information in the 
body of an article. This echoes the findings of our previous chapter, where
position bias is predominant feature being exploited. More efforts on 
modeling the salience of sub-events and knock-on effects of a query event
might be beneficial here. 


