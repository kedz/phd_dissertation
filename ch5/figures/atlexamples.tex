\begin{figure}[p]
    \fbox{\begin{minipage}{\textwidth}
\center

\begin{minipage}{0.98\textwidth}
\begin{subfigure}{\textwidth}
\caption{Example of an alignment training linearization for a \meaningrepresentation~with a list-valued attribute, \textit{genres}. Note also that the \textit{rating} attribute for ViGGO examples is not aligned but always appended after the dialogue act (see \autoref{sec:align} for details). }
\center
\begin{minipage}[t]{0.48\textwidth}
\small
$\mr = \left[\!\!\left[\begin{array}{l} 
    \textsc{Give Opinion} \\ 
    \textrm{name=Little Nightmares} \\
    \textrm{rating=good} \\
    \textrm{genres=[}\\
    \textrm{~~~~adventure,} \\
    \textrm{~~~~platformer,}\\
    \textrm{~~~~puzzle} \\
    \textrm{]} \\
    \textrm{player\_perspective=side view}
\end{array}\right]\!\!\right]$ 
\end{minipage}
\hfill
\begin{minipage}[t]{0.48\textwidth}
\small
$\ls(\mr) = \left[ \begin{array}{l}
\starttok, \\
\textit{give\_opinion},\\
\textit{rating=good},\\ 
\textit{name=Little Nightmares},  \\
\textit{genres=adventure},  \\
\textit{player\_perspective=side view},\\
\textit{genres=platformer},  \\
\textit{genres=puzzle},  \\
\stoptok\end{array} \right]
$ 
\end{minipage}\\
\begin{minipage}[t]{\textwidth}
~\\[0pt]
$\utttoks = $ \textit{Little Nightmares is a pretty cool game that has kept me entertained. It's an adventure side-scrolling platformer with some puzzle elements to give me a bit of a challenge.}
\end{minipage}
\end{subfigure}

~\\~\\


\begin{subfigure}{\textwidth}
\caption{Example of an alignment training linearization with repeated attribute-values. In this case, the \textit{name} attribute is realized twice and 
so it appears twice in the linearization.}
\center
\begin{minipage}[t]{0.48\textwidth}
\small
$\mr = \left[\!\!\left[\begin{array}{l} 
    \textsc{Inform}\\
    {\AV{name}{Aromi}}\\
    {\AV{eat\_type}{coffee shop}}\\
    {\AV{customer\_rating}{5 out of 5}}\\
    {\AV{food}{English}}\\
    {\AV{area}{city centre}}\\
    {\AV{family\_friendly}{yes}}\\
\end{array} \right]\!\!\right]$
\end{minipage}
 \hfill
\begin{minipage}[t]{0.48\textwidth}
\small
$\ls(\mr) = \left[ \begin{array}{l}
\starttok, \\
\textit{inform},\\
\textit{name=Aromi},\\
\textit{eat\_type=coffee shop},\\
\textit{family\_friendly=yes},\\
\textit{food=English},\\
\textit{name=Aromi},\\
\textit{customer\_rating=5 out of 5},\\
\textit{area=city centre},\\
\stoptok\end{array} \right]
$ 
\end{minipage}\\
\begin{minipage}[t]{\textwidth}
~\\[0pt]
$\utttoks = $ \textit{The Aromi coffee shop is family-friendly and serves English food.  Aromi has a customer rating of 5 out of 5 and is located near the center of the city.}
\end{minipage}
\end{subfigure}

~\\~\\


\begin{subfigure}{\textwidth}
\caption{Example alignment training linearization where an attribute-value 
is not grounded in the reference utterance. In this case, \textit{food=Japanese}
is not present in the linearization.}
\center
\begin{minipage}[t]{0.48\textwidth}
\small
$\mr = \left[\!\!\left[\begin{array}{l} 
    \textsc{Inform}\\
    {\AV{name}{The Waterman}}\\
    {\AV{food}{Japanese}}\\
    {\AV{price\_range}{high}}\\
    {\AV{area}{riverside}}\\
\end{array} \right]\!\!\right]$
\end{minipage}
 \hfill
\begin{minipage}[t]{0.48\textwidth}
\small
$\ls(\mr) = \left[ \begin{array}{l}
\starttok, \\
\textit{inform},\\
\textit{area=riverside},\\
\textit{price\_range=high},\\
\textit{name=The Waterman},\\
\stoptok\end{array} \right]
$ 
\end{minipage}\\
\begin{minipage}[t]{\textwidth}
~\\[0pt]
$\utttoks = $ \textit{Near the river there is an expensive sushi place
called the Waterman.}
\end{minipage}
%\end{itemize}
%\label{fig:mr1utt}
\end{subfigure}
\end{minipage}
\end{minipage}}
\caption{Example \meaningrepresentation/utterance pairs ($\mr, \utttoks$) and their alignment
training linearization $\ls(\mr)$.}
\label{fig:atlexamples}
\end{figure}

