\section{Meaning Representations for Task-Oriented Dialogue Generation}
\label{sec:mr4tod}

\subsection{\MeaningRepresentation~Structure}

In this chapter, we use several domain specific \meaningrepresentation s to
formally represent the input to the \surfacerealization~model.  While specifics
of the \meaningrepresentation~can vary from domain to domain, the overall
structure of the \meaningrepresentation~is fairly straightforward, borrowing
from a common format used frequently in  the
\naturallanguagegeneration~literature
\citep{mairesse2010,gasic2014,wen2015,novikova2017,juraska2019}. Each
\meaningrepresentation~has a \dialogueact, which expresses the communicative
goal or intent, and zero or more \attributevalue~pairs which further define the
semantics of the desired utterance. 

See \autoref{fig:informexample} for an example \meaningrepresentation~from the
restaurant domain. The dialogue act, in this case to inform a user, is the
first item and is written in \textsc{SmallCaps} style.  The attributes are
``name,'' ``eat\_type,'' ``customer\_rating,'' ``food,'' ``area,'' and
``family\_friendly.'' Their associated values are ``Aromi,'' ``coffee shop,''
``5 out of 5,'' ``English,'' ``city centre,'' and ``yes'' respectively. In this
case the attributes are referring to the restaurant about which a hypothetical
dialogue agent is trying to inform a user.

In our setting, \dialogueact s are predominantly declarative (e.g.,
\autoref{fig:informexample} or \autoref{fig:giveopinionexample}), but also
include interrogatives  (e.g., \autoref{fig:confirmexample}), and some that may
be a mix of both (e.g., \autoref{fig:compareexample} where the second reference
ends in a question about user preference).  Additionally, we also have
semantically vacuous ``chit-chat'' dialogue acts like \textsc{Greeting} and
\textsc{Goodbye} which are expected to begin and end, respectively, a series of
exchanges with the dialogue agent. 

\begin{figure}
 \begin{subfigure}{\textwidth}
    \begin{minipage}{0.5\textwidth}
\center
$\left[\begin{array}{l} 
    \textsc{Inform} \\ 
    \textrm{name=Portal 2} \\
    \textrm{esrb=E 10+ (for Everyone 10 and Older)} \\
    \textrm{genres=[platformer, puzzle, shooter]} \\
    \textrm{player\_perspective=[first person]} \\
    \textrm{has\_multiplayer=yes}
\end{array}\right]$ 
\end{minipage}
    \begin{minipage}{0.5\textwidth}
        Portal 2 was rated E 10+ (for Everyone 10 and
        Older). It is a puzzle platformer FPS with
        multiplayer.
\end{minipage}

~\\

\caption{An example of list-valued attributes (genres and player\_perspective)
    from the video game domain. Note that the acronym FPS means ``first person
shooter'' which realizes both the
player\_perspective \attributevalue and a genre \attributevalue. }
\end{subfigure}  

~\\
~\\
~\\

\begin{subfigure}{\textwidth}
    \begin{minipage}{0.5\textwidth}
    \center
$\left[\begin{array}{l} 
    \textsc{Request} \\ 
    \textrm{specifier=``dull''} \\
    \textrm{has\_multiplayer=yes}
\end{array}\right]$ 
\end{minipage}\begin{minipage}{0.5\textwidth}
    What's the most dull multiplayer game you've ever played?
\end{minipage}

~\\

\caption{An example of a free-text valued attribute (specifier) from the 
video game domain. The specifier value can be any adjective.}
\end{subfigure}  

~\\
~\\
~\\



\begin{subfigure}{\textwidth}
    \begin{minipage}{0.5\textwidth}
 \center
$\left[\begin{array}{l} 
    \textsc{Inform Count} \\ 
    \textrm{count=$58$} \\
    \textrm{type=laptop} \\
    \textrm{is\_for\_business\_computing=true} \\
    \textrm{weight\_range=don't care} \\
    \textrm{drive\_range=don't care} \\
\end{array}\right]$ 
\end{minipage}
\begin{minipage}{0.5\textwidth}
There are 58 laptops used for business computing if you do not care what 
weight range or drive range they have.
\end{minipage}

~\\


\caption{An example of a numeric-valued attribute (count) from the laptop domain.}
\end{subfigure}  

   
%inform_count(count=58;type=laptop;isforbusinesscomputing=true;weightrange=dontcare;driverange=dontcare
\caption{Examples of various \attributevalue~types paired with an example
realization.}
\end{figure}




The kinds of values that can fill an attribute are typically categorical
variables. For example, in the restaurant domain, the attribute ``food'' may
take values from a closed list of food types such as the set
\[\{\textrm{``Chinese''}, \textrm{``English''}, \textrm{``French''},
\textrm{``Fast food''}, \textrm{``Indian''}, \textrm{``Italian''},
\textrm{``Japanese''}\}.\] Other value types include list-valued attributes,
numerical values, or free text (see \autoref{fig:valtypes} for examples of
each). For list-valued attributes, the value is a list of items drawn from a
closed set.  For example, in the video game domain, a video game can belong to
several genres simultaneously.  For our purposes, we treat each value in the
list as a distinct \attributevalue~pair. So in the case of
\hyperref[fig:valtypes]{Figure \ref{fig:valtypes}a}, we treat it is if it had
the following \meaningrepresentation,
\begin{center} \MR{\textsc{Inform}}{\AV{name}{Portal 2}}{\AV{esrb}{E 10+ (for Everyone 10 and Older)}}
    {\AV{genres}{platformer}} 
    {\AV{genres}{puzzle}} 
    {\AV{genres}{shooter}} 
    {\AV{player\_perspective}{first person}} 
    {\AV{has\_multiplayer}{yes}}
\end{center} 
Additionally, not all attributes need to be specified. In which case, the
utterance should not mention them.

The term ``\meaningrepresentation''~is somewhat of a misnomer as the
representations might better be characterized as a pragmatic construct (i.e. a
representation of the dialogue agent's intentional state).  The
\attributevalues, on the other hand, are a semantic construct, representing the
semantic value or propositional content of the sentences in the utterance.  In
other words, from the perspective of formal semantics, 
\begin{itemize} 
    \item \textit{The Aromi is a coffee shop in the city centre.}  
    \item \textit{Just to confirm, the coffee shop in the city centre is called Aromi?}  
    \item \textit{What
about Aromi, the coffee shop in the city centre?}  
\end{itemize} 
\noindent all share the same semantic value. The ``meaning'' of the above
utterances as a statement of first-order logic might look something like,
\[\exists x : \operatorname{isCoffeeShop}(x) \wedge
\operatorname{namedAromi}(x) \wedge \operatorname{inCityCentre}(x).\] We could
represent this statement in our present setting as a
``\meaningrepresentation~without a dialogue act,'' i.e.,
\begin{center} \MR{\textrm{---}}{\AV{name}{Aromi}}{\AV{eat\_type}{coffee
    shop}}{\AV{area}{city centre}} \end{center} ~\\
    
\noindent which, when combined with one of the dialogue acts \textsc{Inform},
\textsc{Confirm}, or \textsc{Recommend}, yields the pragmatic sense of the
respective utterances above.

\subsection{Relating Between \MeaningRepresentations~and \Utterances}

Let $\mr \in \mrspace$ be a \meaningrepresentation, and let $\utttoks =
\left[\utttok_1,\ldots, \utttok_\uttSize \right] \in \outSpace$ be an
utterance, i.e. sequence of $\uttSize$ tokens from a vocabulary $\uttvocab$ and
$\outSpace = \uttvocab^*$. We say that an utterance $\utttoks$ \textit{denotes}
a meaning representation $\mr$, which we write $\denotes{\utttoks} = \mr$ if
the propositional content of the utterance and the meaning representation are
the same, i.e. the \attributevalues~implied by $\utttoks$ and explicitly listed
by $\mr$ are the same.  We can make similar statements about a sub-sequence of
an utterance. Let $\utttoks_{i:i+j} = \left[\utttok_i, \utttok_{i+1},\ldots,
\utttok_{i+j} \right]$ be a sub-sequence of $j+1$ tokens starting at token $i$.
We then have $\denotes{\utttoks_{i:i+j}} = \mr^\prime$ for some $\mr^\prime \in
\mrspace \cup \emptyset$.  When an utterance or sub-sequence $\utttoks$
contains no propositional content or is otherwise not a meaningful statement,
we have $\denotes{\utttoks} = \emptyset$.

As an example, consider the following \meaningrepresentation,
\begin{singlespace}
\[
    \mr = \left[\!\!\left[ \begin{array}{l}
        \textsc{Inform} \\
        \AV{name}{The Vaults}\\
        \AV{eat\_type}{pub}\\
        \AV{near}{Caf{\'e} Adriatic}\\
        \AV{family\_friendly}{no}\\
\end{array}\right]\!\!\right] 
\]
\end{singlespace}
and the utterance,
\[
    \utttoks = \left[ \textit{The},\,\textit{Vaults},\,\textit{pub},\,\textit{is},\,\textit{near},\,\textit{Caf{\'e}},\,\textit{Adriatic},\,\textit{.},\,\textit{It},\,\textit{is},\,\textit{not},\,\textit{a},\,\textit{good},\,\textit{place},\,\textit{for},\,\textit{families},\,\textit{.}\right]. 
\]
\noindent Clearly, $\denotes{\utttoks} = \mr$. But we can also look at the
meanings of individual phrases,
\begin{singlespace}
    \begin{align*}
        \denotes{\utttoks_{1:2}} = \left[\!\!\left[ \left[\textit{The}, \textit{Vaults}  \right] \right]\!\!\right] & = \left[\!\!\left[
\begin{array}{l} \textrm{---} \\ \AV{name}{The Vaults}\end{array} \right]\!\!\right] \\
    \denotes{\utttoks_{3:3}} = \left[\!\!\left[ \left[\textit{pub}  \right] \right]\!\!\right] &= \left[\!\!\left[
\begin{array}{l} \textrm{---} \\ \AV{eat\_type}{pub}\end{array} \right]\!\!\right] \\
    \denotes{\utttoks_{5:7}} = \left[\!\!\left[ \left[\textit{near}, \textit{Caf{\'e}}, \textit{Adriatic}  \right] \right]\!\!\right] & = \left[\!\!\left[
\begin{array}{l} \textrm{---} \\ \AV{near}{Caf{\'e} Adriatic}\end{array} \right]\!\!\right] \\
    \denotes{\utttoks_{11:16}} = \left[\!\!\left[ \left[\textit{not}, \textit{a}, \textit{good}, \textit{place}, \textit{for}, \textit{families}  \right] \right]\!\!\right] & = \left[\!\!\left[
\begin{array}{l} \textrm{---} \\ \AV{family\_friendly}{no}\end{array} \right]\!\!\right]. 
\end{align*}
\end{singlespace}
Note that it is not the case that $\denotes{\utttoks} = \mr \Rightarrow \denotes{\utttoks_{i:i+j}} \subseteq \mr$. Consider in the example above $\utttoks_{11:16}$ its sub-sequence $\utttoks_{12:16} = \left[\textit{a},\,\textit{good},\,\textit{place},\,\textit{for},\,\textit{families}\right]$ which have
the following denotations,
\begin{singlespace}
   \[
   \denotes{\utttoks_{11:16}} = \left[\!\!\left[\begin{array}{l}\textrm{---}\\ \AV{family\_friendly}{yes} \end{array}\right]\!\!\right] \ne \denotes{\utttoks_{12:16}}= \left[\!\!\left[\begin{array}{l}\textrm{---}\\ \AV{family\_friendly}{no} \end{array}\right]\!\!\right] 
       \]
   \end{singlespace}
\noindent It is also important to note that the \attributevalues~are unordered
and do not necessarily reflect the realization order of the utterance. 

In the datasets used for this chapter, $\mr$ are provided with one or more
reference utterances, $\utttoks^{(1)}, \ldots \utttoks^{(k)}$ and that for each
reference $\utttoks^{(i)}$, we have that each \attributevalue~in $\mr$ can be
mapped to an utterance sub-sequence that denotes it.  Occasionally this is not
the case in the available training data. For example, some \attributevalues~may
have several possible groundings (see
\hyperref[fig:nlgerrors]{\autoref{fig:nlgerrors}a}) or be realized using
inferential knowledge not explicitly represented in the
\meaningrepresentation~(see
\hyperref[fig:nlgerrors]{\autoref{fig:nlgerrors}b}). 

\begin{figure}[t]

\begin{tikzpicture}

    \node[align=left,text width=\textwidth] (a) at (0,0) 
    {(a) The \attributevalue~eat\_type=pub occurs in multiple locations 
    of the utterance.};

    \node[anchor=west] at ($(a.west)+(0,2.5)$) {
        $\left[\!\!\left[\begin{array}{l}
                    \textsc{Inform}\\
            \AV{name}{Wildwood}\\
            \AV{food}{Italian}\\
            \textrm{eat\_type=}\redbox{\textrm{pub}}\\
            \AV{price\_range}{£20-25}\\
            \AV{customer\_rating}{high}
        \end{array}\right]\!\!\right]$};

    \node[draw,dotted,text width=0.5\textwidth,align=left,anchor=east] 
        at ($(a.east)+(0,2.5)$) 
        {\textit{Wildwood is an Italian \redbox{pub} with a price range of
        £20-25. The \redbox{pub} has a very high customer rating.}};

%    \node[align=left,text width=\textwidth] (b) at (0,-6) 
%    {(b) The utterance fails to mention the customer rating or that the 
%    establishment is family friendly.};
%
%    \node[anchor=west] at ($(b.west)+(0,3)$) {
%        $\left[\!\!\left[\begin{array}{l}
%                    \textsc{Inform}\\
%            \AV{name}{Blue Spice}\\
%            \AV{eat\_type}{coffee shop}\\
%            \AV{price\_range}{cheap}\\
%            \AV{customer\_rating}{average}\\
%            \AV{area}{riverside}\\
%            \AV{family\_friendly}{yes}\\
%            \AV{near}{Avalon}
%        \end{array}\right]\!\!\right]$};
%
%    \node[draw,dotted,text width=0.5\textwidth,align=left,anchor=east] 
%        at ($(b.east)+(0,3)$) 
%        {\textit{Blue Spice is a cheap coffee shop located  
%         in the riverside, near Avalon.}};


    \node[align=left,text width=\textwidth] (b) at (0,-5) 
    {(b) The utterance claims the restaurant serves sushi even though this
         is not stated in the \meaningrepresentation. Not all Japanese 
         restaurants serve sushi so this inference is not justified.};

    \node[anchor=west] at ($(b.west)+(0,2.5)$) {
        $\left[\!\!\left[\begin{array}{l}
          \textsc{Inform} \\
          \AV{name}{The Waterman}\\ 
          \textrm{food=}\redbox{\textrm{Japanese}}\\ 
          \AV{price\_range}{high}\\
    %      \AV{customer\_rating}{3 out of 5}\\
          \AV{area}{riverside}%\\
  %        \AV{family\_friendly}{yes}
  \end{array}\right]\!\!\right]$};

    \node[draw,dotted,text width=0.5\textwidth,align=left,anchor=east] 
        at ($(b.east)+(0,2.5)$) 
          {\textit{Near the river there is an expensive \redbox{sushi} 
          place called The Waterman.}}; %It is family friendly and rated 3 
           %out of 5.}};

%

%
%%name[Blue Spice], eatType[coffee shop], priceRange[cheap], customer rating[average], area[riverside], familyFriendly[yes], near[Avalon]"
%
%
%
%
%
%%
%%
%%
%%
%%
%%
%
%
%
%

\end{tikzpicture}

\caption{Example training set errors.}
\label{fig:nlgerrors}
\end{figure}


While such examples may exist in the training data, we consider model
generation of such phenomena to constitute a failure to faithfully generate an
utterance.  In the overwhelming majority of cases, each \attributevalue~is
explicitly and uniquely grounded in the target utterances, this makes
\surfacerealization~from \meaningrepresentations~a useful task to study
faithful generation. The baseline task of correctly generating all
\attributevalues~appropriately for the \dialogueact~is hard enough, and it in
this setting we do not have to worry about ungrounded information or
information that is not explicitly represented in the
\meaningrepresentations~but is deducible from the
\meaningrepresentation~\citep{wiseman2017}.  
