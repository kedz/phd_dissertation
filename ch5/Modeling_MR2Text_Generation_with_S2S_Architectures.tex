\section{Modeling \MeaningRepresentation-to-Text Generation with \SequencetoSequence~Architectures}
\label{mrtproblemdef}

\subsection{\SequencetoSequence~Modeling}

We approach the problem of mapping a \meaningrepresentation~to a natural
language utterance with a variety of popular \sequencetosequence~architectures.
A \sequencetosequence~model is a neural network with parameters $\params$ that
implements a probabilistic mapping, $\gen(\cdot|\cdot;\params) : \inSpace
\times \outSpace \rightarrow (0,1)$, from input sequences 
\[
    \mrtoks = \left[\mrtok_1,\ldots,\mrtok_\mrSize\right] \in \inSpace
\]
to output sequences 
\[
    \utttoks = \left[\utttok_1, \ldots, \utttok_\uttSize\right] \in \outSpace.
\]
Tokens from the input sequence are drawn from a finite vocabulary $\mrvocab$
and input sequences its Kleene closure $\mrvocab^* = \inSpace$.  Analogously,
tokens from the output sequence are drawn from a distinct, finite vocabulary
$\uttvocab$ and output sequences its Kleene closure $\uttvocab^* = \outSpace$.
For clarity we occasionally omit $\params$ in subsequent equations.

Typically, \gen~is implemented as a bipartite network consisting of distinct
encoder and decoder networks $\encMod$~and \decMod~respectively.  The encoder
$\encMod : \inSpace \rightarrow \reals^{*\times\encDim}$ is a mapping of an
input sequence $\mrtoks$ of $\mrSize$ tokens to $\mrSize$ corresponding vectors
$\encState_1,\ldots\encState_\mrSize \in \reals^\encDim$ and
\[ 
    \gen(\utttoks|\mrtoks) = \gen(\utttoks|\encMod(\mrtoks)) = \gen(\utttoks|\encState_1,\ldots,\encState_\mrSize).
\]
The decoder $\decMod : \uttvocab^+ \times \reals^{*\times \encDim} \rightarrow
(0,1)$ then is a mapping of previously generated tokens $\utttoks_{1:i-1} =
\left[\utttok_1,\ldots,\utttok_{i-1}\right]$ and encoder states
$\encState_1,\ldots,\encState_\mrSize$ to a probability distribution over the
output vocabulary $\uttvocab$, where
\[ 
    \gen(\utttok_i|\utttoks_{1:i-1},\mrtoks) = \decMod\left(\utttoks_{1:i}, \encMod(\mrtoks)\right)   \quad \textrm{and} \quad \sum_{\utttok\in \uttvocab} \gen\left(\utttok| \utttoks_{1:i-1}, \encMod(\mrtoks)\right) = 1.  
\]
Hence, $\gen(\cdot|\mrtoks)$ is a conditional language model over utterance
tokens that factorizes in a left-to-right fashion, i.e.,
\[
    \gen\left( \utttoks| \mrtoks \right) = \prod_{i=1}^\uttSize \gen\left(\utttok_i|\utttoks_{1:i-1},\mrtoks\right). 
\]

Notice that the ``inputs'' and ``outputs'' to the \sequencetosequence~model are
sequences of tokens, $\mrtoks$ and $\utttoks$ respectively.  In order to use a
\sequencetosequence~model for \naturallanguagegeneration~from a
\meaningrepresentation~we need only map our desired inputs and outputs to
sequences of discrete tokens. In English, the desired output is relatively
straightforward to represent as a sequence as an English language utterance can
naturally be represented as a sequence of word tokens. In practice, we also
indicate full sentence stops with a special token \senttok\footnote{We add an
explicit sentence stop symbol because occasionally utterances have non-sentence
final periods and we want there to be no ambiguity about sentence stops
produced by the model.} and prepend and append distinguished tokens
\starttok~and \stoptok, respectively, to indicate the start and end of the
utterance, as well as lower-case all tokens.  As an example, the utterance 
\[
    \textit{The Vaults pub is near Caf{\'e} Adriatic. It is not a good place for families.} 
\]
would be represented as 
\[\small \utttoks = \left[\starttok, \textit{the},\,\textit{vaults},\,\textit{pub},\,\textit{is},\,\textit{near},\,\textit{caf{\'e}},\,\textit{adriatic},\,\textit{.},\,\textit{\senttok},\,\textit{it},\,\textit{is},\,\textit{not},\,\textit{a},\,\textit{good},\,\textit{place},\,\textit{for},\,\textit{families},\,\textit{.},\,\textit{\stoptok}\right]. \]

The \meaningrepresentation~is not itself a sequence, however, so we cannot
apply it to a \sequencetosequence~model directly.  Instead it must first be
``linearized,'' or mapped to a linear sequence of tokens. We refer to a
function $\ls : \mrspace \rightarrow \inSpace$, as a \linearizationstrategy.
We experiment with several \linearizationstrategies~in this chapter, however,
all of them operate over the same domain, $\mrvocab^+$, where $\mrvocab$
consists of distinct tokens for each \dialogueact~and \attributevalue~pairs.
As an example consider the following \meaningrepresentation,
\begin{singlespace}
  \[ 
    \mr = \left[\!\!\left[\begin{array}{l} \textsc{Inform} \\ \AV{name}{Aromi}\\\AV{area}{city centre} \\ \AV{eat\_type}{coffee shop} \end{array}\right]\!\!\right] \]
\end{singlespace}
\noindent and some possible linearizations,
\begin{align*}
     \ls_1(\mr) & = \mrtoks= \left[\textit{inform},\,\textit{name=Aromi},\,\textit{eat\_type=coffee shop},\,\textit{area=city centre}\right] \\
     \ls_2(\mr) & = \mrtoks= \left[\textit{inform},\,\textit{eat\_type=coffee shop},\,\textit{name=Aromi},\,\textit{area=city centre}\right] \\
     \ls_3(\mr) & = \mrtoks= \left[\textit{inform},\,\textit{eat\_type=coffee shop},\,\textit{area=city centre},\,\textit{name=Aromi}\right].
\end{align*}
In practice, regardless of the choice of \linearizationstrategy, we prepend a
start token, $\starttok$, and append and a stop token, $\stoptok$, to all
input token sequences, e.g.
\[ \mrtoks = \left[\starttok,\,\textit{inform},\,\textit{name=Aromi},\,\textit{eat\_type=coffee shop},\,\textit{area=city centre},\,\stoptok\right]. \]

The encoder and decoder networks of the \sequencetosequence~model can be
implemented with a variety of architectures. We use two such architectures, the
gated recurrent unit (GRU) \citep{cho2014learning} and the transformer
\citep{vaswani2017}. Since we use standard variants, we defer explicit model
definitions to \hyperref[sec:nlggru]{Appendix \ref{sec:nlggru}} and
\hyperref[sec:nlgtf]{Appendix \ref{sec:nlgtf}} for GRU- and transformer-based
sequence-to-sequence architectures respectively. While the GRU must be given an
explicit linearization to process any input, the transformer variant does not
have to be sensitive to linearization order. However, when the transformer uses
position embeddings, which is the standard configuration, it is sensitive to
linearization order. We always use position embeddings in this work as
\cite{vaswani2017} found the model did not work as well without them.

\subsection{Learning}

Given a dataset of \meaningrepresentation/utterance pairs, $\corpus =
\left\{\left(\mr^{(1)}, \utttoks^{(1)}\right), \ldots,
\left(\mr^{(\corpusSize)},\utttoks^{(\corpusSize)} \right)\right\}$, and a
\linearizationstrategy, $\ls$, the parameters, $\params$, of $\gen$~can be
learned by approximately minimizing the negative log-likelihood of the data,
\[ \hat{\params} \approx \argmin_{\params \in \Params} -\frac{1}{\corpusSize}\sum_{\left(\mr, \utttoks \right) \in \corpus} 
    \log \gen\left(\utttoks|\ls\left(\mr\right);\params\right).
\]
In practice, this is done with some variant of mini-batch stochastic gradient
descent, e.g., Adam \citep{kingma2014adam}.  Additionaly, label smoothing
\citep{szegedy2016} and non-linear learning rates are often necessary in
practice for optimizing the transformer-based \sequencetosequence~model.
Dropout is also typically applied to both the GRU and transformer models during
training. 

\begin{algorithm}[t]
\KwData{~\\$\quad\mrtoks:$ input sequence\\$\quad\nbest:$ beam size\\$\quad T:$ maximum utterance length\\ $\quad\gen :$ generation model for scoring candidates.\\
$\quad\operatorname{rerank-score} :$ reranking hypothesis scoring function. }
$\mathcal{H} \gets \left\{\left[\starttok\right]\right\}$\\
$\mathcal{H}_{complete} \gets \left\{ \right\}$\\
\For{$i =1,\ldots, T$}{
    $\mathcal{H}_{new} \gets \left\{ \right\}$ \\
    \For{$\utttoks = \left[\utttok_1,\ldots,\utttok_i \right]\in \mathcal{H}$}{
        \If{$\utttok_i = \stoptok$}{
            $\mathcal{H}_{complete} \gets \mathcal{H}_{complete} \cup \{ \utttoks \}$\\
            $\mathcal{H} \gets \mathcal{H} \setminus \{ \utttoks \}$\\
        }

        \For{$\utttok^\prime \in \uttvocab$}{
            $\mathcal{H}_{new} \gets \mathcal{H}_{new} \cup \left\{\left[ \utttok_1,\ldots,\utttok_i, \utttok^\prime\right] \right\}$
        }
    }
    $\mathcal{H} \gets \operatorname{Top}_\nbest(\mathcal{H}_{new}, \gen(\cdot|\mrtoks))$\\

}

\KwResult{$\operatorname{Top}_1(\mathcal{H}_{complete}, \operatorname{rerank-score}(\cdot|\mrtoks))$}

\caption{Beam Search}
\label{alg:beamsearch}
\end{algorithm}


\subsection{Inference}
\label{sec:nlginference}

As stated above, $\gen\left(\cdot|\ls\left(\mr\right)\right)$ is a conditional
language model.  Given some \meaningrepresentation~$\mr$, a natural utterance
one might want to infer is the maximum a posteriori (MAP) utterance 
\[
 \predutttoks = \argmax_{\utttoks\in\uttSize} \log\gen\left(\utttoks|\ls\left(\mr\right)\right) 
\] 
under the model.\footnote{Technically, we are only considering valid finite
utterances $\utttoks = \left[\utttok_1,\ldots,\utttok_\uttSize\right] \in
\outSpace$ with $\utttok_1 = \starttok$ and $\utttok_\uttSize = \stoptok$. }
Unfortunately, the search implied by the $\argmax$ is intractable. Instead an
approximate search is performed. The most commonly used search is called
\textit{beam search} \citep{reddy1977} or beam decoding.  Under beam search, a
set $\mathcal{H}$ of $\nbest$-best candidate utterances is maintained
throughout the search. $\nbest$ is referred to as the beam size or beam width.
At each step $i$ of the search, the next word continuations are computed for
each candidate utterance prefix, yielding $\nbest \times \lvert\uttvocab\rvert$
utterances, from which the top-$\nbest$ under some search criterion are
selected, and $\mathcal{H}$ is updated the with the $\nbest$ utterances of
length $i+1$.  When a completed utterance enters $\mathcal{H}$ (i.e.,
$\utttok_{i+1} = \stoptok$), it is added to a set of completed utterances,
$\mathcal{H}_{complete}$, and removed from $\mathcal{H}$.  After exploring the
maximum number of steps $T$ (or some other heuristic stopping criterion),
$\mathcal{H}_{complete}$ is reranked according some heuristic reranking
criteria, and the top-ranked  utterance is returned.  When the beam size is 1,
we refer to the algorithm as greedy search or greedy decoding.  See
\autoref{alg:beamsearch} for a formal description of the algorithm.

Common reranking criteria include the length-normalized log likelihood, \[
\operatorname{rerank-score}(\utttoks,\mr) = \frac{\sum_{i=1}^\setsize{\utttoks}
\log \gen(\utttok_i|\utttoks_{1:i-1},\ls(\mr))}{\setsize{\utttoks}}\] or a
mixture of model likelihood and an auxiliary  language model, $\gen_{LM}$, \[
\operatorname{rerank-score}(\utttoks,\mr) = \log\gen(\utttoks|\ls(\mr)) +
\lambda \log \gen_{LM}(\utttoks).\] The latter method is popular in machine
translation where it is easier to obtain a large monolingual corpus with which
to train a language model than it is to obtain a large parallel corpus for
training the translation model \citep{xie2017neural}. When using
\sequencetosequence~models for the \meaningrepresentation-to-text generation
problem, practitioners often incorporate a discriminative
\meaningrepresentation~parser, $\dmodel(\cdot|\cdot) : \mrspace \times
\outSpace \rightarrow (0,1)$, in the reranker, \[
\operatorname{rerank-score}(\utttoks,\mr) = \log\dmodel(\mr|\utttoks),\] which
can help to select the most semantically correct utterances from the beam
candidates. Under this setting, for a candidate utterance $\predutttoks \in
\mathcal{H}_\textrm{complete}$ obtained with $ \gen\left( \cdot | \ls (\mr)
\right)$ using beam search, $\dmodel\left( \mr | \predutttoks \right)$ gives
the probability that $\denotes{\predutttoks} = \mr$ under the model $\dmodel$.

Despite its wide adoption and empirical success, however, there are many known
issues with beam search.  The output may repeat phrases or words
\citep{holtzman2019}, or may never even terminate
\citep{welleck2020consistency}.  While these issues are often linked to
differences in the maximum likelihood learning objective and test-time search
procedure \citep{lafferty2001,andor2016}, the problems are possibly deeper as
increasing the beam size often leads to worse empirical performance
\citep{koehn2017}. In fact, the biases present in beam search are actually
beneficial when compared to exact search \citep{stahlberg2019}.  Additionally,
it is well known that the set of output beam candidates may lack diversity and
only differ by a small number of words
\citep{sordoni2015,galley2015,li2016,vinyals2015,serban2016}.

\subsection{Sampling}

As an alternative to deterministic decoding, one may sample from the
conditional distribution, $\gen\left(\cdot|\ls(\mr)\right)$. The typical method
for doing this is called \ancestralsampling. \Ancestralsampling~is very similar
to greedy decoding, and works by sequentially sampling the next word
$\utttok_{i+1} \sim \gen\left(\cdot|\utttoks_{1:i},\ls(\mr)\right)$, and
terminating when $\utttok_{i+1} = \stoptok$. There are several modifications
one might make to \ancestralsampling~in practice. To encourage more diversity
in the sampled outputs, a temperature parameter $\temperature$ is sometimes
added to the final $\softmax$ layer, \[ \gen\left(\utttok_{i+1}|
\utttoks_{1:i},\ls(\mr); \temperature) \right) = \softmax\left(
\frac{\weight{o}\tfDecInputRow_i + \bias{o}}{\temperature}
\right)_{\utttok_{i+1}}.\] As $\temperature$ tends toward $+\infty$, the
conditional distribution becomes less peaked and the differences in probability
between any two words diminish, making it easier to sample an unusual
continuation of the utterance sequence.  In the positive limit, each word
becomes equally likely, \[ \lim_{\temperature\rightarrow +\infty}
\gen\left(\utttok| \utttoks_{1:i},\ls(\mr); \temperature) \right) =
\frac{1}{\setsize{\uttvocab}}.\] As $\temperature$ approaches zero, the
distribution becomes a ``one-hot'' distribution, \[
\lim_{\temperature\rightarrow +0} \gen\left(\utttok| \utttoks_{1:i},\ls(\mr);
\temperature) \right) = \mathds{1}\{\utttok = \argmax_{\utttok^\prime}
\gen\left(\utttok^\prime| \utttoks_{1:i},\ls(\mr)\right)  \}\] where the
probability is zero for every word except the most likely next word in the
original distribution, which now has probability one.

While ancestral sampling can lead to more diverse outputs, the next word
distributions are often quite peaked meaning most of the vocabulary accounts
for less than 0.05 of the cumulative distribution function. For a 20 word
sentence, this means that on average at least one word will be sampled from the
long-tail, effectively choosing a word uniformly at random from the vocabulary.
To avoid this issue, two heuristic modifications are often made to ancestral
sampling, \textit{top-$k$} sampling \citep{fan2018,holtzman2018,radford2019}
and \textit{nucleus} sampling \citep{holtzman2019}. 

In top-$k$ sampling, $p(\utttok|\utttoks_{1:i},\ls(\mr))$ is restricted to the
top $k$ most likely words. Let $\topkVocab{i}{k} \subset \uttvocab$ be the set
of $k$ most likely next words at sampling step $i$, i.e.,
\[\topkVocab{i}{k} = \argmax_{S \subset \uttvocab, \setsize{S} = k} \sum_{\utttok \in S} \log\gen(\utttok|\utttoks_{1:i},\ls(\mr)). \]
The next word $\utttok_{i+1}$ is then sampled from the following distribution,
\[
    \gen\left(\utttok_{i+1}|\utttoks_{1:i},\ls(\mr);\topkVocab{i}{k}\right)
    =
    \begin{cases} 
   \frac{
   \gen(\utttok_{i+1}|\utttoks_{1:i},\ls(\mr))
   }{ 
       \sum_{\utttok^\prime \in \topkVocab{i}{k}  }
   \gen(\utttok^\prime|\utttoks_{1:i},\ls(\mr))  
   }  & \utttok_{i+1} \in \topkVocab{i}{k} \\ 
0 & \textrm{otherwise}. \end{cases} 
\]

\citet{holtzman2019} show that picking the right $k$ for top-$k$ sampling is
difficult because the next word distribution can alternate from very flat
(which would suggest a large $k$) to very peaked (which would suggest a small
$k$).  Instead they propose restricting the subset of vocabulary to sample from
to the smallest set of words such that their cumulative probability is greater
than a threshold $\nucleusthr$,
\[ \nucleusVocab{i}{\nucleusthr} = \argmin_{\substack{S \subset \uttvocab \\
\sum_{\utttok \in S} \gen(\utttok|\utttoks_{1:i},\ls(\mr)) \ge \nucleusthr  }}
\setsize{S}. \] The sampling distribution for this method which they call
nucleus sampling, is computed similarly to top-$k$ sampling,
\[
\gen\left(\utttok_{i+1}|\utttoks_{1:i},\ls(\mr); \nucleusVocab{i}{\nucleusthr} \right)
    =
    \begin{cases} 
   \frac{
   \gen(\utttok_{i+1}|\utttoks_{1:i},\ls(\mr))
   }{ 
       \sum_{\utttok^\prime \in \nucleusVocab{i}{\nucleusthr} }
   \gen(\utttok^\prime|\utttoks_{1:i},\ls(\mr))  
   }  & \utttok_{i+1} \in \nucleusVocab{i}{\nucleusthr} \\ 
0 & \textrm{otherwise}. \end{cases} 
\]
Nucleus sampling helps avoid sampling from the long tail of the distribution
while still producing diverse samples.
