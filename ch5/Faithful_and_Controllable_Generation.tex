\chapter{Faithful and Controllable Generation}
\label{ch:nlg}

Up to this point, we have  focused on content selection in a
\texttotext~generation system while relying on a trivial text generation
algorithm, copying or extracting text units verbatim from the input, to perform
the actual generation task.  In this chapter, we move to modeling the actual
language generation process after the content selection stages (i.e.
\hyperref[ch:dlsum]{Chapters 3} and \hyperref[ch:mlsum]{4}) have been
performed.  We focus in particular on the \sequencetosequence~model as a means
of mapping a representation of selected content to an appropriate natural
language utterance. \Sequencetosequence~models are a family of deep learning
models with bi-partite structure, possessing an encoder network which
represents the input and a decoder which generates output from encoder's state
\citep{sutskever2014}.  We tackle two issues, faithfulness and control, which
are necessary prerequisites for any practically useful
\sequencetosequence-based model of \naturallanguagegeneration.  These concepts,
which we define in more detail later in this chapter, can be broadly construed
as ensuring the decoder (i.e., the language model) in a
\sequencetosequence~model is constrained to generate utterances that respect
the semantics of the input (i.e. ensuring model faithfulness) while following a
proscribed discourse structure for the selected content (i.e. ensuring model
control).

When data is plentiful, the \sequencetosequence~paradigm has proven to be
incredibly adaptable to a variety of problem domains, and in the research
literature it has become the standard method for a host of language generation
tasks \citep{xu2015,dusek2016,vaswani2017,fan2018}.  Recent evaluations of
end-to-end trained \sequencetosequence~models for dialogue generation have
shown that they are capable of learning very natural text realizations of
formal \meaningrepresentation. In many cases, they  beat rule and template
based systems on human and automatic measures of quality and naturalness
\citep{dusek2020}.
 
\begin{figure}[p]
\center
%\begin{tabular}{cp{9cm}}
\begin{subfigure}{.35\textwidth}
\center
\MR{\textsc{Inform}}
    {\AV{name}{Aromi}}
    {\AV{eat\_type}{coffee shop}}
    {\AV{customer\_rating}{5 out of 5}}
    {\AV{food}{English}}
    {\AV{area}{city centre}}
    {\AV{family\_friendly}{yes}}
 
\caption{\emph{Inform} dialog act.}\label{fig:informexample}
\end{subfigure}\hfill \begin{subfigure}{.58\textwidth}
\begin{itemize}
\item Aromi coffee shop serves English food in a family-friendly atmosphere near the city center and has a customer rating of 5 out of 5.
\item The Aromi coffee shop is family-friendly and serves English food.  It has a customer rating of 5 out of 5 and is located near the center of the city.
\end{itemize}
\caption{Natural language utterances.}
\label{fig:mr1utt}
\end{subfigure}

%"name[Aromi], eatType[coffee shop], food[English], customer rating[5 out of 5], area[city centre], familyFriendly[yes]",
%"name[Aromi], eatType[coffee shop], food[English], customer rating[5 out of 5], area[city centre], familyFriendly[yes]"
%"name[Aromi], eatType[coffee shop], food[English], customer rating[5 out of 5], area[city centre], familyFriendly[yes]",,0,"name[Aromi], eatType[coffee shop], food[English], customer rating[5 out of 5], area[city centre], familyFriendly[yes]"


~\\~\\

\begin{subfigure}{.35\textwidth}
\center
$\left[\!\!\left[\begin{array}{l} 
    \textsc{Give Opinion} \\ 
    \textrm{name=Little Nightmares} \\
    \textrm{rating=good} \\
    \textrm{genres=[}\\
    \textrm{~~~~adventure,} \\
    \textrm{~~~~platformer,}\\
    \textrm{~~~~puzzle} \\
    \textrm{]} \\
    \textrm{player\_perspective=side view}
\end{array}\right]\!\!\right]$ 
\caption{\emph{Give Opinion} dialog act.}\label{fig:giveopinionexample}
\end{subfigure}\hfill \begin{subfigure}{.58\textwidth}
\begin{itemize}
    \item Adventure games that combine platforming and puzzles can be frustrating to play, but the side view perspective is perfect for them. That's why I enjoyed playing Little Nightmares.

    \item Little Nightmares is a pretty cool game that has kept me entertained. It's an adventure side-scrolling platformer with some puzzle elements to give me a bit of a challenge.
\end{itemize}
\caption{Natural language utterances.}
\label{fig:mr1utt}
\end{subfigure}


~\\~\\


\begin{subfigure}{.35\textwidth}
\center
$\left[\!\!\left[\begin{array}{l} 
    \textsc{Compare} \\ 
    \textrm{name\textsubscript{1}=Erebus 92} \\
    \textrm{resolution\textsubscript{1}=720p} \\
    \textrm{family\textsubscript{1}=W2} \\[8pt]
    
    \textrm{name\textsubscript{2}=Helios 96} \\
    \textrm{resolution\textsubscript{2}=1080p} \\
    \textrm{family\textsubscript{2}=L5} \\
\end{array}\right]\!\!\right]$ 

~\\
~\\

\caption{\emph{Compare} dialog act.}\label{fig:compareexample}
\end{subfigure}\hfill \begin{subfigure}{.58\textwidth}
\begin{itemize}
    \item Tthe Helios 96 tv has a 1080p resolution in the L5 family while the Erebus 92 has a 720p resolution in the W2 family. 
    \item Compared to Erebus 92 which has 720p resolution and is in the W2 product family, Helios 96 has 1080p resolution and is in the L5 product family. Which one do you prefer?
\end{itemize}
\caption{Natural language utterances.}
\label{fig:mr1utt}
\end{subfigure}

~\\~\\

\begin{subfigure}{.35\textwidth}
\center

~\\~\\

$\left[\!\!\left[\begin{array}{l} 
    \textsc{Confirm} \\ 
    \textrm{type=laptop} \\
    \textrm{drive\_range=medium} \\
\end{array}\right]\!\!\right]$ 

~\\~\\~\\

\caption{\emph{Confirm} dialog act.}\label{fig:confirmexample}
\end{subfigure}\hfill \begin{subfigure}{.58\textwidth}
\begin{itemize}
   \item Just to verify. The laptop needs to have a medium drive range, correct?
   \item Let me confirm, a laptop in the medium drive range right?
\end{itemize}
\caption{Natural language utterances.}
\label{fig:mr1utt}
\end{subfigure}



\caption{Example \meaningrepresentation s (left) and their reference utterances (right)
from the restaurant, video game, tv, and laptop domains.}
\label{fig:nlgexamples}
\end{figure}



However, this powerful generation capability comes with a cost;
\deeplearning-based language models are notoriously difficult to control, often
producing quite fluent but  semantically misleading outputs. For instance,
\citet{dusek2020} note that in the E2E Generation Challenge shared-task,
\sequencetosequence~models were both some of the best and worst performers. One
submission, a vanilla \sequencetosequence~model, scored first in human
evaluations of naturalness but last in quality (which they define as semantic
completeness and grammaticality).  In order for such models to truly be useful,
they must be capable of correctly generating utterances for novel
\meaningrepresentation s at test time.  In practice, even with delexicalization
\citep{dusek2016,juraska2018}, copy and coverage mechanisms \citep{elder2018},
and over-generation plus reranking \citep{dusek2016,juraska2018},
\deeplearning-based language generation models still produce errors
\citep{dusek2020}.

We study \sequencetosequence~model faithfulness on the \textit{task-oriented
dialog generation} problem \citep{mairesse2010,wen2015,dusek2018}, where a
\naturallanguagegeneration~model must map a \meaningrepresentation~(i.e., a
dialogue act with an associated set of attribute-value pairs\footnote{In the
literature and in industry, dialogue acts are sometimes called ``intents,'' and
attribute-value pairs as ``slots'' and ``slot-fillers'' or ``entities.''}) to
an appropriate natural language utterance (see \autoref{fig:nlgexamples} for
examples).  In the context of our broader work on \texttotext~generation, we
think of the \meaningrepresentation~input as an idealized representation of the
content selection stage in a \texttotext~generation model.  Studying
faithfulness and control in the closed-world domains of task-oriented dialog
generation allows us to make meaningful progress while minimizing unnecessary
complexity.

For instance, natural language summaries often contain information not
explicitly represented in the input. The source of this content is either from
common sense knowledge, generic or domain specific knowledge, or new facts
deduced from any combination of the input and prior knowledge
\citep{wiseman2017,wang2019}.  Evaluating the faithfulness of a neural language
generation model in this context is extremely difficult because it is not clear
if a generated utterance is due to the decoder language model or the encoder's
representation of the input.

Instead, the task-oriented dialogue generation problems we study are developed
to be closed-world, narrow domain settings, where the totality of the
information needed to generate an utterance is represented by the
\meaningrepresentation. Additionally, the semantics of the
\meaningrepresentation~are explicit; there is no information that needs to be
realized by the language generation component that requires additional
inferences from the input.

We call a \naturallanguagegeneration~model that generates utterances that are
semantically correct with respect to the input \meaningrepresentation, a
\faithfulgeneration~model.  We posit that \sequencetosequence~models do not
learn representations of the input \meaningrepresentation~that correspond to
basic features of the nature of utterance data, chiefly that phrases denote
fragments of \meaningrepresentation~which can be recombined with other
fragments to systematically create new meanings/utterances. Instead, the
learned representations are highly idiosyncratic, and often reflect spurious
correlations and artifacts of the dataset that do not generalize well outside
the training data.  This issue is symptomatic of neural models' lack of
systematicity, which in turn  leads to unfaithful language generation models
\citep{lake18}.

To overcome these issues, we propose a novel data augmentation scheme, called
\textit{noise-injection sampling and self-training}, to create synthetic
\meaningrepresentation/utterance pairs which break spurious correlations in the
training dataset.
%(see \autoref{fig:noiseyexample} for an illustrative example). 
Our method makes use of a vanilla
\sequencetosequence~\naturallanguagegeneration~model, i.e. the kind described
by \citet{dusek2020} that produces natural but semantically incorrect
utterances, and a \meaningrepresentation~parser, both of which can be trained
on the same parallel data.  We then use a noise-injection sampling method
\citep{cho2016} that allows us to generate semantically diverse yet
syntactically well formed utterances from the \naturallanguagegeneration~model.
We obtain corrected \meaningrepresentation s for these sampled utterances using
the \meaningrepresentation~parser.  Using this procedure we can generate a
large collection of synthetic data points which show a reduction in spurious
correlations between the size of a meaning representation and its content or
between different pairs of attribute-values. Training a new
\sequencetosequence~model on the union of the original training and novel
synthetic data yields a more \faithful~generation model with substantially
reduced semantic errors.

\begin{figure}

    \begin{subfigure}[t]{.30\textwidth}
            \centering
    \MR{\textsc{Inform}}{\AV{name}{Aromi}}{\AV{area}{city centre}}{\AV{eat\_type}{coffee shop}}


    ~\\~\\~\\~\\

    \caption{\textit{Inform} dialog act.}
\end{subfigure}
\hfill
\begin{subfigure}[t]{.65\textwidth}
    \small
    \centering
        \begin{tabular}{l}
         %   \toprule
            ~\\
            Plan A:~~~~~~~~name~~~$\rightarrow$~~~eat\_type~~~$\rightarrow$~~~area\\[4pt]
            Realization: \textit{Aromi is a coffee shop in the city 
                centre.}\\
                ~\\
            %\midrule
                Plan B:~~~~~~~~eat\_type~~~$\rightarrow$~~~name~~~$\rightarrow$~~~area\\[4pt]
            Realization: \textit{There is a coffee shop called Aromi in the 
                city centre.}\\
            %\midrule
                ~\\
                Plan C:~~~~~~~~eat\_type~~~$\rightarrow$~~~area~~~$\rightarrow$~~~name\\[4pt]
            Realization: \textit{For coffee in the centre of the city, try 
                Aromi.}\\
        %    \bottomrule
                ~\\
        \end{tabular}
        \caption{Three different utterance plans with example realizations
        for the \textit{Inform} dialogue act (left).}
\end{subfigure}
\caption{Examples of controllable generation. (a) A \meaningrepresentation~of an \textit{Inform} dialogue act. (b) Three hypothetical utterance plans and their realizations for the example dialogue act.}
\label{fig:examplecontrol}
\end{figure}


While a faithful \sequencetosequence~model produces semantically correct
output, in general it is free to let the surface realization order of the
\attributevalues~be determined by its language model. We show that we can
actually control the realization order by properly ``linearizing'' the
\meaningrepresentation, that is, converting the \meaningrepresentation~into a
linear sequence of discrete tokens, before feeding it into the encoder. Our
proposed \textit{alignment training} linearization strategy for converting a
\meaningrepresentation~to an encoder input sequence yields a highly
controllable generation model, effectively moving implicit utterance planning
from the decoder to the encoder.  We consider \controllablegeneration~models to
be a subset of \faithfulgeneration~models that can follow an externally
provided discourse ordering plan.  See \autoref{fig:examplecontrol} for
examples of such plans in the context of  task-oriented dialogue generation.
We find that alignment training endows both recurrent and transformer-based
\sequencetosequence~architectures with the controllable generation property as
well as when fine tuning a large, pretrained \sequencetosequence~model.

While most contemporary research practice prefers end-to-end solutions that
leave planning implicit, we argue that such fine grained control in a
\sequencetosequence~model  is highly desirable.  Not only would it enable
drawing deeper connections between \sequencetosequence~models and the extensive
literature on sentence or utterance level planning for language generation
\citep{reiter2000,walker2001,stone2003}, it would also allow for neural
implementations of various psycho-linguistic theories of discourse (e.g.,
Centering Theory \citep{grosz1995}, or Accessibility Theory \citep{ariel2001}).
This could, in turn, encourage the validation and/or refinement of additional
psychologically plausible models of language production. 

Ensuring robustness of the control behavior is also necessary to reliably
incorporate neural language generation models into larger language generation
pipelines \citep{moryossef2019a,moryossef2019b,castroferreira2019}.  However,
as previously mentioned, neural models do not learn systematic representations
of the input, which can lead to errors in faithfulness or plan following when
generating from ordering plans not well represented in the training data. To
mitigate this, we propose a phrase-based data augmentation scheme to collect
additional examples that give explicit supervision of how constituent phrases
compose, and how that composition can systematically change the meaning (e.g.
prepending ``not'' to a phrase systematically negates its meaning). We show
under extensive stress testing with randomly generated plans that this
data-augmentation improves the robustness of control.

In what follows, we introduce the \meaningrepresentation s used for
task-oriented dialogue generation in more detail (\autoref{sec:mr4tod}) and
provide some background on \sequencetosequence~modeling for
\meaningrepresentation-to-text generation (\autoref{mrtproblemdef}). We then
turn to our main contributions, our noise-injection sampling and self-training
data-augmentation method for \faithfulgeneration~(\autoref{sec:nlgfg}), and
alignment training linearization~for
\controllablegeneration~(\autoref{sec:nlgcg}), before concluding.
