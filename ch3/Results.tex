\begin{table*}[p]
    \center
    %\begin{tabular}{ccggccggccggcc}
    \begin{tabular}{cccccccccccccc}
        \toprule
        \multirow{2}{*}{Extractor} &\multirow{2}{*}{Encoder}  & \multicolumn{2}{c}{\textbf{CNN/DM}} & \multicolumn{2}{c}{\textbf{NYT}} & \multicolumn{2}{c}{\textbf{DUC 2002}} \\
         &  & M & R-2 & M & R-2 & M & R-2 \\
        \midrule
%        Random &  -- & 18.2 & 12.7 & 20.7 & 16.0 & 21.7 & 15.8 & \textbf{20.4} & \textbf{12.9} & 12.7 &  2.0 & 16.9 &  9.3\\
%        \hline
        Lead &  -- & 24.1 & 24.4 & 30.0 & 32.3 & 25.1 & 21.5 \\
%        \hline
%        Tail &  -- & 13.7 &  5.4 & 12.5 &  4.1 & 17.8 &  9.1 & \textbf{19.8} & \textbf{10.0} & 11.5 &  2.7 & 17.0 &  9.9\\
        \hline
        \multirow{3}{*}{RNN} & Avg. & \textbf{25.2} & 25.4 & 29.8 & 34.7 & \textbf{26.8} & 22.7 \\
         & RNN & 25.1 & 25.4 & 29.6 & 34.9 & \textbf{26.8} & 22.6 \\
         & CNN & 25.0 & 25.1 & 29.0 & 33.7 & \textbf{26.7} & \textbf{22.7}\\
        \hline
        \multirow{3}{*}{Seq2Seq} & Avg. & \textbf{25.2} & \textbf{25.6} & \textbf{30.5} & \textbf{35.7} & \textbf{27.0} & \textbf{22.8} \\
         & RNN & \textbf{25.1} & 25.3 & 30.2 & \textbf{35.9} & \textbf{26.7} & 22.5 \\
         & CNN & 25.0 & 25.1 & 29.9 & 35.1 & \textbf{26.7} & \textbf{22.7} \\
        \hline
    \multirow{3}{*}{\begin{tabular}{c} Cheng \\ \& \\ Lapata \end{tabular}} & Avg. & 25.0 & 25.3 & 30.4 & \textbf{35.6} & \textbf{27.1} & \textbf{23.1} \\
         & RNN & 25.0 & 25.0 & \textbf{30.3} & \textbf{35.8} & \textbf{27.0} & \textbf{23.0} \\
         & CNN & \textbf{25.2} & 25.1 & 29.9 & 35.0 & \textbf{26.9} & \textbf{23.0} \\
        \hline
    \multirow{3}{*}{\begin{tabular}{c}Summa \\ Runner \end{tabular} } & Avg. & 25.1 & 25.4 & 30.2 & 35.4 & 26.7 & 22.3 \\
         & RNN & 25.1 & 25.2 & 30.0 & 35.5 & 26.5 & 22.1 \\
         & CNN & 24.9 & 25.0 & 29.3 & 34.4 & 26.4 & 22.2 \\
        \hline
        Oracle & -- & 31.1 & 36.2 & 35.3 & 48.9 & 31.3 &  31.8 \\
        \bottomrule
    \end{tabular}

    \caption{News domain \meteor~(M) and \rouge-2 recall (R-2)  results across all 
        extractor/encoder pairs.
           Results that are statistically indistinguishable from the best 
           system are shown in bold face.}
  \label{tab:newsresults}
\end{table*}

\begin{table*}[p]
    \center
    %\begin{tabular}{ccggccggccggcc}
    \begin{tabular}{cccccccccccccc}
        \toprule
        \multirow{2}{*}{Extractor} &\multirow{2}{*}{Encoder}  &  \multicolumn{2}{c}{\textbf{Reddit}} & \multicolumn{2}{c}{\textbf{AMI}} & \multicolumn{2}{c}{\textbf{PubMed}}\\
         &  & M & R-2 & M & R-2 & M & R-2 \\
        \midrule
%        Random &  -- & 18.2 & 12.7 & 20.7 & 16.0 & 21.7 & 15.8 & \textbf{20.4} & \textbf{12.9} & 12.7 &  2.0 & 16.9 &  9.3\\
%        \hline
        Lead &  -- &  \textbf{20.1} & \textbf{10.9} & 12.3 &  2.0 & 15.9 &  9.3\\
%        \hline
%        Tail &  -- & 13.7 &  5.4 & 12.5 &  4.1 & 17.8 &  9.1 & \textbf{19.8} & \textbf{10.0} & 11.5 &  2.7 & 17.0 &  9.9\\
        \hline
        \multirow{3}{*}{RNN} & Avg. &  \textbf{20.4} & \textbf{11.4} & \textbf{17.0} & \textbf{ 5.5} & 19.8 & 17.0\\
         & RNN &  \textbf{20.2} & \textbf{11.4} & 16.2 & \textbf{ 5.2} & 19.7 & 16.6\\
         & CNN & \textbf{20.9} & \textbf{12.8} & 14.4 &  3.2 & 19.9 & 16.8\\
        \hline
        \multirow{3}{*}{Seq2Seq} & Avg. &  \textbf{20.9} & \textbf{13.6} & \textbf{17.0} & \textbf{ 5.5} & \textbf{20.1} & \textbf{17.7}\\
         & RNN &  \textbf{20.5} & \textbf{12.0} & 16.1 & \textbf{ 5.3} & 19.7 & 16.7\\
         & CNN &  \textbf{20.7} & \textbf{13.2} & 14.2 &  2.9 & 19.8 & 16.9\\
        \hline
    \multirow{3}{*}{\begin{tabular}{c} Cheng \&  Lapata \end{tabular}} & Avg. &  \textbf{20.9} & \textbf{13.6} & \textbf{16.7} & \textbf{ 6.1} & \textbf{20.1} & \textbf{17.7}\\
         & RNN &  \textbf{20.3} & \textbf{12.6} & \textbf{16.3} & \textbf{ 5.0} & 19.7 & 16.7\\
         & CNN &  \textbf{20.5} & \textbf{13.4} & 14.3 &  2.8 & 19.9 & 16.9\\
        \hline
    \multirow{3}{*}{\begin{tabular}{c}Summa Runner \end{tabular} } & Avg. & \textbf{21.0} & \textbf{13.4} & \textbf{17.0} & \textbf{ 5.6} & 19.9 & 17.2\\
         & RNN &\textbf{20.9} & \textbf{12.5} & \textbf{16.5} & \textbf{ 5.4} & 19.7 & 16.5\\
         & CNN &  \textbf{20.4} & \textbf{12.3} & 14.5 &  3.2 & 19.8 & 16.8\\
        \hline
        Oracle & -- &  24.3 &  16.2 &17.8 &  8.7  & 24.1 &25.0 \\
        \bottomrule
    \end{tabular}

    \caption{Non-news domain \meteor~(M) and \rouge-2 recall (R-2)  results across all 
        extractor/encoder pairs.
           Results that are statistically indistinguishable from the best 
           system are shown in bold face.}
  \label{tab:otherresults}
\end{table*}


\section{Results}

The results of our main experiment comparing the different extractors/encoders
on news and non-news domains are shown in \autoref{tab:newsresults} and
\autoref{tab:otherresults} respectively.  Overall, we find no major advantage
when using the \convolutionalneuralnetwork~and \recurrentneuralnetwork~sentence
encoders over the averaging encoder. The best performing encoder/extractor pair
either uses the averaging encoder (five out of six datasets) or the differences
are not statistically significant. 

When looking at extractors, the Seq2Seq extractor is either part of the best
performing system (three out of six datasets) or is not statistically
distinguishable from the best extractor. 

Overall, on the news and medical journal domains, the differences are quite
small with the differences between worst and best systems on the CNN/DM dataset
spanning only .56 of a ROUGE point. While there is more performance variability
in the Reddit and AMI data, there is less distinction among systems: no
differences are significant on Reddit and every extractor has at least one
configuration that is indistinguishable from the best system on the AMI corpus.
This is probably due to the small test size of these datasets.

\input{ch3/tables/table_embs.tex}

\subsection{Ablation Experiments}

In addition to our main evaluation above, we also perform several ablation
experiments to further understand how the various summarization models perform
when certain information is witheld from the model. In particular, we evaluate
the effect of fine-tuning word embeddings, part-of-speech (POS) based
ablations, and sentence-order shuffling.

\paragraph{Word Embedding Fine-tuning} Given that learning a sentence encoder
(averaging has no learned parameters) does not yield significant improvement,
it is natural to consider whether fine-tuning word embeddings is also
necessary.  In \autoref{tab:embeddings} we compare the performance of different
extractors using the averaging encoder, when the word embeddings are held fixed
or fine-tuned during training. In both cases, word embeddings are initialized
with GloVe embeddings trained on a combination of Gigaword and Wikipedia.  When
fine-tuning embeddings, words occurring fewer than three times in the training
data are mapped to an unknown token (with learned embedding).
 
In all but one case, fixed embeddings are as good or better than the fine-tuned
embeddings.  This is a somewhat surprising finding on the CNN/DM data since it
is reasonably large, and learning embeddings should give the models more
flexibility to identify important word features.\footnote{The AMI corpus is an
exception here where learning \emph{does} lead to small performance boosts,
however, only in the Seq2Seq extractor is this diference significant; it is
quite possible that this is an artifact of the very small test set size.} This
suggests that we cannot extract much generalizable learning signal from the
content other than what is already present from initialization.  Even on
PubMed, where the language is quite different from the news/Wikipedia articles
the GloVe embeddings were trained on, fine-tuning leads to significantly worse
results.

\input{ch3/tables/table_pos.tex}

\paragraph{POS Ablation} It is also not well explored what word features are
being used by the encoders.  To understand which classes of words were most
important we ran an ablation study, selectively removing nouns, verbs
(including participles and auxiliaries), adjectives \& adverbs, and function
words (adpositions, determiners, conjunctions).  All datasets were
automatically tagged using the spaCy POS
tagger\footnote{https://github.com/explosion/spaCy}.   The embeddings of
removed words were replaced with a zero vector, preserving the order and
position of the non-ablated words in the sentence.  Ablations were performed on
training, validation, and test partitions, using the RNN extractor with
averaging encoder.  Note that while the input to the models has redacted word
classes, the produced summaries are evaluated with all words present. See
\autoref{fig:wordabl} for example inputs under the different word class
ablations.

\begin{figure}
\fbox{\begin{minipage}{\textwidth}
~\\
\begin{minipage}[t]{0.45\textwidth}
\begin{center}\textbf{Nouns Redacted}\end{center}~\\[-50pt]
\begin{singlespace}
\small
\begin{enumerate}
\item \colorbox{black}{hurricane} \colorbox{black}{gilbert} swept toward the \colorbox{black}{dominican} \colorbox{black}{republic} \colorbox{black}{sunday}, and the \colorbox{black}{civil} \colorbox{black}{defense} alerted its heavily populated south \colorbox{black}{coast} to prepare for high \colorbox{black}{winds}, heavy \colorbox{black}{rains} and high \colorbox{black}{seas}.
\item the \colorbox{black}{storm} was approaching from the \colorbox{black}{southeast} with sustained \colorbox{black}{winds} of 75 \colorbox{black}{mph} gusting to 92 \colorbox{black}{mph} .
\item ``There is no \colorbox{black}{need} for \colorbox{black}{alarm},'' civil \colorbox{black}{defense} \colorbox{black}{director} \colorbox{black}{eugenio} \colorbox{black}{cabral} said in a \colorbox{black}{television} \colorbox{black}{alert} shortly before \colorbox{black}{midnight} \colorbox{black}{saturday}.
\item \colorbox{black}{cabral} said \colorbox{black}{residents} of the \colorbox{black}{province} of \colorbox{black}{barahona} should closely follow \colorbox{black}{gilbert}'s \colorbox{black}{movement}.
\end{enumerate}
\end{singlespace}
\end{minipage}~~~~~~~\begin{minipage}[t]{0.45\textwidth}
\begin{center}\textbf{Verbs Redacted}\end{center}~\\[-50pt]
\begin{singlespace}
\small
\begin{enumerate}
\item Hurricane Gilbert \colorbox{black}{swept} toward the Dominican Republic Sunday, and the Civil Defense \colorbox{black}{alerted} its heavily populated south coast \colorbox{black}{to} \colorbox{black}{prepare} for high winds, heavy rains and high seas.
\item The storm \colorbox{black}{was} \colorbox{black}{approaching} from the southeast with sustained winds of 75 mph \colorbox{black}{gusting} to 92 mph.
\item ``There \colorbox{black}{is} no need for alarm,'' Civil Defense Director Eugenio Cabral \colorbox{black}{said} in a television alert shortly before midnight Saturday.
\item Cabral \colorbox{black}{said} residents of the province of Barahona \colorbox{black}{should} closely \colorbox{black}{follow} Gilbert \colorbox{black}{'s} movement.
\end{enumerate}
\end{singlespace}
\end{minipage}

~\\
~\\

\begin{minipage}{0.45\textwidth}
\begin{center}\textbf{Adjectives/Adverbs Redacted}\end{center}~\\[-50pt]
\begin{singlespace}
\small
\begin{enumerate}
\item Hurricane Gilbert swept toward the Dominican Republic Sunday, and the Civil Defense alerted \colorbox{black}{its} \colorbox{black}{heavily} \colorbox{black}{populated} \colorbox{black}{south} coast to prepare for \colorbox{black}{high} winds, \colorbox{black}{heavy} rains and \colorbox{black}{high} seas.
\item The storm was approaching from the southeast with \colorbox{black}{sustained} winds of 75 mph gusting to 92 mph.
\item ``\colorbox{black}{there} is no need for alarm,'' \colorbox{black}{civil} Defense Director Eugenio Cabral said in a television alert \colorbox{black}{shortly} before midnight Saturday.
\item Cabral said residents of the province of Barahona should \colorbox{black}{closely} follow Gilbert's movement.
\end{enumerate}
\end{singlespace}
\end{minipage}~~~~~~~\begin{minipage}{0.45\textwidth}
\begin{center}\textbf{Function Words Redacted}\end{center}~\\[-50pt]
\begin{singlespace}
\small
\begin{enumerate}
\item Hurricane Gilbert swept \colorbox{black}{toward} \colorbox{black}{the} Dominican Republic Sunday, \colorbox{black}{and} \colorbox{black}{the} Civil Defense alerted its heavily populated south coast to prepare \colorbox{black}{for} high winds, heavy rains \colorbox{black}{and} high seas.
\item \colorbox{black}{the} storm was approaching \colorbox{black}{from} \colorbox{black}{the} southeast \colorbox{black}{with} sustained winds \colorbox{black}{of} 75 mph gusting \colorbox{black}{to} 92 mph .
\item ``There is \colorbox{black}{no} need \colorbox{black}{for} alarm,'' Civil Defense Director Eugenio Cabral said \colorbox{black}{in} \colorbox{black}{a} television alert shortly \colorbox{black}{before} midnight Saturday.
\item Cabral said residents \colorbox{black}{of} \colorbox{black}{the} province \colorbox{black}{of} Barahona should closely follow Gilbert's movement.
\end{enumerate}
\end{singlespace}
\end{minipage}

~\\
\end{minipage}}
\caption{The first four sentences from a DUC 2002 article (id: d061j-AP880911-0016) under the different word class ablations. }
\label{fig:wordabl}
\end{figure}


\autoref{tab:ablations} shows the results of the POS tag ablation experiments.
While removing any word class from the representation generally hurts
performance (with statistical significance), on the news domains, the absolute
values of the differences are quite small (.18 on CNN/DM, .41 on NYT, .3 on
DUC) suggesting that the model's predictions are not overly dependent on any
particular word types. 

Qualitatively, we see little difference in outputs produced under the redacted
models.  For example, see \autoref{fig:sumwordabl} where we show output
summaries from the different word class redacted models. All four variants
identify the lead two sentences as most important, while also including
sentences five and six. Three of the summaries also selected sentence 13.  The
summarizers appear to have selected most content from the front of the article
suggesting the lead is identfiable even when different content is ablated.

\begin{figure}[p]
    \fbox{   \begin{minipage}{\textwidth}
 \begingroup
  \setlength{\fboxsep}{0pt}%  
\small 
\textbf{Nouns Redacted}\\
\colorbox{Orchid!25!}{\begin{minipage}{\textwidth}(1) Hurricane Gilbert swept toward the Dominican Republic Sunday, and the Civil Defense alerted its heavily populated south coast to prepare for high winds, heavy rains and high seas.\end{minipage}}\\
\colorbox{Dandelion!25!}{\begin{minipage}{\textwidth}(2) The storm was approaching from the southeast with sustained winds of 75 mph gusting to 92 mph.\end{minipage}}\\
\colorbox{GreenYellow!35!}{\begin{minipage}{\textwidth}(5) An estimated 100,000 people live in the province, including 70,000 in the city of Barahona, about 125 miles west of Santo Domingo.\end{minipage}}\\
\colorbox{ProcessBlue!15!}{\begin{minipage}{\textwidth}(6) Tropical Storm Gilbert formed in the eastern Caribbean and strengthened into a hurricane Saturday night.\end{minipage}}\\
%\colorbox{red!25!}{
\begin{minipage}{\textwidth}(7) The National Hurricane Center in Miami reported its position at 2 a.m. Sunday at latitude 16.1 north, longitude 67.5 west, about 140 miles south of Ponce, Puerto Rico, and 200 miles southeast of Santo Domingo.\end{minipage}\\

\textbf{Verbs Redacted}\\
\colorbox{Orchid!25!}{\begin{minipage}{\textwidth}(1) Hurricane Gilbert swept toward the Dominican Republic Sunday, and the Civil Defense alerted its heavily populated south coast to prepare for high winds, heavy rains and high seas.\end{minipage}}\\
\colorbox{Dandelion!25!}{\begin{minipage}{\textwidth}(2) The storm was approaching from the southeast with sustained winds of 75 mph gusting to 92 mph.\end{minipage}}\\
\colorbox{green!25!}{\begin{minipage}{\textwidth}(13) On Saturday, Hurricane Florence was downgraded to a tropical storm and its remnants pushed inland from the U.S. Gulf Coast.\end{minipage}}\\
\colorbox{ProcessBlue!15!}{\begin{minipage}{\textwidth}(6) Tropical Storm Gilbert formed in the eastern Caribbean and strengthened into a hurricane Saturday night.\end{minipage}}\\
(12) San Juan, on the north coast, had heavy rains and gusts Saturday, but they subsided during the night.\\
\colorbox{GreenYellow!35!}{\begin{minipage}{\textwidth}(5) An estimated 100,000 people live in the province, including 70,000 in the city of Barahona, about 125 miles west of Santo Domingo.\end{minipage}}\\

\textbf{Adjectives/Adverbs Redacted}\\
\colorbox{Dandelion!25!}{\begin{minipage}{\textwidth}(2) The storm was approaching from the southeast with sustained winds of 75 mph gusting to 92 mph.\end{minipage}}\\
\colorbox{Orchid!25!}{\begin{minipage}{\textwidth}(1) Hurricane Gilbert swept toward the Dominican Republic Sunday, and the Civil Defense alerted its heavily populated south coast to prepare for high winds, heavy rains and high seas.\end{minipage}}\\
\colorbox{GreenYellow!35!}{\begin{minipage}{\textwidth}(5) An estimated 100,000 people live in the province, including 70,000 in the city of Barahona, about 125 miles west of Santo Domingo.\end{minipage}}\\
\colorbox{green!25!}{\begin{minipage}{\textwidth}(13) On Saturday, Hurricane Florence was downgraded to a tropical storm and its remnants pushed inland from the U.S. Gulf Coast.\end{minipage}}\\
\colorbox{ProcessBlue!15!}{\begin{minipage}{\textwidth}(6) Tropical Storm Gilbert formed in the eastern Caribbean and strengthened into a hurricane Saturday night.\end{minipage}}\\
(14) Residents returned home, happy to find little damage from 80 mph winds and sheets of rain.\\

\textbf{Function Words Redacted}\\
\colorbox{Orchid!25!}{\begin{minipage}{\textwidth}(1) Hurricane Gilbert swept toward the Dominican Republic Sunday, and the Civil Defense alerted its heavily populated south coast to prepare for high winds, heavy rains and high seas.\end{minipage}}\\
\colorbox{Dandelion!25!}{\begin{minipage}{\textwidth}(2) The storm was approaching from the southeast with sustained winds of 75 mph gusting to 92 mph.\end{minipage}}\\
\colorbox{green!25!}{\begin{minipage}{\textwidth}(13) On Saturday, Hurricane Florence was downgraded to a tropical storm and its remnants pushed inland from the U.S. Gulf Coast.\end{minipage}}\\
\colorbox{ProcessBlue!15!}{\begin{minipage}{\textwidth}(6) Tropical Storm Gilbert formed in the eastern Caribbean and strengthened into a hurricane Saturday night.\end{minipage}}\\
(10) Strong winds associated with the Gilbert brought coastal flooding, strong southeast winds and up to 12 feet feet to Puerto Rico's south coast.\\
\colorbox{GreenYellow!35!}{\begin{minipage}{\textwidth}(5) An estimated 100,000 people live in the province, including 70,000 in the city of Barahona, about 125 miles west of Santo Domingo.\end{minipage}}\\
\endgroup
\end{minipage}}
\caption{Outputs from the word class ablated models when given the document as  input from
\autoref{fig:wordabl}. Original document positions of the extracted sentences are  shown in parenthesis; sentences that are selected by multiple systems
are highlighted in color.}
\label{fig:sumwordabl}
\end{figure}


On the non-news datasets, the ablations have a larger effect (max differences
are 1.89 on Reddit, 2.56 on AMI, and 1.3 on PubMed).  Removing nouns leads to
the largest drop on AMI and PubMed.  Removing adjectives and adverbs leads to
the largest drop on Reddit, suggesting the intensifiers and descriptive words
are useful for identifying important content in personal narratives.
Curiously, removing the function word POS class yields a significant
improvement on DUC 2002 and AMI.

\input{ch3/tables/table_order.tex}

\textbf{Sentence Order Shuffling} Sentence position is a well known and
powerful feature for news summarization \citep{hong2014improving}, owing to the
intentional lead bias in the news article
writing\footnote{\url{https://en.wikipedia.org/wiki/Inverted_pyramid_(journalism)}};
it also explains the difficulty in beating the lead baseline for
single-document summarization \citep{nenkova2005b,brandow1995}.  In examining
the generated summaries, we found most of the selected sentences in the news
domain came from the lead paragraph of the document. This is despite the fact
that there is a long tail of sentence extractions from later in the document in
the ground truth extract summaries (31\%, 28.3\%, and 11.4\% of DUC, CNN/DM,
and NYT training extract labels come from the second half of the document).
Because this lead bias is so strong, it is questionable whether the models are
learning to identify important content or just find the start of the document.
We conduct a sentence order experiment where each document's sentences are
randomly shuffled during training. We then evaluate each model performance on
the unshuffled test data, comparing to the model trained on unshuffled data; if
the models trained on shuffled data drop in performance, then this indicates
the lead bias is the relevant factor.

\autoref{tab:shuffle} shows the results of the shuffling experiments.  The news
domains and PubMed suffer a significant drop in performance when the document
order is shuffled. By comparison, there is no significant difference between
the shuffled and in-order models on the Reddit domain, and shuffling actually
improves performance on AMI.  This suggest that position is being learned by
the models in the news/journal article domain even when the model has no
explicit position features, and that this feature is more important than either
content or function words.
