\section{Problem Definition}

We now formally define the sentence extractive, single document summarization
task as a sequence tagging problem, following \cite{conroy2001}.  Let a
document $\doc\in \docSpace$ be a sequence of $\docSize$ sentences, 
\[ 
    \doc = \left[ \sent_1, \sent_2, \ldots, \sent_\docSize\right].
\] 
Sentences are themselves sequences of words,
\[
    \sent_i=\left[
        \word_{i,1}, \word_{i,2}, \ldots, \word_{i,{\sentSize_i}}\right],
\]
where $\sentSize_i\in\naturals$ is the length of sentence $\sent_i$ in words.
The words themselves are drawn from a finite vocabulary $\wordVocab$.

The binary \salience~of a sentence $\sent_i$ is $\bsal_i \in\{0,1\}$.
$\bsal_i=1$ indicates that sentence $\sent_i$ is \salient~and should be
included in the extract summary while $\bsal_i=0$ is assigned to
non-\salient~sentences that should be excluded from the summary.  We indicate
the vector of salience judgements for the $\docSize$ sentences in $\doc$ as
$\bsals = \left[\bsal_1,\ldots,\bsal_\docSize\right]\in  \labelSpace$.  The
objective of this sequence tagging problem is to learn a function $f :
\docSpace \rightarrow \labelSpace$ which maps a document $\doc$ to a sequence
of \salience~labels $\bsals$. In this work, we learn a probabilistic mapping
$\model(\bsals|\doc;\params)$ where $\model$ is a neural network with
parameters $\theta$ and $\model(\cdot|\doc;\params) : \labelSpace \rightarrow
(0,1)$. 

Prediction is achieved by finding $\predbsals = \operatorname{arg\;max}_{\bsals
\in \labelSpace} \model(\bsals|\doc;\params)$, either by approximation or when
the structure of $\model$ allows, exactly.  Additionally, a typical constraint
on summarization is that the extract summary not exceed a word budget
$\wordbudget \in \naturals$, that is, $\sum_{i=1}^\docSize \hat{\bsal}_i \cdot
\sentSize_i \le \wordbudget$.  Since it is not trivial to incorporate this
constraint into the sequence labeling formulation, we instead rely on a greedy
heuristic to enforce the budget constraint in practice. More details on test
time inference can be found in \autoref{sec:inference}.
