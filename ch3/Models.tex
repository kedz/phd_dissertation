\section{Models}

We implement our salience estimation model $\nnmodel(\nnsals|\nndoc;\params)$
hierarchically following the analogous structure of the documents we are
modeling.  Every model $\nnmodel$ proposed in this chapter consists of three
modules or layers: \textit{(i)} the word embedding layer, \textit{(ii)} the
sentence encoder layer, and \textit{(iii)} the sentence extractor layer.  The
word embedding layer maps the words in a sentence to a sequence of word
embeddings. The sentence encoder layer similarly maps sequences of word
embeddings to a sentence embedding. Finally, the sentence extractor maps
sequences of sentence embeddings to sequence of salience labels.

Choosing an architecture for each of the modules defines the model. We define
several architectures for the sentence encoder and extractor layers and show
how particular settings of each correspond to prior summarization models
proposed by \citet{cheng2016neural} and \citet{nallapati2017summarunner}.
Additionally, we propose two novel sentence extractor layers, and in
experiments consider all combinations of sentence encoder/extractor pairings.
In the next subsections, we describe each layer in more detail, and conclude
this section showing how certain configurations of each layer maps to
previously proposed or novel salience estimation models and how the models
generate an extract summary at test time (which we refer to as inference).

\subsection{Word Embedding Layer}

The word embedding layer, $\embLayer(\cdot\,;\embParams) : \wordVocab^*
\rightarrow \reals^{* \times \embDim},$ maps a sequence of words 
\[
    \sent_i = \left[\word_{i,1},\ldots, \word_{i,\sentSize_i}\right]
\] 
to a sequence of word embeddings 
\[
\nnwordEmbs_i = \left[ \wordEmb_{i,1}, \ldots, \wordEmb_{i,\sentSize_i} \right]
    \in \reals^{\sentSize_i \times \embDim}
\]
where the sole parameter $\embParams \in \mathbb{R}^{|\wordVocab| \times
\embDim}$ is a $\embDim$-dimensional embedding matrix and $\wordEmb_{i,j} =
\embParams_{\word_{i,j}}$ is the word embedding for word $\word_{i,j}$.
$\embParams$ is initialized prior to training the full model with embeddings
obtained using the unsupervised Global Vector (GloVe) embedding method on a
large collection of text \citep{pennington2014glove}.  Additionally,
$\embParams$ can be held fixed during training or updated with other model
parameters. We use $\embDim = 200$-dimensional embeddings in our models.

\begin{figure}
\begin{subfigure}{\textwidth}
  \centering
  %\includegraphics[width=.8\linewidth]{images/ch2/avgsentencoder.pdf}
  \includegraphics{images/ch2/avgsentencoder.pdf}
  \caption{Averaging Sentence Encoder}
  \label{fig:sfig1}
\end{subfigure}

\begin{subfigure}{\textwidth}
  \centering
  %\includegraphics[width=.8\linewidth]{images/ch2/cnnsentencoder.pdf}
  \includegraphics{images/ch2/rnnsentencoder.pdf}
  \caption{\RecurrentNeuralNetwork~Sentence Encoder}
  \label{fig:sfig2}
\end{subfigure}

\begin{subfigure}{\textwidth}
  \centering
  %\includegraphics[width=.8\linewidth]{images/ch2/cnnsentencoder.pdf}
  \includegraphics{images/ch2/cnnsentencoder.pdf}
  \caption{\convolutionalneuralnetwork~Sentence Encoder}
  \label{fig:sfig2}
\end{subfigure}
    \caption{Schematics for the averageing, \recurrentneuralnetwork,
    and \convolutionalneuralnetwork~sentence encoders.}
\label{fig:sentenceEncoders}
\end{figure}




\subsection{Sentence Encoder Layer} \label{sec:senc}

The sentence encoder layer, \sentEncFuncDef, maps a sequence of word embeddings
\[
\nnwordEmbs_i = \left[\wordEmb_{i,1},\ldots, \wordEmb_{i,\sentSize_i}\right]
\] 
to a $\sentDim$-dimensional embedding representation of sentence $\sent_i$.
The set of associated parameters, $\sentEncParams$, depends on the exact
architecture for implementing the encoder. We experiment with three
architectures for mapping sequences of word embeddings to a fixed length
vector: averaging, recurrent neural networks, and convolutional neural
networks.

We describe each variant now, and also briefly discuss the trade-offs
associated with each architecture. The main distinction amongst the encoders is
to how they can exploit the context and structure of the words they are
encoding. With the averaging encoder, the resulting sentence embedding captures
all words equally, but is insensitive to phrase structure phenomenon like
negation. The convolutional encoder can capture local phrase structure but may
not be able to capture long range phrase structure. The recurrent encoder can
capture long range phrase structure and context but is computationally more
expensive than the recurrent encoder.  Schematics of each encoder architecture
can be found in \autoref{fig:sentenceEncoders}.

\subsubsection{Averaging Sentence Encoder} 

Under the averaging encoder, a sentence embedding $\sentEmb_i \in
\reals^{\sentDim}$ is simply the average of its word embeddings,
\begin{align} 
    \sentEmb_i & = \sentEnc(\nnwordEmbs_i;\sentEncParams) =  
        \frac{1}{\sentSize_i} \sum_{j=1}^{\sentSize_i} \wordEmb_{i,j}.
\end{align}
The sentence representation here is effectively a bag of word embeddings.
There are no parameters associated with this encoder (i.e.  $\sentEncParams =
\emptyset$). The size of the sentence embedding is simply $\sentDim = \embDim =
200$. Dropout with drop probability 0.25 is applied to each word embedding
$\wordEmb_{i,j}$ during training. 

\subsubsection{\RecurrentNeuralNetwork~Sentence Encoder} 

When using the \recurrentneuralnetwork~encoder we apply both forward and
backward \recurrentneuralnetwork s over the word embedding sequences produced
by the embedding layer. To obtain the actual sentence embedding, we concatenate
the final output step of the forward and backward networks.  For the actual
recurrence function, we use the \gatedrecurrentunit~(GRU)
\citep{cho2014learning}.  The GRU function, $\gruFuncDef{\embDim}{\hidDim}$, is
defined as
\begin{align}
\fgru(\wordEmb, \sentEmb; \varphi) & 
    = (1-\GRUupdategate) \odot \GRUcandgate + \GRUupdategate \odot \sentEmb \label{eqn:gru}\\
    \textit{(Reset gate)} & \nonumber \\
\GRUresetgate &= 
    \sigma\left(
      \GRUWeight^{(\wordEmbChar\GRUresetchar)} \wordEmb 
        + \GRUBias^{(\wordEmbChar\GRUresetchar)} +
      \GRUWeight^{(\sentEmbChar\GRUresetchar)} \sentEmb 
        + \GRUBias^{(\sentEmbChar\GRUresetchar)} 
    \right)\\
    \textit{(Update gate)} & \nonumber \\
\GRUupdategate &= 
    \sigma\left(
      \GRUWeight^{(\wordEmbChar\GRUupdatechar)} \wordEmb 
        + \GRUBias^{(\wordEmbChar\GRUupdatechar)} +
      \GRUWeight^{(\sentEmbChar\GRUupdatechar)} \sentEmb 
        + \GRUBias^{(\sentEmbChar\GRUupdatechar)} 
    \right)\\
    \textit{(Candidate state)} & \nonumber \\
\GRUcandgate &= \tanh\left(
    \GRUWeight^{(\wordEmbChar\GRUcandchar)} \wordEmb 
        + \GRUBias^{(\wordEmbChar\GRUcandchar)} + 
    \GRUresetgate \odot \left(
        \GRUWeight^{(\sentEmbChar\GRUcandchar)} \sentEmb 
            + \GRUBias^{(\sentEmbChar\GRUcandchar)} 
    \right) \right)
\end{align}
where $\GRUparams = \left\{   \GRUWeight^{(ab)}, \GRUBias^{(ab)} \Big\vert a
\in \{v,h\}, b \in \{r,u,o\} \right\}$ is the set of \gru~parameters with
$\GRUWeight^{(\wordEmbChar\cdot)} \in \reals^{\hidDim \times \embDim}$,
$\GRUWeight^{(\sentEmbChar\cdot)} \in \reals^{\hidDim \times \hidDim}$, and
$\GRUBias^{(\cdot)} \in \reals^{\hidDim }$, and $\sigma(x) =
\frac{1}{1+e^{-x}}$, $\tanh(x) = \frac{e^x-1}{e^x+1}$, and $\odot$ is the
Hadamard product.
 

Under the \recurrentneuralnetwork~encoder, a sentence embedding $\sentEmb_i$ is
then defined as
\begin{align} 
  \sentEmb_i = \sentEnc\Big(\nnwordEmbs_i; \sentEncParams\Big) & = \left[
                \begin{array}{l}\rSentEmb_{i,\sentSize_i}\\ \lSentEmb_{i,1}
    \end{array}
  \right] \\
  \textit{(Forward GRU)} \nonumber & \\
  \rSentEmb_{i,0} &= \zeroEmb, \\ 
  \rSentEmb_{i,j} &= 
      \fgru(\wordEmb_{i,j}, \rSentEmb_{i,j-1}; \rSentGRUParams) 
      & \forall j \in \{1,\ldots,\sentSize_i\} \\  
  \textit{(Backward GRU)} \nonumber & \\
  \lSentEmb_{i,\sentSize_i + 1} &= \zeroEmb, \\
  \lSentEmb_{i,j} &= 
      \fgru(\wordEmb_{i,j},\lSentEmb_{i,j+1}; \lSentGRUParams)
      & \forall j \in \{\sentSize_i, \ldots, 1\} 
\end{align}
where $[\cdots]$ is the vector concatenation operator and $\rSentGRUParams$ and
$\lSentGRUParams$ are distinct parameters for the forward and backward GRUs
respectively.  Collectively the set of parameters for the recurrent neural
network sentence encoder is $\sentEncParams = \left\{ \rSentGRUParams,
\lSentGRUParams \right\}$. We use $\hidDim=300$ dimensional hidden layers for
each \gru, making the size of the sentence embedding $\sentDim=2\hidDim=600$.
Dropout with drop probability $0.25$ is applied to \gru~outputs
$\rSentEmb_{i,j}$ and $\lSentEmb_{i,j}$ for $j \in \{1,\ldots,\sentSize_i\}$
during training.

\subsubsection{\ConvolutionalNeuralNetwork~Sentence Encoder} 
\label{sec:sentconvenc}

The \convolutionalneuralnetwork~sentence encoder uses a series of convolutional
feature maps to encode each sentence. This encoder is similar to the
convolutional architecture of \citet{kim2014convolutional} used for text
classification tasks. It performs a series of ``one-dimensional'' convolutions
over word embeddings. The \kernelwidth~$\ckernelWidth \in \naturals$ of a
feature map determines the number of contiguous words that a feature map is
sensitive to. For $\ckernelWidth=3$, for example, the feature map would
function as a trigram feature detector essentially.  We denote a single
convolutional feature map of kernel width $\ckernelWidth$ as
$\ckernel_\ckernelWidth : \reals^{*\times \embDim} \rightarrow \reals$ with
\begin{align}
    \ckernel_\ckernelWidth(\wordEmb_i;\upsilon,\beta)  & 
= \max_{j \in \{ 
    1 - \left\lfloor \frac{\ckernelWidth}{2} \right\rfloor, 
    \ldots, \sentSize_i +  
    \left\lfloor \frac{\ckernelWidth}{2} \right\rfloor - \ckernelWidth + 1 \}}
  \relu\left( \cBias + \cMatrix \cdot \left[ \begin{array}{c} \wordEmb_{i,j}\\ \wordEmb_{i,j+1} \\ \vdots \\ \wordEmb_{i,j+k-1} \end{array} \right] \right),
\end{align}
where $\relu(x) = \max(0, x)$ is the rectified linear unit \citep{nair2010},
$\left\lfloor\cdot\right\rfloor$ is the floor operator, and $\cMatrix \in
\reals^{\ckernelWidth \embDim}$ and $\cBias \in \reals$ are learned parameters.
Note that we use a ``zero-padded'' convolution \citep{dumoulin2016}. That is,
the $\max$ operator ranges over  $j \in \left\{1 - \left\lfloor
\frac{\ckernelWidth}{2} \right\rfloor, \ldots, \sentSize_i +  \left\lfloor
\frac{\ckernelWidth}{2} \right\rfloor - \ckernelWidth + 1 \right\}$ instead of
$\left\{1,\ldots,\sentSize_i - \ckernelWidth + 1\right\}$, and $\wordEmb_{i,j}
= \zeroEmb$ for $j < 1$ and $j > \sentSize_i$. Padded convolutions help
alleviate the problem of reduced receptive fields on the boundaries of the
sequence. See \autoref{fig:paddedconv} for a visual example.

\begin{figure}
\centering 
\fbox{\resizebox{\textwidth}{!}{
\begin{tikzpicture}[x=0.25cm, y=0.25cm,
    embs/.style={step=0.25cm,red},
    bigram/.style={step=0.25cm,blue,line width=0.5mm},
  convcon/.style={black,line width=0.5mm},
  hid/.style 2 args={
    rectangle split,
    draw=#2,
    rectangle split parts=#1,
    fill=#2!20,
    outer sep=1mm}]

\def\SEP{10}
\def\YSEP{15}

\foreach \a [count=\step from 1,
             evaluate=\step as \smo using int(\step - 1),
             evaluate=\step as \fnwx using \SEP*\step+3,
             evaluate=\step as \fnwy using 4-\step+2,
             evaluate=\step as \fsex using \SEP*\step,
             evaluate=\step as \fsey using 4-\step,
             evaluate=\step as \anwx using \SEP*\step+7,
             evaluate=\step as \anwy using 5-\step+1-0.5,
             evaluate=\step as \asex using \SEP*\step+6,
             evaluate=\step as \asey using 5-\step-0.5,
] in {10,25,50,35,30} {

    \foreach \i [count=\w from 1] in {1,...,4} {
        \node[font=\tiny] at (\SEP*\step-1,5-\i-0.5) {$w_\w$};
    }
    \fill[red!20!] (\SEP*\step,0) rectangle (\SEP*\step+3,4);
    \draw[embs] (\SEP*\step,0) grid (\SEP*\step+3,4);
    
    \draw[bigram] (\fsex,\fsey) grid (\fnwx,\fnwy);
    
    \draw (\fsex,\fnwy) -- (\asex,\anwy);
    \draw (\fsex,\fsey) -- (\asex,\asey);
    \draw (\fnwx,\fnwy) -- (\anwx,\anwy);
    \draw (\fnwx,\fsey) -- (\anwx,\asey);

    \fill[blue!\a!] (\asex,\asey) rectangle (\anwx,\anwy);
    \draw[step=0.25cm,black,yshift=-0.125cm] 
        (\SEP*\step+6,0) grid (\SEP*\step+7,5); 

    \node at (\SEP*\step+3.5,-3.0) {$j=\smo$};
}

\foreach \a [count=\step from 1,
             evaluate=\step as \smo using int(\step - 1),
             evaluate=\step as \anwx using \SEP*6.5+1,
             evaluate=\step as \anwy using 5-\step+1-0.5,
             evaluate=\step as \asex using \SEP*6.5,
             evaluate=\step as \asey using 5-\step-0.5,
] in {10,25,50,35,30} {
    \fill[blue!\a!] (\asex,\asey) rectangle (\anwx,\anwy);
}
\draw[step=0.25cm,black,yshift=-0.125cm] (\SEP*6.5,0) grid (\SEP*6.5+1,5); 
\node[align=center] at (\SEP*6.5+0.5,8) {Convolutional\\Feature Values};
\draw[->] (\SEP*6-1.25,2.25) -- (\SEP*6.25,2.25);
\node[align=left,anchor=north west] at (\SEP*1,9) 
    {Bigram Feature Detector ($k=2$)};


\node[font=\tiny] (zp1) at (\SEP*.5,4.5) {zero padding};
\draw[->] (zp1.east) -- ($(zp1.east)+(1.5,0)$);
\node[font=\tiny] (zp1) at (\SEP*1.5,4.5-\YSEP) {zero padding};
\draw[->] (zp1.east) -- ($(zp1.east)+(1.5,0)$);
\node[font=\tiny] (zp1) at (\SEP*4.5,-1.0-\YSEP) {zero padding};
\draw[->] (zp1.east) -- ($(zp1.east)+(1.5,0.5)$);
\node[font=\tiny] (zp1) at (\SEP*4.5,-1.0) {zero padding};
\draw[->] (zp1.east) -- ($(zp1.east)+(1.5,0.5)$);

\foreach \a [count=\step from 2,
             evaluate=\step as \smo using int(\step - 2),
             evaluate=\step as \fnwx using \SEP*\step+3,
             evaluate=\step as \fnwy using 4-\step+3-\YSEP,
             evaluate=\step as \fsex using \SEP*\step,
             evaluate=\step as \fsey using 4-\step-\YSEP,
             evaluate=\step as \anwx using \SEP*\step+7,
             evaluate=\step as \anwy using 5-\step+1-\YSEP,
             evaluate=\step as \asex using \SEP*\step+6,
             evaluate=\step as \asey using 5-\step-\YSEP,
] in {25,27,45,30} {


    \foreach \i [count=\w from 1] in {1,...,4} {
        \node[font=\tiny] at (\SEP*\step-1,5-\i-0.5-\YSEP) {$w_\w$};
    }
    \fill[red!20!] (\SEP*\step,0-\YSEP) rectangle (\SEP*\step+3,4-\YSEP);
    \draw[embs] (\SEP*\step,0-\YSEP) grid (\SEP*\step+3,4-\YSEP);
    
    \draw[bigram] (\fsex,\fsey) grid (\fnwx,\fnwy);
    
    \draw (\fsex,\fnwy) -- (\asex,\anwy);
    \draw (\fsex,\fsey) -- (\asex,\asey);
    \draw (\fnwx,\fnwy) -- (\anwx,\anwy);
    \draw (\fnwx,\fsey) -- (\anwx,\asey);

    \fill[blue!\a!] (\asex,\asey) rectangle (\anwx,\anwy);
    \draw[step=0.25cm,black] 
        (\SEP*\step+6,0-\YSEP) grid (\SEP*\step+7,4-\YSEP); 

    \node at (\SEP*\step+3.5,-3.0-\YSEP) {$j=\smo$};
}

\foreach \a [count=\step from 2,
             evaluate=\step as \smo using int(\step - 1),
             evaluate=\step as \anwx using \SEP*6.5+1,
             evaluate=\step as \anwy using 5-\step+1-\YSEP,
             evaluate=\step as \asex using \SEP*6.5,
             evaluate=\step as \asey using 5-\step-\YSEP,
] in {25,27,45,30} {
    \fill[blue!\a!] (\asex,\asey) rectangle (\anwx,\anwy);
}
\draw[step=0.25cm,black] (\SEP*6.5,0-\YSEP) grid (\SEP*6.5+1,4-\YSEP); 
\node[align=center] at (\SEP*6.5+0.5,8-\YSEP) {Convolutional\\Feature Values};
\draw[->] (\SEP*6-1.25,2.25-\YSEP) -- (\SEP*6.25,2.25-\YSEP);
\node[align=left,anchor=north west] at (\SEP*1,9-\YSEP) 
    {Trigram Feature Detector ($k=3$)};

\end{tikzpicture}}}

\caption{Examples of zero padding with bigram ($k=2$) and trigram ($k=3$) features for a sequence of length $l_i=4$.}
\label{fig:paddedconv}

\end{figure}


The final sentence embedding $\sentEmb_i$ is a concatenation of many
convolutional feature maps ranging over multiple kernel widths with each filter
having its own distinct sets of parameters.  Let $\ckernelWidths =
\{\ckernelWidth_1, \ldots, \ckernelWidth_m \} \subset \naturals$ be the set of
the sentence encoder's $m$ kernel widths, and $\cFeatureMaps_\ckernelWidth \in
\naturals$ be the number of feature maps for kernel width $\ckernelWidth$. The
final sentence embedding produced by the \convolutionalneuralnetwork~sentence
encoder is defined as 
\begin{align}
\sentEmb_i & = \sentEnc(\wordEmb_i; \sentEncParams) = \left[  
    \ckernel^{\left(1\right)}_{\ckernelWidth_1},
    \ldots, 
    \ckernel^{\left(\cFeatureMaps_{\ckernelWidth_1}\right)}_{\ckernelWidth_1}, 
    \ckernel^{\left(1\right)}_{\ckernelWidth_2},
    \ldots, 
    \ckernel^{\left(\cFeatureMaps_{\ckernelWidth_2}\right)}_{\ckernelWidth_2}, 
    \ldots, 
    \ckernel^{\left(1\right)}_{\ckernelWidth_m},
    \ldots, 
    \ckernel^{\left(\cFeatureMaps_{\ckernelWidth_m}\right)}_{\ckernelWidth_m}
  \right]  
\end{align}
where $\ckernel^{(j)}_{\ckernelWidth} =
\ckernel^{(j)}_{\ckernelWidth}(\nnwordEmbs_i;\cMatrix^{(j,\ckernelWidth)},\cBias^{(j,\ckernelWidth)})$
and  $\sentEncParams = \left\{ \cMatrix^{\left(k,l\right)},
\cBias^{\left(k,l\right)}\Big\vert \forall k,l : k \in \ckernelWidths, l \in
\{1,\ldots,\cFeatureMaps_k\}\right\}$ are the sentence encoder's learned
parameters.  In our instantiation, we use kernel widths $\ckernelWidths =
\{1,\ldots, 6\}$ with corresponding feature maps sizes $\cFeatureMaps_1=25$,
$\cFeatureMaps_2=25$, $\cFeatureMaps_3=50$, $\cFeatureMaps_4=50$,
$\cFeatureMaps_5=50$, and $\cFeatureMaps_6=50$, making the resulting sentence
embedding dimensionality $\sentDim=250$.  Dropout with drop probability $0.25$
is also applied to $\sentEmb_i$ during training.

\subsubsection{Sentence Encoder Trade-offs}

The sentence encoder's role is to obtain a vector representation of a finite
sequence of word embeddings that is useful for the sentence extraction stage.
Therefore, it must aggregate features in the word embedding space that are
predictive of salience. Averaging embeddings is not an unreasonable approach to
this. Empirically there is evidence that word embedding averaging is a fairly
competitive sentence representation generally
\citep{iyyer2015,wieting2015,arora2017,wieting2017}. In the context of
summarization, averaging can be thought of as a noisy OR; if any of the words
in a sentence are indicative of salience, this representation should capture
them.  Computationally, the averaging encoder is the fastest to compute and
does not require learning of parameters, reducing the memory and computation
time during training. 

The \recurrentneuralnetwork~sentence encoder can in theory capture some
compositional features of a word sequence that would be difficult or impossible
to represent in the averaging encoder (e.g. negation or co-reference).
However, this comes at a much heavier computational cost, as
\recurrentneuralnetwork s cannot be fully parallelized due to the inherently
sequential nature of their computation.

The \convolutionalneuralnetwork~encoder represents a middle ground between the
averaging and \recurrentneuralnetwork~encoders. When using modestly sized
kernel widths (e.g., 1-5), the receptive window should be sensitive short
phrases and some locally scoped negation. It will not be able to capture the
longer ranged dependencies that the \recurrentneuralnetwork~encoder would.
However, it is much faster to compute than the \recurrentneuralnetwork~as the
individual feature maps can be computed completely in parallel. 

\subsection{Sentence Extraction Layer} \label{sec:sext}

The role of the sentence extractor, $\sentExt(\cdot; \xParams) : \reals^{*
\times \sentDim} \rightarrow \labelSpace$, is to map a sequence of sentence
embeddings $\sentEmb_1,\ldots,\sentEmb_\docSize$ produced by the sentence
encoder layer to a sequence of salience judgements 
\[
    \nnsals = \left[ \bsal_1,\ldots, \bsal_\docSize\right].
\] %y_1, ..., y_n.%
The proposed sentence extractors do this by first implementing a probability
distribution over salience label sequences conditioned on
$\sentEmb_1,\ldots,\sentEmb_\docSize$,
$\nnmodel(\nnsals|\sentEmb_1,\ldots,\sentEmb_\docSize; \xParams)$, and then
inferring the (approximate) maximum likelihood sequence, i.e., 
\[ 
    \sentExt(\sentEmb_1,\ldots,\sentEmb_\docSize; \xParams) = \nnpredsals \approx \argmax_{\nnsals \in \labelSpace} \nnmodel(\nnsals|\sentEmb_1,\ldots,\sentEmb_\docSize;\xParams).
\]
Previous neural network approaches to sentence extraction have assumed an
\autoregressive~model, leading to the following factorization of the salience
label distribution
\[
    \nnmodel(\nnsal_{1},\ldots,\nnsal_\docSize|\sentEmb_1,\ldots,\sentEmb_\docSize;\xParams)=
      \prod_{i=1}^\docSize 
        \model(\nnsal_i|\nnsal_1,\ldots,\nnsal_{i-1},\sentEmb_1,\ldots,\sentEmb_\docSize;\xParams),
\]
where each prediction $\nnsal_i$ is dependent on \emph{all} previous $\bsal_j$
for all $j < i$. We compare two such models proposed by
\citet{cheng2016neural} and \citet{nallapati2017summarunner}. 

While intuitively it makes sense that previous extraction decisions might
affect the probability of extracting subsequent sentences, (e.g., highly
salient sentences might cluster together), it has not been empirically
investigated whether this dependence is necessary for deep learning models in
practice. For example, in the models of \citet{cheng2016neural} and
\citet{nallapati2017summarunner}, individual predictions of $\nnsal_i$ are made
using information from some or all of the sentence embeddings $\sentEmb_1,
\ldots, \sentEmb_\docSize$, such that information about neighboring sentences
could be propagated through sentence embedding interactions rather than on
previous salience decisions.  Additionally, from an efficiency perspective, the
autoregressive design prevents parallelization of individual $\nnsal_i$
predictions, since they must now be sequentially computed.  Motivated by these
considerations, we propose two \nonautoregressive~sentence extractor
architectures where individual salience labels $\nnsal_i$ are independent of
each other, that is, 
\[
    \nnmodel(\nnsal_1,\ldots,\nnsal_\docSize|\sentEmb_1,\ldots,\sentEmb_\docSize;\xParams) = \prod_{i=1}^\docSize 
\nnmodel(\nnsal_i|\sentEmb_1,\ldots,\sentEmb_\docSize;\xParams).
\]

While the sentence extractor architectures are quite different in their
details, under our unified treatment of them here, we view them as producing in
their intermediate computations a sequence of contextual sentence embeddings
$\xHid_1,\ldots,\xHid_\docSize$.  Unlike the ``context free'' sentence
embeddings $\sentEmb_i$ which are constructed only using words from sentence
$\sent_i$,  each $\xHid_i$ is computed using information information propagated
from neighboring sentence embeddings $\sentEmb_1, \ldots, \sentEmb_{i-1}$ and
$\sentEmb_{i+1}, \ldots, \sentEmb_\docSize$ and in the case of the
autoregressive models, salience estimates $\nnpsal_1,\ldots,\nnpsal_{i-1}$
where 
\[
    \nnpsal_i = \nnmodel(\nnsal_i=1|\nnsal_1,\ldots,\nnsal_{i-1},\sentEmb_1, \ldots, \sentEmb_\docSize;\xParams).
\]
Additionally, the SummaRunner extractor produces an embedding representation of
the document, $\srDocEmb$, as well as iterative representations of the summary
$\srSum_1,\ldots, \srSum_{\docSize-1}$ which also affect the creation of the
contextual sentence embeddings.

We now describe in detail how  the \autoregressive~sentence extractors (the
\clext~extractor and  the \srext~extractor) and our proposed
\nonautoregressive~ones (the \rnnext~extractor and the \stsext~extractor),
produce the these various representations and make sentence salience estimates. 

\begin{figure}
\center
\fbox{\resizebox{\textwidth}{!}{
\begin{tikzpicture}[
  dep/.style ={
    ->,line width=0.3mm
  },
  hid/.style 2 args={
    rectangle split,
    draw=#2,
    rectangle split parts=#1,
    fill=#2!20,
    minimum width=5mm,
    minimum height=5mm,
    outer sep=2mm},
  mlp/.style 2 args={
    rectangle split,
    rectangle split horizontal,
    draw=#2,
    rectangle split parts=#1,
    fill=#2!20,
    outer sep=2mm},
  sal/.style={
    circle, 
    minimum width=8mm,
    outer sep=2mm,
    draw=#1, 
    fill=#1!20},
]

  \def\stepsize{2}%
  \def\lvlbase{0}%
  \def\lvlheight{3}%

    % Sentence Embeddings    
  \foreach \step in {1,...,3} {
    \node[hid={3}{sentemb}] (s\step) at (\stepsize*\step, \lvlbase) {};    
    \node at (\stepsize*\step, \lvlbase) {$\sentEmb_\step$};    
   }
    \foreach \step [count=\i from 1] in {5,6} {
        \node[hid={3}{salsentemb}] (s\step) at (\stepsize*\step, \lvlbase) {};    
        \node at (\stepsize*\step, \lvlbase) {$\salSentEmb_\i$};    
    %\draw[->] (i\step.north) -> (e\step.south);
    }

       \node[hid={3}{red}] (s4) at (\stepsize*4, \lvlbase) {};    
       \node at (\stepsize*4, \lvlbase) {$\salSentEmb_0$};    

    % RNN hidden states
    \foreach \step in {1,...,3} {
        \node[hid={3}{encemb}] (rnn_\step) 
            at (\stepsize *\step, \lvlbase + \lvlheight) {};    
        \node at (\stepsize *\step, \lvlbase + \lvlheight) {$\xEncHid_\step$}; 
    }
    \foreach \step [count=\i from 1] in {4,...,6} {
        \node[hid={3}{decemb}] (rnn_\step) 
            at (\stepsize *\step, \lvlbase + \lvlheight) {};    
        \node at (\stepsize *\step, \lvlbase + \lvlheight) {$\xDecHid_\i$}; 
    }

    \foreach \step in {1,...,6} {
        \draw[dep] (s\step.north) to (rnn_\step.south);
    }
    \foreach \start [count=\stop from 2] in {1,...,5} {
        \draw[dep] (rnn_\start.east) to (rnn_\stop.west);
    }


    \foreach \step [count=\i from 1] in {4,...,6} {
        \node[hid={3}{ctxemb}] (ctx_\i) 
            at (\stepsize *\step, \lvlbase + 2*\lvlheight) {};    
        \node at (\stepsize *\step, \lvlbase + 2*\lvlheight) {$\xPredHid_\i$}; 
        \draw[dep] (rnn_\step.north) to (ctx_\i.south);
        \draw[dep] (rnn_\i.north) to (ctx_\i.south west);
        \node[sal={sal}] (sal_\i) at (\stepsize * \step,\lvlbase + 3*\lvlheight) {};
        \node at (\stepsize * \step,\lvlbase + 3*\lvlheight) {$\psal_\i$};
        \draw[dep] (ctx_\i.north) to (sal_\i.south);
    }

    \foreach \step [count=\i from 1] in {5,...,6} {
        \draw[dep] (sal_\i) -- (\stepsize * \step - \stepsize / 2,
                            \lvlbase + 3*\lvlheight) 
                     -- (\stepsize * \step - \stepsize / 2,
                            \lvlbase + 0* \lvlheight) -> (s\step) ;

    }
    \draw[dep] (s1.south) -- ($ (s1.south) + (0,-0.3)$) --($ (s5.south) + (0,-0.3)$) -- (s5.south);

    \draw[dep] (s2.south) -- ($ (s2.south) + (0,-0.5)$) --($ (s6.south) + (0,-0.5)$) -- (s6.south);

    \draw[rectangle,draw=black,dotted] 
        (\stepsize * 3.5,\lvlbase + 3.5*\lvlheight) -- 
        (\stepsize*6.5, \lvlbase + 3.5*\lvlheight) -- 
        (\stepsize*6.5, \lvlbase + 2.6*\lvlheight) --
        (\stepsize*3.5, \lvlbase + 2.6*\lvlheight) --
        (\stepsize*3.5, \lvlbase + 3.5*\lvlheight) ;

    \node[align=left,anchor=north west] 
        at (\stepsize * 3.5,\lvlbase + 3.5*\lvlheight) 
        {\textit{(e) Salience Estimates}};

    \draw[rectangle,draw=black,dotted] 
        (\stepsize*0.5,\lvlbase + 2.5*\lvlheight) -- 
        (\stepsize*6.5, \lvlbase + 2.5*\lvlheight) -- 
        (\stepsize*6.5, \lvlbase + 1.5*\lvlheight) --
        (\stepsize*0.5, \lvlbase + 1.5*\lvlheight) --
        (\stepsize*0.5, \lvlbase + 2.5*\lvlheight) ;

    \node[align=left,anchor=north west] 
        at (\stepsize * 0.5,\lvlbase + 2.5*\lvlheight) 
        {\textit{(d) Contextual Sentence Embeddings}};

    \draw[rectangle,draw=black,dotted] 
        (\stepsize*-1.5,\lvlbase + 0.5*\lvlheight) -- 
        (\stepsize*3.4, \lvlbase + 0.5*\lvlheight) -- 
        (\stepsize*3.4, \lvlbase + -0.9*\lvlheight) --
        (\stepsize*-1.5, \lvlbase + -0.9*\lvlheight) --
        (\stepsize*-1.5, \lvlbase + 0.5*\lvlheight) ;

    \node[align=left,anchor=south west] 
        at (\stepsize * -1.5,\lvlbase + -0.9*\lvlheight) 
        {\textit{Sentence Embeddings (Sentence Encoder Output)}};

    \draw[rectangle,draw=black,dotted] 
        (\stepsize*-1.5,\lvlbase + 1.4*\lvlheight) -- 
        (\stepsize*3.4, \lvlbase + 1.4*\lvlheight) -- 
        (\stepsize*3.4, \lvlbase + 0.6*\lvlheight) --
        (\stepsize*-1.5, \lvlbase + 0.6*\lvlheight) --
        (\stepsize*-1.5, \lvlbase + 1.4*\lvlheight) ;

    \node[align=left,anchor=north west] 
        at (\stepsize * -1.5,\lvlbase + 1.4*\lvlheight) 
        {\textit{(a) Extractor -- Encoder}};

    \draw[rectangle,draw=black,dotted] 
        (\stepsize*3.6,\lvlbase + 1.4*\lvlheight) -- 
        (\stepsize*8.5, \lvlbase + 1.4*\lvlheight) -- 
        (\stepsize*8.5, \lvlbase + 0.6*\lvlheight) --
        (\stepsize*3.6, \lvlbase + 0.6*\lvlheight) --
        (\stepsize*3.6, \lvlbase + 1.4*\lvlheight) ;

    \node[align=left,anchor=north east] 
        at (\stepsize * 8.5,\lvlbase + 1.4*\lvlheight) 
        {\textit{(c) Extractor -- Decoder}};

    \draw[rectangle,draw=black,dotted] 
        (\stepsize*4.6,\lvlbase + 0.5*\lvlheight) -- 
        (\stepsize*8.5, \lvlbase + 0.5*\lvlheight) -- 
        (\stepsize*8.5, \lvlbase + -0.9*\lvlheight) --
        (\stepsize*4.6, \lvlbase + -0.9*\lvlheight) --
        (\stepsize*4.6, \lvlbase + 0.5*\lvlheight) ;

    \node[align=left,anchor=south west] 
        at (\stepsize * 4.6,\lvlbase + -0.9*\lvlheight) 
        {\textit{(b) Salience Gated Sentence Embeddings}};

\end{tikzpicture}
}}

\caption{Schematic for the \clext~sentence extractor.}
\label{fig:clext} 
\end{figure}


\subsubsection{\clext~Extractor}

The \clext~extractor \citep{cheng2016neural} is built around a somewhat
idiosyncratic \unidirectional~\sequencetosequence~model. A schematic outlining
the structure of the encoder and closely following the subsequent equations can
be found in \autoref{fig:clext}.

The encoder is fairly standard. The initial state is initialized to a zero
embedding, $\zeroEmb$, and each sentence embedding $\sentEmb_i$ is fed into the
encoder, to obtain the final encoder hidden state $\xEncHid_\docSize \in
\reals^\xhidSize$. That is,\\

\noindent\fcolorbox{encemb}{encemb!20!}{\textit{(\hyperref[fig:clext]{Figure~\ref{fig:clext}.a}) Extractor -- Encoder}}\\[-40pt]
\begin{align}
    \xEncHid_0 & = \zeroEmb \\
    \xEncHid_i &= \fgru(\sentEmb_i, \xEncHid_{i-1};\clEncParams) & 
    \forall i : i \in \{1,\ldots,\docSize\}. 
\end{align}

The initial decoder hidden state $\xDecHid_0 \in \reals^{\xhidSize}$ is
initialized with the last encoder hidden state, $\xEncHid_\docSize$.  The
inputs to decoder step $i$, for $i >1$, are the salience gated
$(i-1)^\textrm{th}$ sentence embeddings,\\

\noindent\fcolorbox{salsentemb}{salsentemb!20!}{\textit{(\hyperref[fig:clext]{Figure~\ref{fig:clext}.b}) Salience Gated Sentence Embeddings}}\\[-40pt]
\begin{align}
    \salSentEmb_{i-1} & = \nnpsal_{i-1}\sentEmb_{i-1} &
    \forall i: i \in \{2,\dots,\docSize\},
\end{align}
where the salience gate is $\nnpsal_i =
\nnmodel(\nnsal_i|\nnsal_1,\ldots,\nnsal_{i-1},
\sentEmb_1,\ldots,\sentEmb_\docSize;\xParams)$, is the salience estimate
computed for sentence $\sent_{i-1}$.  For the first decoder step (i.e. $i=1$),
since there is no $\nnpsal_0$, $\salSentEmb_0$ is a special learned parameter.

The extractor decoder outputs are then computed as,\\

\noindent\fcolorbox{decemb}{decemb!20!}{\textit{(\hyperref[fig:clext]{Figure~\ref{fig:clext}.c}) Extractor -- Decoder}}\\[-40pt]
\begin{align}
    \xDecHid_0  &= \xEncHid_\docSize  \\
   \xDecHid_i &= \fgru(\salSentEmb_{i-1}, \xDecHid_{i-1};\clDecParams)
   &  
    \forall i : i \in \{1,\ldots,\docSize\}  \label{eq:cl1} 
\end{align}
Note in \autoref{eq:cl1} that the decoder side \gru~input is the sentence
embedding from the previous time step, $\sentEmb_{i-1}$, weighted by its
probability of extraction, $\nnpsal_{i-1}$, from the previous step, inducing
dependence of each output $\nnsal_i$ on all previous outputs
$\nnsal_1,\ldots,\nnsal_{i-1}$.

The contextual sentence embeddings $\xPredHid_i$ are then computed by
concatenating the encoder and decoder outputs $\xEncHid_i$ and $\xDecHid_i$ and
running them through a \feedforward~layer with $\relu$ activation,\\

\noindent\fcolorbox{ctxemb}{ctxemb!20!}{\textit{(\hyperref[fig:clext]{Figure~\ref{fig:clext}.d}) Contextual Sentence Embeddings}}\\[-30pt]
\begin{align}
\xPredHid_i &= \relu\left(\xpWeight^{(1)} \left[\begin{array}{c}\xEncHid_i \\ \xDecHid_i \end{array}\right] + \xpBias^{(1)} \right) & 
    \forall i : i \in \{1,\ldots,\docSize\}.
\end{align}

The actual salience estimate for sentence $\sent_i$ is then computed by feeding
$\xPredHid_i$ through another \feedforward~layer with logistic sigmoid
activation,\\

\noindent\fcolorbox{sal}{sal!20!}{\textit{(\hyperref[fig:clext]{Figure~\ref{fig:clext}.e}) Salience Estimates}}\\[-40pt]
\begin{align}
 p_i = \model(\bsal_i =1|\bsal_1,\ldots,\bsal_{i-1}, \sentEmb_1,\ldots,\sentEmb_\docSize; \xParams) &= \sigma\left(\xpWeight^{(2)}\xPredHid_i + \xpBias^{(2)}  \right) & 
    \forall i : i \in \{1,\ldots,\docSize\}. 
\end{align}

The contextual embedding and salience estimate layers have parameters are
$\xpWeight^{(1)} \in \reals^{\xpHidSize \times 2 \xhidSize}$, $\xpBias^{(1)}
\in \reals^{\xpHidSize}$, $\xpWeight^{(2)} \in \reals^{1 \times \xpHidSize}$,
and $\xpBias^{(2)}\in \reals$.  The entire set of learned parameters for the
\clext~extractor are
\[
    \chi = \left\{ 
    \clEncParams, \clDecParams,
    \salSentEmb_0,
    \xpWeight^{(1)}, \xpBias^{(1)}, \xpWeight^{(2)}, \xpBias^{(2)}\right\}.
\]
The hidden layer dimensionality of the \gru~and the contextual embedding layer
is $\xhidSize = 300$ and $\xpHidSize=100$, respectively.  Dropout with drop
probability $0.25$ is applied to the \gru~outputs ($\xEncHid_i$ and
$\xDecHid_i$),   and to $\xPredHid_i$.

\subsubsection{\srext~Extractor}

\citet{nallapati2017summarunner} proposed a sentence extractor, which we refer
to as the \srext~Extractor, that factorizes the salience estimates for each
sentence into contributions from five different sources, which we refer to as
\saliencefactors.  The \saliencefactors~take into account interactions between
contextual sentence embeddings and document embeddings or summary embeddings,
as well as sentence position embeddings. Salience estimates are made
sequentially, starting with the first sentence $\sent_1$ and preceding to the
last $\sent_\docSize$. When computing the salience estimate of sentence
$\sent_i$, the previous $i-1$ salience estimates are used to update the summary
representation.

In order to construct the contextual sentence embeddings, document embeddings,
and summary embeddings, the \srext~extractor first runs a
\bidirectional~\gru~over the sentence embeddings created by the sentence
encoder (visually depicted in \autoref{fig:sr1}),\\

\begin{figure}[h!]
    \fbox{\begin{minipage}{\textwidth}
\center
\scalebox{0.75}{
\begin{tikzpicture}[
  dep/.style ={
    ->,line width=0.3mm
  },
  hid/.style 2 args={
    rectangle split,
    draw=#2,
    rectangle split parts=#1,
    fill=#2!20,
    minimum width=5mm,
    minimum height=5mm,
    outer sep=2mm},
  mlp/.style 2 args={
    rectangle split,
    rectangle split horizontal,
    draw=#2,
    rectangle split parts=#1,
    fill=#2!20,
    outer sep=2mm},
  sal/.style={
    circle, 
    minimum width=8mm,
    outer sep=2mm,
    draw=#1, 
    fill=#1!20},
]

  \def\stepsize{1.5}%
  \def\lvlbase{0}%
  \def\lvlheight{3}%
 

    % Sentence Embeddings    
  \foreach \step in {1,...,3} {
    \node[hid={3}{sentemb}] (s\step) at (\stepsize*\step, \lvlbase) {};    
    \node at (\stepsize*\step, \lvlbase) {$\sentEmb_\step$};    
   }
    \draw[rectangle,draw=black,dotted] 
        (\stepsize*0.0,\lvlbase + 2*\lvlheight) -- 
        (\stepsize*3.9, \lvlbase + 2*\lvlheight) -- 
        (\stepsize*3.9, \lvlbase + 0.5*\lvlheight) --
        (\stepsize*0.0, \lvlbase + 0.5*\lvlheight) --
        (\stepsize*0.0, \lvlbase + 2*\lvlheight) ;

    \node[align=left,anchor=north west] 
        at (\stepsize * 0.0,\lvlbase + 2*\lvlheight) 
        {\textit{(a) Forward and Backward \gru}\\\textit{\phantom{(a) }Outputs}};

    \draw[rectangle,draw=black,dotted] 
        (\stepsize*0.0,\lvlbase + 0.4*\lvlheight) -- 
        (\stepsize*3.9, \lvlbase + 0.4*\lvlheight) -- 
        (\stepsize*3.9, \lvlbase -0.75*\lvlheight) --
        (\stepsize*0.0, \lvlbase -0.75*\lvlheight) --
        (\stepsize*0.0, \lvlbase + 0.4*\lvlheight) ;

    \node[align=left,anchor=south west] 
        at (\stepsize * 0.0,\lvlbase + -0.75*\lvlheight) 
        {\textit{Sentence Embeddings}\\\textit{(Sentence Encoder Output)}};


    \draw[rectangle,draw=black,dotted] 
        (\stepsize*4.0,\lvlbase + 2*\lvlheight) -- 
        (\stepsize*7.9, \lvlbase + 2*\lvlheight) -- 
        (\stepsize*7.9, \lvlbase + 0.5*\lvlheight) --
        (\stepsize*4.0, \lvlbase + 0.5*\lvlheight) --
        (\stepsize*4.0, \lvlbase + 2*\lvlheight) ;

    \node[align=left,anchor=north west] 
        at (\stepsize * 4.0,\lvlbase + 2*\lvlheight) 
        {\textit{(b) Contextual Sentence}\\ \textit{\phantom{(b) } Embeddings}};

    \draw[rectangle,draw=black,dotted] 
        (\stepsize*8.0,\lvlbase + 2*\lvlheight) -- 
        (\stepsize*11.9, \lvlbase + 2*\lvlheight) -- 
        (\stepsize*11.9, \lvlbase + 0.5*\lvlheight) --
        (\stepsize*8.0, \lvlbase + 0.5*\lvlheight) --
        (\stepsize*8.0, \lvlbase + 2*\lvlheight) ;

    \node[align=left,anchor=north west] 
        at (\stepsize * 8.0,\lvlbase + 2*\lvlheight) 
        {\textit{(c) Document Embedding}};




%    % RNN hidden states
    \foreach \step in {1,...,3} {
        \node[hid={3}{rencemb}] (rrnn_\step) 
            at (\stepsize *\step-0.3, \lvlbase + \lvlheight) {};    
        \node at (\stepsize *\step-0.3, \lvlbase + \lvlheight) 
            {$\rnnextRHid_\step$}; 
        \node[hid={3}{lencemb}] (lrnn_\step) 
            at (\stepsize *\step+0.3, \lvlbase + \lvlheight + 1.0) {};    
        \node at (\stepsize *\step+0.3, \lvlbase + \lvlheight+ 1.0) 
            {$\rnnextLHid_\step$}; 
        \draw[dep] (s\step.north) -- (rrnn_\step.south);
        \draw[dep] (s\step.north) -- (lrnn_\step.south);

    }
    \foreach \start [count=\stop from 2] in {1,...,2} {
        \draw[dep] ($ (rrnn_\start.east) - (0,0.3)$) 
            -- ($ (rrnn_\stop.west) - (0,0.3) $);
        \draw[dep] ($(lrnn_\stop.west) + (0,0.3)$) 
            -- ($ (lrnn_\start.east) + (0,0.3)   $);
    }

  \def\stepsize{1.5}%
  \def\lvlbase{0}%
  \def\lvlheight{3}%
 

    \foreach \step in {1,...,3} {
        \node[hid={3}{rencemb}] (rrnn_\step) 
            at (\stepsize *\step-0.35 + 5 + 1, \lvlbase + 0*\lvlheight) {};    
        \node at (\stepsize*\step-0.35 + 5 + 1, \lvlbase + 0*\lvlheight) 
            {$\rnnextRHid_\step$}; 
        \node[hid={3}{lencemb}] (lrnn_\step) 
            at (\stepsize *\step+5.35 + 1, \lvlbase + 0*\lvlheight + 0.5) {};    
        \node at (\stepsize *\step+5.35 + 1, \lvlbase + 0*\lvlheight+ 0.5) 
            {$\rnnextLHid_\step$}; 
        \node[hid={3}{ctxemb}] (ctx_\step) 
            at (\stepsize *\step + 5 + 1, \lvlbase + 1*\lvlheight) {};    
        \node at (\stepsize*\step + 5 + 1, \lvlbase + 1*\lvlheight) 
            {$\srHid_\step$}; 

        \draw[dep] (rrnn_\step.north) to (ctx_\step.south);
        \draw[dep] (lrnn_\step.north) to (ctx_\step.south);
    }


        \node[hid={3}{doc}] (doc) 
            at (\stepsize *2 + 10+2, \lvlbase + 1*\lvlheight) {};    
        \node at (\stepsize*2 + 10+2, \lvlbase + 1*\lvlheight) 
            {$\srDocEmb$}; 



    \foreach \step in {1,...,3} {
        \node[hid={3}{rencemb}] (rrnn_\step) 
            at (\stepsize *\step-0.35 + 10 + 2, \lvlbase + 0*\lvlheight) {};    
        \node at (\stepsize*\step-0.35 + 10 + 2, \lvlbase + 0*\lvlheight) 
            {$\rnnextRHid_\step$}; 
        \node[hid={3}{lencemb}] (lrnn_\step) 
            at (\stepsize *\step+10.35+2, \lvlbase + 0*\lvlheight + 0.5) {};    
        \node at (\stepsize *\step+10.35+2, \lvlbase + 0*\lvlheight+ 0.5) 
            {$\rnnextLHid_\step$}; 

        \draw[dep] (rrnn_\step.north) to (doc.south);
        \draw[dep] (lrnn_\step.north) to (doc.south);
    }


\end{tikzpicture}}


\caption{\srext~contextual sentence embedding and document embeddings.}
\label{fig:sr1}\end{minipage}}
\end{figure}


\noindent\fcolorbox{rencemb}{rencemb!20!}{\textit{(\hyperref[fig:sr1]{Figure~\ref{fig:sr1}.a}) Forward and Backward \gru~Outputs}}\\[-40pt]
\begin{align}
    \srrHid_0 &= \zeroEmb, \\
    \srrHid_i &= \fgru(\sentEmb_i, \srrHid_{i-1}; \srRRNNParams) &
    \forall i : i \in \{1,\ldots,\docSize\},\\
    \srlHid_{\docSize + 1} &= \zeroEmb,\\
    \srlHid_i &= \fgru(\sentEmb_i, \srlHid_{i+1}; \srLRNNParams)
    & \forall i : i \in \{1,\ldots,\docSize\},
\end{align}
where $\srrHid_i,\srlHid_i \in \reals^{\srRNNDim}$ and $\srRRNNParams$ and
$\srLRNNParams$ are the forward and backward \gru~parameters respectively.

The \gru~output is concatenated and run through a feed-forward layer to obtain
a contextual sentence embedding representation $\srHid_i \in
\reals^{\srRepDim}$,\\

\noindent\fcolorbox{ctxemb}{ctxemb!20!}{\textit{(\hyperref[fig:sr1]{Figure~\ref{fig:sr1}.b}) Contextual Sentence Embeddings}} 
\begin{align}
\srHid_i  & = \relu\left(\srSentBias + \srSentWeight \left[ \begin{array}{c} \srrHid_i \\ \srlHid_i  \end{array} \right]  \right) &
    \forall i : i \in \{1,\ldots,\docSize\},
\end{align}
where $\srSentWeight \in \reals^{\srRepDim \times 2\srRNNDim}$ and $\srSentBias
\in \reals^{\srRepDim}$ are learned parameters.

To construct the document embedding $\srDocEmb$, the forward and backward
\gru~outputs are concatenated and averaged before running through a different
\feedforward~layer,\\

\noindent\fcolorbox{doc}{doc!20!}{\textit{(\hyperref[fig:sr1]{Figure~\ref{fig:sr1}.c}) Document Embedding}}\\[-40pt]
\begin{align}
\srDocEmb  & = \tanh\left(\srDocBias + \srDocWeight \left(\frac{1}{\docSize}\sum_{i=1}^{\docSize} \left[ \begin{array}{c} \srrHid_i \\ \srlHid_i  \end{array} \right] \right) \right)
\end{align}
where $\srDocWeight \in \reals^{\srRepDim \times 2\srRNNDim}$ and $\srDocBias
\in \reals^{\srRepDim}$ are learned parameters.

Additionally, an iterative representation of the extract summary at step $i$,
$\srSum_i$, is constructed by summing the $i-1$ contextual sentence embeddings
weighted by their salience estimates,\\

\noindent\fcolorbox{sum}{sum!20!}{\textit{(\autoref{fig:sr2}) Summary Embeddings}}\\[-40pt]
\begin{align}
\srSum_1 & = \zeroEmb, \\
\srSum_i & = \tanh\left(\sum_{j=1}^{i-1} \nnpsal_j \cdot \srHid_j\right)
& \forall i: i \in \{2,\ldots,\docSize-1 \},
\end{align}
where $\nnpsal_j=
\nnmodel\left(\nnsal_j=1|\nnsal_1,\ldots,\nnsal_{j-1},\sentEmb_1, \ldots,
\sentEmb_\docSize; \xParams\right)$ are previously computed salience estimates
for sentences $\sent_1,\ldots,\sent_{i-1}$.

\begin{figure}[t]
    \fbox{\begin{minipage}{\textwidth}
\center
\scalebox{0.75}{
\begin{tikzpicture}[
  dep/.style ={
    ->,line width=0.3mm
  },
  hid/.style 2 args={
    rectangle split,
    draw=#2,
    rectangle split parts=#1,
    fill=#2!20,
    minimum width=5mm,
    minimum height=5mm,
    outer sep=2mm},
  mlp/.style 2 args={
    rectangle split,
    rectangle split horizontal,
    draw=#2,
    rectangle split parts=#1,
    fill=#2!20,
    outer sep=2mm},
  sal/.style={
    circle, 
    minimum width=8mm,
    outer sep=2mm,
    draw=#1, 
    fill=#1!20},
]

  \def\stepsize{1.5}%
  \def\lvlbase{0}%
  \def\lvlheight{3}%
 
  \def\stepsize{1.5}%
  \def\lvlbase{0}%
  \def\lvlheight{3}%


        \node[hid={3}{ctxemb}] (ctx_1) 
            at (\stepsize *6 , \lvlbase + -5*\lvlheight) {};    
        \node at (\stepsize*6, \lvlbase + -5*\lvlheight) 
            {$\srHid_1$}; 

        \node[sal={sal}] (p_1) 
            at (\stepsize *7 , \lvlbase + -5*\lvlheight) {};    
        \node at (\stepsize*7, \lvlbase + -5*\lvlheight) 
            {$\psal_1$}; 

        \node[hid={3}{ctxemb}] (ctx_2) 
            at (\stepsize *8 , \lvlbase + -5*\lvlheight) {};    
        \node at (\stepsize*8, \lvlbase + -5*\lvlheight) 
            {$\srHid_2$}; 

        \node[sal={sal}] (p_2) 
            at (\stepsize *9 , \lvlbase + -5*\lvlheight) {};    
        \node at (\stepsize*9, \lvlbase + -5*\lvlheight) 
            {$\psal_2$}; 


        \node[hid={3}{sum}] (summary_3) 
            at (\stepsize *9 , \lvlbase + -4*\lvlheight) {};    
        \node at (\stepsize*9, \lvlbase + -4*\lvlheight) 
            {$\srSum_{i}$}; 


            \node at (\stepsize *9 , \lvlbase + -3.65*\lvlheight) {\textit{Summary Embedding}};

        \node[hid={3}{ctxemb}] (ctx_n) 
            at (\stepsize *11 , \lvlbase + -5*\lvlheight) {};    
        \node at (\stepsize*11, \lvlbase + -5*\lvlheight) 
            {$\srHid_{i-1}$}; 

        \node[sal={sal}] (p_n) 
            at (\stepsize *12 , \lvlbase + -5*\lvlheight) {};    
        \node at (\stepsize*12, \lvlbase + -5*\lvlheight) 
            {$\psal_{i-1}$}; 

        \node at (\stepsize*10, \lvlbase + -5*\lvlheight) {\Large $\cdots$};





        \draw[dep] (ctx_1.north) to (summary_3.south);
        \draw[dep] (p_1.north) to (summary_3.south);

        \draw[dep] (ctx_2.north) to (summary_3.south);
        \draw[dep] (p_2.north) to (summary_3.south);

        \draw[dep] (ctx_n.north) to (summary_3.south);
        \draw[dep] (p_n.north) to (summary_3.south);













%        \node[hid={3}{yellow}] (ctx_1) 
%            at (\stepsize *4 , \lvlbase + -5*\lvlheight) {};    
%        \node at (\stepsize*4, \lvlbase + -5*\lvlheight) 
%            {$\srHid_1$}; 
%        \node[hid={3}{yellow}] (ctx_2) 
%            at (\stepsize *5 , \lvlbase + -5*\lvlheight) {};    
%        \node at (\stepsize*5, \lvlbase + -5*\lvlheight) 
%            {$\srHid_2$}; 
%
%
%

%    \foreach \start [count=\stop from 2] in {1,...,2} {
%        \draw[dep] ($ (rrnn_\start.east) - (0,0.3)$) 
%            -- ($ (rrnn_\stop.west) - (0,0.3) $);
%        \draw[dep] ($(lrnn_\stop.west) + (0,0.3)$) 
%            -- ($ (lrnn_\start.east) + (0,0.3)   $);
%    }

%    \draw[rectangle,draw=black,dotted] 
%        (\stepsize*-2.5,\lvlbase + 3.5*\lvlheight) -- 
%        (\stepsize*3.5, \lvlbase + 3.5*\lvlheight) -- 
%        (\stepsize*3.5, \lvlbase + 2.7*\lvlheight) --
%        (\stepsize*-2.5, \lvlbase + 2.7*\lvlheight) --
%        (\stepsize*-2.5, \lvlbase + 3.5*\lvlheight) ;
%
%    \node[align=left,anchor=north west] 
%        at (\stepsize * -2.5,\lvlbase + 3.5*\lvlheight) 
%        {\textit{Salience Estimates}};
%
%    \draw[rectangle,draw=black,dotted] 
%        (\stepsize*-2.5,\lvlbase + 2.6*\lvlheight) -- 
%        (\stepsize*3.5, \lvlbase + 2.6*\lvlheight) -- 
%        (\stepsize*3.5, \lvlbase + 1.8*\lvlheight) --
%        (\stepsize*-2.5, \lvlbase + 1.8*\lvlheight) --
%        (\stepsize*-2.5, \lvlbase + 2.6*\lvlheight) ;
%
%    \node[align=left,anchor=north west] 
%        at (\stepsize * -2.5,\lvlbase + 2.6*\lvlheight) 
%        {\textit{Contextual Sentence Embeddings}};
%
%    \draw[rectangle,draw=black,dotted] 
%        (\stepsize*-2.5,\lvlbase + 0.5*\lvlheight) -- 
%        (\stepsize*3.5, \lvlbase + 0.5*\lvlheight) -- 
%        (\stepsize*3.5, \lvlbase + -0.50*\lvlheight) --
%        (\stepsize*-2.5, \lvlbase + -0.50*\lvlheight) --
%        (\stepsize*-2.5, \lvlbase + 0.5*\lvlheight) ;
%
%    \node[align=left,anchor=north west] 
%        at (\stepsize * -2.5,\lvlbase + 0.5*\lvlheight) 
%        {\textit{Sentence Embeddings}\\\textit{(Sentence Encoder Output)}};
%
%
%    \draw[rectangle,draw=black,dotted] 
%        (\stepsize*-2.5,\lvlbase + 1.7*\lvlheight) -- 
%        (\stepsize*3.5, \lvlbase + 1.7*\lvlheight) -- 
%        (\stepsize*3.5, \lvlbase + 0.6*\lvlheight) --
%        (\stepsize*-2.5, \lvlbase + 0.6*\lvlheight) --
%        (\stepsize*-2.5, \lvlbase + 1.7*\lvlheight) ;
%
%
%    \node[align=left,anchor=north west] 
%        at (\stepsize * -2.5,\lvlbase + 1.7*\lvlheight) 
%        {\textit{Forward and Backward Partial}\\\textit{Contexual Sentence Embeddings}};

\end{tikzpicture}}

        \caption{\srext~iterative summary embeddings.}
        \label{fig:sr2}

\end{minipage}}
\end{figure}


When computing the salience of a sentence $\nnpsal_i$, the SummaRunner model
uses the contextual sentence embedding $\xHid_i$, the document embedding
$\srDocEmb$, the summary embedding $\srSum_i$, and the document position $i$.
These representations are used to compute five different factors, which are
then summed and run through a logistic sigmoid to the salience estimate.  The
five factors are named the content factor ($\srContentFactor$), the centrality
factor ($\srSalienceFactor$),\footnote{\citet{nallapati2017summarunner} refer
to this as the salience factor, but we rename it here to avoid confusion with
the model's final predictions which we call salience estimates.} the novelty
factor ($\srNoveltyFactor$), the fine grained position factor
($\srFinePositionFactor$), and the coarse grained position factor
($\srCoarsePositionFactor$).

First, there is the content factor which is simply the dot product of the
contextual sentence embedding with a learned parameter vector,\\

\noindent\fcolorbox{factor}{factor!20!}{\textit{(\hyperref[fig:sr3]{Figure~\ref{fig:sr3}.a}) Content Factor}}\\[-40pt]
\begin{align}
    \srContentFactor_i &={\srContentWeight}^\T \srHid_i & \forall i : i \in \{1,\ldots,\docSize\}, 
\end{align}
where $\srContentWeight \in \reals^{\srRepDim}$. This term is intended to
represent contributions of the individual sentences to their salience.

The centrality factor is intended to capture a sentence's similarity to the
document embedding. This interaction computed as\\

\noindent\fcolorbox{factor}{factor!20!}{\textit{(\hyperref[fig:sr3]{Figure~\ref{fig:sr3}.b}) Centrality Factor}}\\[-40pt]
\begin{align}
    \srSalienceFactor_i & = \srHid_i^\T\srSalienceWeight \srDocEmb & \forall i : i \in \{1,\ldots,\docSize\}, 
\end{align}
and is mediated by a learned parameter matrix, $\srSalienceWeight \in \reals^{\srRepDim \times \srRepDim}$.

The third factor, novelty, is similarly an interaction between the contextual
sentence embedding and the $i$-th summary embedding,\\

\noindent\fcolorbox{factor}{factor!20!}{\textit{(\hyperref[fig:sr3]{Figure~\ref{fig:sr3}.c}) Novelty Factor}}\\[-40pt]
\begin{align}
    \srNoveltyFactor_i &= -\srHid_i^\T \srNoveltyWeight \srSum_i
    & \forall i : i \in \{1,\ldots,\docSize\},
    \label{eq:srnov} 
\end{align}
where $\srNoveltyWeight \in \reals^{\srRepDim \times \srRepDim}$ is a learned
parameter matrix. Note also this factor is multiplied by negative to indicate
that high levels of similarity between $\srHid_i$ and $\srSum_i$ should be
discouraged. 


\begin{figure}
\centering
\fbox{\begin{minipage}{\textwidth}\centering\begin{tikzpicture}[
  dep/.style ={
    ->,line width=0.3mm
  },
  hid/.style 2 args={
    rectangle split,
    rectangle split horizontal,
    draw=#2,
    rectangle split parts=#1,
    fill=#2!20,
    minimum width=5mm,
    minimum height=5mm,
    outer sep=2mm},
  mlp/.style 2 args={
    rectangle split,
    rectangle split horizontal,
    draw=#2,
    rectangle split parts=#1,
    fill=#2!20,
    outer sep=2mm},
  sal/.style={
    circle, 
    minimum width=8mm,
    outer sep=2mm,
    draw=#1, 
    fill=#1!20},
]

  \def\stepsize{3}%
  \def\lvlbase{0}%
  \def\lvlheight{3}%

   \node[sal={sal}] (sal) at (\stepsize * 3.55,\lvlbase + -4.25*\lvlheight)   {}; 
    \node at (\stepsize * 3.55,\lvlbase + -4.25*\lvlheight) {$\psal_i$}; 



        \node[hid={3}{ctxemb}] (ctx_i) 
            at (\stepsize *1 , \lvlbase + -2*\lvlheight) {};    
        \node at (\stepsize*1, \lvlbase + -2*\lvlheight) 
            {$\srHid_i$}; 

        \node[hid={1}{factor}] (content_i) 
            at (\stepsize *2, \lvlbase + -2*\lvlheight) {};    
        \node at (\stepsize*2, \lvlbase + -2*\lvlheight) 
            {$\srContentFactor_i$}; 
        \draw[->] (content_i) -- (sal);

        \draw[dep] (ctx_i) to (content_i);
    \draw[rectangle,draw=black,dotted] 
        (\stepsize*0.5,\lvlbase -1.6*\lvlheight) -- 
        (\stepsize*2.25, \lvlbase -1.6*\lvlheight) -- 
        (\stepsize*2.25, \lvlbase - 2.3*\lvlheight) --
        (\stepsize*0.5, \lvlbase -2.3*\lvlheight) --
        (\stepsize*0.5, \lvlbase -1.6*\lvlheight) ;
%

    \node[align=left,anchor=north east] 
        at (\stepsize * 2.25,\lvlbase + -1.6*\lvlheight) 
            {\textit{(a) Content Factor}};

        \node[hid={3}{ctxemb}] (ctx_i) 
            at (\stepsize *1 , \lvlbase + -2.75*\lvlheight) {};    
        \node at (\stepsize*1, \lvlbase + -2.75*\lvlheight) 
            {$\srHid_i$}; 

        \node[hid={1}{factor}] (salience_i) 
            at (\stepsize *2, \lvlbase + -3*\lvlheight) {};    
        \node at (\stepsize*2, \lvlbase + -3*\lvlheight) 
            {$\srSalienceFactor_i$}; 

        \draw[->] (salience_i) -- (sal);
        \node[hid={3}{doc}] (doc) 
            at (\stepsize *1, \lvlbase + -3.25*\lvlheight) {};    
        \node at (\stepsize*1, \lvlbase + -3.25*\lvlheight) 
            {$\srDocEmb$}; 

        \draw[dep] (ctx_i.east) to (salience_i);
        \draw[dep] (doc.east) to (salience_i);

    \draw[rectangle,draw=black,dotted] 
        (\stepsize*0.5,\lvlbase -2.4*\lvlheight) -- 
        (\stepsize*2.25, \lvlbase -2.4*\lvlheight) -- 
        (\stepsize*2.25, \lvlbase - 3.5*\lvlheight) --
        (\stepsize*0.5, \lvlbase -3.5*\lvlheight) --
        (\stepsize*0.5, \lvlbase -2.4*\lvlheight) ;

    \node[align=left,anchor=north east] 
        at (\stepsize * 2.25,\lvlbase + -2.4*\lvlheight) 
            {\textit{(b) Centrality Factor}};

        \node[hid={3}{ctxemb}] (ctx_i) 
            at (\stepsize *1 , \lvlbase + -4.0*\lvlheight) {};    
        \node at (\stepsize*1, \lvlbase + -4.0*\lvlheight) 
            {$\srHid_i$}; 

        \node[hid={1}{factor}] (salience_i) 
            at (\stepsize *2, \lvlbase + -4.25*\lvlheight) {};    
        \node at (\stepsize*2, \lvlbase + -4.25*\lvlheight) 
            {$\srNoveltyFactor_i$}; 

        \draw[->] (salience_i) -- (sal);
        \node[hid={3}{sum}] (doc) 
            at (\stepsize *1, \lvlbase + -4.5*\lvlheight) {};    
        \node at (\stepsize*1, \lvlbase + -4.5*\lvlheight) 
            {$\srSum_i$}; 

        \node[hid={3}{black}] (fg) 
            at (\stepsize *1, \lvlbase + -5.5*\lvlheight) {};    
        \node at (\stepsize*1, \lvlbase + -5.5*\lvlheight) 
            {$\srFinePositionEmb_i$}; 

        \node[hid={1}{factor}] (fine_i) 
            at (\stepsize *2, \lvlbase + -5.5*\lvlheight) {};    
        \node at (\stepsize*2, \lvlbase + -5.5*\lvlheight) 
            {$\srFinePositionFactor_i$}; 

        \draw[->] (fine_i) -- (sal);

    \node[anchor=north east,align=right] 
        at (\stepsize * 2.25,\lvlbase + -4.84*\lvlheight) 
            {\textit{(d) Fine-Grained}\\\textit{Position Factor}};

    \node[align=right,anchor=north east] 
        at (\stepsize * 2.25,\lvlbase + -5.84*\lvlheight) 
            {\textit{(e) Coarse-Grained}\\\textit{Position Factor}};


        \node[hid={3}{black}] (cg) 
            at (\stepsize *1, \lvlbase + -6.5*\lvlheight) {};    
        \node at (\stepsize*1, \lvlbase + -6.5*\lvlheight) 
            {$\srCoarsePositionEmb_i$}; 
        \node[hid={1}{factor}] (coarse_i) 
            at (\stepsize *2, \lvlbase + -6.5*\lvlheight) {};    
        \node at (\stepsize*2, \lvlbase + -6.5*\lvlheight) 
            {$\srCoarsePositionFactor_i$}; 

        \draw[->] (coarse_i) -- (sal);



        \draw[dep] (ctx_i.east) to (salience_i);
        \draw[dep] (doc.east) to (salience_i);


    \draw[rectangle,draw=black,dotted] 
        (\stepsize*0.5,\lvlbase -3.6*\lvlheight) -- 
        (\stepsize*2.25, \lvlbase -3.6*\lvlheight) -- 
        (\stepsize*2.25, \lvlbase - 4.75*\lvlheight) --
        (\stepsize*0.5, \lvlbase -4.75*\lvlheight) --
        (\stepsize*0.5, \lvlbase -3.6*\lvlheight) ;
    \draw[rectangle,draw=black,dotted] 
        (\stepsize*2.95,\lvlbase -3.6*\lvlheight) -- 
        (\stepsize*4.25, \lvlbase -3.6*\lvlheight) -- 
        (\stepsize*4.25, \lvlbase - 4.75*\lvlheight) --
        (\stepsize*2.95, \lvlbase -4.75*\lvlheight) --
        (\stepsize*2.95, \lvlbase -3.6*\lvlheight) ;


    \node[align=left,anchor=north east] 
        at (\stepsize * 2.25,\lvlbase + -3.6*\lvlheight) 
            {\textit{(c) Novelty Factor}};

    \node[align=right,anchor=north east] 
        at (\stepsize * 4.25,\lvlbase + -3.6*\lvlheight) 
            {\textit{(f) Salience}\\\textit{Estimate}};




    \draw[rectangle,draw=black,dotted] 
        (\stepsize*0.5,\lvlbase -4.84*\lvlheight) -- 
        (\stepsize*2.25, \lvlbase -4.84*\lvlheight) -- 
        (\stepsize*2.25, \lvlbase - 5.75*\lvlheight) --
        (\stepsize*0.5, \lvlbase -5.75*\lvlheight) --
        (\stepsize*0.5, \lvlbase -4.84*\lvlheight) ;

    \draw[rectangle,draw=black,dotted] 
        (\stepsize*0.5,\lvlbase -5.84*\lvlheight) -- 
        (\stepsize*2.25, \lvlbase -5.84*\lvlheight) -- 
        (\stepsize*2.25, \lvlbase - 6.75*\lvlheight) --
        (\stepsize*0.5, \lvlbase -6.75*\lvlheight) --
        (\stepsize*0.5, \lvlbase -5.84*\lvlheight) ;



        \draw[dep] (fg) to (fine_i);
        \draw[dep] (cg) to (coarse_i);



\end{tikzpicture}\end{minipage}}
\caption{Schematic of salience estimation in the SummaRunner extractor.}
\label{fig:sr3}
\end{figure}


Finally, there are two factors for the fine- and coarse-grained position,\\

\noindent\fcolorbox{factor}{factor!20!}{\textit{(\hyperref[fig:sr3]{Figure~\ref{fig:sr3}.d}) Fine-grained Position Factor}}\\[-40pt]
\begin{align}
    \srFinePositionFactor& = {\srFinePositionWeight}^\T \srFinePositionEmb_i
& \forall i : i \in \{1,\ldots,\docSize\}, 
\end{align}
\noindent\fcolorbox{factor}{factor!20!}{\textit{(\hyperref[fig:sr3]{Figure~\ref{fig:sr3}.e}) Coarse-grained Position Factor}}\\[-40pt]
\begin{align}
    \srCoarsePositionFactor& = {\srCoarsePositionWeight}^\T \srCoarsePositionEmb_i & \forall i : i \in \{1,\ldots,\docSize\}, 
\end{align}
where $\srFinePositionEmb_i$ and $\srCoarsePositionEmb_i$ are embeddings
associated with the sentence position and sentence position quartile of the
$i$-th sentence (e.g., sentence $s_7$ in a document with 12 sentences, would
have embeddings $\srFinePositionEmb_7$ and $\srCoarsePositionEmb_2$
corresponding to the seventh sentence position and $2^\textrm{nd}$ sentence
position quartile respectively).  Both $\srFinePositionWeight,
\srCoarsePositionWeight \in \reals^{\srPosDim}$, and
$\srFinePositionEmb_1,\ldots,\srFinePositionEmb_\docSizeMax,\srCoarsePositionEmb_1,\ldots,\srCoarsePositionEmb_4
\in \reals^{\srPosDim}$ are learned parameters of the \srext~extractor, and
$\docSizeMax\in\naturals$ is the maximum document size in sentences (when
handling unusually long documents, sentences with positions greater than
$\docSizeMax$ are all mapped to $\srFinePositionEmb_\docSizeMax$).

Each salience estimate $\psal_i$ is calculated as the sum of those five
factors~run through a logistic sigmoid function,\\~\\

\noindent\fcolorbox{sal}{sal!20!}{\textit{(\hyperref[fig:sr3]{Figure~\ref{fig:sr3}.f}) Salience Estimates}}\\[-40pt]
\begin{align}
  \nnpsal_i =  \nnmodel(\nnsal_i=1|\nnsal_1,\dots,\nnsal_{i-1},\sentEmb_1,\ldots,\sentEmb_\docSize;\xParams)
         & = 
        \sigma\left(\srContentFactor_i 
        + \srSalienceFactor_i + \srNoveltyFactor_i
    + \srFinePositionFactor_i + \srCoarsePositionFactor_i   \right) \nonumber \\
    & \phantom{= } \quad \forall i : i \in \{1,\ldots,\docSize\}.
\end{align}
We should give a bit of caution about interpreting the individual factors along
the lines of their given names as the learning procedure does not enforce that
their actual values meaningfully correspond to their names. For example, it
could be that a sentence's similarity to the summary embedding is actually
positively correlated with high salience. In such a case, the model could learn
to produce a large negative value for $\srHid_i^\T \srNoveltyWeight \srSum_i$,
such that the ``novelty'' factor $\srNoveltyFactor_i$ now has large positive
values when the $i$-th sentence is not novel to the summary. In this sense, for
a sufficiently over-parameterized network, the negative in novelty term is
superfluous as the model can learn around it. 

The complete set of parameters for the \srext~extractor is 
\[
    \xParams = \left\{\srRRNNParams,\srLRNNParams,\srSentWeight,\srSentBias,
\srDocWeight,\srDocBias, \srContentWeight, \srSalienceWeight, \srNoveltyWeight,
\srFinePositionWeight, \srCoarsePositionWeight, \srFinePositionEmb_1,\ldots,\srFinePositionEmb_{\docSizeMax}, \srCoarsePositionEmb_1,\ldots,\srCoarsePositionEmb_4,\right\}. 
\]
In our experiments, we set $\srRNNDim=300$, $\srRNNDim=100$, and
$\srPosDim=16$. Dropout with drop probability of $0.25$ is applied to the
\gru~outputs $\srrHid_i$ and $\srlHid_i$, as well as the contextual sentence
embeddings $\srHid_i$ for all $i \in \{1,\ldots,\docSize\}$.

\subsubsection{\rnnext~Extractor}


\begin{figure}[t]
    \fbox{\begin{minipage}{\textwidth}
\center
\scalebox{0.75}{
\begin{tikzpicture}[
  dep/.style ={
    ->,line width=0.3mm
  },
  hid/.style 2 args={
    rectangle split,
    draw=#2,
    rectangle split parts=#1,
    fill=#2!20,
    minimum width=5mm,
    minimum height=5mm,
    outer sep=2mm},
  mlp/.style 2 args={
    rectangle split,
    rectangle split horizontal,
    draw=#2,
    rectangle split parts=#1,
    fill=#2!20,
    outer sep=2mm},
  sal/.style={
    circle, 
    minimum width=8mm,
    outer sep=2mm,
    draw=#1, 
    fill=#1!20},
]

  \def\stepsize{2}%
  \def\lvlbase{0}%
  \def\lvlheight{3}%
 

    % Sentence Embeddings    
  \foreach \step in {1,...,3} {
    \node[hid={3}{sentemb}] (s\step) at (\stepsize*\step, \lvlbase) {};    
    \node at (\stepsize*\step, \lvlbase) {$\sentEmb_\step$};    
   }
%    \foreach \step [count=\i from 1] in {5,6} {
%        \node[hid={3}{green}] (s\step) at (\stepsize*\step, \lvlbase) {};    
%        \node at (\stepsize*\step, \lvlbase) {$\sentEmb_\i$};    
%    %\draw[->] (i\step.north) -> (e\step.south);
%    }
%
%       \node[hid={3}{red}] (s4) at (\stepsize*4, \lvlbase) {};    
%       \node at (\stepsize*4, \lvlbase) {$\sentEmb_0$};    
%
%    % RNN hidden states
    \foreach \step in {1,...,3} {
        \node[hid={3}{rencemb}] (rrnn_\step) 
            at (\stepsize *\step-0.5, \lvlbase + \lvlheight) {};    
        \node at (\stepsize *\step-0.5, \lvlbase + \lvlheight) 
            {$\rnnextRHid_\step$}; 
        \node[hid={3}{lencemb}] (lrnn_\step) 
            at (\stepsize *\step+0.5, \lvlbase + \lvlheight + 1.0) {};    
        \node at (\stepsize *\step+0.5, \lvlbase + \lvlheight+ 1.0) 
            {$\rnnextLHid_\step$}; 
        \draw[dep] (s\step.north) -- (rrnn_\step.south);
        \draw[dep] (s\step.north) -- (lrnn_\step.south);


        \node[hid={3}{ctxemb}] (ctx_\step) 
            at (\stepsize *\step, \lvlbase + 2.25*\lvlheight) {};    
        \node at (\stepsize *\step, \lvlbase + 2.25*\lvlheight) 
            {$\rnnextHid_\step$}; 
        \draw[dep] (rrnn_\step.north) -- (ctx_\step.south);
        \draw[dep] (lrnn_\step.north) -- (ctx_\step.south);


        \node[sal={sal}] (sal_\step) 
            at (\stepsize *\step, \lvlbase + 3*\lvlheight) {};    
        \node at (\stepsize *\step, \lvlbase + 3*\lvlheight) 
            {$\psal_i$}; 
        \draw[dep] (ctx_\step.north) -- (sal_\step.south);
        %\draw[dep] (lrnn_\step.north) -- (ctx_\step.south);


    }
    \foreach \start [count=\stop from 2] in {1,...,2} {
        \draw[dep] ($ (rrnn_\start.east) - (0,0.3)$) 
            -- ($ (rrnn_\stop.west) - (0,0.3) $);
        \draw[dep] ($(lrnn_\stop.west) + (0,0.3)$) 
            -- ($ (lrnn_\start.east) + (0,0.3)   $);
    }

    \draw[rectangle,draw=black,dotted] 
        (\stepsize*-2.5,\lvlbase + 3.5*\lvlheight) -- 
        (\stepsize*3.5, \lvlbase + 3.5*\lvlheight) -- 
        (\stepsize*3.5, \lvlbase + 2.7*\lvlheight) --
        (\stepsize*-2.5, \lvlbase + 2.7*\lvlheight) --
        (\stepsize*-2.5, \lvlbase + 3.5*\lvlheight) ;

    \node[align=left,anchor=north west] 
        at (\stepsize * -2.5,\lvlbase + 3.5*\lvlheight) 
        {\textit{(c) Salience Estimates}};

    \draw[rectangle,draw=black,dotted] 
        (\stepsize*-2.5,\lvlbase + 2.6*\lvlheight) -- 
        (\stepsize*3.5, \lvlbase + 2.6*\lvlheight) -- 
        (\stepsize*3.5, \lvlbase + 1.8*\lvlheight) --
        (\stepsize*-2.5, \lvlbase + 1.8*\lvlheight) --
        (\stepsize*-2.5, \lvlbase + 2.6*\lvlheight) ;

    \node[align=left,anchor=north west] 
        at (\stepsize * -2.5,\lvlbase + 2.6*\lvlheight) 
        {\textit{(b) Contextual Sentence Embeddings}};

    \draw[rectangle,draw=black,dotted] 
        (\stepsize*-2.5,\lvlbase + 0.5*\lvlheight) -- 
        (\stepsize*3.5, \lvlbase + 0.5*\lvlheight) -- 
        (\stepsize*3.5, \lvlbase + -0.50*\lvlheight) --
        (\stepsize*-2.5, \lvlbase + -0.50*\lvlheight) --
        (\stepsize*-2.5, \lvlbase + 0.5*\lvlheight) ;

    \node[align=left,anchor=north west] 
        at (\stepsize * -2.5,\lvlbase + 0.5*\lvlheight) 
        {\textit{Sentence Embeddings}\\\textit{(Sentence Encoder Output)}};


    \draw[rectangle,draw=black,dotted] 
        (\stepsize*-2.5,\lvlbase + 1.7*\lvlheight) -- 
        (\stepsize*3.5, \lvlbase + 1.7*\lvlheight) -- 
        (\stepsize*3.5, \lvlbase + 0.6*\lvlheight) --
        (\stepsize*-2.5, \lvlbase + 0.6*\lvlheight) --
        (\stepsize*-2.5, \lvlbase + 1.7*\lvlheight) ;


    \node[align=left,anchor=north west] 
        at (\stepsize * -2.5,\lvlbase + 1.7*\lvlheight) 
        {\textit{(a) Left and Right Partial}\\\textit{\phantom{(a) }Contexual Sentence Embeddings}};


\end{tikzpicture}}

\caption{Schematic for the \rnnext~sentence extractor.}
\label{fig:rnnext}
\end{minipage}}
\end{figure}


Our first proposed non-autoregressive model is a very simple
\bidirectional~\recurrentneuralnetwork~based tagging model
\citep{graves2005,wang2015}, which we refer to as the \rnnext~extractor.  See
\autoref{fig:rnnext} for a visual depiction of   the extractor.  As in the
\srext~extractor, the first step of the \rnnext~extractor is to run a
\bidirectional~\recurrentneuralnetwork~over the sentence embeddings produced by
the sentence encoder layer, which produces left and right partial contextual
embeddings $\rnnextRHid_i$ and $\rnnextLHid_i$ respectively,\\

\noindent\fcolorbox{rencemb}{rencemb!20!}{\textit{(\hyperref[fig:rnnext]{Figure~\ref{fig:rnnext}.a}) Left and Right Partial Contexual Sentence Embeddings}}\\[-40pt]
\begin{align}
    \rnnextRHid_0 & = \zeroEmb,\\ 
 \rnnextRHid_i &= \fgru\left(\sentEmb_i, \rnnextRHid_{i-1}; \rnnextRParams\right) &
    \forall i : i \in \{1,\ldots,\docSize\}, \\
  \rnnextLHid_{\docSize+1} & = \zeroEmb, \\
 \rnnextLHid_i &= \fgru\left(\sentEmb_i, \rnnextLHid_{i+1};\rnnextLParams\right) &
    \forall i : i \in \{\docSize, \ldots, 1\}, 
\end{align}
where $\rnnextRHid_i,\rnnextLHid_i \in \reals^{\rnnextRNNDim}$,
and $\rnnextRParams$ and $\rnnextLParams$ are the forward and backward
\gru~parameters.

The left and right partial contextual embeddings of each sentence 
are then passed through a \feedforward~layer to produce contextual
sentence embeddings $\rnnextHid_i$,\\

\noindent\fcolorbox{ctxemb}{ctxemb!20!}{\textit{(\hyperref[fig:rnnext]{Figure~\ref{fig:rnnext}.b}) Contexual Sentence Embeddings}}\\[-35pt]
\begin{align}
   \rnnextHid_i &= \relu\left(
    \rnnextHidWeight
    \left[ \begin{array}{c} 
        \rnnextRHid_i \\
        \rnnextLHid_i \end{array}\right] + \rnnextHidBias \right)
    & \forall i :\;\; i \in \{1,\ldots,\docSize\},
\end{align}
where $\rnnextHidWeight \in \reals^{\rnnextHidDim \times 2 \rnnextRNNDim}$
and $\rnnextHidBias \in \reals^{\rnnextHidDim}$ are learned parameters.

Another \feedforward~layer with a logistic sigmoid activation computes the
actual salience estimates $\nnpsal_1,\ldots,\nnpsal_\docSize$ where $\nnpsal_i
= \nnmodel(\nnsal_i=1|\sentEmb_1,\ldots,\sentEmb_\docSize;\xParams)$ and \\

\noindent\fcolorbox{sal}{sal!20!}{\textit{(\hyperref[fig:rnnext]{Figure~\ref{fig:rnnext}.c}) Salience Estimates}}\\[-40pt]
\begin{align}
    \nnpsal_i &= \nnmodel(\nnsal_i=1|\sentEmb_i,\ldots,\sentEmb_\docSize; \xParams) = \sigma\left(\rnnextPredWeight\rnnextHid_i + \rnnextPredBias  \right) &
    \forall i :  i \in \{1,\ldots,\docSize\},
\end{align}
where $\rnnextPredWeight \in \reals^{1 \times \rnnextHidDim}$
and $\rnnextPredBias \in \reals$ are learned parameters.

The complete set of parameters for the extractor is 
\[ \xParams = \left\{\rnnextRParams, \rnnextLParams, \rnnextHidWeight, \rnnextHidBias, \rnnextPredWeight, \rnnextPredBias \right\}.\]
In our experiments, we set $\rnnextRNNDim=300$ and $\rnnextHidDim=100$.
Dropout with drop probability of $0.25$ is applied to $\rnnextRHid_i,
\rnnextLHid_i,$ and $\rnnextHid_i$ for $i \in \{1,\ldots,\docSize\}$.

\FloatBarrier

\subsubsection{\sts~Extractor} 

One shortcoming of the RNN extractor is that long range information from one
end of the document may not easily be able to affect extraction probabilities
of sentences at the other end.  Our second proposed model, the \sts~extractor
mitigates this problem with an attention mechanism commonly used for neural
machine translation \citep{bahdanau2015,luong2015} and abstractive
summarization \citep{see2017}. The \sts~extractor has distinct encoder and
decoder bidirectional GRUs  which create distinct sequences of encoder
contextual embeddings and decoder contextual embeddings (see
\autoref{fig:stsext1} and \autoref{fig:stsext2} respectively).  Using the
attention mechanism, each decoder contextual embedding $\stsextDecHid_i$
attends to the encoder contextual embeddings,
$\stsextEncHid_1,\ldots,\stsextEncHid_\docSize$, to create the final contextual
sentence embedding $\stsextHid_i$. The final contextual embedding,
$\stsextHid_i$, thus contains a representation of sentence $\sent_i$ as well as
its relation to the other contextual representations of the remaining
sentences.  The salience estimate for $\sent_i$ is then produces by running
$\stsextHid_i$ through a feed-forward layer with logistic sigmoid output (see
\autoref{fig:stsext3}). We now describe in detail the encoder, decoder, and
attention/salience estimation layers.

The sentence embeddings produced by the sentence encoder are first encoded by a
\bidirectional~\gru, which produces left and right partial contextual sentence
embeddings, \\

\begin{figure}[t]
\fbox{\begin{minipage}{\textwidth}
\center
\scalebox{0.75}{
\begin{tikzpicture}[
  dep/.style ={
    ->,line width=0.3mm
  },
  hid/.style 2 args={
    rectangle split,
    draw=#2,
    rectangle split parts=#1,
    fill=#2!20,
    minimum width=5mm,
    minimum height=5mm,
    outer sep=2mm},
  mlp/.style 2 args={
    rectangle split,
    rectangle split horizontal,
    draw=#2,
    rectangle split parts=#1,
    fill=#2!20,
    outer sep=2mm},
  sal/.style={
    circle, 
    minimum width=8mm,
    outer sep=2mm,
    draw=#1, 
    fill=#1!20},
]

  \def\stepsize{2}%
  \def\lvlbase{0}%
  \def\lvlheight{3}%
 

    % Sentence Embeddings    
  \foreach \step in {1,...,3} {
    \node[hid={3}{sentemb}] (s\step) at (\stepsize*\step, \lvlbase) {};    
    \node at (\stepsize*\step, \lvlbase) {$\sentEmb_\step$};    
   }
%    \foreach \step [count=\i from 1] in {5,...,7} {
%        \node[hid={3}{green}] (s\step) at (\stepsize*\step, \lvlbase) {};    
%        \node at (\stepsize*\step, \lvlbase) {$\sentEmb_\i$};    
%    %\draw[->] (i\step.north) -> (e\step.south);
%    }

%       \node[hid={3}{red}] (s4) at (\stepsize*4, \lvlbase) {};    
%       \node at (\stepsize*4, \lvlbase) {$\sentEmb_0$};    
%
%    % RNN hidden states
    \foreach \step in {1,...,3} {
        \node[hid={3}{rencemb}] (rrnn_\step) 
            at (\stepsize *\step+0.5, \lvlbase + \lvlheight) {};    
        \node at (\stepsize *\step+0.5, \lvlbase + \lvlheight) 
            {$\stsextREncHid_\step$}; 

        \node[hid={3}{lencemb}] (lrnn_\step) 
            at (\stepsize *\step-0.5, \lvlbase + \lvlheight+1.0) {};    
        \node at (\stepsize *\step-0.5, \lvlbase + \lvlheight+1.0) 
            {$\stsextLEncHid_\step$}; 


        \node[hid={3}{encctxemb}] (enc_ctx_\step) 
            at (\stepsize *\step, \lvlbase + 2.5*\lvlheight) {};    
        \node at (\stepsize *\step, \lvlbase + 2.5*\lvlheight) 
            {$\stsextEncHid_\step$}; 

        \draw[dep] (s\step.north) -- (lrnn_\step.south);
        \draw[dep] (s\step.north) -- (rrnn_\step.south);

        \draw[dep] (lrnn_\step.north) -- (enc_ctx_\step.south);
        \draw[dep] (rrnn_\step.north) -- (enc_ctx_\step.south);
    }
    \foreach \start [count=\stop from 2] in {1,...,2} {
        \draw[dep] ($ (rrnn_\start.east) - (0,0.3)$) 
            -- ($ (rrnn_\stop.west) - (0,0.3) $);
        \draw[dep] ($(lrnn_\stop.west) + (0,0.3)$) 
            -- ($ (lrnn_\start.east) + (0,0.3)   $);
    }


    \draw[rectangle,draw=black,dotted] 
        (\stepsize*-2.75,\lvlbase + 0.5*\lvlheight) -- 
        (\stepsize*3.5, \lvlbase + 0.5*\lvlheight) -- 
        (\stepsize*3.5, \lvlbase + -0.50*\lvlheight) --
        (\stepsize*-2.75, \lvlbase + -0.50*\lvlheight) --
        (\stepsize*-2.75, \lvlbase + 0.5*\lvlheight) ;

    \node[align=left,anchor=north west] 
        at (\stepsize * -2.75,\lvlbase + 0.5*\lvlheight) 
        {\textit{Sentence Embeddings}\\\textit{(Sentence Encoder Output)}};

    \draw[rectangle,draw=black,dotted] 
        (\stepsize*-2.75,\lvlbase + 3.0*\lvlheight) -- 
        (\stepsize*3.5, \lvlbase + 3.0*\lvlheight) -- 
        (\stepsize*3.5, \lvlbase + 2.0*\lvlheight) --
        (\stepsize*-2.75, \lvlbase + 2.0*\lvlheight) --
        (\stepsize*-2.75, \lvlbase + 3.0*\lvlheight) ;

    \node[align=left,anchor=north west] 
        at (\stepsize * -2.75,\lvlbase + 3.0*\lvlheight) 
        {\textit{(b) Encoder Contextual Sentence Embeddings}};

    \draw[rectangle,draw=black,dotted] 
        (\stepsize*-2.75,\lvlbase + 1.8*\lvlheight) -- 
        (\stepsize*3.5, \lvlbase + 1.8*\lvlheight) -- 
        (\stepsize*3.5, \lvlbase + 0.6*\lvlheight) --
        (\stepsize*-2.75, \lvlbase + 0.6*\lvlheight) --
        (\stepsize*-2.75, \lvlbase + 1.8*\lvlheight) ;

    \node[align=left,anchor=north west] 
        at (\stepsize * -2.75,\lvlbase + 1.8*\lvlheight) 
        {\textit{(a) Encoder Left and Right Partial}\\\textit{\phantom{(a) }Contextual Sentence Embeddings}};


\end{tikzpicture}}
\caption{Schematic for the encoder contextual sentence embeddings as 
computed in the \stsext~sentence extractor.}
\label{fig:stsext1}
\end{minipage}}
\end{figure}



\noindent\fcolorbox{rencemb}{rencemb!20!}{\textit{(\hyperref[fig:stsext1]{Figure~\ref{fig:stsext1}.a}) Encoder Left and Right Partial Contextual Sentence Embeddings}}\\[-40pt]
\begin{align}
        \stsextREncHid_0  &= \zeroEmb, \\
\stsextREncHid_i & = \fgru\left(
            \sentEmb_i, \stsextREncHid_{i-1}; 
            \stsextREncParams\right) &
\forall i : i \in \{1,\ldots,\docSize\}, \\
        \stsextLEncHid_{\docSize + 1}  &= \zeroEmb, \\
\stsextLEncHid_i &= \fgru\left(
            \sentEmb_i, \stsextLEncHid_{i+1}; 
            \stsextLEncParams\right) &
\forall i : i \in \{\docSize,\ldots,1\}, 
\end{align}
where $\stsextREncHid_i,\stsextLEncHid_i \in \reals^{\stsextRNNDim}$ and
$\stsextREncParams$ and $\stsextLEncParams$ are the forward and backward
encoder \gru~parameters respectively. The encoder contextual sentence
embeddings are then formed by simply concatenating the encoder left and right
partial contextual embeddings,\\

\noindent\fcolorbox{encctxemb}{encctxemb!20!}{\textit{(\hyperref[fig:stsext1]{Figure~\ref{fig:stsext1}.b}) Encoder Contextual Sentence Embeddings}}\\[-20pt]
\begin{align}
        \stsextEncHid_i & = \left[\begin{array}{c}
            \stsextREncHid_i\\ 
            \stsextLEncHid_i\end{array}\right]  &
\forall i :  i \in \{1,\ldots,\docSize\}.
\end{align}


\begin{figure}[t]
\fbox{\begin{minipage}{\textwidth}
\center
\scalebox{0.75}{
\begin{tikzpicture}[
  dep/.style ={
    ->,line width=0.3mm
  },
  hid/.style 2 args={
    rectangle split,
    draw=#2,
    rectangle split parts=#1,
    fill=#2!20,
    minimum width=5mm,
    minimum height=5mm,
    outer sep=2mm},
  mlp/.style 2 args={
    rectangle split,
    rectangle split horizontal,
    draw=#2,
    rectangle split parts=#1,
    fill=#2!20,
    outer sep=2mm},
  sal/.style={
    circle, 
    minimum width=8mm,
    outer sep=2mm,
    draw=#1, 
    fill=#1!20},
]

  \def\stepsize{2}%
  \def\lvlbase{0}%
  \def\lvlheight{3}%
 

    \node[hid={3}{decsentemb}] (s0) at (\stepsize*0, \lvlbase) {};    
    \node at (\stepsize*0, \lvlbase) {$\stsextSent_>$};    


    \node[hid={3}{decsentemb}] (s4) at (\stepsize*4, \lvlbase) {};    
    \node at (\stepsize*4, \lvlbase) {$\stsextSent_{<}$};    

    % Sentence Embeddings    
  \foreach \step in {1,...,3} {
    \node[hid={3}{sentemb}] (s\step) at (\stepsize*\step, \lvlbase) {};    
    \node at (\stepsize*\step, \lvlbase) {$\sentEmb_\step$};    
   }
%    \foreach \step [count=\i from 1] in {5,...,7} {
%        \node[hid={3}{green}] (s\step) at (\stepsize*\step, \lvlbase) {};    
%        \node at (\stepsize*\step, \lvlbase) {$\sentEmb_\i$};    
%    %\draw[->] (i\step.north) -> (e\step.south);
%    }

%       \node[hid={3}{red}] (s4) at (\stepsize*4, \lvlbase) {};    
%       \node at (\stepsize*4, \lvlbase) {$\sentEmb_0$};    
%
%    % RNN hidden states

        \node[hid={3}{rdecemb}] (rrnn_0) 
            at (\stepsize *0+0.5, \lvlbase + \lvlheight) {};    
        \node at (\stepsize *0+0.5, \lvlbase + \lvlheight) 
            {$\stsextRDecHid_0$}; 

        \node[hid={3}{ldecemb}] (lrnn_4) 
            at (\stepsize *4-0.5, \lvlbase + \lvlheight+1.0) {};    
        \node at (\stepsize *4-0.5, \lvlbase + \lvlheight+1.0) 
            {$\stsextLDecHid_4$}; 

        \node[hid={3}{rencemb}] (rrnn_m1) 
            at (-\stepsize +0.5, \lvlbase + \lvlheight) {};    
        \node at (-\stepsize +0.5, \lvlbase + \lvlheight) 
            {$\stsextREncHid_3$}; 

        \node[hid={3}{lencemb}] (lrnn_m1) 
            at (\stepsize*5 -0.5, \lvlbase + \lvlheight+1.0) {};    
        \node at (\stepsize*5 -0.5, \lvlbase + \lvlheight+1.0) 
            {$\stsextLEncHid_1$}; 

            \node[align=left,anchor=north west,text width=4.3cm] (lbl)  
                at (\stepsize*5.0, \lvlbase + \lvlheight*1.65) 
                {\textit{Encoder Right Partial Contextual Sentence Embedding}};    
    %\draw[dotted] (lbl.east) to (s0);
        \draw[rectangle,draw=black,dotted] 
        (\stepsize*4.5,\lvlbase + 1.65*\lvlheight) -- 
        (\stepsize*7.5, \lvlbase + 1.65*\lvlheight) -- 
        (\stepsize*7.5, \lvlbase + 0.65*\lvlheight) --
        (\stepsize*4.5, \lvlbase + 0.65*\lvlheight) --
        (\stepsize*4.5, \lvlbase + 1.65*\lvlheight) ;


        \draw[rectangle,draw=black,dotted] 
        (\stepsize*-3.0,\lvlbase + 1.65*\lvlheight) -- 
        (\stepsize*-0.5, \lvlbase + 1.65*\lvlheight) -- 
        (\stepsize*-0.5, \lvlbase + 0.65*\lvlheight) --
        (\stepsize*-3.0, \lvlbase + 0.65*\lvlheight) --
        (\stepsize*-3.0, \lvlbase + 1.65*\lvlheight) ;

            \node[align=left,anchor=north west,text width=4.3cm] (lbl)  
                at (\stepsize*-3.0, \lvlbase + \lvlheight*1.65) 
                {Encoder Left Partial Contextual Sentence Embedding};    





    \foreach \step in {1,...,3} {
        \node[hid={3}{rdecemb}] (rrnn_\step) 
            at (\stepsize *\step+0.5, \lvlbase + \lvlheight) {};    
        \node at (\stepsize *\step+0.5, \lvlbase + \lvlheight) 
            {$\stsextRDecHid_\step$}; 

        \node[hid={3}{ldecemb}] (lrnn_\step) 
            at (\stepsize *\step-0.5, \lvlbase + \lvlheight+1.0) {};    
        \node at (\stepsize *\step-0.5, \lvlbase + \lvlheight+1.0) 
            {$\stsextLDecHid_\step$}; 


        \node[hid={3}{decctxemb}] (enc_ctx_\step) 
            at (\stepsize *\step, \lvlbase + 2.5*\lvlheight) {};    
        \node at (\stepsize *\step, \lvlbase + 2.5*\lvlheight) 
            {$\stsextDecHid_\step$}; 

        \draw[dep] (s\step.north) -- (lrnn_\step.south);
        \draw[dep] (s\step.north) -- (rrnn_\step.south);

        \draw[dep] (lrnn_\step.north) -- (enc_ctx_\step.south);
        \draw[dep] (rrnn_\step.north) -- (enc_ctx_\step.south);
    }
    \foreach \start [count=\stop from 2] in {1,...,2} {
        \draw[dep] ($ (rrnn_\start.east) - (0,0.3)$) 
            -- ($ (rrnn_\stop.west) - (0,0.3) $);
        \draw[dep] ($(lrnn_\stop.west) + (0,0.3)$) 
            -- ($ (lrnn_\start.east) + (0,0.3)   $);
    }

        \draw[dep] ($ (rrnn_m1.east) - (0,0.3)$) 
            -- ($ (rrnn_0.west) - (0,0.3) $);
        \draw[dep] ($ (rrnn_0.east) - (0,0.3)$) 
            -- ($ (rrnn_1.west) - (0,0.3) $);

            \draw[dep] (s0.north) to (rrnn_0.south);
            \draw[dep] (s4.north) to (lrnn_4.south);

        \draw[dep] ($ (lrnn_m1.west) + (0,0.3)$) 
            -- ($ (lrnn_4.east) + (0,0.3) $);

        \draw[dep] ($ (lrnn_4.west) + (0,0.3)$) 
            -- ($ (lrnn_3.east) + (0,0.3) $);

%    \foreach \step [count=\i from 0] in {4,...,7} {
%        \node[hid={3}{orange}] (rnn_\step) 
%            at (\stepsize *\step, \lvlbase + \lvlheight) {};    
%        \node at (\stepsize *\step, \lvlbase + \lvlheight) {$\xDecHid_\i$}; 
%    }
%
%    \foreach \step in {1,...,6} {
%        \draw[dep] (s\step.north) to (rnn_\step.south);
%    }
%    \foreach \start [count=\stop from 2] in {1,...,5} {
%        \draw[dep] (rnn_\start.east) to (rnn_\stop.west);
%    }
%
%
%%    \foreach \step [count=\i from 1] in {4,...,6} {
%%        \node[hid={3}{green}] (ctx_\i) 
%%            at (\stepsize *\step, \lvlbase + 2*\lvlheight) {};    
%%        \node at (\stepsize *\step, \lvlbase + 2*\lvlheight) {$\xPredHid_\i$}; 
%%        \draw[dep] (rnn_\step.north) to (ctx_\i.south);
%%        \draw[dep] (rnn_\i.north) to (ctx_\i.south west);
%%        \node[sal={blue}] (sal_\i) at (\stepsize * \step,\lvlbase + 3*\lvlheight) {};
%%        \node at (\stepsize * \step,\lvlbase + 3*\lvlheight) {$\bsal_\i$};
%%        \draw[dep] (ctx_\i.north) to (sal_\i.south);
%%    }
%%
%%    \foreach \step [count=\i from 1] in {5,...,6} {
%%        \draw[dep] (sal_\i) -- (\stepsize * \step - \stepsize / 2,
%%                            \lvlbase + 3*\lvlheight) 
%%                     -- (\stepsize * \step - \stepsize / 2,
%%                            \lvlbase + 0* \lvlheight) -> (s\step) ;
%%
%%    }
%%    \draw[dep] (s1.south) -- ($ (s1.south) + (0,-0.3)$) --($ (s5.south) + (0,-0.3)$) -- (s5.south);
%%
%%    \draw[dep] (s2.south) -- ($ (s2.south) + (0,-0.5)$) --($ (s6.south) + (0,-0.5)$) -- (s6.south);
%%
%%    \draw[rectangle,draw=black,dotted] 
%%        (\stepsize * 3.5,\lvlbase + 3.5*\lvlheight) -- 
%%        (\stepsize*6.5, \lvlbase + 3.5*\lvlheight) -- 
%%        (\stepsize*6.5, \lvlbase + 2.6*\lvlheight) --
%%        (\stepsize*3.5, \lvlbase + 2.6*\lvlheight) --
%%        (\stepsize*3.5, \lvlbase + 3.5*\lvlheight) ;
%%
%%    \node[align=left,anchor=north west] 
%%        at (\stepsize * 3.5,\lvlbase + 3.5*\lvlheight) 
%%        {\textit{Salience Estimates}};
%%
%%    \draw[rectangle,draw=black,dotted] 
%%        (\stepsize*0.5,\lvlbase + 2.5*\lvlheight) -- 
%%        (\stepsize*6.5, \lvlbase + 2.5*\lvlheight) -- 
%%        (\stepsize*6.5, \lvlbase + 1.5*\lvlheight) --
%%        (\stepsize*0.5, \lvlbase + 1.5*\lvlheight) --
%%        (\stepsize*0.5, \lvlbase + 2.5*\lvlheight) ;
%%
%%    \node[align=left,anchor=north west] 
%%        at (\stepsize * 0.5,\lvlbase + 2.5*\lvlheight) 
%%        {\textit{Contextual Sentence Embeddings}};
%%
%%    \draw[rectangle,draw=black,dotted] 
%%        (\stepsize*-0.5,\lvlbase + 0.5*\lvlheight) -- 
%%        (\stepsize*3.5, \lvlbase + 0.5*\lvlheight) -- 
%%        (\stepsize*3.5, \lvlbase + -0.9*\lvlheight) --
%%        (\stepsize*-0.5, \lvlbase + -0.9*\lvlheight) --
%%        (\stepsize*-0.5, \lvlbase + 0.5*\lvlheight) ;
%%
%%    \node[align=left,anchor=south west] 
%%        at (\stepsize * -0.5,\lvlbase + -0.9*\lvlheight) 
%%        {\textit{Sentence Embeddings}\\\textit{(Sentence Encoder Output)}};
%%
%%    \draw[rectangle,draw=black,dotted] 
%%        (\stepsize*4.3,\lvlbase + 0.5*\lvlheight) -- 
%%        (\stepsize*6.5, \lvlbase + 0.5*\lvlheight) -- 
%%        (\stepsize*6.5, \lvlbase + -0.9*\lvlheight) --
%%        (\stepsize*4.3, \lvlbase + -0.9*\lvlheight) --
%%        (\stepsize*4.3, \lvlbase + 0.5*\lvlheight) ;
%%
%%    \node[align=left,anchor=south west] 
%%        at (\stepsize * 4.3,\lvlbase + -0.9*\lvlheight) 
%%        {\textit{Salience Gated}\\\textit{Sentence Embeddings}};
%%

    

    \draw[rectangle,draw=black,dotted] 
        (\stepsize*-0.75,\lvlbase + 3.0*\lvlheight) -- 
        (\stepsize*3.5, \lvlbase + 3.0*\lvlheight) -- 
        (\stepsize*3.5, \lvlbase + 2.0*\lvlheight) --
        (\stepsize*-0.75, \lvlbase + 2.0*\lvlheight) --
        (\stepsize*-0.75, \lvlbase + 3.0*\lvlheight) ;

    \node[align=left,anchor=north west] 
        at (-0.75 *\stepsize,\lvlbase + 3.0*\lvlheight) 
        {\textit{(b) Decoder Contextual Sentence Embeddings}};

        \draw[rectangle,draw=black,dotted] 
        (\stepsize*0.5,\lvlbase + 0.5*\lvlheight) -- 
        (\stepsize*3.5, \lvlbase + 0.5*\lvlheight) -- 
        (\stepsize*3.5, \lvlbase - 0.75*\lvlheight) --
        (\stepsize*0.5, \lvlbase - 0.75*\lvlheight) --
        (\stepsize*0.5, \lvlbase + 0.5*\lvlheight) ;

    \node[align=left,anchor=south west] 
        at (0.5 *\stepsize,\lvlbase - 0.75*\lvlheight) 
        {\textit{Sentence Embeddings}\\\textit{(Sentence Encoder Outputs)}};

        \draw[rectangle,draw=black,dotted] 
        (\stepsize*-0.25,\lvlbase + 1.65*\lvlheight) -- 
        (\stepsize*4.25, \lvlbase + 1.65*\lvlheight) -- 
        (\stepsize*4.25, \lvlbase + 0.65*\lvlheight) --
        (\stepsize*-0.25, \lvlbase + 0.65*\lvlheight) --
        (\stepsize*-0.25, \lvlbase + 1.65*\lvlheight) ;

    \node[align=left,anchor=south west,text width=6.4cm] 
    (lbl) at (4.5 *\stepsize,\lvlbase + 2.25*\lvlheight) 
        {\textit{(a) Decoder Left and Right Partial \phantom{(a) }Contextual Sentence Embeddings}};

    \draw[dotted] (\stepsize*4.25, \lvlbase + 1.65*\lvlheight) to (lbl);


    \node[align=left,anchor=west,text width=4.2cm] (lbl)  at (\stepsize*5, \lvlbase) {\textit{Right Begin Decoding Embedding}};    
    \draw[dotted] (lbl.west) to (s4);

    \node[align=left,anchor=east,text width=4.5cm] (lbl)  at (\stepsize-3, \lvlbase) {\textit{Left Begin Decoding Embedding}};    
    \draw[dotted] (lbl.east) to (s0);


\end{tikzpicture}}
\end{minipage}}
\caption{Schematic for the decoder contextual sentence embeddings as computed by the \stsext~sentence extractor.}
\label{fig:stsext2}

\end{figure}




The final output of each encoder \gru~initializes a separate decoder \gru~which
is then run over the sentence embeddings a second time, \\


\noindent\fcolorbox{rdecemb}{rdecemb!20!}{\textit{(\hyperref[fig:stsext2]{Figure~\ref{fig:stsext2}.a}) Decoder Left and Right Partial Contextual Sentence Embeddings}}\\[-40pt]
\begin{align}
\stsextRDecHid_0 & = \fgru\left(\stsextSent_>,   \stsextREncHid_\docSize; \stsextRDecParams\right),\\ 
        \stsextRDecHid_i & = \fgru(
            \sentEmb_i,  \stsextRDecHid_{i-1}; 
            \stsextRDecParams) &
    \forall i : i \in \{1,\ldots,\docSize\}, \\
\stsextLDecHid_{\docSize+1} & = \fgru\left(\stsextSent_<, \stsextLEncHid_1;\stsextLDecParams\right),\\ 
        \stsextLDecHid_i & = \fgru(
            \sentEmb_i,  \stsextLDecHid_{i+1}; 
            \stsextLDecParams) &
    \forall i : i \in \{1,\ldots,\docSize\}, 
\end{align}
where $\stsextSent_>$ and $\stsextSent_<$ are special ``begin decoding'' input
embeddings for the forward and backward decoder respectively,
$\stsextRDecHid_i, \stsextLDecHid_i \in \reals^{\stsextRNNDim}$, and
$\stsextRDecParams$ and $\stsextLDecParams$ are the parameters for the forward
and backward decoder \gru s respectively.

The decoder left and right partial contextual sentence embeddings
($\stsextRDecHid_i$ and $\stsextLDecHid_i$) are then concatenated to form the
decoder contextual sentence embeddings,\\

\noindent\fcolorbox{decctxemb}{decctxemb!20!}{\textit{(\hyperref[fig:stsext2]{Figure~\ref{fig:stsext2}.b}) Decoder Contextual Sentence Embeddings}}\\[-20pt]
\begin{align}
        \stsextDecHid_i & = \left[\begin{array}{c}
            \stsextRDecHid_i\\ 
            \stsextLDecHid_i\end{array}\right] &
    \forall i :  i \in \{1,\ldots,\docSize\}.
\end{align}


\begin{figure}[p]
\fbox{\begin{minipage}{\textwidth}
\center
\scalebox{0.75}{
\begin{tikzpicture}[
  dep/.style ={
    ->,line width=0.3mm
  },
  hid/.style 2 args={
    rectangle split,
    draw=#2,
    rectangle split parts=#1,
    fill=#2!20,
    minimum width=5mm,
    minimum height=5mm,
    outer sep=2mm},
  mlp/.style 2 args={
    rectangle split,
    rectangle split horizontal,
    draw=#2,
    rectangle split parts=#1,
    fill=#2!20,
    outer sep=2mm},
  sal/.style={
    circle, 
    minimum width=8mm,
    outer sep=2mm,
    draw=#1, 
    fill=#1!20},
]

  \def\stepsize{3}%
  \def\lvlbase{0}%
  \def\lvlheight{3}%
 


    % Sentence Embeddings    
  \foreach \step in {1,...,3} {
    \node[hid={3}{encctxemb}] (enc\step) at (\stepsize*\step, \lvlbase) {};    
    \node at (\stepsize*\step, \lvlbase) {$\stsextEncHid_\step$};    
   }

  \foreach \step in {1,...,3} {
    \node[hid={1}{yellow}] (a\step) at (\stepsize*\step-0.5*\stepsize, \lvlbase) {};    
    \node at (\stepsize*\step-0.5*\stepsize, \lvlbase) {$\stsextAttn_{i,\step}$};    

    \draw[dep] (enc\step.west) to (a\step.east);
   }


    \node[hid={3}{decctxemb}] (deci) at (\stepsize*4, \lvlbase) {};    
    \node at (\stepsize*4, \lvlbase) {$\stsextDecHid_i$};    


    \draw[dep] (deci.south) to [out=270,in=270] (a3.south);
    \draw[dep] (deci.south) to [out=270,in=270] (a2.south);
    \draw[dep] (deci.south) to [out=270,in=270] (a1.south);

    \node[hid={3}{yellow}] (encattn) at (\stepsize*2, \lvlbase+\lvlheight) {};    
    \node at (\stepsize*2, \lvlbase+\lvlheight) {$\stsextAttnHid_i$};    


    \node[hid={3}{ctxemb}] (ctxi) at (\stepsize*4, \lvlbase+\lvlheight) {};    
    \node at (\stepsize*4, \lvlbase+\lvlheight) {$\stsextHid_i$};    

        \node[sal={sal}] (sal) at (\stepsize * 5,\lvlbase + \lvlheight) {};
        \node at (\stepsize * 5,\lvlbase + \lvlheight) {$\psal_i$};

    \draw[dep] (ctxi) to (sal);
    \draw[dep] (deci.north) to (ctxi.south);
    \draw[dep] (encattn.east) to (ctxi.west);

    \node[text width=2.3cm] (info) at (0,\lvlbase + \lvlheight) {\textit{Encoder Contextual Sentence Embeddings}};
    \draw[dotted] (enc1.north west) to (info);
    \draw[dotted] (enc2.north west) to (info);
    \draw[dotted] (enc3.north west) to (info);
    \node (info) at (0,\lvlbase - 1.5*\lvlheight) {\textit{(a) Attention Weights}};
    \draw[dotted] (a1.south west) to (info);
    \draw[dotted] (a2.south west) to (info);
    \draw[dotted] (a3.south west) to (info);

    \foreach \step in {1,...,3} {
        \draw[dep] (a\step.north) to (encattn.south);
        \draw[dep] (enc\step.north) to (encattn.south);
    }


    \node[align=left,anchor=north west,text width=5.5cm] at (\stepsize*2-2.1,2.0*\lvlheight+\lvlbase)
    {\textit{(b) Attention-weighted \phantom{(b) }Encoder Sentence \phantom{(b) }Embedding}};
\draw[rectangle,draw=black,dotted] 
        (\stepsize*2.0-2.1, \lvlbase + 2.0*\lvlheight) -- 
        (\stepsize*2.0+2.1, \lvlbase + 2.0*\lvlheight) -- 
        (\stepsize*2.0+2.1, \lvlbase + 0.6*\lvlheight) --
        (\stepsize*2.0-2.1, \lvlbase + 0.6*\lvlheight) --
        (\stepsize*2.0-2.1, \lvlbase + 2.0*\lvlheight) ;




    \node[align=left,anchor=north west,text width=2.5cm] at (\stepsize*4-1.4,2.0*\lvlheight+\lvlbase)
    {\textit{(c) Contextual \phantom{(c) }Sentence \phantom{(c) }Embedding}};
\draw[rectangle,draw=black,dotted] 
        (\stepsize*4.0-1.4, \lvlbase + 2.0*\lvlheight) -- 
        (\stepsize*4.0+1.4, \lvlbase + 2.0*\lvlheight) -- 
        (\stepsize*4.0+1.4, \lvlbase + 0.6*\lvlheight) --
        (\stepsize*4.0-1.4, \lvlbase + 0.6*\lvlheight) --
        (\stepsize*4.0-1.4, \lvlbase + 2.0*\lvlheight) ;


    \node[align=left,anchor=north west,text width=2.5cm] at (\stepsize*5-1.4,2.0*\lvlheight+\lvlbase)
{\textit{(d) Salience \phantom{(d) }Estimate}};
\draw[rectangle,draw=black,dotted] 
        (\stepsize*5.0-1.4, \lvlbase + 2.0*\lvlheight) -- 
        (\stepsize*5.0+1.4, \lvlbase + 2.0*\lvlheight) -- 
        (\stepsize*5.0+1.4, \lvlbase + 0.6*\lvlheight) --
        (\stepsize*5.0-1.4, \lvlbase + 0.6*\lvlheight) --
        (\stepsize*5.0-1.4, \lvlbase + 2.0*\lvlheight) ;


    \draw[rectangle,draw=black,dotted] 
        (\stepsize*4.0-1.4, \lvlbase + 0.4*\lvlheight) -- 
        (\stepsize*5.0+1.4, \lvlbase + 0.4*\lvlheight) -- 
        (\stepsize*5.0+1.4, \lvlbase - 0.6*\lvlheight) --
        (\stepsize*4.0-1.4, \lvlbase - 0.6*\lvlheight) --
        (\stepsize*4.0-1.4, \lvlbase + 0.4*\lvlheight) ;


    \node[align=left,anchor=north east,text width=2.5cm] at (\stepsize*5+1.4,0.4*\lvlheight + \lvlbase)
    {\textit{\phantom{(a) }Decoder \phantom{(a) }Contextual \phantom{(a) }Sentence \phantom{(a) }Embedding}};


\end{tikzpicture}}
\caption{Attention layer, contextual sentence embedding, and salience
estimation layer for the \stsext~extractor.}
\label{fig:stsext3}
\end{minipage}}
\end{figure}



\clearpage

Each decoder contextual sentence embedding $\stsextDecHid_i$ then attends to
the encoder contextual sentence embeddings $\stsextEncHid_1,\ldots,
\stsextEncHid_\docSize$, to produce an attention-weighted encoder sentence
embedding $\stsextAttnHid_i$,\\
    
\noindent\fcolorbox{yellow}{yellow!20!}{\textit{(\hyperref[fig:stsext3]{Figure~\ref{fig:stsext3}.a}) Attention Weights}}\\[-20pt]
\begin{align}
    \stsextAttn_{i,j} & = 
        \frac{\exp \left(\stsextDecHid_i \cdot  \stsextEncHid_j \right)}{
            \sum_{j^\prime=1}^{\docSize}\exp\left(  
                \stsextDecHid_i \cdot  \stsextEncHid_{j^\prime} \right)} &
    \forall i : i \in \{1,\ldots,\docSize\}, 
\end{align}
\noindent\fcolorbox{yellow}{yellow!20!}{\textit{(\hyperref[fig:stsext3]{Figure~\ref{fig:stsext3}.b}) Attention-weighted Encoder Sentence Embeddings}}\\[-20pt]
\begin{align}
    \stsextAttnHid_i & = 
        \sum_{j=1}^{\docSize} \alpha_{i,j} \left[\begin{array}{c}
            \stsextREncHid_j\\ 
            \stsextLEncHid_j\end{array}\right] & 
    \forall i : i \in \{1,\ldots,\docSize\}.
\end{align}

The attention-weighted encoder sentence embeddings and the decoder contextual
sentence embedding are then concatenated and fed through a \feedforward~layer
to produce the $i^\textrm{th}$ contextual sentence embedding $\stsextHid_i$,
which is itself fed through a final \feedforward~layer to compute the
$i^\textrm{th}$ salience estimate $\nnpsal_i =
\nnmodel(\nnsal_i=1|\sentEmb_1,\ldots,\sentEmb_\docSize;\xParams)$,\\

\noindent\fcolorbox{ctxemb}{ctxemb!20!}{\textit{(\hyperref[fig:stsext3]{Figure~\ref{fig:stsext3}.c}) Contextual Sentence Embeddings}}\\[-20pt]
\begin{align}
    \stsextHid_i &= \relu\left(\stsextHidWeight \left[\begin{array}{c}
        \stsextAttnHid_i \\ 
        \stsextDecHid_i \end{array}\right] 
        + \stsextHidBias \right) &
    \forall i : i \in \{1,\ldots,\docSize\},
\end{align}


\noindent\fcolorbox{sal}{sal!20!}{\textit{(\hyperref[fig:stsext3]{Figure~\ref{fig:stsext3}.d}) Salience Estimates}}\\[-40pt]
\begin{align}
\nnpsal_i =    \nnmodel(\nnsal_i=1|\sentEmb_1, \ldots,\sentEmb_\docSize;\xParams)& = 
            \sigma\left(\stsextPredWeight \stsextHid_i + \stsextPredBias  
            \right) &
    \forall i : i \in \{1,\ldots,\docSize\}.
\end{align}
where $\stsextHidWeight \in \reals^{\stsextHidDim \times 3\stsextRNNDim}$,
$\stsextHidBias \in \reals^{\stsextHidDim}$, 
$\stsextPredWeight \in \reals^{1 \times \stsextHidDim}$, and
$\stsextPredBias \in \reals$ are model parameters.
The complete set of \stsext~extractor parameters is 
\[\xParams = \left\{ \stsextREncParams, \stsextLEncParams,
        \stsextRDecParams, \stsextLDecParams, \stsextHidWeight, \stsextHidBias, \stsextPredWeight, \stsextPredBias, \right\}. \]
In our experiments, we set $\stsextRNNDim=300$, $\stsextHidDim=100$.  Dropout
with drop probability of 0.25 is applied to $\stsextREncHid_i,
\stsextLEncHid_i, \stsextRDecHid_i, \stsextLDecHid_i$, and $\stsextHid_i$ for
all $i \in \{1,\ldots,\docSize\}$.

\subsubsection{Comparison of Sentence Extractors}

All of the extractors rely on running at least one \gru~over the sentence
embeddings to propagate contextual information of neighboring sentences to
$\xHid_i$. However, only the \clext~and \srext~extractors rely on
$\psal_1,\ldots,\psal_{i-1}$ to compute $\psal_i$.  After the \gru~layers are
computed in the \rnnext~and \stsext~extracts, all salience estimates can be
computed in parallel; the \clext~and \srext~must make their salience
predictions sequentially.  Additionally, we hypothesize that including this
dependence is not necessary to achieve good performance, as similar
information about the salience of neighboring sentences will already be
captured in the left and right partial contextual sentence embeddings used to
construct $\xHid_i$.

\subsection{Inference and Summary Generation} \label{sec:inference}

Given a setting of encoder and extractor architectures, and the appropriate
embedding, encoder, and extractor parameters ($\embParams$, $\sentEncParams$,
and $\xParams$ respectively), we can compute the probability of a salience
label sequence given a document, $\model\left(\bsals|\doc;\theta\right)$, as 
\begin{align*}
\model\left(\bsals|\doc;\theta\right) &= \model\Big(\bsals|
\sentEnc\big(\embLayer\left(\sent_1;\embParams\right),\ldots,\embLayer\left(\sent_\docSize;\embParams\right)  ;\sentEncParams\big);\xParams\Big) \\
    & = \prod_{i=1}^\docSize \nnpsal_i^{\nnsal_i} (1 - \nnpsal_i)^{(1- \nnsal_i)}.
\end{align*}
Since even in the autoregressive case discrete extraction decisions are never
fed back into the model\footnote{In the autoregressive extractors, the salience
estimates $\nnpsal_i$ are fed back into the model to compute
$\nnpsal_{i+1},\ldots, \nnpsal_\docSize$. However, since these quantities are
not determined by whether or not the model actually extracts sentence
$\sent_i$, they do not create a combinatorial search space.} we can easily find
the most likely extractive sequence 
\[ \nnpredsals = \argmax_{\nnsals\in\labelSpace} \model\left(\nnsals|\doc;\params\right),
\] 
where 
\[\nnpredsal_i = \begin{cases}
       1 & \textrm{if $\nnpsal_i > 0.5$} \\ 0 & \textrm{otherwise}. 
\end{cases}
\]

This method of inference does not take into account the summary word budget
$\wordbudget$ and the resulting $\predbsals$ may imply an extract summary that
is either too short or too long with respect to the budget constraint. Overly
long summaries are less of a problem because they can always be truncated.
Short summaries, however, will suffer more in evaluation as \rouge-recall
monotonically increases with summary size until reaching the word budget.
Additionally, for fair comparisons between systems, summary lengths should
always be the same size \citep{napoles2011}.

To account for this and obtain a summary of length $\wordbudget$, we first
compute $\nnpsal_1,\ldots,\nnpsal_\docSize$. Then we sort the sentences  in
descending order $\sent_{\pi_1}, \ldots, \sent_{\pi_\docSize}$ where
$\psal_{\pi_i} \ge \psal_{\pi_{i+1}}$. We then construct the extract summary by
selecting the first $k$ sentences such that $\sum_{i=1}^{k-1} \sentSize_{\pi_i}
< \wordbudget \le \sum_{i=1}^{k} \sentSize_{\pi_i}$ (i.e., the first $k$
sentences that  meet or exceed the length budget). When evaluating or
displaying summaries, we show the extracted sentences in the original document
order since this improves the coherence, and we truncate the final sentence
where it exceeds $\wordbudget$.
