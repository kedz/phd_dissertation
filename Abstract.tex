%Abstract Page

\begin{titlepage}
\begin{center}

\vspace*{5\baselineskip}
\textbf{\large Abstract}
\begin{singlespace}
\textbf{Salience Estimation and Faithful Generation}\\
Modeling Methods for Text Summarization and Generation\\
\end{singlespace}

Chris Kedzie
\end{center}

%\begin{flushleft}
%\hspace{10mm}
This thesis is focused on a particular text-to-text generation problem,
automatic summarization, where the goal is to map a large input text to a much
shorter summary text. The research presented aims to both understand and
tame existing machine learning models, hopefully paving the way for more
reliable text-to-text~generation algorithms.  Somewhat against the prevailing
trends, we eschew end-to-end training of an abstractive summarization model,
and instead break down the text summarization problem into its constituent
tasks.  At a high level, we divide these tasks into two categories:
\textit{content selection}, or ``what to say'' and \textit{content
realization}, or ``how to say it'' \citep{mckeown1985}.  Within these
categories we propose models and learning algorithms for the problems of 
\textit{salience estimation} and \textit{faithful generation}.

Salience estimation, that is, determining the importance of a piece of text
relative to some context, falls into a problem of the former category,
determining what should be selected for a summary.  In particular, we
experiment with a variety of popular or novel deep learning models for
salience estimation in a single document summarization setting, and design several ablation experiments to gain some
insight into which input signals are most important for making predictions.
Understanding these signals is critical for designing reliable summarization
models. 

We then consider a more difficult problem of estimating salience in a large
document stream, and propose two alternative approaches using classical
machine learning techniques from both unsupervised clustering and structured
prediction. These models incorporate salience estimates into larger text
extraction algorithms that also consider redundancy and previous extraction
decisions.

Overall, we find that when simple, position based heuristics are available,
as in single document news or research summarization, deep learning models
of salience often exploit them to make predictions, while ignoring the
arguably more important content features of the input. In more demanding
environments, like stream summarization, where heuristics are unreliable,
more semantically relevant features become key to identifying salience 
content.

In part two, content realization, we assume content selection has already been
performed and focus on methods for faithful generation (i.e., ensuring that
output text utterances respect the semantics of the input content). Since they
can generate very fluent and natural text, deep learning-based natural language generation models are
a popular approach to this problem. However, they often omit, misconstrue, or
otherwise generate text that is not semantically correct given the input
content. In this section, we develop a data augmentation and self-training
technique to mitigate this problem. Additionally, we propose a training method
for making deep learning-based natural language generation models capable of following a content plan,
allowing for more control over the output utterances generated by the model.
Under a stress test evaluation protocol, we demonstrate some
empirical limits on several neural natural language generation models' ability to encode and properly
realize a content plan.

Finally, we conclude with some remarks on future directions for abstractive
summarization outside of the end-to-end deep learning paradigm. Our aim here
is to suggest avenues for constructing  abstractive summarization systems
with transparent, controllable, and reliable behavior when it comes to text
understanding, compression, and generation. Our hope is that this thesis inspires more research in this direction, and, ultimately, real tools that are broadly
useful outside of the natural language processing community.
 
%\end{flushleft}
\vspace*{\fill}
\end{titlepage}
