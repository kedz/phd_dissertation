\subsection{A Practical Data-Augmentation Protocol}
\label{sec:daprotos}



\newtcbox{\truttbox}[1][]{colframe=green, colback=green!15, boxrule=0.1mm,
                       nobeforeafter, tcbox raise base, shrink tight, extrude
                       by=0.32mm, #1}
\newtcbox{\mrdistbox}[1][]{colframe=orange, colback=orange!15, boxrule=0.1mm,
                       nobeforeafter, tcbox raise base, shrink tight, extrude
                       by=0.32mm, #1}

\newtcbox{\ninjbox}[1][]{colframe=purple, colback=purple!15, boxrule=0.1mm,
                       nobeforeafter, tcbox raise base, shrink tight, extrude
                       by=0.32mm, #1}
\newtcbox{\correctbox}[1][]{colframe=blue, colback=blue!15, boxrule=0.1mm,
                       nobeforeafter, tcbox raise base, shrink tight, extrude
                       by=0.32mm, #1}
\newtcbox{\filterbox}[1][]{colframe=cyan, colback=cyan!15, boxrule=0.1mm,
                       nobeforeafter, tcbox raise base, shrink tight, extrude
                       by=0.32mm, #1}

\newtcbox{\returnbox}[1][]{colframe=violet, colback=violet!15, boxrule=0.1mm,
                       nobeforeafter, tcbox raise base, shrink tight, extrude
                       by=0.32mm, #1}

\begin{figure}[t]
\begin{singlespace}
\begin{algorithm}[H]
\truttbox{$\gen_0 \gets \operatorname{Train}_\utttoks\left(\corpus\right)$} \\
\truttbox{$\dmodel \gets \operatorname{Train}_\mr\left(\corpus\right)$} \\
$\augdata \gets \left\{ \right\}$\\
\While{$\setsize{\augdata} < \numSamples$}{
\mrdistbox{$\samplmr \sim \pdaMrDist$}\\
\ninjbox{$\pdaCandUtts\gets \left\{ \pdaCandUtt \sim 
            \gen_0\left(\cdot|\ls(\samplmr), \pdaCandEps\right) 
            \quad \forall i : i \in \{1,\ldots, k\} \right\} $}\label{lst:pdacand}\\
\ninjbox{$\predutttoks \gets
    \argmax_{\pdaCandUtt \in \pdaCandUtts} 
    \frac{\log \gen_0\left(\pdaCandUtt|\ls(\samplmr),\pdaCandEps \right)}{\setsize{\pdaCandUtt}}$}\label{lst:pdacandselect}\\
\correctbox{$\pdaPredMr \gets \dmodel\left(\predutttoks\right)$}\\
\If{\filterbox{$\lnot \operatorname{Filter}\left(\pdaPredMr, \predutttoks\right)$}}{
    $\augdata \gets \augdata \cup \left\{ \left(\pdaPredMr, \predutttoks\right) \right\}$
}
~\\
}
\returnbox{$\gen_1 \gets \operatorname{Train}_\utttoks(\corpus \cup \augdata)$}\\
\KwResult{$\gen_1$}
    \caption{Data Augmentation with Noise-Injection Sampling and Self-Training }
%\label{alg:idealda}
\end{algorithm}
\end{singlespace}
\caption{A comparison of greedy decoding, ancestral sampling, and noise injection sampling.}
\label{fig:practda}
\end{figure}


Because of its ability to generate semantically divergent and novel
outputs while maintaining fluency, we adopt this noise-injection
sampling as our method of sampling utterances, $\daUttDist$, for 
data-augmentation. We show our actual data-augmentation scheme in
\autoref{fig:practda} and now walk through some of the implementation details.

\paragraph{\truttbox{Train base generator $\gen_0$ and \meaningrepresentation~parser $\dmodel$.}}
The algorithm begins by training the base generator, i.e. na{\"i}ve 
\sequencetosequence~model, and \meaningrepresentation~parser $\dmodel$. 
Both models are trained on the same data, with the only real change to the 
$\operatorname{Train}$ sub-routine being which part of a training example
is the ouput and which is the input. Alternatively, $\dmodel$ can 
also be implemented using regular-expression-based rules. We defer detailed 
explanation of $\dmodel$ until the experiments; it suffices to understand
$\dmodel$ as a mapping from utterances to \meaningrepresentations.




\paragraph{\mrdistbox{Sampling a \meaningrepresentation, $\samplmr$}}
We use the meaning representation described in \autoref{sec:ideal} to implement
 the distribution $\corpus_\mr^{-1}$. 

\paragraph{\ninjbox{Generating a novel utterance with noise-injection 
    sampling.}}
    In \autoref{lst:pdacand} we take $200$ noise-injection samples to construct
    a candidate set of utterances, $\pdaCandUtts_{200}$. 
    From these we use only the top 20 utterances, $\pdaCandUtts_{20}$ by average log-likelihood,
    $\frac{\log\gen(\pdaCandUtt|\ls(\samplmr), \boldsymbol{\epsilon}^{(i)})}{\setsize{\pdaCandUtt}}$ (\autoref{lst:pdacandselect}).
%    From $\pdaCandUtts$ we
%    select as our noise-injection sample, the utterance $\predutttoks$
%    which has the highest average token log-likelihood (\autoref{lst:pdacandselect}). 
    We do this selection step so as to be extra cautious and avoid adding
    any potentially disfluent utterances to $\augdata$.
    
    \paragraph{\correctbox{Predict \meaningrepresentation~$\pdaPredMr$ from
    $\predutttoks$.}} Because the noise-injection sampling produces highly
semanticly divergent utterances, it is unlikely that $\denotes{\predutttoks } = \samplmr$. Instead we use the \meaningrepresentation~parser, $\dmodel$, to
recover the most likely \meaningrepresentation, $\pdaPredMr = \dmodel\left(\predutttoks\right)$.

\paragraph{\filterbox{Check synthetic datapoint $(\pdaPredMr,\predutttoks)$.}}
We do one last quality check on the synthetic example $(\pdaPredMr,\predutttoks)$ before adding it to the augmented dataset, $\augdata$. We make sure that
the probability of $\pdaPredMr$ under $\dmodel$ is above 0.5 when using
a model-based \meaningrepresentation~parser. When using a rule-based
\meaningrepresentation~parser, we check to make sure that there are no
repeated \attributevalue-pairs in $\predutttoks$, e.g., ``Aromi is a 
coffee shop and it is a coffee shop.'' Meaning representation/utterances that
have been previously generated are also discarded. If the \meaningrepresentation/utterance
pair passes these final quality checks, we add it to $\augdata$.

\paragraph{\returnbox{Train an augmented generator $\auggen$ on $\corpus \cup \augdata$.}} After generating a synthetic dataset, $\augdata$, we train a new 
generation model, $\auggen$, on the union of the original training data
and the newly generated synthetic data. We refer to this model as 
the augmented generator and, as we will show empirically, the augmented
generator is more faithful than the base generator, $\gen_0$.
We call this process self-training because $\auggen$ and $\gen_0$ share 
the same architecture, and $\auggen$ is trained on data produced by $\gen_0$.


