
\subsection{Generation Models}
We examine the effects of linearization strategy and data augmentation
on biGRU (see \autoref{sec:nlggru})
and transformer (see \autoref{sec:nlgtf}) based \sequencetosequence~models.
See \autoref{tab:nlghpsspace} for the set of hyper-parameters that we explored for
each model and \autoref{tab:gruparams} and \autoref{tab:tfparams} for the winning hyper-parameter settings for the biGRU and transformer models respectively.
Hyper-parameters were found using grid-search, selecting the model
with best validation \textsc{Bleu} score. We performed a separate
grid-search for each architecture-linearization strategy pairing in case
there was no one best hyper-parameter setting.
We used a batch size of 128 for all biGRU and
Transformer models and trained for at most 700 epochs.



\begin{table}
    \centering
    \begin{tabular}{cp{4.25cm}p{4.25cm}}
            \toprule
            Hyperparameter & biGRU & Transformer\\
            \midrule
            Layers & $1$, $2$ & $1$, $2$\\
            Label Smoothing & $0.0$, $0.1$ & $0.0$, $0.1$\\
            Weight Decay & $0$, $10^-5$ & --- \\
Optimizer/Learning Rate & Adam/$10^{-3}$, Adam/$10^{-4}$, Adam/$10^{-5}$,
            SGD/$0.5$, SGD/$0.25$, SGD/$0.1$ & Adam with the learning
            rate schedule from \cite{rush2018} (factor=1, warmup=8000)\\
        Tied Decoder Embeddings & tied, untied & tied, untied\\
        Attention & Bahdanau, General & ---\\
        \bottomrule
\end{tabular}
\caption{Hyperparameter search space for biGRU and transformer architectures.}
\label{tab:nlghpsspace}
\end{table}


\begin{table}
\small
\center
\begin{tabular}{clccc ccc cc ccccc}
\toprule
&Model & L & LS & WD & Optim. & LR & Attn & $\embDim$ & $\hidDim$ & $\encDim$ & $\decDim$ & Drop. & Params \\
\midrule
    \parbox[t]{2mm}{\multirow{6}{*}{\rotatebox[origin=c]{90}{E2E}}} 
 & \textsc{Rnd} & 2 & 0.1 & $10^{-5}$ & Adam & $10^{-5}$ &  Bahd. & 512 & 512 & 1024 & 512 & 0.1 & 14,820,419  \\
 & \textsc{Fp} & 2 & 0.1 & $10^{-5}$ & SGD & $0.1$ &  Bahd. & 512 & 512 & 1024 & 512 & 0.1 & 14,820,003 \\ 
 & \textsc{If} & 2 & 0.1 & $0.0$ & SGD & $0.5$ &  Gen. & 512 & 512 & 1024 & 512  & 0.1 & 14,557,763 \\
 & \textsc{If+p} & 2 & 0.1 & $0.0$ & SGD & $0.5$ &  Gen. &512 & 512 & 1024 & 512  & 0.1 &14,557,763 \\
 & \textsc{At} & 2 & 0.1 & $10^{-5}$ & Adam & $10^{-5}$ &  Bahd. & 512 & 512 & 1024 & 512  & 0.1 & 14,820,419  \\
 & \textsc{At+p} & 2 & 0.1 & $10^{-5}$ & Adam & $10^{-5}$ &  Bahd. & 512 & 512 & 1024 & 512  & 0.1 & 14,820,419  \\
\midrule
    \parbox[t]{2mm}{\multirow{6}{*}{\rotatebox[origin=c]{90}{ViGGO}}} 
 & \textsc{Rnd} & 2 & 0.1 & $10^{-5}$ & SGD & $0.25$ &  Gen. & 512 & 512 & 1024 & 512 & 0.1 & 14,274,865 \\
 & \textsc{Fp} & 1 & 0.1 & $10^{-5}$ & Adam & $10^{-5}$ &  Bahd. & 512 & 512 & 1024 & 512 & 0.1 & 7,718,193 \\ 
 & \textsc{If} & 1 & 0.0 & $0.0$ & SGD & $0.5$ &  Bahd. &  512 & 512 & 1024 & 512  & 0.1 & 7,712,049 \\ 
 & \textsc{If+} & 1 & 0.0 & $0.0$ & SGD & $0.5$ &  Bahd. & 512 & 512 & 1024 & 512  & 0.1 & 7,712,049 \\ 
 & \textsc{At} & 2 & 0.1 & $0.0$ & Adam & $10^{-5}$ &  Bahd. &  512 & 512 & 1024 & 512  & 0.1 & 14,537,521 \\ 
 & \textsc{At+p} & 2 & 0.1 & $0.0$ & Adam & $10^{-5}$ &  Bahd. &  512 & 512 & 1024 & 512  & 0.1 & 14,537,521 \\ 
\bottomrule
\end{tabular}

\caption{Winning hyperparameter settings for biGRU models. L, LS, and WD 
indicate number of layers, label smoothing, and weight decay respectively. 
All models use untied embeddings. Drop. indicates dropout (i.e. drop probability).}
\label{tab:gruparams}
\end{table}

\begin{table}
\center
\begin{tabular}{cl cccc ccccc}
\toprule
&Model & Layers & LS & Emb. & Params & $\embDim$ & $\hidDim$ & $\encDim$ & $\decDim$ & Dropout\\
\midrule
    \parbox[t]{2mm}{\multirow{6}{*}{\rotatebox[origin=c]{90}{E2E}}} 
 & \textsc{Rnd} & 1 & 0.1 & tied & 7,966,787 & 512 & 2048 & 512 & 512 & 0.1\\
 & \textsc{Fp} & 1 & 0.1 & tied & 7,970,371 & 512 & 2048 & 512 & 512 & 0.1\\
 & \textsc{If} & 1 & 0.1 & untied & 8,525,379 & 512 & 2048 & 512 & 512 & 0.1 \\
 & \textsc{If+p} & 1 & 0.1 & untied & 8,525,379 & 512 & 2048 & 512 & 512 & 0.1 \\
 & \textsc{At} & 2 & 0.1 & untied & 15,881,795 & 512 & 2048 & 512 & 512 & 0.1 \\
 & \textsc{At+p} & 2 & 0.1 & untied & 15,881,795 & 512 & 2048 & 512 & 512 & 0.1 \\
\midrule
    \parbox[t]{2mm}{\multirow{6}{*}{\rotatebox[origin=c]{90}{ViGGO}}} 
 & \textsc{Rnd} & 2 & 0.0 & untied & 15,598,897 & 512 & 2048 & 512 & 512 & 0.1\\
 & \textsc{Fp} & 2 & 0.1 & untied & 15,605,041 & 512 & 2048 & 512 & 512 & 0.1\\
 & \textsc{If} & 2 & 0.1 & untied & 15,598,897 & 512 & 2048 & 512 & 512 & 0.1 \\
 & \textsc{If+p} & 2 & 0.1 & untied & 15,598,897 & 512 & 2048 & 512 & 512 & 0.1\\
 & \textsc{At} & 2 & 0.1 & untied & 15,598,897 & 512 & 2048 & 512 & 512 & 0.1 \\
 & \textsc{At+p} & 2 & 0.1 & untied & 15,598,897 & 512 & 2048 & 512 & 512 & 0.1 \\
\bottomrule
\end{tabular}

\caption{Winning hyperparameter settings for transformer models 
(trained from scratch). L and  LS indicate number of layers and label smoothing respectively. Drop. indicates dropout (i.e. drop probability).
All models trained with the Adam optimizir with the learning
            rate schedule from \cite{rush2018} (factor=1, warmup=8000).
}
\label{tab:tfparams}
\end{table}

%lowest validation set cross-entropy. 
%
%\paragraph{Transformer}
%We used the Transformer S2S as implemented in
%\href{https://pytorch.org/}{PyTorch}.
%The input embedding dimension is 512 and inner hidden layer size is 2048.
%We used 8 heads in all multi-head attention layers.
%We used Sinusoidal position embeddings following those described in
%\citet{rush2018annotated}. Additionally, we used Adam with the learning
%rate schedule provided in that work (factor=1, warmup=8000).
%Dropout was set to 0.1.
%
%
\newcommand{\utt}{\ensuremath{\mathbf{y}}}
\newcommand{\uttVocab}{\ensuremath{\mathcal{W}}}
\newcommand{\da}{\ensuremath{a}}
\newcommand{\inseq}{\mathbf{x}}
\newcommand{\Attrs}{\ensuremath{\mathcal{V}}}
\newcommand{\inSize}{m}
\newcommand{\outSize}{n}

\newcommand{\mmhAttn}{\operatorname{maskedMHAttn}}
\newcommand{\mhAttn}{\operatorname{MHAttn}}

\newcommand{\mrEmb}{\mathbf{W}}
\newcommand{\uttEmb}{\mathbf{V}}
\newcommand{\decInput}{\mathbf{G}}
\newcommand{\decInputi}{\mathbf{g}_i}


\newcommand{\tfeA}{\boldsymbol{\check{\encInput}}^{(i)}}
\newcommand{\tfeB}{\boldsymbol{\bar{\encInput}}^{(i)}}
\newcommand{\tfeC}{\boldsymbol{\hat{\encInput}}^{(i)}}
\newcommand{\tfeD}{\boldsymbol{\dot{\encInput}}^{(i)}}
\newcommand{\tfeE}{\boldsymbol{\ddot{\encInput}}^{(i)}}

\newcommand{\tfdA}{\boldsymbol{\check{\decInput}}^{(i)}}
\newcommand{\tfdB}{\boldsymbol{\bar{\decInput}}^{(i)}}
\newcommand{\tfdC}{\boldsymbol{\hat{\decInput}}^{(i)}}
\newcommand{\tfdD}{\boldsymbol{\grave{\decInput}}^{(i)}}
\newcommand{\tfdE}{\boldsymbol{\tilde{\decInput}}^{(i)}}
\newcommand{\tfdF}{\boldsymbol{\acute{\decInput}}^{(i)}}
\newcommand{\tfdG}{\boldsymbol{\dot{\decInput}}^{(i)}}
\newcommand{\tfdH}{\boldsymbol{\ddot{\decInput}}^{(i)}}


%\subsection{Transformer Model Definition}
%
%Each Transformer layer is divided into blocks which each have three
%parts, (i) layer norm, (ii) feed-forward/attention, and  (iii) skip-connection.
%We first define the components used in the transformer blocks before
%describing the overall S2S transformer. 
%Starting with layer norm \cite{ba2016}, let $\encInput \in \reals^{m\times n}$, then we have
%$\layerNorm : \reals^{m \times n} \rightarrow \reals^{m \times n}$,
%\[\layerNorm(\encInput; \lnweightv, \lnbias) = \lnweight \odot (\encInput - \boldsymbol{\mu}) \odot \Lambda + \mathbf{b} \]
%
%where $\lnweightv, \lnbias \in \reals^n$ are learned parameters, $\odot$ is the elementwise product, $\lnweight = \left[\lnweightv,\ldots,\lnweightv\right] \in \reals^{m\times n}$ is a tiling of the parameter vector, $\lnweightv$, $m$ times, and  $\boldsymbol{\mu}, \boldsymbol{\Lambda} \in \reals^{m\times n}$ are
%defined elementwise as
%\[\boldsymbol{\mu}_{i,j} = \frac{1}{n} \sum_{k=1}^n \encInput_{i,k}\]
%and 
%\[\boldsymbol{\Lambda}_{i,j} = \left(
%    \sqrt{ \frac{1}{n-1} \sum_{k=1}^n \left( 
%\encInput_{i,k} - \boldsymbol{\mu}_{i,j} \right)^2  + \epsilon}\right)^{-1}\]
%respectively. The $\epsilon$ term is a small constant for numerical stability,
%set to $10^{-5}$.
%
%The inplace feed-forward layer, $\feedforward$, is a simple single-layer perceptron
%with $\relu$ activation 
%($\relu(\encInput) = \max\left(\zeroEmb, \encInput\right)$) \cite{nair2010}, applied to each row of an $m \times n$ input matrix, i.e. a sequence of $m$ objects
%with $n$ features,\\
%
%
%\noindent $\feedforward\left(\encInput;\weight{i},\weight{j},\bias{i},\bias{j}\right) =$
%\[  \relu\left(\encInput\weight{i} + \bias{i}\right)\weight{j} + \bias{j}     \]
%where $\weight{i} \in \reals^{\embDim \times \hidDim}$, $\bias{i} \in \reals^{\hidDim}$,
%$\weight{j} \in \reals^{\hidDim \times \embDim}$, $\bias{j} \in \reals^{\embDim}$ are learned parameters and 
%matrix-vector additions (i.e. $\mathbf{X} + \mathbf{b}$) are broadcast across
%the matrix rows.
%
%
%The final component to be defined is the multi-head attention, $\MultiAttn$ which is defined
%as\\
%
%\noindent  $\MultiAttn(\Query, \Key; \weight{a_1}, \weight{a_2}) =$
%\[ \left[ \begin{array}{c} 
%and $\bias{o} \in \reals^{\embDim}$ are learned parameters. 
%
%
%%We used the Transformer S2S as implemented in 
%%\href{https://pytorch.org/}{PyTorch}.
%The input embedding dimension is $\embDim= 512$ and inner hidden layer size 
%is $\hidDim=2048$. The encoder and decoder have separate parameters.
%We used $H=8$ heads in all multi-head attention layers. 
% We used Adam with the learning
%rate schedule provided in  \citet{rush2018} (factor=1, warmup=8000).
%Dropout was set to 0.1 was applied to input embeddings and each skip 
%connection (i.e. the third line in each block definition). As a 
%hyperparameter, we optionally tie the decoder input and output embeddings,
%i.e. $\decEmbs = \weight{o}$.

Additionally, we
fine-tune BART \cite{lewis2019bart}, a large pretrained transformer based 
\sequencetosequence~model. We stop fine-tuning after validation set cross-entropy stops decreasing.
We use the same settings as the fine-tuning for the CNN-DailyMail
summarization task,
although we modify the maximum number of updates to be roughly to be
equivalent to 10 epochs on the training set when using a 500 token batch
size, since
the number of updates effects
the learning rate scheduler. We selected the model iterate with
lowest validation set cross-entropy.

While BART is unlikely to have seen any linearized MR in its pretraining
data, its use of sub-word encoding  allows it to encode
arbitrary strings. Rather than extending it's encoder input vocabulary to
add the MR tokens, we simply format the input MR as a string
(in the correpsonding linearization order), e.g. ``inform rating=good name=NAME platforms=PC platforms=Xbox''.


