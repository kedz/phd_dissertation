\subsection{Alignment Training Linearization}
%As was mentioned in \autoref{mrtproblemdef}, there are many choices of
%\linearizationstrategy~$\ls$ when modeling \surfacerealization~with
%a \sequencetosequence~archictecture. One particular \linearizationstrategy,
%which we call the \alignmenttraining~linearization, yields a controllable \sequencetosequence~model which is capable of following an utterance plan indicating
% the \surfacerealization~order of a \meaningrepresentation's \attributevalues.
%
%
Unlike the {\color{red}arbitrary linearization} used in \autoref{somewhere}, \alignmenttraining~linearization~is not soley a function of $\mr$, but is determined by both $\mr$ and a reference utterance $\utttoks$. Given a $\left(\mr, \utttoks\right)$ pair, the \alignmenttraining~linearization finds a linearization
$\ls$ such that the order of the \attributevalues~in $\ls(\mr)$ corresponds
to the order in which they are realized in $\utttoks$. %Formally, this means that 
%for any $\left(i^{(\ls_k)},j^{(\ls_k)}\right), \left(i^{(\ls_{k+1})},j^{(\ls_{k+1})}\right) \in \denotationset_{\mr,\utttoks}$, we have $j^{(\ls_k)} < i^{(\ls_{k+1})}$.


\autoref{fig:atexamples} shows some examples of the 
\alignmenttraining~linearization, including some special cases. When
linearizing list-valued attributes, for instance, we treat them as distinct
\attributevalue~pairs. Occasionally, we encounter repeated \attributevalues~in the training set, and in that case we include extra \attributevalue~pairs
in the correpsonding location in the linearization. We ignore any instances
of  ungrounded information, as in example \autoref{} where (such and such)
has no explicit representation.  

%\subsubsection{Alignment Training Implementation}

%\begin{algorithm}[t]
    \caption{Alignment Training Linearization Algorithm}
    \label{alg:at}
    \DontPrintSemicolon
    \KwData{\utterance~$\utttoks \in \outSpace$, \attributevalue~tagger $\tagger$}  
    Tag the utterance, $\mathbf{t} \gets \tagger(\utttoks)$.  \label{alg:attag} \\
Split $\mrtags$ into the minimal number of subsequences with the same distinct
tag value, i.e. $\mrtags = \left[\left[ \mrtag_1, \ldots mrtag_j \right] \right]$\\
    Segment $\mathbf{t}$ into $k$ spans by splitting $\mathbf{t}$ on tags mapped to no \attributevalue~(i.e. $t_i = \emptyset$): $\boldsymbol{\hat{t}} = \left[\ldots,\left[t_{{i_k}},t_{i_{k}+1},\ldots, t_{j_k}\right],\ldots\right]$ \\
    Segment each of the $k$ sub-segments into maximal length contiguous
    spans sharing the same tag.
\end{algorithm}




%\hyperref[alg:at]{Algorithm \ref{alg:at}} shows the alignment training algorithm for a training example $(\mr,\utttoks)$. 

In \autoref{fig:at} we show the steps of our procedure for obtaining
the alignment training linearization, given a reference utterance $\utttoks$.
%the intermediate outputs of each line in \hyperref[alg:at]{Algorithm \ref{alg:at}}. %In practice, there are two distinct and important issues when implementing the 
%\alignmenttraining~linearization. 
The first step is 
to tag the utterance tokens 
$\utttoks = \left[\utttok_1,\ldots,\utttok_\uttSize\right]$ 
with a corresponding tag sequence 
$\mrtags=\left[\mrtag_1,\ldots, \mrtag_\uttSize\right]$ where each 
tag $\mrtag_i$ is equal to an \attributevalue~$\mrtok_j \in \mr$ or 
the null tag $\nulltag$. We assume that we have access to such a tagger
$\tagger : \outSpace \rightarrow \inSpace$ (see \autoref{sec:tagger}
for implementation details). After producing the tag sequence $\mrtags^{(1)} = \tagger\left(\utttoks\right)$ 
%(\hyperref[alg:attag]{Lines \ref{alg:attag}} in \hyperref[alg:at]{Algorithm \ref{alg:at}} and \hyperref[fig:at]{\autoref{fig:at}.b}),
(\hyperref[fig:at]{\autoref{fig:at}b}),
we then group contiguous sequences of tags sharing the same tag value, discarding any null tag sequences to obtain the sequence of subsequences 
$\mrtags^{(2)} = \left[\mrtags^{(1)}_{i_1:j_1},\ldots,\mrtags^{(1)}_{i_\mrSize:j_\mrSize}, \right]$ (\hyperref[fig:at]{\autoref{fig:at}c}).
Finally, $\mrtoks$ is constructed by 
by prepending the \dialogueact~$\mrtok_0$ of $\mr$ to the ordered sequence 
of \attributevalue~pairs $\mrtok_1,\ldots,\mrtok_\mrSize$ implied by $\mrtags^{(1)}_{i_1},\ldots,\mrtags^{(1)}_{i_\mrSize}$ (\hyperref[fig:at]{\autoref{fig:at}d}). 



%~\\~\\
%we then segment $\mrtags^{(1)}$ into contiguous sequences of tags containing no null tags, $\nulltag$, to obtain the sequence of subsequences
%$\mrtags^{(2)}$ (\hyperref[fig:at]{\autoref{fig:at}c}). Each subsequence of
%$\mrtags^{(2)}$ is further split into the longest contiguous tag sequences 
%such that all 
%tags of a segment are equal to the same \attributevalue~pair(\hyperref[fig:at]{\autoref{fig:at}d}). 
%Finally, $\mrtoks$ is constructed by 
%by prepending the \dialogueact~$\mrtok_0$ of $\mr$ to the ordered sequence 
%of \attributevalue~pairs $???$ implied by $\mrtoks^{(3)}$ (\hyperref[fig:at]{\autoref{fig:at}e}). 
%
\tikzset{mynode/.style={anchor=center, minimum height=1.5em, 
      text height=1.5ex, text depth=.25ex, align=center}}

      \begin{figure}[p]
    \fbox{\begin{minipage}{\textwidth}
            ~\\
        \center
        \scalebox{0.75}{
    \begin{tikzpicture}
        \def\titlex{-3.0}
        \def\colwidth{1.25}
        \def\utttitleheight{2}
        \def\uttheightA{0.0}
        \def\uttheightB{-1.0}
        \def\tagtitleheight{-3.0}
        \def\tagheightA{-5.0}
        \def\tagheightB{-6.0}
        \def\nullsegtitleheight{-12.0}
        \def\nullsegheightA{-14.0}
        \def\mrsegtitleheight{-12.0}
        \def\mrsegheightA{-14.0}
        \def\attitleheight{-16.0}
        \def\atheightA{-18.0}
        \def\atheightB{-19.0}



        \def\city{{area=city centre}};
        \def\coffee{{eat\_type=coffee shop}};
       % \def\nulltag{{$\emptyset$}};
        \def\name{{name=Aromi}};
        \def\words{For, coffee, in, the, centre, of, the, city, {,}, try, 
                   Aromi, .};
        \def\tags{\coffee, \coffee, \city, \city, \city, \city, \city, \city,
                     \nulltag, \nulltag, \name, \nulltag};

        %%% Input Utterance
        \node[mynode,anchor=west] at (\titlex*\colwidth,\utttitleheight) 
            {\Large (a) Input Utterance};

        \foreach \w [count=\wi from 1] in \words {
            \node[mynode] at (\wi*\colwidth,\uttheightA) 
                {$\utttok_{\wi}\ifthenelse{\wi = 12}{}{,}$};
            \node[mynode] at (\wi*\colwidth,\uttheightB) {\w};
        }
        \node[mynode] at (0*\colwidth,\uttheightA) {$\utttoks = \Bigg[$};
        \node[mynode] at (12*\colwidth+0.5,\uttheightA) {$\Bigg]$};


        %%% Tagged Utterance
        \node[mynode,anchor=west] at (\titlex*\colwidth,\tagtitleheight) 
            {\Large (b) Tagged Utterance};
        \foreach \t [count=\ti from 1] in \tags {
            \node[mynode] at (\ti*\colwidth,\tagheightA) 
                {$\mrtag_{\ti}\ifthenelse{\ti = 12}{}{,}$};
            \node[mynode,rotate=90,anchor=east] at 
                (\ti*\colwidth,\tagheightB) {\t};
        }
        \node[mynode] at (0*\colwidth,\tagheightA) {$\mrtags^{(1)} = \Bigg[$};
        \node[mynode] at (12*\colwidth+0.5,\tagheightA) {$\Bigg]$};


%        %%% Null Segmented Tags
%        \node[mynode,anchor=west] at (\titlex*\colwidth,\nullsegtitleheight) 
%            {\Large (c) Null Segmented Tags};
%        \node[mynode] at (0*\colwidth,\nullsegheightA) 
%            {$\mrtags^{(2)} = \Bigg[$};
%        \node[mynode] at (12*\colwidth+0.5,\nullsegheightA) {$\Bigg]$};        
%        \foreach \tprime in {1,...,8} {
%            \node[mynode] at (\tprime*\colwidth,\nullsegheightA) 
%            {$\mrtag_{\tprime}\ifthenelse{\tprime = 8}{}{,}$};
%        }
%        \foreach \tprime in {11} {
%            \node[mynode] at (\tprime*\colwidth,\nullsegheightA) 
%                {$\mrtag_{\tprime}$};
%        }
%
%        \node[mynode] at (\colwidth-0.50,\nullsegheightA) {$\bigg[$};
%        \node[mynode] at (8*\colwidth+0.50,\nullsegheightA) {$\bigg],$};
%        \node[mynode] at (11*\colwidth-0.50,\nullsegheightA) {$\bigg[$};
%        \node[mynode] at (11*\colwidth+0.50,\nullsegheightA) {$\bigg]$};


        %%% MR Segmented Tags
        \node[mynode,anchor=west] at (\titlex*\colwidth,\mrsegtitleheight) 
            {\Large (c) MR Segmented Tags};
        \node[mynode] at (0*\colwidth,\mrsegheightA) 
            {$\mrtags^{(2)} = \Bigg[$};
        \node[mynode] at (12*\colwidth+0.5,\mrsegheightA) {$\Bigg]$};        
        \foreach \tprime in {1,...,2} {
            \node[mynode] at (\tprime*\colwidth,\mrsegheightA) 
            {$\mrtag_{\tprime}\ifthenelse{\tprime = 2}{}{,}$};
        }
        \foreach \tprime in {3,...,8} {
            \node[mynode] at (\tprime*\colwidth,\mrsegheightA) 
            {$\mrtag_{\tprime}\ifthenelse{\tprime = 8}{}{,}$};
        }
        \foreach \tprime in {11} {
            \node[mynode] at (\tprime*\colwidth,\mrsegheightA) 
                {$\mrtag_{\tprime}$};
        }

        \node[mynode] at (\colwidth-0.50,\mrsegheightA) {$\bigg[$};
        \node[mynode,anchor=center] at (2*\colwidth+0.65,\mrsegheightA) {$\Bigg]_1, \Bigg[$};
    \node[mynode] at (8*\colwidth+0.50,\mrsegheightA) {$\bigg]_2,$};
        \node[mynode] at (11*\colwidth-0.50,\mrsegheightA) {$\bigg[$};
        \node[mynode] at (11*\colwidth+0.50,\mrsegheightA) {$\bigg]_3$};


        %%% Alignment Training Linearization
        \node[mynode,anchor=west] at (\titlex*\colwidth,\attitleheight) 
            {\Large (d) Alignment Training Linearization};
        \node[mynode,anchor=east] at (-1.5*\colwidth-0.5,\atheightA) 
            {$\mrtoks = \Bigg[$};
        \node[mynode] at (-1.5*\colwidth,\atheightA) {$\mrtok_0,$};
        \node[mynode] at (1.5*\colwidth,\atheightA) {$\mrtok_1,$};
        \node[mynode] at (5.5*\colwidth,\atheightA) {$\mrtok_2,$};
        \node[mynode] at (11*\colwidth,\atheightA) {$\mrtok_3$};
        \node[mynode] at (11*\colwidth+0.5,\atheightA) {$\Bigg]$};
%
        \node[mynode] at (-1.5*\colwidth,\atheightB) {inform,};
        \node[mynode] at (1.5*\colwidth,\atheightB) {eat\_type=coffee shop,};
        \node[mynode] at (5.5*\colwidth,\atheightB) {area=city centre,};
        \node[mynode] at (11*\colwidth,\atheightB) {name=Aromi};



    \end{tikzpicture}
    }
    ~\\
    \caption{Example steps of the \alignmenttraining~linearization algorithm
        for producing a linearized \meaningrepresentation~$\mrtoks$.}
        \label{fig:at}
\end{minipage}}
\end{figure}




At test time, the generation model is only
presented with a \meaningrepresentation~$\mr$ and we don't have a reference 
utterance $\utttoks$ with which to apply the alignment training 
linearization. In this case,
we can use an utterance planning model $\planner : \mrspace \rightarrow \inSpace$ to map a \meaningrepresentation~$\mr$ to a
linear sequence $\mrtoks$. Alternatively, we can use the test set reference
to obtain the alignment training linearization; this represents an unrealistically
optimistic case where the model has clairvoyant known of the discourse
ordering preferred by a human.
%
%a human reference utterance. In the latter case, this represents an
%unrealistic case where the model has clairvoyant knowledge of the 
%discourse ordering of the desired utterance but is also representative
%of discourse orderings prefered by humans. 
In either case, we refer to 
a linearization $\mrtoks$ obtained either from $\planner$ or a human reference
as an \utteranceplan~since a generation model trained with
\alignmenttraining~linearizations will attempt to follow it during 
the generation of the utterance.


