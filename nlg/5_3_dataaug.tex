\subsection{Phrase-based Data Augmentation}

\begin{figure}
    \centering


        \begin{tikzpicture}

            \def\th{5mm};
            \def\td{2mm};
    \node[text height=\th,text depth=\td] (aromi) at (-5,-4) {Aromi};
            \node[text height=\th,text depth=\td] (is) at (-1.5,-4) {is};
            \node[text height=\th,text depth=\td] (not) at (0.0,-4) {not};

            \node[text height=\th,text depth=\td] (a) at (1.5,-4) {a};
            \node[text height=\th,text depth=\td] (ff) at (4,-4) {family-friendly};
            \node[text height=\th,text depth=\td] (est) at (6.5,-4) {establishment};


            \node (root) at (-2.5,0) {S};
            \node[draw,circle,font=\small,inner sep=0] at ($(root)+(-0.5,0.5)$) {4};
            \node (rootNP) at (-5,-1) {NP};
            \node[draw,circle,font=\small,inner sep=0] at ($(rootNP)+(-0.5,0.5)$) {3};
            \node (rootNPNNP) at (-5,-2) {NNP};
            \node (r1c2) at (0.0,-1) {VP};
            \node[draw,circle,font=\small,inner sep=0] at ($(r1c2)+(0.5,0.5)$) {2};
                \node (r2c3) at (4,-2) {NP};
            \node[draw,circle,font=\small,inner sep=0] at ($(r2c3)+(0.5,0.5)$) {1};
                \node (det) at (1.5,-3) {DET};
                \node (jj) at (4,-3) {JJ};
                \node (nn) at (6.5,-3) {NN};
                \node (r2c1) at (-1.5,-2) {VB};

                \node (r2c2) at (0.0,-2) {RB};
                \draw[-] (r1c2) -- (r2c1);
                \draw[-] (r1c2) -- (r2c2);
                \draw[-] (r1c2) -- (r2c3);


                \draw[-] (r2c1) -- (is);

            \draw[-] (rootNPNNP) -- (aromi);
            \draw[-] (root.south west) -- (rootNP.north east);
            \draw[-] (rootNP) -- (rootNPNNP);
            \draw[-] (root.south east) -- (r1c2.north west);
                \draw[-] (r2c3) -- (det);
                \draw[-] (r2c3) -- (jj);
                \draw[-] (r2c3) -- (nn);
                \draw[-] (det) -- (a);
                \draw[-] (jj) -- (ff);
                \draw[-] (nn) -- (est);
                \draw[-] (r2c2) -- (not);
        

      %      \node[anchor=north west,align=left,inner sep=0,outer sep=0,text height=0mm,text width=11cm] at (0.7,0.5) {
                    %\begin{enumerate}
                    %    \item Parse training examples.
                %\item<3-> Create additional training examples from constituent phrases.
                %\end{enumerate}};

        \end{tikzpicture}

        ~\\~\\

        \begin{tabular}{ccc}
            \toprule
            &         \MeaningRepresentation~($\mr$) & Utterance ($\utttoks$) \\
            \midrule
            \raisebox{0.5pt}{\textcircled{\raisebox{-0.9pt} {1}}} &  
        $\left[\!\!\left[ \begin{array}{l} 
            \textsc{Inform}\\ 
            \AV{family\_friendly}{yes} 
        \end{array}   \right]\!\!\right]$  & 
            $\left[\textit{<<s>>}, \textit{a}, \textit{family-friendly}, \textit{establishment}, \textit{<<e>>}\right]$\\
        \raisebox{0.5pt}{\textcircled{\raisebox{-0.9pt} {2}}} &
        $\left[\!\!\left[ \begin{array}{l} 
            \textsc{Inform}\\ 
            \AV{family\_friendly}{no} 
        \end{array}\right]\!\!\right]$  & 
            $\left[\textit{<<s>>}, \textit{is}, \textit{not}, \textit{a}, 
            \textit{family-friendly}, \textit{establishment}, \textit{<<e>>}\right]$ \\
        \raisebox{0.5pt}{\textcircled{\raisebox{-0.9pt} {3}}} &
        $\left[\!\!\left[ \begin{array}{l} 
            \textsc{Inform}\\ 
            \AV{name}{Aromi} \end{array}   
        \right]\!\!\right]$ &  
        $\left[\textit{<<s>>}, \textit{aromi}, \textit{<<e>>}\right]$ \\
        \raisebox{0.5pt}{\textcircled{\raisebox{-0.9pt} {4}}} &
        $\left[\!\!\left[ \begin{array}{l}
            \textsc{Inform}\\ 
            \AV{name}{Aromi} \\ 
            \AV{family\_friendly}{no} 
        \end{array}\right]\!\!\right]$  & 
        $\left[\textit{<<s>>}, \textit{aromi}, \textit{is}, \textit{not}, 
               \textit{a}, \textit{family-friendly}, \textit{establishment}, 
               \textit{<<e>>}\right]$ \\
        \bottomrule
        \end{tabular}
    \caption{Example training instances produced from the phrase-based
            data augmentation protocol. The constituent parse is shown
        above. Numbered phrase nodes correspond to the phrase examples
    created in the table below.}
            \label{fig:pbdaexample}
\end{figure}


While the alignment training linearization leads to a controllable model,
there is evidence that sequence-to-sequence models do not behave very
systemically with standard training methods 
\cite{lake2018generalization,loula2018rearranging}. By systematic, we mean that
$h = enc_\theta(\ls(\mr))$ is highly training data dependent and 
 small changes in $\ls(\mr)$ may lead to large, non-linear
changes in h, and as a consequence, the decoder may succeed on
on one $\ls(\mr)$ but fail on another $\ls^\prime(\mr)$ even when the 
differences between the permutations are small. With this in mind, we propose
two data augmentations to make explicit examples of  modular pairwise 
transitions and constituent structure reuse so that models trained with alignment training
may systemically recombine these elements to generate any ordering.
See \autoref{tab:augdat} for statistics of the generated training data.


We augment the training data with MR/utterance pairs taken from constituent
phrases in the original training data. 
We parse all training utterances and enumerate all constituent phrases
governed by 
NP, VP, ADJP, ADVP, PP, S, Sbar
non-terminals.\footnote{We used the \href{https://stanfordnlp.github.io/CoreNLP/}{Stanford CoreNLP parser v3.9.2}.} We then apply the attribute-value 
matching rules
used for alignment training (see \autoref{sec:align})
to obtain a corresponding MR, keeping the dialog act 
of the original utterance. We discard
phrases with no realized attributes.
See \autoref{tab:main.dataset.stats} for augmented data statistics.

Because 
we reclassify the MR of phrases using the matching rules, 
the augmented  data includes examples of
how to invert binary attributes, e.g. from the phrase
``is not on Mac,'' which implies \AV{has\_mac\_release}{no},
we obtain the phrase ``on Mac'' which implies 
\AV{has\_mac\_release}{yes}.
When presenting the linearized MR of phrase examples to the model encoder
we prepend and append phrase specific \textit{start} and \textit{stop} tokens respectively
(e.g., \textit{start-NP} and \textit{stop-NP}) to prevent the model
from ever producing an incomplete sentence when generating for a complete MR.



