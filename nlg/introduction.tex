Up to this point, we have  focused on content selection
%(through the lens of \salienceestimation) 
in a \texttotext~generation system
while relying on a trivial text generation algorithm, copying or extracting text units verbatim from the input, to perform the actual generation task.
In this chapter, we move to modeling the actual 
language generation process after the content selection stages (i.e. \hyperref[dlsal]{Chapters 1} and \hyperref[mlsal]{2}) have been performed. We focus on two particular issues of great 
importance to 
neural models of \naturallanguagegeneration, \faithfulness~and \control.
These concepts, which we define in more detail in later in this chapter,
can be broadly
construed as ensuring the decoder (i.e., the language model) 
in a neural \naturallanguagegeneration~model is contrained to generate utterances
that respects the 
semantics and discourse structure of the selected content.


\Deeplearning~architectures have become the standard modeling method for a
host of language generation tasks \citep{all,of,the,things}. When data is
plentiful, the \deeplearning~\sequencetosequence~framework has proven to be
incredibly adaptable to a variety of problem domains.  Recent evaluations of
end-to-end trained \deeplearning~models for dialogue generation have shown
that they are capable of learning very natural text realizations of formal
\meaningrepresentation s. In many cases, they  beat
rule and template based systems on human and automatic measures of quality
\citep{dusek2020}.
 
\begin{figure}[p]
\center
%\begin{tabular}{cp{9cm}}
\begin{subfigure}{.35\textwidth}
\center
\MR{\textsc{Inform}}
    {\AV{name}{Aromi}}
    {\AV{eat\_type}{coffee shop}}
    {\AV{customer\_rating}{5 out of 5}}
    {\AV{food}{English}}
    {\AV{area}{city centre}}
    {\AV{family\_friendly}{yes}}
 
\caption{\emph{Inform} dialog act.}\label{fig:informexample}
\end{subfigure}\hfill \begin{subfigure}{.58\textwidth}
\begin{itemize}
\item Aromi coffee shop serves English food in a family-friendly atmosphere near the city center and has a customer rating of 5 out of 5.
\item The Aromi coffee shop is family-friendly and serves English food.  It has a customer rating of 5 out of 5 and is located near the center of the city.
\end{itemize}
\caption{Natural language utterances.}
\label{fig:mr1utt}
\end{subfigure}

%"name[Aromi], eatType[coffee shop], food[English], customer rating[5 out of 5], area[city centre], familyFriendly[yes]",
%"name[Aromi], eatType[coffee shop], food[English], customer rating[5 out of 5], area[city centre], familyFriendly[yes]"
%"name[Aromi], eatType[coffee shop], food[English], customer rating[5 out of 5], area[city centre], familyFriendly[yes]",,0,"name[Aromi], eatType[coffee shop], food[English], customer rating[5 out of 5], area[city centre], familyFriendly[yes]"


~\\~\\

\begin{subfigure}{.35\textwidth}
\center
$\left[\!\!\left[\begin{array}{l} 
    \textsc{Give Opinion} \\ 
    \textrm{name=Little Nightmares} \\
    \textrm{rating=good} \\
    \textrm{genres=[}\\
    \textrm{~~~~adventure,} \\
    \textrm{~~~~platformer,}\\
    \textrm{~~~~puzzle} \\
    \textrm{]} \\
    \textrm{player\_perspective=side view}
\end{array}\right]\!\!\right]$ 
\caption{\emph{Give Opinion} dialog act.}\label{fig:giveopinionexample}
\end{subfigure}\hfill \begin{subfigure}{.58\textwidth}
\begin{itemize}
    \item Adventure games that combine platforming and puzzles can be frustrating to play, but the side view perspective is perfect for them. That's why I enjoyed playing Little Nightmares.

    \item Little Nightmares is a pretty cool game that has kept me entertained. It's an adventure side-scrolling platformer with some puzzle elements to give me a bit of a challenge.
\end{itemize}
\caption{Natural language utterances.}
\label{fig:mr1utt}
\end{subfigure}


~\\~\\


\begin{subfigure}{.35\textwidth}
\center
$\left[\!\!\left[\begin{array}{l} 
    \textsc{Compare} \\ 
    \textrm{name\textsubscript{1}=Erebus 92} \\
    \textrm{resolution\textsubscript{1}=720p} \\
    \textrm{family\textsubscript{1}=W2} \\[8pt]
    
    \textrm{name\textsubscript{2}=Helios 96} \\
    \textrm{resolution\textsubscript{2}=1080p} \\
    \textrm{family\textsubscript{2}=L5} \\
\end{array}\right]\!\!\right]$ 

~\\
~\\

\caption{\emph{Compare} dialog act.}\label{fig:compareexample}
\end{subfigure}\hfill \begin{subfigure}{.58\textwidth}
\begin{itemize}
    \item Tthe Helios 96 tv has a 1080p resolution in the L5 family while the Erebus 92 has a 720p resolution in the W2 family. 
    \item Compared to Erebus 92 which has 720p resolution and is in the W2 product family, Helios 96 has 1080p resolution and is in the L5 product family. Which one do you prefer?
\end{itemize}
\caption{Natural language utterances.}
\label{fig:mr1utt}
\end{subfigure}

~\\~\\

\begin{subfigure}{.35\textwidth}
\center

~\\~\\

$\left[\!\!\left[\begin{array}{l} 
    \textsc{Confirm} \\ 
    \textrm{type=laptop} \\
    \textrm{drive\_range=medium} \\
\end{array}\right]\!\!\right]$ 

~\\~\\~\\

\caption{\emph{Confirm} dialog act.}\label{fig:confirmexample}
\end{subfigure}\hfill \begin{subfigure}{.58\textwidth}
\begin{itemize}
   \item Just to verify. The laptop needs to have a medium drive range, correct?
   \item Let me confirm, a laptop in the medium drive range right?
\end{itemize}
\caption{Natural language utterances.}
\label{fig:mr1utt}
\end{subfigure}



\caption{Example \meaningrepresentation s (left) and their reference utterances (right)
from the restaurant, video game, tv, and laptop domains.}
\label{fig:nlgexamples}
\end{figure}



However, this powerful generation capability comes with a cost;
\deeplearning-based language models are notoriously difficult to control,
often producing quite fluent but  semantically misleading outputs. For
instance, \citet{dusek2020} note that in the E2E Generation Challenge
shared-task, a vanilla \sequencetosequence~model scored first in human
evaluations of naturalness but last in quality (which they define as semantic
completeness and grammaticality).  In order for such models to truly be
useful, they must be capable of correctly generating utterances for novel
\meaningrepresentation s at test time.  In practice, even with
delexicalization \citep{dusek2016,juraska2018}, copy and coverage mechanisms
\citep{elder2018}, and over-generation plus reranking
\citep{dusek2016,juraska2018}, \deeplearning-based language generation models
still produce errors \citep{dusek2020}.


We tackle these issues on the task-oriented dialog generation problem
\citep{mairesse2010,wen2015,dusek2018}, where a
\naturallanguagegeneration~model must map a \meaningrepresentation~(i.e., a
dialogue act with an associated set of attribute-value pairs\footnote{In the
literature and in industry, dialogue acts are sometimes called ``intents,''
and attribute-value pairs as ``slots'' and ``slot-fillers'' or ``entities.''})
to an appropriate natural language utterance (see \autoref{fig:nlgexamples}
for examples).  In the context of our broader work on \texttotext~generation,
we think of the \meaningrepresentation~input as an idealized representation of
the content selection stage in a \texttotext~generation model.  Studying
faithfulness and control in the closed-world domains of task-oriented dialog
generation allows us to make meaningful progress while minimizing unnecessary
complexity.

For instance, natural language summaries often contain information not
explicitly represented in the input. The source of this content is either from
common sense knowledge, generic or domain specific knowledge, or new facts
deduced from any combination of the input and prior knowledge
\citep{wiseman2017,wang2019}.  Evaluating the faithfulness of a neural language
generation model in this context is extremely difficult because it is not
clear if a generated utterance is due to the decoder language model or 
the encoder's representation of the input.

Instead, the task-oriented dialogue generation problems we study are developed
to be closed-world, narrow domain settings, where the totality of the
information needed to generate an utterance is represented by the
\meaningrepresentation.\footnote{Mention an exception} Additionally, the
semantics of the \meaningrepresentation~are explicit and there is no
information that needs to be realized by the language generation component
that requires secondary facts deduced from the \meaningrepresentation.

We call a \naturallanguagegeneration~model that generates utterances 
that are semantically correct with respect to the input 
\meaningrepresentation, a \faithfulgeneration~model.
We posit that neural \sequencetosequence~models do not learn a representation
of the input that are parsimonious with the nature of utterance data, chiefly
that phrases that denote fragemnts of \meaningrepresentation s can be combinatorially
recombined to make new meanings/utterances. Instead, the learned representations are highly idiosyncratic, and often reflect spurious correlations and 
artifacts of the dataset that do not generalize well outside the training data.
This issue is symptomatic of neural
\sequencetosequence~models' lack of systematicity \citep{lake18}.
%fail to compositionally
%%reuse information in the training data, and that this is leads to failures
%in generating faithful utterances on novel test-set \meaningrepresentation s.


%to generalize to novel test-set
%\meaningrepresentation s and give some empirical evidence for this
%from the E2E Challenge dataset. 

To overcome these issues, we propose a novel data augmentation scheme to
create synthetic \meaningrepresentation/utterance pairs which break spurious
correlations in the training dataset. Our method makes use of a vanilla
\sequencetosequence~\naturallanguagegeneration~model, i.e. the kind described
by \citet{dusek2020} that produces natural but semantically incorrect
utterances, and a \naturallanguageunderstanding~model, both of which can
trained on the same parallel data.  We then use a noise-injection sampling
method \citep{noise} that allows us to generate semantically diverse yet
syntactically well formed utterance from the \naturallanguagegeneration~model.
We obtain \meaningrepresentation s for these sampled utterances using the
\naturallanguageunderstanding~model.  Using this procedure we can generate a
large collection of synthetic data points. Training a new
\sequencetosequence~model on the union of the original training and novel
synthetic data yields a more \faithfulgeneration~model, with substantially
reduced semantic errors.


While a
\faithfulgeneration~model produces semantically correct output, in general
it is free to let the surface realization order of output utterances
be determined by its language model. We further define a 
\controllablegeneration~model as a \naturallanguagegeneration~model that can
follow a externally provided discourse ordering plan. 
\Controllablegeneration~models are a subset of \faithfulgeneration~models.
See \autoref{fig:examplecontrol} for examples of utterance plans in the
context of \meaningrepresentation~s for task-oriented dialogue generation.



\begin{figure}

    \begin{subfigure}[t]{.30\textwidth}
            \centering
    \MR{\textsc{Inform}}{\AV{name}{Aromi}}{\AV{area}{city centre}}{\AV{eat\_type}{coffee shop}}


    ~\\~\\~\\~\\

    \caption{\textit{Inform} dialog act.}
\end{subfigure}
\hfill
\begin{subfigure}[t]{.65\textwidth}
    \small
    \centering
        \begin{tabular}{l}
         %   \toprule
            ~\\
            Plan A:~~~~~~~~name~~~$\rightarrow$~~~eat\_type~~~$\rightarrow$~~~area\\[4pt]
            Realization: \textit{Aromi is a coffee shop in the city 
                centre.}\\
                ~\\
            %\midrule
                Plan B:~~~~~~~~eat\_type~~~$\rightarrow$~~~name~~~$\rightarrow$~~~area\\[4pt]
            Realization: \textit{There is a coffee shop called Aromi in the 
                city centre.}\\
            %\midrule
                ~\\
                Plan C:~~~~~~~~eat\_type~~~$\rightarrow$~~~area~~~$\rightarrow$~~~name\\[4pt]
            Realization: \textit{For coffee in the centre of the city, try 
                Aromi.}\\
        %    \bottomrule
                ~\\
        \end{tabular}
        \caption{Three different utterance plans with example realizations
        for the \textit{Inform} dialogue act (left).}
\end{subfigure}
\caption{Examples of controllable generation. (a) A \meaningrepresentation~of an \textit{Inform} dialogue act. (b) Three hypothetical utterance plans and their realizations for the example dialogue act.}
\label{fig:examplecontrol}
\end{figure}



%\sequencetosequence~models
%prevents 










 While a
\faithfulgeneration~model produces semantically correct output, in general
it is free to let the surface realization order of output utterances
be determined by its language model. We further define a 
\controllablegeneration~model as a \naturallanguagegeneration~model that can
follow a externally provided discourse ordering plan. 
\Controllablegeneration~models are a subset of \faithfulgeneration~models.

We show how different ways of mapping a \meaningrepresentation~to a flat sequence of tokens, which we refer to as a \linearizationstrategies,
for use as input to a neural \sequencetosequence~model can affect
the model's \faithfulness~at test time.
What's more, we show  that
linearizing the \meaningrepresentation~in a way that aligns with the surface
realization order of the corresponding reference utterance that during training  yields very 
controllable models at test time. Such a model can now follow an arbitrary
utterance plan at test time. Additionally, we propose a phrase-based data augmentation
technique to improve the models ability to follow arbitrary or difficult plans. We find
this controllable generation ability holds for a variety of popular architectures and a large, pretrained conditional language generation model.


%In what follows, we introduce the \meaningrepresentation s used 
%for task-oriented dialogue generation (\autoref{sec:mr4tod}). Then we give
%some background on how neural \naturallanguagegeneration~are used to generate
%utterances from the provided \meaningrepresentation~(\autoref{mrtproblemdef}).
%
%
In what follows, we introduce the \meaningrepresentation s for task-oriented
dialogue genertation in more detail (\autoref{sec:mr4tod}) and then develop some
background on neural \sequencetosequence~modeling for 
\meaningrepresentation-to-text generation (\autoref{2}). We then turn
to our main contributions,
our data-augmentation method for \faithfulgeneration~(\autoref{3}),
and \linearizationstrategies~for \controllablegeneration~(\autoref{4}).

%~\\~\\
%
%\linearizationstrategies,
%that is, mappings from \meaningrepresentation~to 
%\naturallanguagegeneration~model input sequences, can effect the 
%\faithfulness~of the model. In particular, we show that during training that
%linearizing the \meaningrepresentation~in a way that aligns with the surface
%realization order of the corresponding reference utterance yields very 
%controllable models at test time. Such a model can now follow an arbitrary
%order at test time. Additionally, we propose a phrase-based data augmentation
%technique to improve the models ability to follow arbitrary orders. We find
%this controllable generation ability holds for a variety of popular architectures and a large, pretrained conditional language generation model.
%
%
%
%
%~\\~\\
%
%
%
% Instead this knowledge is either common sense
%knowledge, domain specific knowledge, or knowl
%
%derived from background
%knowledge of the world that is not represented in the input, and it 
%becomes difficult to evaluate the faithfulness in this context
%
%~\\~\\
%
%
%by making the 
%knowledge base explicit (the model can only use facts represented in the
%input \meaningrepresentation) and keeping the semantics simple and easy
%to check by a non-domain expert (the mapping from \meaningrepresentation to phrases in the utterance is roughly one-to-one, and does not require making a deductive leap).
%
%
%
%Even in narrow domains like generating sports summaries,
%making distinctions between knowledge represented in the input and what
%is simply part of the language model can be difficult \cite{wiseman2017,wang2019}.
%~\\~\\
%
%
%Faithfulness
%and control are incredibly difficult to study in open-domain settings
%like news. 
%in open-domain summarization
%where making distinctions between 
%
%
%  treating the  
%\meaningrepresentation s as an idealized representation of the content 
%selection stage in a \texttotext~generation model. 
%
%
%, i.e. dialogue acts with an associated set of
%attribute-value pairs 
%
%
%
%
%
%%model 
%In this problem setting, we work from
%an idealized representation
%of content 
%
%(as if we were receiving output from a a hypothetical content selection model from the previous chapter) , we now focus on the text generation process in more detail. 
%
%
%%In particular, we focus on ensuring the language generation component is more
%%than just a good language model producing fluent text but also produces 
%%semantically correct or consistent output given its conditioning inputs.
%%We 
%
%
% 
%%\input{example.tex}
%
%
%We call a \naturallanguagegeneration~model that generates utterances 
%that are semantically correct with respect to the input 
%\meaningrepresentation~a \faithfulgeneration~model. While a
%\faithfulgeneration~model produces semantically correct output, in general
%it is free to let the surface realization order of output utterances
%be determined by its language model. We further define a 
%\controllablegeneration~model as a \naturallanguagegeneration~model that can
%follow a externally provided discourse ordering plan. 
%\Controllablegeneration~models are a subset of \faithfulgeneration~models.
%
%
%
%In this chapter we develop training strategies to produce both \faithful~and
%\controllablegeneration~models. In \autoref{sec:fg}, we propose a 
%data augmentation strategy for using an 
%un\faithful~\naturallanguagegeneration~model and 
%a \naturallanguageunderstanding~model to 
%generate novel \meaningrepresentation/utterance pairs that are not well
%represented in the original training data. Crucially, we use a noise-injection
%sampling method that allows us to generate semantically diverse 
%yet syntactically well formed outputs. Using this procedure we can generate
%a large collection of synthetic data points. Training a new 
%\sequencetosequence~model on the union of the original training and novel
%synthetic data yields a more \faithfulgeneration~model, with substantially
%reduced semantic errors relative to very competitive baselines. 
%
%
%
%In the next sections, we present related works, before formally defining the 
%\meaningrepresentation~to~text generation task and models, before developing
%the data augmentation techniques for \faithfulgeneration~\autoref{somesec} 
%and \linearizationstrategies~for~\controllablegeneration~\autoref{somesec}.
%
%
% 
%
%
