\section{Meaning Representations for Task-Oriented Dialogue Generation}
\label{sec:mr4tod}
\subsubsection{On \MeaningRepresentations}

In this chapter, we use several domain specific \meaningrepresentation s to
formally represent the input to the \surfacerealization~model.  While
specifics of the \meaningrepresentation~can vary from domain to domain, the
overall structure of the \meaningrepresentation~is fairly straightforward,
borrowing from a common format used frequently in the
\naturallanguagegeneration~literature
\citep{mairesse2010,gasic2014,wen2015,novikova2017,juraska2019}. Each
\meaningrepresentation~of an utterance is defined by the \dialogueact, which
expresses the communicative goal or intent, and zero or more
\attributevalue~pairs which further define the semantics of the desired
utterance. 

See \autoref{fig:informexample} for an example \meaningrepresentation~from
the restaurant domain. The dialogue act, in this case \textit{Inform},
is the first item and is written in \textsc{SmallCaps} style.
The attributes are ``name,'' ``eat\_type,'' ``customer\_rating,'' ``food,'' 
``area,'' and ``family\_friendly;''
their associated values are ``Aromi,'' ``coffee shop,'' ``5 out of 5,'' 
``English,'' ``city centre,'' and ``yes'' respectively. In this
case the attributes are referring to the restaurant about which a hypothetical
dialogue agent is trying to inform a user.

In our setting, \dialogueact s are predominantly declaritive (e.g.,
\autoref{fig:informexample} or \autoref{fig:giveopinionexample}), but also
include iterogatives (e.g., \autoref{fig:confirmexample}), and some that may be
a mix of both (e.g., \autoref{fig:compareexample} where the second reference
ends in a question about user preference).  Additionally, we also have
semantically vacuous ``chit-chat'' like \textit{Greeting} and \textit{Goodbye}.


The term \meaningrepresentation~is somewhat of a misnomer as the
representations might better be characterized as a pragmatic construct, i.e. a
representation of the dialogue agent's intentional state.  The
\attributevalues, on the other hand, are a semantic construct, representing the
semantic value or propositional content of the sentences in the utterance. In
other words, from the perspective of formal semantics, \begin{itemize} 
    \item The Aromi is a coffe shop in the city centre.  \item Just to
    confirm, the coffee shop in the city centre is called Aromi?  \item What
about Aromi, the coffee shop in the city centre?  \end{itemize} \noindent all
share the same semantic value. The ``meaning'' of the above statements
in first-order logic might look something like, \[\exists x :
\operatorname{isCoffeeShop}(x) \wedge \operatorname{namedAromi}(x) \wedge
\operatorname{inCityCentre}(x).\] which we might represent in our present
setting as a ``\meaningrepresentation~without a dialogue act,'' i.e.,
\begin{center} \MR{\textrm{---}}{\AV{name}{Aromi}}{\AV{eat\_type}{coffee
    shop}}{\AV{area}{city centre}} \end{center} ~\\
    
    \noindent and which combined with one of the dialogue acts
    \textit{Inform}, \textit{Confirm},
    or \textit{Recommend} yields the pragmatic meaning of the respective sentences above.



The kinds of values that can fill an attribute are typically categorical
variables. For example, in the restaurant domain, the attribute ``food'' may
take values from a closed list of food types such as the set
$\{\textrm{``Chinese''}, \textrm{``English''}, \textrm{``French''},
\textrm{``Fast food''}, \textrm{``Italian''}, \textrm{``Japanese''}\}$. Other
value types include list-valued attributes, numerical values, or free text (see
\autoref{fig:valtypes} for examples of each). For list-valued attributes, the
value is a list of items drawn from a closed set.  For example, in the video
game domain, a video game can belong to several genres simultaneously.
For purposes, we treat each value in the 
list as a distinct \attributevalue~pair. So in the case of \autoref{fig:valtypes},
we treat the genres attribute, it is as if we have \AV{genres}{platformer}, 
\AV{genres}{puzzle}, and \AV{genres}{shooter} present in the \meaningrepresentation.
Additionally, not all attributes need to be specified. In which case, the
utterance should not mention them.

\begin{figure}
 \begin{subfigure}{\textwidth}
    \begin{minipage}{0.5\textwidth}
\center
$\left[\begin{array}{l} 
    \textsc{Inform} \\ 
    \textrm{name=Portal 2} \\
    \textrm{esrb=E 10+ (for Everyone 10 and Older)} \\
    \textrm{genres=[platformer, puzzle, shooter]} \\
    \textrm{player\_perspective=[first person]} \\
    \textrm{has\_multiplayer=yes}
\end{array}\right]$ 
\end{minipage}
    \begin{minipage}{0.5\textwidth}
        Portal 2 was rated E 10+ (for Everyone 10 and
        Older). It is a puzzle platformer FPS with
        multiplayer.
\end{minipage}

~\\

\caption{An example of list-valued attributes (genres and player\_perspective)
    from the video game domain. Note that the acronym FPS means ``first person
shooter'' which realizes both the
player\_perspective \attributevalue and a genre \attributevalue. }
\end{subfigure}  

~\\
~\\
~\\

\begin{subfigure}{\textwidth}
    \begin{minipage}{0.5\textwidth}
    \center
$\left[\begin{array}{l} 
    \textsc{Request} \\ 
    \textrm{specifier=``dull''} \\
    \textrm{has\_multiplayer=yes}
\end{array}\right]$ 
\end{minipage}\begin{minipage}{0.5\textwidth}
    What's the most dull multiplayer game you've ever played?
\end{minipage}

~\\

\caption{An example of a free-text valued attribute (specifier) from the 
video game domain. The specifier value can be any adjective.}
\end{subfigure}  

~\\
~\\
~\\



\begin{subfigure}{\textwidth}
    \begin{minipage}{0.5\textwidth}
 \center
$\left[\begin{array}{l} 
    \textsc{Inform Count} \\ 
    \textrm{count=$58$} \\
    \textrm{type=laptop} \\
    \textrm{is\_for\_business\_computing=true} \\
    \textrm{weight\_range=don't care} \\
    \textrm{drive\_range=don't care} \\
\end{array}\right]$ 
\end{minipage}
\begin{minipage}{0.5\textwidth}
There are 58 laptops used for business computing if you do not care what 
weight range or drive range they have.
\end{minipage}

~\\


\caption{An example of a numeric-valued attribute (count) from the laptop domain.}
\end{subfigure}  

   
%inform_count(count=58;type=laptop;isforbusinesscomputing=true;weightrange=dontcare;driverange=dontcare
\caption{Examples of various \attributevalue~types paired with an example
realization.}
\end{figure}





\subsubsection{Relating Between \MeaningRepresentations~and \Utterances}

Let $\mr \in \mrspace$ be a \meaningrepresentation, and let $\utttoks = \left[\utttok_1,\ldots,
\utttok_\uttSize \right] \in \outSpace$ be an utterance, i.e. sequence of $\uttSize$ tokens from a vocabulary 
$\uttvocab$ and $\outSpace = \uttvocab^*$. We say that an utterance $\utttoks$ \textit{denotes} a \meaningrepresentation~$\mr$,
which we write $\denotes{\utttoks} = \mr$ if the propositional content of the utterance and the
meaning representation are the same, i.e. the \attributevalues~implied by $\utttoks$ and explicitly 
listed by $\mr$ are the same. 
We can make similar statements about a sub-sequence of an utterance. Let 
$\utttoks_{i:i+j} = \left[\utttok_i, \utttok_{i+1},\ldots, \utttok_{i+j-1} \right]$ be 
sub-sequence of $j$ tokens starting at token $i$. 
We then have $\denotes{\utttoks_{i:i+j}} = \mr^\prime$ for some $\mr^\prime \in \mrspace$.  


As an example, consider the following \meaningrepresentation,
\begin{singlespace}
\[
    \mr = \left[\!\!\left[ \begin{array}{l}
        \textsc{Inform} \\
        \AV{name}{The Vaults}\\
        \AV{eat\_type}{pub}\\
        \AV{near}{Caf{\'e} Adriatic}\\
        \AV{family\_friendly}{no}\\
\end{array}\right]\!\!\right] 
\]
\end{singlespace}
and the utterance,
\[\utttoks = \left[ \textit{The},\,\textit{Vaults},\,\textit{pub},\,\textit{is},\,\textit{near},\,\textit{Caf{\'e}},\,\textit{Adriatic},\,\textit{.},\,\textit{It},\,\textit{is},\,\textit{not},\,\textit{a},\,\textit{good},\,\textit{place},\,\textit{for},\,\textit{families},\,\textit{.}\right]. \]
\noindent Clearly, $\denotes{\utttoks} = \mr$. But we can also look at the meanings of individual
phrases,\begin{singlespace}
    \begin{align*}
        \denotes{\utttoks_{1:3}} = \left[\!\!\left[ \left[\textit{The}, \textit{Vaults}  \right] \right]\!\!\right] & = \left[\!\!\left[
\begin{array}{l} \textrm{---} \\ \AV{name}{The Vaults}\end{array} \right]\!\!\right] \\
    \denotes{\utttoks_{3:4}} = \left[\!\!\left[ \left[\textit{pub}  \right] \right]\!\!\right] &= \left[\!\!\left[
\begin{array}{l} \textrm{---} \\ \AV{eat\_type}{pub}\end{array} \right]\!\!\right] \\
    \denotes{\utttoks_{5:8}} = \left[\!\!\left[ \left[\textit{near}, \textit{Caf{\'e}}, \textit{Adriatic}  \right] \right]\!\!\right] & = \left[\!\!\left[
\begin{array}{l} \textrm{---} \\ \AV{near}{Caf{\'e} Adriatic}\end{array} \right]\!\!\right] \\
    \denotes{\utttoks_{11:17}} = \left[\!\!\left[ \left[\textit{not}, \textit{a}, \textit{good}, \textit{place}, \textit{for}, \textit{families}  \right] \right]\!\!\right] & = \left[\!\!\left[
\begin{array}{l} \textrm{---} \\ \AV{family\_friendly}{no}\end{array} \right]\!\!\right]. 
\end{align*}
\end{singlespace}
Note that it is not the case that $\denotes{\utttoks} = \mr \Rightarrow \denotes{\utttoks_{i:i+j}} \subseteq \mr$. Consider in the example above $\utttoks_{11:17}$ its sub-sequence $\utttoks_{12:17} = \left[\textit{a},\,\textit{good},\,\textit{place},\,\textit{for},\,\textit{families}\right]$ which have
the following denotations,
\begin{singlespace}
   \[
   \denotes{\utttoks_{11:17}} = \left[\!\!\left[\begin{array}{l}\textrm{---}\\ \AV{family\_friendly}{yes} \end{array}\right]\!\!\right] \ne \denotes{\utttoks_{12:17}}= \left[\!\!\left[\begin{array}{l}\textrm{---}\\ \AV{family\_friendly}{no} \end{array}\right]\!\!\right] 
       \]
   \end{singlespace}
\noindent It is important to note that the \attributevalues~are
unordered and do not necessarily reflect the realization order of the 
utterance. 


   In the datasets used for this chapter, $\mr$ are provided with one or more reference uttterances, $\utttoks^{(1)}, \ldots \utttoks^{(k)}$ and that for each reference $\utttoks^{(i)}$, we have that each 
\attributevalue~in $\mr$ can be mapped to an 
utterance sub-sequence that denotes it.
Occasionally this is not the case in the available
training data. For example, some \attributevalues~may have several possible
groundings (see \autoref{somefig}), 
no possible grounding (\autoref{somefig2}),
or be realized using inferential knoweldge not explicitly 
represented in the \meaningrepresentation~(\autoref{somefig3}). While we 
do not remove such examples for the training data, we consider model 
generation of such phenomena to constitute a failure to faithfully generate
an utterance.  
 
Because each \attributevalue~is explicitly and uniquely grounded in the 
target utterances, this makes \surfacerealization~from 
\meaningrepresentations~a useful task to study faithful generation. The
baseline task of correctly generating all \attributevalues~appropriately 
for the \dialogueact~is hard enough, and it in this setting we do not have
to worry about ungrounded information or information that is not explicitly
represented in the \meaningrepresentations~but is deducable from
the \meaningrepresentation~\citep{cleanuprotowire}.  
 
\clearpage

%
%pairs are provided for training a
%\naturallanguagegeneration~model such that each \attributevalue~can be mapped to an 
%utterance sub-sequence that denotes it.
%of words 
%(or subwords) in the 
%reference utterance.
%
%
%
%
%
%
%
%
%
%
%%\begin{figure}
%%\center
%%%\begin{tabular}{cp{9cm}}
%%\begin{subfigure}{.35\textwidth}
%%\center
%%$\left[\begin{array}{l} 
%%    \textsc{Inform} \\ 
%%    \textrm{name=The Vaults} \\
%%    \textrm{eat\_type=pub} \\
%%    \textrm{customer\_rating=5 out of 5} \\
%%    \textrm{near=Caf{\'e} Adriatic} \\
%%    \textrm{price\_range=more than \pounds 30}
%%\end{array}\right]$ 
%%\caption{\emph{Inform} dialog act.}
%%\end{subfigure}\hfill \begin{subfigure}{.58\textwidth}
%%\begin{itemize}
%%    \item The Vaults pub near Caf{\'e} Adriatic has a 5 star rating. Prices start at \pounds 30.
%%    \item Located near Caf{\'e} Adriatic, the Vaults is a 5 star rated pub. Prices range above \pounds 30.
%%\end{itemize}
%%\caption{Natural language utterances.}
%%\label{fig:mr1utt}
%%\end{subfigure}
%%\caption{An example \meaningrepresentation~(a) and example utterances (b)
%%from the restaurant domain.}
%%\label{fig:mr1}
%%\end{figure}
%
%
%\autoref{fig:mr1utt} shows some possible realizations of the 
%\meaningrepresentation. It is important to note that the \attributevalues~are
%unordered and do not necessarily reflect the realization order of the 
%utterance. In the datasets used for this chapter,  \meaningrepresentations~are
% paired with at least one reference utterance, and for the most part
%each \attributevalue~uniquelt maps to a span of words (or subwords) in the 
%reference utterance. See \autoref{fig:mr2align} where the mapping is made
%explicit for the first utterance realization from \autoref{fig:mr1utt}.
%
%
%%\begin{figure}
%%\center
%%%\begin{tabular}{cp{9cm}}
%%\begin{subfigure}{.35\textwidth}
%%\center
%%$\left[\begin{array}{l} 
%%    \textsc{Inform} \\ 
%%    \textrm{(1) name=The Vaults} \\
%%    \textrm{(2) eat\_type=pub} \\
%%    \textrm{(3) customer\_rating=5 out of 5} \\
%%    \textrm{(4) near=Caf{\'e} Adriatic} \\
%%    \textrm{(5) price\_range=more than \pounds 30}
%%\end{array}\right]$ 
%%\caption{\emph{Inform} dialog act.}
%%\end{subfigure}\hfill \begin{subfigure}{.58\textwidth}
%%  [The Vaults]\textsubscript{(1)} [pub]\textsubscript{(2)} [near Caf{\'e} Adriatic]\textsubscript{(4)} [has a 5 star rating.]\textsubscript{(3)} [Prices start at \pounds 30.]\textsubscript{(5)}
%%\caption{Natural language utterances.}
%%
%%\end{subfigure}
%%\caption{An example \meaningrepresentation~(a) and example utterances (b)
%%from the restaurant domain.}
%%\label{fig:mr2align}
%%\end{figure}
%
%
%
%
%
%
%\begin{figure}
% \begin{subfigure}{\textwidth}
%    \begin{minipage}{0.5\textwidth}
%\center
%$\left[\begin{array}{l} 
%    \textsc{Inform} \\ 
%    \textrm{name=Portal 2} \\
%    \textrm{esrb=E 10+ (for Everyone 10 and Older)} \\
%    \textrm{genres=[platformer, puzzle, shooter]} \\
%    \textrm{player\_perspective=[first person]} \\
%    \textrm{has\_multiplayer=yes}
%\end{array}\right]$ 
%\end{minipage}
%    \begin{minipage}{0.5\textwidth}
%        Portal 2 was rated E 10+ (for Everyone 10 and
%        Older). It is a puzzle platformer FPS with
%        multiplayer.
%\end{minipage}
%
%~\\
%
%\caption{An example of list-valued attributes (genres and player\_perspective)
%    from the video game domain. Note that the acronym FPS means ``first person
%shooter'' which realizes both a genre \attributevalue~and the 
%player\_perspective \attributevalue.}
%\end{subfigure}  
%
%~\\
%~\\
%~\\
%
%\begin{subfigure}{\textwidth}
%    \begin{minipage}{0.5\textwidth}
%    \center
%$\left[\begin{array}{l} 
%    \textsc{Request} \\ 
%    \textrm{specifier=``dull''} \\
%    \textrm{has\_multiplayer=yes}
%\end{array}\right]$ 
%\end{minipage}\begin{minipage}{0.5\textwidth}
%    What's the most dull multiplayer game you've ever played?
%\end{minipage}
%
%~\\
%
%\caption{An example of a free-text valued attribute (specifier) from the 
%video game domain. The specifier value can be any adjective.}
%\end{subfigure}  
%
%~\\
%~\\
%~\\
%
%\begin{subfigure}{\textwidth}
%    \begin{minipage}{0.5\textwidth}
% \center
%$\left[\begin{array}{l} 
%    \textsc{Inform Count} \\ 
%    \textrm{count=$58$} \\
%    \textrm{type=laptop} \\
%    \textrm{is\_for\_business\_computing=true} \\
%    \textrm{weight\_range=don't care} \\
%    \textrm{drive\_range=don't care} \\
%\end{array}\right]$ 
%\end{minipage}
%\begin{minipage}{0.5\textwidth}
%There are 58 laptops used for business computing if you do not care what 
%weight range or drive range they have.
%\end{minipage}
%
%~\\
%
%
%\caption{An example of a numeric-valued attribute (count) from the laptop domain.}
%\end{subfigure}  
%
%   
%%inform_count(count=58;type=laptop;isforbusinesscomputing=true;weightrange=dontcare;driverange=dontcare
%\caption{Examples of various \attributevalue~types paired with an example
%realization.}
%\end{figure}
%
%
