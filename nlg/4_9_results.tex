\begin{table}
\centering
\begin{tabular}{rcccc}
\toprule
Model  & BLEU & R.-L & MET. \\ \midrule
Slug   & 66.19 & 67.72 & 44.54  \\  
DANGNT & 59.90 & 66.34  & 43.46  \\
TUDA   & 56.57 & 66.14  & 45.29  \\
        \midrule
delex. \basegen~~~~~~greedy & 66.91 & 68.27 & 44.95 \\
                      beam & \textbf{67.13} &  \textbf{68.91} &  45.15  \\
        \auggen~\ruleclf~greedy   &  65.57 & 67.71 & 45.56  \\
                             beam & 66.28 &  68.08 &  \textbf{45.78}  \\
     \learnedclf~greedy & 63.76 &  67.31 &  44.94  \\
   beam   & 
 64.23 &  67.54 &  45.17  \\
\midrule
   lex. \basegen~~~~~~greedy & 60.35 &  64.51 & 41.82  \\
                      beam   & 61.81 &  65.83 & 42.69  \\
    \auggen~\ruleclf~greedy & 64.74 &  68.21 & 44.46  \\
       beam & 64.81 &  67.83 &    44.39  \\
\bottomrule
\end{tabular}
\caption{BLEU, ROUGE-L, and METEOR metrics on the E2E test set. Baseline methods all rely
on at least partial delexicalization, puting our lexicalized models at a relative disadvantage.}

\label{table:autoqual}
\end{table}





\begin{table*}
\setlength{\tabcolsep}{5pt}
\center
  \begin{tabular}{rrrr ccccc ccc ccc}
    \toprule
 \multicolumn{4}{c}{
\multirow{2}{*}{
Model} }
       & \multirow{2}{*}{Name} & \multirow{2}{*}{Near}    
            &  Family  & 
           \multirow{2}{*}{Area}    & Customer & \multirow{2}{*}{Food} 
      & Price & Eat & \multirow{2}{*}{All} \\
  & & & &  &  & Friendly & 
        & Rating &  & Range & Type &  \\
\midrule
\multicolumn{4}{r}{Slug}  
                    & 0 & 0 & 6  & 1  & 6  &  10  & 35  & 9  & 67 \\ 
\multicolumn{4}{r}{DANGNT}
                    & 0 & 0 & 18  & 0  & 0  & 0  & 0  & 58  & 76 \\ 
\multicolumn{4}{r}{TUDA}  
                    & 0 & 0 & 0  & 0  & 0  & 0  & 0  & 0    & \textbf{0} \\
\midrule
delex. & \basegen & & greedy 
                    & 0   & 0   & 23 & 23 & 16 & 26 & 27 & 0 & 115 \\ 
& & & beam          & 0   & 0   & 60 & 3  & 9  & 3  & 8  & 0 & 83  \\ 
& \auggen & \ruledmodel & greedy 
                    & 0   & 0   & 0  & 0  & 0  & 0  & 0  & 0 & \textbf{0} \\ 
 & & & beam         & 0   & 0   & 0  & 0  & 0  & 0  & 0  & 0 & \textbf{0} \\
 & & \learndmodel & greedy 
                    & 0   & 0   & 1  & 0  & 8  & 1  & 9  & 0 & 19 \\
 & &  & beam        & 0   & 0   & 0  & 0  & 3  & 0  & 0  & 0 & 3 \\
\midrule
lex. & \basegen & & greedy 
                    & 145 & 141 & 14 & 15 & 2   & 14 & 2  & 0 & 333 \\
 & & & beam         & 155 & 124 & 62 & 0  & 0   & 0  & 0  & 0 & 341 \\ 
 & \auggen & \ruledmodel  & greedy 
                    & 0   & 0   & 2  & 0  & 0  & 125 & 0  & 0 & 127 \\
&  &  & beam        & 0   & 2   & 0  & 0  & 0  & 119 & 0  & 0 & 121 \\
\bottomrule
    \end{tabular}
\caption{Attribute realization errors on the E2E test set. The Slug model and our delexicalized models delexicalize the NAME and NEAR slots, 
    thus making 0 errors on these attributes. DANGNT and TUDA models perform complete delexicalization. } 
\label{table:autosem}
\end{table*}





\subsection{Results}
\subsubsection{E2E Challenge}

Automatic evaluation metrics are shown in \autoref{tab:fgautoqual}.
Surprisingly, \basegen~using greedy decoding surpases all of the 
baseline systems on all three automatic metrics. 
This is quite shocking as the Slug baseline ensembles
three different sequence-to-sequence models producing 10 outputs each using beam search and reranking based on slot alignment to select the final generation
output. The $\auggen$/$\ruledmodel$~model remains competitive with Slug, 
again even using greedy decoding.
The $\auggen$/$\learndmodel$~starts underperforming Slug on \textsc{Bleu} 
score but
remains competitive on \textsc{Rouge-L} and \textsc{Meteor} again when using 
greedy decoding.
Overall the augmented training data tends to hurt generation with respect
to automatic quality measures.
In this regard, the added noise of the model-based parser, $\learndmodel$,
 exacerbates things
as it reduces quality more than the rule-based parser, $\ruledmodel$. 

In the lexicalized setting,
$\basegen$~produces lower quality output than the Slug system.
However, the augmented training procedure increases
the quality of the lexicalized $\auggen$~model which beats Slug on \textsc{Rouge-L}.


The automatic quality evaluations are somewhat misleading, however. To gain
more insight into model performance
we apply our rule based parser to estimate attribute realization error
for all system outputs on the test set,
similarly to \cite{dusek2019}
(e.g., if the MR specifies \AV{food}{French},
we check to make sure the generated utterance says so). 
The results of this evaluation are shown in \autoref{table:autosem}.
Immediately, it is revealed that \basegen~is far worse than the baseline 
methods making 115 and 83 errors using greedy and beam decoding respectively.

It is here that we see the benefits of the data-augmentation.
The $\auggen$/$\ruledmodel$~model achieves zero test set
errors even when using the greedy 
decoding. The $\auggen$/$\learndmodel$~model is slightly worse (in agreement 
with the automatic quality measurements), but its greedy search is still
superior to the more sophisticated Slug decoder, achieving 19 total
test set errors compared to Slug's 67 errors.



The lexicalized $\basegen$~model has especially high error rates, 
particularly on the \Atr{name} and \Atr{near} attributes.
With augmented data training, the $\auggen$~model reduces these errors
 to zero when using greedy search and 2 with beam search. Unfortunately,
the augmented training is more unstable in the lexicalized setting, 
as it produces a large spike in \Atr{food} attribute errors, although
the $\auggen$~models still have lower overall error than  $\basegen$.


\subsubsection{Laptops and TVs}
The results are more mixed here. Our \textsc{Bleu} scores are about 15 points 
below the baselines on the Laptops dataset and 20 points below the 
baselines on the TVs dataset. 
Upon examing  the evaluation script in detail we see that 
\textsc{Bleu} score is calculated using 5 model outputs which \citet{juraska2018}
and \citet{wen2016} do. We only produce the 1-best output
at test time,
perhaps explaining the difference.

Looking through our model
outputs we see mostly good utterances, often nearly exactly matching the 
references.
Our models outperform the state of the art models on errors. The best state of the art models  make errors by generating sentences that do not match the input representation 0.79\%  and 1.67\% of the time on the Laptops and TVs datasets
respectively. Our \auggen~model reduces that error to only 0.13\% and 
0.20\%.

%We do achieve the lowest error rates, with \auggen~beam
%decoding achieving a 0.2\% error rate versus the previous best of 1.67\%
%on the TVs dataset, and \auggen~greedy decoding achieving a 0.13\% error
%rate compared the 0.79\% baseline rate.

\begin{table}
\centering
    \begin{tabular}{llrrrr}
        \toprule
       & & \multicolumn{2}{c}{Laptops} & \multicolumn{2}{c}{TVs} \\
        Model& & \textsc{Bleu} & Err. & \textsc{Bleu} & Err.  \\
        \midrule
        SCLSTM&    & 51.16 &   0.79\% & \textbf{52.65} &   2.31\%\\
        Slug && \textbf{52.38}  &  1.55\% & 52.26  &  1.67\% \\
      Base Gen. (\basegen) ~~~~ & beam  &37.13  &0.72\% & 32.63 & 0.72\% \\
      Aug. Gen. (\auggen)~ Rule Parser (\ruledmodel) &greedy & 37.21 &  \textbf{0.13\%} & 32.43 & 0.28\%\\
          &beam   & 37.19 & 0.14\% & 32.59 &  \textbf{0.20\%} \\
\bottomrule
    \end{tabular}

    \caption{\textsc{Bleu} and automatic attribute error on the Laptops and TVs
    datasets.}
    \label{table:laptoptvautoqual}
\end{table}





\subsection{Experiment Human Evaluation} 

\paragraph{E2E Dataset} We had two undergraduate students not involved with 
the research look at 100 random test set utterances for six
of our model variants. They were shown
both the Slug output and one of our model outputs and asked to select
which output was of better linguistic quality and correctness or 
indicate that they were equally good.
 We resolved disagreements in favor of the baseline,
i.e. if any annotator thought the baseline was better we considered it so.
If an annotator marked one of our 
systems as better and the other marked it as equal, we considered it 
equal to the baseline. Inter-annotator agreement was high, with 92\% agreement on correctness
and 88\% agreement on quality.

\autoref{humane2e} shows the results of the evaluation.
We find that the \auggen~model outputs are indistinguishable from the Slug
model in terms of linguistic quality, regardless of the setting.
In terms of correctness, the lexicalized \auggen~model is as good as or better than the Slug model 98\%
of the time. 
%We again stress that the \auggen~model only uses beam search 
%with beam size 8, while Slug is delexicalized, and consists of an ensemble of 
%three DNN models each producing 10 outputs via beam search, and reranking 
%to select the candidate that minimizes attribute realization error.
When using the delexicalized models, we don't even need beam search.
The delexicalized \auggen~greedy decoder is as good as or better 
than Slug 100\% of the time.


\begin{table}
\center
    \begin{tabular}{r ccc| ccc}
\toprule
        & \multicolumn{3}{c}{Correct.} & \multicolumn{3}{c}{Quality} \\
    Model & $>$ & $=$ & $<$ & $>$ & $=$ & $<$ \\
\midrule
delex.~\basegen~b & 7 & 89 & 4 & 1 & 96 & 3 \\
delex.~\auggen~\ruledmodel~g & 7 & 93 & 0 & 0 & 100 & 0 \\
delex.~\auggen~\ruledmodel~b & 7 & 93 & 0 & 0 & 100 & 0 \\
delex.~\auggen~\learndmodel~g & 5 & 95 & 0 & 0 & 100 & 0 \\
delex.~\auggen~\learndmodel~b & 8 & 92 & 0 & 0 & 100 & 0 \\
lex.~\auggen~\ruledmodel~g & 8 & 90 & 2 & 0 & 100 & 0 \\
\bottomrule
\end{tabular}
\caption{Human correctness and quality judgments (\%). Comparisons are better
than ($>$), equal to ($=$), and worse than ($<$) the baseline Slug model.
(g) and (b) indicate greedy and beam decoding respectively.}
\label{humane2e}
\end{table}

\begin{table}
    \center
    \begin{tabular}{c ccc| ccc}
\toprule
        & \multicolumn{3}{c}{Correct.} & \multicolumn{3}{c}{Quality} \\
    Model & $>$ & $=$ & $<$ & $>$ & $=$ & $<$ \\
\midrule
delex. \auggen~\ruledmodel~g & 0 & 100 & 0 & 2 & 91 & 7 \\

\bottomrule
\end{tabular}
\caption{Human correctness and quality judgments (\%). Comparisons are better
than ($>$), equal to ($=$), and worse than ($<$) the test set references.}
\label{humanlaptop}
\end{table}

  %"A": Path("base_delex_beam.tsv"),

%4 2 44
%1 3 46

    %"B": Path("aug_rule_delex_greedy.tsv"),
%4 0 46
%0 0 50


%    "C": Path("aug_rule_delex_beam.tsv"),
%4 1 45
%0 0 50


%    "D": Path("aug_clf_delex_greedy.tsv"),
%1 1 48
%0 0 50


%    "E": Path("aug_clf_delex_beam.tsv"),
%5 1 44
%0 0 50


%    "F": Path("aug_rule_lex_beam.tsv"),
%6 1 43
%0 0 50


\paragraph{Laptops Dataset} We had the same annotators look at 100 random 
\textit{Inform}
DAs from the Laptops test set since they are the majority DA type and we 
could use the same annotator
guidelines from the E2E experiment. We do not have access to the Slug
or SCLSTM outputs on this dataset, so we compared to one of the two
test set reference sentences (picking at random) vs. 
the $\auggen$/$\ruledmodel$~with greedy decoding. \autoref{humanlaptop}
shows the results. Despite the low BLEU scores, we find our outputs
to be of comparable quality to references 91\% of the time. Moreover,
they are equally as correct as the human references 100\% of the time.
Annotators agreed 99\% and 87\% of the time on correctness and quality 
respectively.



