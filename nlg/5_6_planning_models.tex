\subsection{Utterance Planner Model} We experiment with three approaches to
creating a test-time utterance plan for the alignment training models. The
first is a bigram language model (\BgUP) over attribute-value sequences.
Attribute-value bigram counts are estimated from the training data 
(using Lidstone smoothing \citep{chen1996} with $\lidstone=10^{-6}$)  according to the
ordering determined by the matching rules (i.e. the alignment-training ordering). 

\begin{table}
\centering
\begin{tabular}{cc}
\toprule
Hyperparameter & Search Space\\
\midrule
Layers & 1, 2\\
Learning Rate &$10^{-3}$, $10^{-4}$, $10^{-5}$\\
RNN Cell & GRU, LSTM \\
Encoder direction & uni-, bi- \\
Label Smoothing & 0.0, 0.1\\
\bottomrule
\end{tabular}
\caption{Hyperparameter search space for the neural utterance planner (NUP).}
\label{tab:nuphps}
\end{table}


\begin{table}
\centering
\begin{tabular}{c ccc ccc ccccc}
\toprule
Dataset & L & Enc. Dir. & RNN Cell& LR & LS & Attn. &$\embDim$ & $\hidDim$  & $\encDim$ & $\decDim$ & Dropout \\
\midrule
   E2E & 1 & bi- &LSTM& $10^{-5}$ & 0.1 & Bahd. & 512 & 512 & 1024 & 512 & 0.1\\
ViGGO & 1 & uni- & LSTM & $10^{-4}$ & 0.1 & Bahd.  & 512 & 512 & 1024 & 512 & 0.1\\ 
\bottomrule
\end{tabular}
\caption{Winning hyperparameter options for the neural utterance planner (NUP)
model.}
\label{tab:nuphp}
\end{table}



The second model is a recurrent neural network based \sequencetosequence~model, which we refer to 
as the neural
utterance planner (\NUP). We train the \NUP~to map \textsc{If} ordered
attribute-values to the alignment training ordering. We grid-search model
hyperparameters, selecting the model with highest average Kendall's $\tau$
\citep{kendall1938} on the validation set alignment training orderings. See
\autoref{tab:nuphps} for the hyperparameter search space and \autoref{tab:nuphp} to see the chosen hyperameter setting. We used a batch size of 128, the Adam optimizer, and trained for at most 50 epochs.   Unlike the
\BgUP~model, the \NUP~model also conditions on the dialogue act, so it can
learn ordering preferences that differ across dialogue acts.

For both \BgUP~and \NUP, we use beam search (with beam size 32) to generate
candidate utterance plans. The beam search is constrained to only generate
attribute-value pairs that are given in the supplied \meaningrepresentation, and to avoid
generating repeated attributes. The search is not allowed to terminate until
all attribute-values in the \meaningrepresentation~are generated.  Beam candidates are ranked by
log likelihood. We show validation and test set Kendall's $\tau$ to the 
reference utterance for both planning models in \autoref{tab:uptau}.
A Kendall's $\tau$ of 1.0 indicates that the planner exactly follows the
human reference order while 0.0 indicates a random order relative to the
human reference. $\tau=-1$ indicates the model produces the reverse order
of the human reference plan. We see that the NUP produces utterance plans that are closer in order to the human reference on both the E2E Challenge and ViGGO datasets

\begin{table}
\centering

\begin{tabular}{ll c c}
\toprule
Dataset & Model & Valid & Test \\
\midrule
ViGGO & \textsc{BgUP} & 0.417 & 0.347 \\
                       & \textsc{NUP} & 0.739 & 0.651 \\
\midrule
E2E & \textsc{BgUP} & 0.433 & 0.432 \\
                       & \textsc{NUP} & 0.502 & 0.447 \\
\bottomrule
\end{tabular}


\caption{Validation and test set Kendall's $\tau$ for \textsc{BgUP} and 
NUP models.}
\label{tab:uptau}
\end{table}


The final ordering we propose is the \Oracle~ordering, i.e. the utterance plan
implied by the human-authored test-set reference utterances. This plan
represents the model performance if it had \textit{a priori}  knowledge of the
reference utterance plan. When a test example has multiple references, we
select the most frequent ordering in the references, breaking ties according
to \BgUP~log-likelihood.
