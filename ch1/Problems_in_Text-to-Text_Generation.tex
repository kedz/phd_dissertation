\section{Problems in Text-to-Text Generation}
  
Amongst the natural language processing (NLP) research community, machine
learning, and especially deep learning, has become the de facto methodological
framework  for solving text-to-text generation problems.  One need only to
obtain a large collection of input-output texts, and under this paradigm, the
details of the particular generation task can be abstracted away. It is
sufficient to re-pose the generation problem as a one of optimization, for
which a general purpose neural network model can be trained such that it
minimizes the relevant loss function, and in so doing, learns a mapping from
the input text to the output text
\citep{sutskever2014,bahdanau2015,rush2015,nallapati2016,see2017}. 

That is to say, the deep learning framework de-emphasizes possessing an
algorithmic solution to a problem, and prefers instead a hands off approach:
the algorithmic solution is implicit in the dataset, so let the neural network,
whose inductive bias is very different from that of the human researcher, learn
directly from the data the representations that are most useful to its
satisfaction of the optimization criteria.

The logical extension of this central dogma is that the best data is more data
\citep{halevy2009}, and indeed, that is precisely what has happened. Large
language model pre-training, where a language model, typically using a
transformer architecture \citep{vaswani2017}, is trained in a self-supervised
way on web-scale text, has dramatically expanded the quality of natural
language utterances that can be generated by a computational model
\citep{radford2019,brown2020}, while also capturing the attention of the
popular press \citep{simonite2019,vincent2019}. In the NLP research community,
pretrained sequence-to-sequence models, the family of deep learning model most
commonly used for text-to-text generation, like PEGASUS \citep{zhang2019}, T5
\citep{raffel2020}, or BART \citep{lewis2020}, have led to impressive gains in
the fluency and coherence of modestly-sized paragraphs, as well as automatic
task metrics like \bleu~\citep{papineni2002} and \rouge~\citep{lin2004}.

There are downsides to this approach however. By learning only from surface
level forms, it is possible for the models to appear like they contain
knowledge, but in reality, they are only modeling the probability of word
sequences \citep{bender2020}. This can cause them to hallucinate information
not present in the input or imply propositions contrary to what was given
\citep{wiseman2017,kryscinski2019,maynez2020,kryscinski2020}.  Without an
understanding of pragmatics, they cannot know how an argument or chain of
propositions holds together, only that they are statistically likely.
Moreover, they will learn many implicit and harmful biases of the society that
produced the corpora they are trained on, including but not limited to negative
stereotypical word associations \citep{bolukbasi2016,nissim2020} and outright
hate speech \citep{lee2016}.

With all of that in mind, we would like this thesis to emphasize the importance
of articulating understandable steps or sub-tasks on the way to solving the
larger summarization or generation problems.  While we do not abandon machine
learning and deep learning, we instead show how they may be used parsimoniously
to solve the actual problems at hand and not simply learn the dataset. Where we
do use machine learning models, we try to design experiments which reveal which
signals relevant to the actual problem are being used to make predictions, and
to establish some empirical limitations on their ability to represent the input
data and perform successfully. 
 
We now describe some of the central problems of summarization and text-to-text
generation that we are interested in solving.  In most summarization tasks, we
are generally interested in a text's \textit{salience}, that is to say, the
general importance or relevance of a given text unit with respect to its
context. Salience is usually the primary dimension of the input data that we
wish to measure or predict for determining summary content.  

In \autoref{ch:dlsum}, we study salience estimation in the \emph{sentence
extractive, single document summarization task}, where the goal is to classify
which sentences in an input document should be included in an extract summary.
In this case, the input document is the context, and the units of text for
which we are estimating salience are the document sentences.  An extractive
summary is a subset of sentences that have maximum salience while satisfying a
length budget constraint, typically in the summary word or byte length. While
the constrained subset selection problem is interesting and has been studied
previously \citep{goldstein1998,mcdonald2007,lin2010}, we focus on modeling the
salience estimation task specifically. 
 
We study a variety of popular and novel deep learning architectures for
implementing the salience prediction task. Our key contributions here are not
only a model architecture, but also our systematic study of the combination of
sentence and context (i.e. document) level encoders as well as the manipulation
of the input documents to ablate which surface features are available to a
given model. Through these input data ablations, we can gain a better
understanding of how the salience prediction mechanism is working.
 
But what about more difficult summarization tasks? In \autoref{ch:mlsum}, we
study salience estimation for \emph{query focused, streaming, sentence
extractive summarization}. In this task, we add a search query and time as
additional elements to the summarization problem. As input we are given a
time-ordered stream of news articles and a query, typically a notable
real-world event, e.g., \texttt{``hurricane sandy''}. Our objective is to
extract sentences that are relevant to the query event while minimally
redundant to previously extracted sentences. Unlike the previous problem, the
salience of a given sentence is not constant but monotonically decreases as
time progresses. Due to the large volume of input texts, in attempting to model
the salience of a sentence we must now also model the redundancy between
sentences and previous extraction decisions to perform competently.
      
We incorporate a salience estimation model into two possible approaches to
extracting query-relevant summary sentences from the news stream.  The first
method uses the salience estimates to bias an affinity propagation clustering
algorithm \citep{dueck2009} to identify exemplar sentences which we extract for
the summary.  The clustering algorithm must trade off representativeness versus
the salience of individual sentences when selecting exemplars, i.e. an exemplar
must be a good representative of its cluster but also be important under the
salience estimator.  This approach works in hourly batches, predicting the
salience of sentences, clustering them, and then adding the exemplars to a
rolling update summary. The salience estimates also adaptively control the
number of clusters produced, allowing the model to adapt the number of updates
to fit the volume of salient sentences found in the stream. 

The first method has several drawbacks. The summary selection is not done to
optimize the final summary evaluation measure and the predictions of salience
are static and cannot take into account previous decisions the summarizer has
made. Additionally, the use of clustering means that there will always be some
latency between when important information is known and when we can extract it
for the summary. In domains like crisis informatics, minimizing these delays
are critical \citep{starbird2013}.

To address these limitations, we recast stream summarization as a sequential
decision-making problem \citep{littman1996}, where we learn a policy for
extracting a sentence based on its estimated salience as well as its relation
to previously extracted sentences.  The sequential decision-making view opens
our model up to exposure bias as the learned policy will suffer from a
train-test distribution mismatch if our reference policies are overly
optimistic or pessimistic.  To mitigate this, we employ a learning-to-search
style training regime \citep{chang2015} to train a policy to make locally
optimal decisions when following either a noisy learned policy or an oracle
reference policy.  The result is a fully online summarization model whose local
decisions positively correlate with good overall summary evaluation measures.
Additionally, because the model is greedy, it does not suffer as much from the
negative effects of latency as the clustering model does.
      
In our experiments with single-document extractive summarization, we found that
neural models heavily exploited position-based heuristics (i.e., did the
sentence occur in the lead paragraph) to determine sentence salience, which
arguably does not capture the essence of the summarization problem. In our work
on stream summarization, we show that with careful feature and model design, we
can capture salience beyond such heuristics.  In particular, we show that
content, location, and redundancy features can be used to predict salience in
this more challenging scenario.

In \autoref{ch:nlg}, we move to problems of content realization, after the
content selection process has been performed.  Here we focus on the related
goals of \textit{faithful} and \textit{controllable} generation using neural
NLG models. A neural NLG model is faithful if it can generate utterances that
are semantically correct with respect to the information extracted in the
content selection stage.  One of the central tensions in a neural NLG model is
that between the encoder, which creates a representation of the input, and the
decoder, which is functionally a language model conditioned on the encoder
representation.  The decoder language model must simultaneously place high
probability on output word sequences that are likely given the training corpus,
but also prefer output word sequences that are correlated with the encoder
representation of the input sequence. This conflict can lead to hallucinations
as the language model may occasionally put more probability mass on a sequence
of words that is frequently observed in the training data, but not necessarily
licensed by a particular input.

We hypothesize that this failure mode happens in part because the decoder, by
predicting next word continuations, is both implicitly planning the layout of
the utterance but also trying to satisfy the constraints given by the input,
and that alleviating the decoder language model of the planning task may
improve the faithfulness.  We propose developing controllable neural NLG
models, i.e. models that can follow an explicit plan determining  the surface
realization order of the intended utterance.  Controllable models learn to
represent the layout of the intended utterance implicitly in their encoder, and
thus the decoder language model has less flexibility in selecting the next
words, which can lower the chances of hallucinating text and improve the
overall faithfulness of the generation model.

We also believe that controllable generation has additional benefits beyond
increased faithfulness. For one, it will enable more integration of neural NLG
models into large NLG pipelines, \citep{castroferreira2019}.  Controllable
generation at the level of shallow phrase chunk ordering like we are proposing
may also lead to implementations of cognitively plausible discourse ordering
theories like Centering Theory \citep{grosz1995} or Accessibility Theory
\citep{ariel2001}, which place constraints on the discourse ordering of
entities.

Evaluating the faithfulness of a language generation model for open-world
summary generation tasks is non-trivial \citep{kryscinski2020,maynez2020}. In
order to simplify things, we study faithful and controllable generation in the
context of task-oriented dialogue generation, where, given an explicit
representation of a dialogue agent's belief state and goals, we must generate
an appropriate natural language utterance. Because the input is an explicit,
formalized representation of the meaning of the intended utterance, manual and
even automatic checking of the faithfulness of an utterance/meaning
representation pair becomes much simpler. Since the concerns of faithful and
controllable generation are still incredibly important to generating summaries,
we consider the explicit meaning representations as an idealized version of
summarization system's content selection stage.  Faithfulness is critical for
any real summarization application; the reader has to be able to trust that
content is correct or it is functionally useless. Controllable generation will
further increase reliability and faithfulness, while also allowing the
tailoring of the summary to focus on particular user needs, for example
targeting generation to focus on a particular set of entities. 
      
For our contribution to faithful generation, we propose a novel data
augmentation method for sequence-to-sequence models.  We observe that a popular
sequence-to-sequence NLG model trained on a task-oriented dialogue generation
dataset produces fluent and natural utterances that are unfortunately
frequently semantically incorrect with respect to the input meaning
representation.  We then present evidence that the reason for these errors are
spurious correlations between the inputs and outputs in the training data.  We
also note that by injecting random noise into the unfaithful NLG model we can
cause it behave in an uncontrolled but useful manner: it generates utterances
that are not faithful to the input but do not exhibit the spurious correlations
or exhibit them to a lesser degree.  We generate a synthetic corpus of these
utterances, and then use another semantic parsing model to give them correct
meaning representations.  Remarkably, sequence-to-sequence models trained on
the union of original training data and the synthetically generated training
examples exhibit increased faithfulness without hurting their fluency. We also
find that in the union dataset, the problematic spurious correlations are
diminished.

While this data augmentation method helps reduce semantic errors, it leaves the
surface realization of utterances up to the decoder language model.  Our second
contribution to neural NLG models is an alignment training method that reliably
produces a model capable of following a discourse ordering plan when generating
utterances.  Our method works by aligning the individual components of a
meaning representation to their reference utterances on the training set. Given
this alignment, we then map a meaning representation into a linear sequence of
tokens, such that the order of the individual components corresponds to the
realization order in the training reference. Training an arbitrary
sequence-to-sequence model to map this linearized meaning representation to its
reference utterance induces the ability to control the model at test time
(i.e., we can use a planning model to propose an ordering of the meaning
representation's sub-components, and the controllable sequence-to-sequence
model will attempt to realize them in that order).  To achieve a different
ordering, one need only permute the input sequence.

In our experiments, we also evaluate how well models are able to follow
adversarially generated plans that do not have human, English language ordering
preferences and show that models struggle to realize utterances correctly in
this setting. We finally propose another data augmentation scheme to generate
constituent phrase data that gives explicit examples of how phrases can be
composed into larger units and how that systematically changes the meaning
representation. We find that this additional data improves the robustness of
the control behavior on these more difficult to follow plans.
