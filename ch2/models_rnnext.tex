\subsubsection{\rnnext~Extractor}



\begin{figure}[t]
    \fbox{\begin{minipage}{\textwidth}
\center
\scalebox{0.75}{
\begin{tikzpicture}[
  dep/.style ={
    ->,line width=0.3mm
  },
  hid/.style 2 args={
    rectangle split,
    draw=#2,
    rectangle split parts=#1,
    fill=#2!20,
    minimum width=5mm,
    minimum height=5mm,
    outer sep=2mm},
  mlp/.style 2 args={
    rectangle split,
    rectangle split horizontal,
    draw=#2,
    rectangle split parts=#1,
    fill=#2!20,
    outer sep=2mm},
  sal/.style={
    circle, 
    minimum width=8mm,
    outer sep=2mm,
    draw=#1, 
    fill=#1!20},
]

  \def\stepsize{2}%
  \def\lvlbase{0}%
  \def\lvlheight{3}%
 

    % Sentence Embeddings    
  \foreach \step in {1,...,3} {
    \node[hid={3}{sentemb}] (s\step) at (\stepsize*\step, \lvlbase) {};    
    \node at (\stepsize*\step, \lvlbase) {$\sentEmb_\step$};    
   }
%    \foreach \step [count=\i from 1] in {5,6} {
%        \node[hid={3}{green}] (s\step) at (\stepsize*\step, \lvlbase) {};    
%        \node at (\stepsize*\step, \lvlbase) {$\sentEmb_\i$};    
%    %\draw[->] (i\step.north) -> (e\step.south);
%    }
%
%       \node[hid={3}{red}] (s4) at (\stepsize*4, \lvlbase) {};    
%       \node at (\stepsize*4, \lvlbase) {$\sentEmb_0$};    
%
%    % RNN hidden states
    \foreach \step in {1,...,3} {
        \node[hid={3}{rencemb}] (rrnn_\step) 
            at (\stepsize *\step-0.5, \lvlbase + \lvlheight) {};    
        \node at (\stepsize *\step-0.5, \lvlbase + \lvlheight) 
            {$\rnnextRHid_\step$}; 
        \node[hid={3}{lencemb}] (lrnn_\step) 
            at (\stepsize *\step+0.5, \lvlbase + \lvlheight + 1.0) {};    
        \node at (\stepsize *\step+0.5, \lvlbase + \lvlheight+ 1.0) 
            {$\rnnextLHid_\step$}; 
        \draw[dep] (s\step.north) -- (rrnn_\step.south);
        \draw[dep] (s\step.north) -- (lrnn_\step.south);


        \node[hid={3}{ctxemb}] (ctx_\step) 
            at (\stepsize *\step, \lvlbase + 2.25*\lvlheight) {};    
        \node at (\stepsize *\step, \lvlbase + 2.25*\lvlheight) 
            {$\rnnextHid_\step$}; 
        \draw[dep] (rrnn_\step.north) -- (ctx_\step.south);
        \draw[dep] (lrnn_\step.north) -- (ctx_\step.south);


        \node[sal={sal}] (sal_\step) 
            at (\stepsize *\step, \lvlbase + 3*\lvlheight) {};    
        \node at (\stepsize *\step, \lvlbase + 3*\lvlheight) 
            {$\psal_i$}; 
        \draw[dep] (ctx_\step.north) -- (sal_\step.south);
        %\draw[dep] (lrnn_\step.north) -- (ctx_\step.south);


    }
    \foreach \start [count=\stop from 2] in {1,...,2} {
        \draw[dep] ($ (rrnn_\start.east) - (0,0.3)$) 
            -- ($ (rrnn_\stop.west) - (0,0.3) $);
        \draw[dep] ($(lrnn_\stop.west) + (0,0.3)$) 
            -- ($ (lrnn_\start.east) + (0,0.3)   $);
    }

    \draw[rectangle,draw=black,dotted] 
        (\stepsize*-2.5,\lvlbase + 3.5*\lvlheight) -- 
        (\stepsize*3.5, \lvlbase + 3.5*\lvlheight) -- 
        (\stepsize*3.5, \lvlbase + 2.7*\lvlheight) --
        (\stepsize*-2.5, \lvlbase + 2.7*\lvlheight) --
        (\stepsize*-2.5, \lvlbase + 3.5*\lvlheight) ;

    \node[align=left,anchor=north west] 
        at (\stepsize * -2.5,\lvlbase + 3.5*\lvlheight) 
        {\textit{(c) Salience Estimates}};

    \draw[rectangle,draw=black,dotted] 
        (\stepsize*-2.5,\lvlbase + 2.6*\lvlheight) -- 
        (\stepsize*3.5, \lvlbase + 2.6*\lvlheight) -- 
        (\stepsize*3.5, \lvlbase + 1.8*\lvlheight) --
        (\stepsize*-2.5, \lvlbase + 1.8*\lvlheight) --
        (\stepsize*-2.5, \lvlbase + 2.6*\lvlheight) ;

    \node[align=left,anchor=north west] 
        at (\stepsize * -2.5,\lvlbase + 2.6*\lvlheight) 
        {\textit{(b) Contextual Sentence Embeddings}};

    \draw[rectangle,draw=black,dotted] 
        (\stepsize*-2.5,\lvlbase + 0.5*\lvlheight) -- 
        (\stepsize*3.5, \lvlbase + 0.5*\lvlheight) -- 
        (\stepsize*3.5, \lvlbase + -0.50*\lvlheight) --
        (\stepsize*-2.5, \lvlbase + -0.50*\lvlheight) --
        (\stepsize*-2.5, \lvlbase + 0.5*\lvlheight) ;

    \node[align=left,anchor=north west] 
        at (\stepsize * -2.5,\lvlbase + 0.5*\lvlheight) 
        {\textit{Sentence Embeddings}\\\textit{(Sentence Encoder Output)}};


    \draw[rectangle,draw=black,dotted] 
        (\stepsize*-2.5,\lvlbase + 1.7*\lvlheight) -- 
        (\stepsize*3.5, \lvlbase + 1.7*\lvlheight) -- 
        (\stepsize*3.5, \lvlbase + 0.6*\lvlheight) --
        (\stepsize*-2.5, \lvlbase + 0.6*\lvlheight) --
        (\stepsize*-2.5, \lvlbase + 1.7*\lvlheight) ;


    \node[align=left,anchor=north west] 
        at (\stepsize * -2.5,\lvlbase + 1.7*\lvlheight) 
        {\textit{(a) Left and Right Partial}\\\textit{\phantom{(a) }Contexual Sentence Embeddings}};


\end{tikzpicture}}

\caption{Schematic for the \rnnext~sentence extractor.}
\label{fig:rnnext}
\end{minipage}}
\end{figure}


    Our first proposed model is a very simple 
    \bidirectional~\recurrentneuralnetwork~based tagging
    model \citep{sometagfolks}, which we refer to as the \rnnext~extractor.
See \autoref{fig:rnnext} for a visual depiction of   
the extractor.
As in the \srext~extractor, the first step of the \rnnext~extractor
is to run a \bidirectional~\recurrentneuralnetwork~over the sentence 
embeddings produced by the sentence encoder layer, which produces
left and right partial contextual embeddings $\rnnextRHid_i$ and $\rnnextLHid_i$
respectively,

  \vspace{10pt} \noindent\textit{(Fig.~\ref{fig:rnnext}.a) Left and Right Partial Contexual Sentence Embeddings}
\begin{align}
    \rnnextRHid_0 = \zeroEmb, &\quad \rnnextLHid_{\docSize+1} = \zeroEmb, \\
    \forall i : \;\; i \in \{1,\ldots,\docSize\}& \nonumber \\
 \rnnextRHid_i &= \fgru\left(\sentEmb_i, \rnnextRHid_{i-1}; \rnnextRParams\right), \\
 \rnnextLHid_i &= \fgru\left(\sentEmb_i, \rnnextLHid_{i+1};\rnnextLParams\right), 
\end{align}
where $\rnnextRHid_i,\rnnextLHid_i \in \reals^{\rnnextRNNDim}$,
and $\rnnextRParams$ and $\rnnextLParams$ are the forward and backward
\gru~parameters.

%As in the \recurrentneuralnetwork~sentence encoder we use a \gru~cell
%to implement the forward and backward \recurrentneuralnetwork s.
The left and right partial contextual embeddings of each sentence 
are then passed through a \feedforward~layer to produce contextual
sentence embeddings $\rnnextHid_i$,

\vspace{10pt}\noindent\textit{(Fig.~\ref{fig:rnnext}.b) Contextual Sentence Embeddings}
\begin{align}
    \forall i :\;\; i \in \{1,\ldots,\docSize\}& \nonumber \\
   \rnnextHid_i &= \relu\left(
    \rnnextHidWeight
    \left[ \begin{array}{c} 
        \rnnextRHid_i \\
        \rnnextLHid_i \end{array}\right] + \rnnextHidBias \right),
 %p(\bsal_i=1|\sentEmb_i,\ldots,\sentEmb_\docSize) &= \sigma\left(\rnnextPredWeight\rnnextHid_i + \rnnextPredBias  \right)
\end{align}
where $\rnnextHidWeight \in \reals^{\rnnextHidDim \times 2 \rnnextRNNDim}$
and $\rnnextHidBias \in \reals^{\rnnextHidDim}$ are learned parameters.

Another \feedforward~layer
with a logsitic sigmoid activation computes the actual salience 
estimates $\psal_1,\ldots,\psal_\docSize$ where $\psal_i = \model(\bsal_i=1|\sentEmb_1,\ldots,\sentEmb_\docSize;\xParams)$ and 

\vspace{10pt}\noindent\textit{(Fig.~\ref{fig:rnnext}.c) Salience Estimates}
\begin{align}
    \forall i : \;\; i \in \{1,\ldots,\docSize\}& \nonumber \\
    \psal_i &= \model(\bsal_i=1|\sentEmb_i,\ldots,\sentEmb_\docSize; \xParams) = \sigma\left(\rnnextPredWeight\rnnextHid_i + \rnnextPredBias  \right)
\end{align}
where $\rnnextPredWeight \in \reals^{1 \times \rnnextHidDim}$
and $\rnnextPredBias \in \reals$ are learned parameters.

The complete set of parameters for the extractor is 
\[ \xParams = \left\{\rnnextRParams, \rnnextLParams, \rnnextHidWeight, \rnnextHidBias, \rnnextPredWeight, \rnnextPredBias \right\}.\]
In our experiments, we set $\rnnextRNNDim=300$ and $\rnnextHidDim=100$.
Dropout with drop probability of $0.25$ is applied to $\rnnextRHid_i, \rnnextLHid_i,$ and $\rnnextHid_i$ for $i \in \{1,\ldots,\docSize\}$.

\FloatBarrier


