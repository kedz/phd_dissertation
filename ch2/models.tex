\section{Models}

We implement our salience estimation model $\model(\bsals|\doc;\params)$ 
hierarchically following the analogous structure of the documents we are 
modeling.  Every model $\model$ proposed in this chapter consists of three modules
or layers: \textit{(i)} the word embedding layer, \textit{(ii)} the sentence
encoder layer, and \textit{(iii)} the sentence extractor layer. 
The word embedding layer maps the words in a sentence to a sequence of 
word embeddings. The sentence encoder layer similarly maps sequences of 
word embeddings to a sentence embedding. Finally, the sentence extractor 
maps sequences of sentence embeddings to sequence of \salience~labels.

Choosing
an architecture for each of the modules defines the model. We define several
architectures for the sentence encoder and extractor layers and show how
particular settings of each correspond to prior summarization models proposed 
by \citet{cheng2016neural} and \citet{nallapati2017summarunner}. 
Additionally, we propose two novel sentence extractor layers, 
and in experiments 
consider all combinations of sentence encoder/extractor pairings.
%combinations of these components as well. 
In the next subsections, we 
describe each layer in more detail, and conclude ths section showing
how certain configurations of each layers maps to previously proposed or
novel \salience~estimation models and how the models generate an extract 
summary at test time (which we refer to as inference).

\subsection{Word Embedding Layer}

    The word embedding layer, $\embLayer(\cdot\,;\Lambda) : \wordVocab^* \rightarrow \mathbb{R}^{* \times \embDim}$ maps a sequence of words $\sent_i = \word_{i,1},\ldots,
\word_{i,\sentSize_i}$ to a sequence of word embeddings $\wordEmb_i = \wordEmb_{i,1},
\ldots, \wordEmb_{i,\sentSize_i} \in \mathbb{R}^{\sentSize_i \times \embDim}$
where the sole parameter $\Lambda \in \mathbb{R}^{|\wordVocab| \times \embDim}$ 
is a $\embDim$-dimensional embedding matrix and $\wordEmb_{i,j} = \Lambda_{\word_{i,j}}$ is the word embedding for word $\word_{i,j}$.
$\Lambda$ is initialized prior to training the full model with 
embeddings obtained using the unsupervised Global Vector (GloVe) embedding 
method on a large collection of text \citep{pennington2014glove}. 
Additionally, $\Lambda$ can be held fixed during training or updated 
with other model parameters. We use $\embDim = 200$-dimensional embeddings
in our models.







%For a typical deep learning model of sentence extractive 
%summarization there are two main design decisions:
%%At a high level, all the models considered in this paper share the same two part structure: 
%\textit{a)}  the choice of \textit{sentence encoder} 
%which maps each sentence $\sent_i$
%%(treated as a sequence of word embeddings) 
%to an embedding $\sentEmb_i$, 
%%\hal{notation class, you used $d$ already for number of sentences} 
%and 
%\textit{b)} the choice of \textit{sentence extractor} 
%which maps a sequence of sentence embeddings 
%$\sentEmb = \sentEmb_1,\ldots, \sentEmb_{\docSize}$  
%to a sequence of extraction
%decisions $\bsal = \bsal_1,\ldots,\bsal_{\docSize}$.
%%and predicts which sentences to extract to produce the 
%%extract summary. 
%


\begin{figure}
\begin{subfigure}{\textwidth}
  \centering
  %\includegraphics[width=.8\linewidth]{images/ch2/avgsentencoder.pdf}
  \includegraphics{images/ch2/avgsentencoder.pdf}
  \caption{Averaging Sentence Encoder}
  \label{fig:sfig1}
\end{subfigure}

\begin{subfigure}{\textwidth}
  \centering
  %\includegraphics[width=.8\linewidth]{images/ch2/cnnsentencoder.pdf}
  \includegraphics{images/ch2/rnnsentencoder.pdf}
  \caption{\RecurrentNeuralNetwork~Sentence Encoder}
  \label{fig:sfig2}
\end{subfigure}

\begin{subfigure}{\textwidth}
  \centering
  %\includegraphics[width=.8\linewidth]{images/ch2/cnnsentencoder.pdf}
  \includegraphics{images/ch2/cnnsentencoder.pdf}
  \caption{\convolutionalneuralnetwork~Sentence Encoder}
  \label{fig:sfig2}
\end{subfigure}
    \caption{Schematics for the averageing, \recurrentneuralnetwork,
    and \convolutionalneuralnetwork~sentence encoders.}
\label{fig:sentenceEncoders}
\end{figure}











\subsection{Sentence Encoders} \label{sec:senc}

    The sentence encoder layer, \sentEncFuncDef~maps a sequence of word 
    embeddings $\wordEmb_i = \wordEmb_{i,1},\ldots, \wordEmb_{i,\sentSize_i}$ 
    to a $\sentDim$-dimensional embedding representation of sentence $\sent_i$.
    The set of associated parameters, $\sentEncParams$, depends on the exact 
    architecture for implementing the encoder. We experiment with three 
    architectures for mapping sequences of word embeddings to a fixed length 
    vector: averaging, \recurrentneuralnetwork s, and 
    \convolutionalneuralnetwork s. We describe each variant now, and also 
    briefly discuss the tradeoffs associated with each architecture. Schematics
    of each encoder architecture can be found in 
    \autoref{fig:sentenceEncoders}.


\subsubsection{Averaging Sentence Encoder} 

    Under the averaging encoder, a sentence embedding $\sentEmb_i \in 
    \reals^{\sentDim}$ is simply the average of its word embeddings,
    \begin{align} 
    \sentEmb_i & = \sentEnc(\wordEmb_i;\sentEncParams) =  
        \frac{1}{\sentSize_i} \sum_{j=1}^{\sentSize_i} \wordEmb_{i,j}.
    \end{align}
    There are no parameters associated with this encoder (i.e. 
    $\sentEncParams = \emptyset$). The size of the sentence embedding is simply
$\sentDim = \embDim = 200$. Dropout with drop probability 0.25 is applied to each word embedding $\wordEmb_{i,j}$ during training.

\subsubsection{\RecurrentNeuralNetwork~Sentence Encoder} 

    When using the \recurrentneuralnetwork~encoder we apply both forward 
    and backward \recurrentneuralnetwork s over the word embedding sequences
    produced by the embedding layer. To obtain the actual sentence embedding, 
    we concate the final output step of
    the forward and backward networks. 
    For the actual recurrence function, we use the 
    \gatedrecurrentunit~\citep{cho2014gru}. 
    The \gatedrecurrentunit~function $\gruFuncDef{\embDim}{\hidDim}$~is defined as
\begin{align}
\fgru(\wordEmb, \sentEmb; \varphi) & 
    = (1-\GRUupdategate) \odot \GRUcandgate + \GRUupdategate \odot \sentEmb \label{eqn:gru}\\
    \textit{(Reset gate)} & \nonumber \\
\GRUresetgate &= 
    \sigma\left(
      \GRUWeight^{(\wordEmbChar\GRUresetchar)} \wordEmb 
        + \GRUBias^{(\wordEmbChar\GRUresetchar)} +
      \GRUWeight^{(\sentEmbChar\GRUresetchar)} \sentEmb 
        + \GRUBias^{(\sentEmbChar\GRUresetchar)} 
    \right)\\
    \textit{(Update gate)} & \nonumber \\
\GRUupdategate &= 
    \sigma\left(
      \GRUWeight^{(\wordEmbChar\GRUupdatechar)} \wordEmb 
        + \GRUBias^{(\wordEmbChar\GRUupdatechar)} +
      \GRUWeight^{(\sentEmbChar\GRUupdatechar)} \sentEmb 
        + \GRUBias^{(\sentEmbChar\GRUupdatechar)} 
    \right)\\
    \textit{(Candidate state)} & \nonumber \\
\GRUcandgate &= \tanh\left(
    \GRUWeight^{(\wordEmbChar\GRUcandchar)} \wordEmb 
        + \GRUBias^{(\wordEmbChar\GRUcandchar)} + 
    \GRUresetgate \odot \left(
        \GRUWeight^{(\sentEmbChar\GRUcandchar)} \sentEmb 
            + \GRUBias^{(\sentEmbChar\GRUcandchar)} 
    \right) \right)
\end{align}
where \[\GRUparams = \left\{
    \GRUWeight^{(\wordEmbChar\GRUresetchar)}, 
    \GRUBias^{(\wordEmbChar\GRUresetchar)}, 
    \GRUWeight^{(\sentEmbChar\GRUresetchar)}, 
    \GRUBias^{(\sentEmbChar\GRUresetchar)}, 
    \GRUWeight^{(\wordEmbChar\GRUupdatechar)}, 
    \GRUBias^{(\wordEmbChar\GRUupdatechar)}, 
    \GRUWeight^{(\sentEmbChar\GRUupdatechar)}, 
    \GRUBias^{(\sentEmbChar\GRUupdatechar)},
    \GRUWeight^{(\wordEmbChar\GRUcandchar)}, 
    \GRUBias^{(\wordEmbChar\GRUcandchar)}, 
    \GRUWeight^{(\sentEmbChar\GRUcandchar)}, 
    \GRUBias^{(\sentEmbChar\GRUcandchar)}
  \right\}\] is the set of \gru~parameters with $\GRUWeight^{(\wordEmbChar\cdot)} \in \reals^{\hidDim \times \embDim}$, 
    $\GRUWeight^{(\sentEmbChar\cdot)} \in \reals^{\hidDim \times \hidDim}$, 
    and $\GRUBias^{(\cdot)} \in \reals^{\hidDim }$, 
    and $\sigma(x) = \frac{1}{1+e^{-x}}$, 
    $\tanh(x) = \frac{e^x-1}{e^x+1}$, and $\odot$ is the Hadamard product.
 


    Under the \recurrentneuralnetwork~encoder, a sentence embedding 
    $\sentEmb_i$ is then defined as
    \begin{align} 
        \sentEmb_i 
        = \sentEnc\Big(\wordEmb_i; \sentEncParams\Big) & = \left[
                \rSentEmb_{i,\sentSize_i}, \lSentEmb_{i,1}
            \right] \\
            %\textit{(Forward RNN~encoder)} \nonumber\\
            \rSentEmb_{i,0}  &= \zeroEmb, \\ 
            \lSentEmb_{i,\sentSize_i + 1} &= \zeroEmb, \\
            \forall j \in \{1,\ldots,\sentSize_i\}& \nonumber \\  
           \rSentEmb_{i,j} & = 
                \fgru(\wordEmb_{i,j}, \rSentEmb_{i,j-1}; \rSentGRUParams), \\
%                 \textit{(Backward RNN~encoder)} \nonumber & \\
%     \lSentEmb_{i,\sentSize_i + 1} & = \zeroEmb, \\
%            \forall j \in \{1,\ldots,\sentSize_i\}& \nonumber \\  
           \lSentEmb_{i,j} &= 
                \fgru(\wordEmb_{i,j},\lSentEmb_{i,j+1};\lSentGRUParams ),
    %%\lSentVec_i &= \lgru(w_i, \lSentVec_{i+1}) 
    \end{align}
    where $[\cdots]$ is the vector concatenation operator and
    $\rSentGRUParams$ and $\lSentGRUParams$ are distinct parameters for 
    the forward and backward \recurrentneuralnetwork s respectively. 
    Collectively the set of parameters for the \recurrentneuralnetwork~sentence
    encoder is $\sentEncParams = \left\{ \rSentGRUParams, \lSentGRUParams
    \right\}$. We use $\hidDim=300$ dimensional hidden layers for each \gru, 
    making the size of the sentence embedding $\sentDim=2\hidDim=600$.
    Dropout with drop probability $0.25$ is applied to \gru~outputs $\rSentEmb_{i,j}$ and $\lSentEmb_{i,j}$ for 
$j \in \{1,\ldots,\sentSize_i\}$ during training.




\subsubsection{\ConvolutionalNeuralNetwork~Sentence Encoder} 

The \convolutionalneuralnetwork~sentence encoder uses a series of 
convolutional feature maps to encode each sentence. This encoder is similar
to the convolutional architecture of \citet{kim2014convolutional} used for 
text classification tasks. It performs a series of ``one-dimensional'' 
convolutions over word embeddings. The \kernelwidth~$\ckernelWidth \in 
\naturals$ of a feature map determines the number of contiguous words that a 
feature map is sensitive to. For $\ckernelWidth=3$, for example, the feature 
map would function as a trigram feature detector essentially. 
We denote a single convolutional feature map of kernel width 
$\ckernelWidth$ as $\ckernel_\ckernelWidth : \reals^{*\times \embDim} \rightarrow \reals$ with
\begin{align}
    \ckernel_\ckernelWidth(\wordEmb_i;\upsilon,\beta)  & 
= \max_{j \in \{ 
    1 - \left\lfloor \frac{\ckernelWidth}{2} \right\rfloor, 
    \ldots, \sentSize_i +  
    \left\lfloor \frac{\ckernelWidth}{2} \right\rfloor - \ckernelWidth + 1 \}}
  \relu\left( \cBias + \cMatrix^T \left[ \begin{array}{c} \wordEmb_{i,j}\\ \wordEmb_{i,j+1} \\ \vdots \\ \wordEmb_{i,j+k-1} \end{array} \right] \right),
\end{align}
where $\relu(x) = \max(0, x)$ is the rectified linear unit \citep{relu},
$\left\lfloor\cdot\right\rfloor$ is the floor operator,
and $\cMatrix \in \reals^{1 \times \ckernelWidth \embDim}$
and $\cBias \in \reals$ are parameters. Note that we use a ``zero-padded'' 
convolution \cite{cnnarithmatic}. That is, the $\max$ operator ranges
over  $j \in \left\{1 - \left\lfloor \frac{\ckernelWidth}{2} \right\rfloor, \ldots, 
    \sentSize_i +  \left\lfloor \frac{\ckernelWidth}{2} \right\rfloor - \ckernelWidth + 1
\right\}$ instead of $\left\{1,\ldots,\sentSize_i - \ckernelWidth + 1\right\}$, and $\wordEmb_{i,j} = \zeroEmb$ for
$j < 1$ and $j > \sentSize_i$. Padded convolutions help
alleviate the problem of reduced receptive fields on the boundaries 
of the sequence. See \autoref{fig:paddedconv} for a visual example.

The final sentence embedding $\sentEmb_i$ is 
a concatenation of many convolutional feature maps ranging over muliple kernel widths with each filter having its own distinct sets of parameters.
Let $\ckernelWidths = \{\ckernelWidth_1, \ldots, \ckernelWidth_m \} \subset \naturals$ be the set of the sentence encoder's
$m$ kernel widths, and $\cFeatureMaps_\ckernelWidth \in \naturals$ be the number of feature
maps for kernel width $\ckernelWidth$. The final sentence embedding produced
by the \convolutionalneuralnetwork~sentence encoder is defined as 
\begin{align}
\sentEmb_i & = \sentEnc(\wordEmb_i; \sentEncParams) = \left[  
    \ckernel^{\left(1\right)}_{\ckernelWidth_1},
  %  \ckernel^{(2)}_{\ckernelWidth_1},
    \ldots, 
    \ckernel^{\left(\cFeatureMaps_{\ckernelWidth_1}\right)}_{\ckernelWidth_1}, 
    \ckernel^{\left(1\right)}_{\ckernelWidth_2},
%    \ckernel^{(2)}_{\ckernelWidth_2},
    \ldots, 
    \ckernel^{\left(\cFeatureMaps_{\ckernelWidth_2}\right)}_{\ckernelWidth_2}, 
    \ldots, 
    \ckernel^{\left(1\right)}_{\ckernelWidth_m},
 %   \ckernel^{(2)}_{\ckernelWidth_m},
    \ldots, 
    \ckernel^{\left(\cFeatureMaps_{\ckernelWidth_m}\right)}_{\ckernelWidth_m}
  \right]  \\
    \ckernel^{(j)}_{\ckernelWidth} &=
\ckernel^{(j)}_{\ckernelWidth}(\wordEmb_i;\cMatrix^{(j,\ckernelWidth)},\cBias^{(j,\ckernelWidth)})
\end{align}
where \[ \sentEncParams = \left\{ 
    \cMatrix^{\left(1,\ckernelWidth_1\right)}, \cBias^{\left(1,\ckernelWidth_1\right)}, \ldots,
    \cMatrix^{\left(\cFeatureMaps_{\ckernelWidth_1},\ckernelWidth_1\right)}, 
            \cBias^{\left(\cFeatureMaps_{\ckernelWidth_1},\ckernelWidth_1\right)}, \ldots,
    \cMatrix^{\left(1,\ckernelWidth_m\right)}, \cBias^{\left(1,\ckernelWidth_m\right)}, \ldots,
    \cMatrix^{\left(\cFeatureMaps_{\ckernelWidth_m},\ckernelWidth_m\right)}, 
            \cBias^{\left(\cFeatureMaps_{\ckernelWidth_m},\ckernelWidth_m\right)}
    \right\}\] are the sentence encoder parameters. 
In our instantiation, we use kernel widths $\ckernelWidths = \{1, 2, 3, 4, 5, 6\}$ with corresponding feature maps sizes 
$\cFeatureMaps_1=25$, $\cFeatureMaps_2=25$, $\cFeatureMaps_3=50$, $\cFeatureMaps_4=50$, $\cFeatureMaps_5=50$, and $\cFeatureMaps_6=50$, making 
the resulting sentence embedding dimensionality $\sentDim=250$. 
Dropout with drop probability $0.25$ is also applied to $\sentEmb_i$ during 
training.


\subsubsection{Sentence Encoder Tradeoffs}


The sentence encoder's role is to obtain a vector representation of a finite
sequence of word embeddings that is useful for the sentence extraction
stage. Therefore it must aggregate features in the word embedding space that
are predictive of salience. Averaging embeddings is not an unreasonable
approach to this. Empirically there is evidence that word embedding
averaging is a fairly competitive sentence representation generally \citep{someone,someoneelse}. In the context of summarization, averaging can be thought of
as a noisy OR; if any of the words in a the sentence are indicative 
of salience, this representation should capture them.
Computationally, the averaging encoder is the fastest to compute and does
not require learning of parameters, reducing the memory and computation time
during training. 

The \recurrentneuralnetwork~sentence encoder can in theory 
capture some compositional features of a word sequence 
 that would be difficult or impossible 
to represent in the averaging encoder
(e.g. negation or coreference).
However, this comes at a much heavier 
computational cost, as \recurrentneuralnetwork s cannot be fully parallelized 
due to the inherently sequential nature of their computation. {\color{red} The \recurrentneuralnetwork~has $\mathcal{O}(?)$ number of parameters which
must be learned.}

The \convolutionalneuralnetwork~encoder represents a middle ground between
the averaging and \recurrentneuralnetwork~encoders. When using modestly sized
kernel widths (e.g., 1-5), the recepetive window should be sensitive short
phrases and some locally scoped negation. It will not be able to capture
the longer ranged dependencies that the \recurrentneuralnetwork~encoder would.
However, it is much faster to compute than the \recurrentneuralnetwork~as 
the individual feature maps can be computed completely in parallel. 
{\color{red} The \convolutionalneuralnetwork~has $\mathcal{O}(?)$ number of parameters which
must be learned.}










\subsection{Sentence Extractors} \label{sec:sext}

The role of the sentence extractor, 
$\sentExt(\cdot; \xParams) : \reals^{* \times \sentDim} \rightarrow \labelSpace$, is to map a
sequence of sentence 
embeddings 
    $\sentEmb_1,\ldots,\sentEmb_\docSize$ %h_1, ... , h_n%
produced by the sentence encoder layer to a sequence of salience judgements 
    $\bsals = \bsal_1,\ldots, \bsal_\docSize$. %y_1, ..., y_n.%
    The proposed sentence extractors do this by first implementing a 
    probability distribution over salience label sequences conditioned 
    on, $\model(\bsals|\sentEmb_1,\ldots,\sentEmb_\docSize; \xParams)$, and then inferring the (approximate) maximum likelilhood
    sequence $\predbsals \approx \argmax_{\bsals \in \labelSpace} \model(\bsals|\sentEmb_1,\ldots,\sentEmb_\docSize;\xParams)$.
%    a mapping of sentence embeddings to a probability
%The sentence extractor is essentially a discriminative classifier 
%   $p(\bsal_1,\ldots, \bsal_\docSize | \sentEmb_1, \ldots, \sentEmb_\docSize)$.
   %p(y_1,...,y_n|h_1,...,h_n).%
Previous neural network approaches to sentence extraction have assumed 
an \autoregressive~model, leading to a semi-Markovian factorization of the 
salience label distribution
  \[\model(\bsal_{1},\ldots,\bsal_\docSize|\sentEmb_1,\ldots,\sentEmb_\docSize;\xParams)=
      \prod_{i=1}^\docSize 
        \model(\bsal_i|\bsal_1,\ldots,\bsal_{i-1},\sentEmb_1,\ldots,\sentEmb_\docSize;\xParams),\]
where each prediction $\bsal_i$ is dependent on 
\emph{all} 
previous $\bsal_j$ for
all $j < i$. We compare two such models proposed by \citet{cheng2016neural}
and \citet{nallapati2017summarunner}. 

While intuitively it makes sense that previous extraction decisions might 
affect the probability of extracting subsequent sentences, (e.g., highly 
salient sentences might cluster together), it has not been empirically 
investigated whether this dependence is necessary for \deeplearning~models in 
practice. For example, in the models of 
\citet{cheng2016neural} and \citet{nallapati2017summarunner}, individual
predictions of $\bsal_i$ are made using information from some or all of 
the sentence embeddings $\sentEmb_1, \ldots, \sentEmb_\docSize$, such that 
information about neighboring sentences could be propagated through sentence
embedding interactions rather than on previous salience decisions.

The semi-Markovian factorization is not without
other consequences. Because intermediate sentence extractor representations
depend on prior extraction decisions $\bsal_i$, exact test time inference of
$\predbsals = \argmax_{\bsals \in \bSalSpace} 
\model(\bsals|\sentEmb_1,\ldots,\sentEmb_\docSize;\xParams)$
becomes intractable. Additionally, from an efficiency perspective, 
the semi-Markovian factorization prevents parallelization of individual
$\bsal_i$ predictions, since they must now be sequentially computed.

Motivated by these considerations, we propose two \nonautoregressive~sentence
extractor architectures where inidividual salience labels $\bsal_i$ 
are independent of each other, that is, 
\[\model(\bsal_1,\ldots,\bsal_\docSize|\sentEmb_1,\ldots,\sentEmb_\docSize;\xParams) = \prod_{i=1}^\docSize 
\model(\bsal_i|\sentEmb_1,\ldots,\sentEmb_\docSize;\xParams).\]
We now describe  the
previously proposed \autoregressive~sentence extractors (the \clext~extractor and  the \srext~extractor) before describing
our proposed \nonautoregressive~ones (the \rnnext~extractor and the \stsext~extractor).


%  which we explore in two proposed extractor models that we refer to as the 
%    \naseOne~and \naseTwo~sentence extractors.  
%Since the hidden state pre
%A simpler approach that does not allow interaction among the $y_{1:n}$
%is to 
%%\hal{a simpler approach (explain why simpler) is a fully factored representation 
%  model $p(\bsal_{1:n}|\sentEmb) = \prod_{i=1}^n p(y_i|h)$, 
%Implementation details for all extractors are in \autoref{app:sentextractors}.
%



%\begin{figure}
\begin{subfigure}{0.5\textwidth}
  \centering
  %\includegraphics[width=.8\linewidth]{images/ch2/avgsentencoder.pdf}
  \includegraphics{images/ch2/rnnextractor.pdf}
  \caption{Averaging Sentence Encoder}
  \label{fig:sfig1}
\end{subfigure}
\begin{subfigure}{0.5\textwidth}
  \centering
  %\includegraphics[width=.8\linewidth]{images/ch2/cnnsentencoder.pdf}
  \includegraphics{images/ch2/s2s_extractor.pdf}
  \caption{\RecurrentNeuralNetwork~Sentence Encoder}
  \label{fig:sfig2}
\end{subfigure}
    \caption{Schematics for the novel sentence extractors.}
\label{fig:sentenceEncoders}
\end{figure}



%\begin{figure}
\begin{subfigure}{0.5\textwidth}
  \centering
  %\includegraphics[width=.8\linewidth]{images/ch2/avgsentencoder.pdf}
  \includegraphics{images/ch2/clextractor.pdf}
  \caption{Averaging Sentence Encoder}
  \label{fig:sfig1}
\end{subfigure}
\begin{subfigure}{0.5\textwidth}
  \centering
  %\includegraphics[width=.8\linewidth]{images/ch2/cnnsentencoder.pdf}
  \includegraphics{images/ch2/sr_extractor.pdf}
  \caption{\RecurrentNeuralNetwork~Sentence Encoder}
  \label{fig:sfig2}
\end{subfigure}
    \caption{Schematics for the novel sentence extractors.}
\label{fig:sentenceEncoders}
\end{figure}




%
\begin{figure}[t]
    \fbox{\begin{minipage}{\textwidth}
\center
\scalebox{0.75}{
\begin{tikzpicture}[
  dep/.style ={
    ->,line width=0.3mm
  },
  hid/.style 2 args={
    rectangle split,
    draw=#2,
    rectangle split parts=#1,
    fill=#2!20,
    minimum width=5mm,
    minimum height=5mm,
    outer sep=2mm},
  mlp/.style 2 args={
    rectangle split,
    rectangle split horizontal,
    draw=#2,
    rectangle split parts=#1,
    fill=#2!20,
    outer sep=2mm},
  sal/.style={
    circle, 
    minimum width=8mm,
    outer sep=2mm,
    draw=#1, 
    fill=#1!20},
]

  \def\stepsize{2}%
  \def\lvlbase{0}%
  \def\lvlheight{3}%
 

    % Sentence Embeddings    
  \foreach \step in {1,...,3} {
    \node[hid={3}{sentemb}] (s\step) at (\stepsize*\step, \lvlbase) {};    
    \node at (\stepsize*\step, \lvlbase) {$\sentEmb_\step$};    
   }
%    \foreach \step [count=\i from 1] in {5,6} {
%        \node[hid={3}{green}] (s\step) at (\stepsize*\step, \lvlbase) {};    
%        \node at (\stepsize*\step, \lvlbase) {$\sentEmb_\i$};    
%    %\draw[->] (i\step.north) -> (e\step.south);
%    }
%
%       \node[hid={3}{red}] (s4) at (\stepsize*4, \lvlbase) {};    
%       \node at (\stepsize*4, \lvlbase) {$\sentEmb_0$};    
%
%    % RNN hidden states
    \foreach \step in {1,...,3} {
        \node[hid={3}{rencemb}] (rrnn_\step) 
            at (\stepsize *\step-0.5, \lvlbase + \lvlheight) {};    
        \node at (\stepsize *\step-0.5, \lvlbase + \lvlheight) 
            {$\rnnextRHid_\step$}; 
        \node[hid={3}{lencemb}] (lrnn_\step) 
            at (\stepsize *\step+0.5, \lvlbase + \lvlheight + 1.0) {};    
        \node at (\stepsize *\step+0.5, \lvlbase + \lvlheight+ 1.0) 
            {$\rnnextLHid_\step$}; 
        \draw[dep] (s\step.north) -- (rrnn_\step.south);
        \draw[dep] (s\step.north) -- (lrnn_\step.south);


        \node[hid={3}{ctxemb}] (ctx_\step) 
            at (\stepsize *\step, \lvlbase + 2.25*\lvlheight) {};    
        \node at (\stepsize *\step, \lvlbase + 2.25*\lvlheight) 
            {$\rnnextHid_\step$}; 
        \draw[dep] (rrnn_\step.north) -- (ctx_\step.south);
        \draw[dep] (lrnn_\step.north) -- (ctx_\step.south);


        \node[sal={sal}] (sal_\step) 
            at (\stepsize *\step, \lvlbase + 3*\lvlheight) {};    
        \node at (\stepsize *\step, \lvlbase + 3*\lvlheight) 
            {$\psal_i$}; 
        \draw[dep] (ctx_\step.north) -- (sal_\step.south);
        %\draw[dep] (lrnn_\step.north) -- (ctx_\step.south);


    }
    \foreach \start [count=\stop from 2] in {1,...,2} {
        \draw[dep] ($ (rrnn_\start.east) - (0,0.3)$) 
            -- ($ (rrnn_\stop.west) - (0,0.3) $);
        \draw[dep] ($(lrnn_\stop.west) + (0,0.3)$) 
            -- ($ (lrnn_\start.east) + (0,0.3)   $);
    }

    \draw[rectangle,draw=black,dotted] 
        (\stepsize*-2.5,\lvlbase + 3.5*\lvlheight) -- 
        (\stepsize*3.5, \lvlbase + 3.5*\lvlheight) -- 
        (\stepsize*3.5, \lvlbase + 2.7*\lvlheight) --
        (\stepsize*-2.5, \lvlbase + 2.7*\lvlheight) --
        (\stepsize*-2.5, \lvlbase + 3.5*\lvlheight) ;

    \node[align=left,anchor=north west] 
        at (\stepsize * -2.5,\lvlbase + 3.5*\lvlheight) 
        {\textit{(c) Salience Estimates}};

    \draw[rectangle,draw=black,dotted] 
        (\stepsize*-2.5,\lvlbase + 2.6*\lvlheight) -- 
        (\stepsize*3.5, \lvlbase + 2.6*\lvlheight) -- 
        (\stepsize*3.5, \lvlbase + 1.8*\lvlheight) --
        (\stepsize*-2.5, \lvlbase + 1.8*\lvlheight) --
        (\stepsize*-2.5, \lvlbase + 2.6*\lvlheight) ;

    \node[align=left,anchor=north west] 
        at (\stepsize * -2.5,\lvlbase + 2.6*\lvlheight) 
        {\textit{(b) Contextual Sentence Embeddings}};

    \draw[rectangle,draw=black,dotted] 
        (\stepsize*-2.5,\lvlbase + 0.5*\lvlheight) -- 
        (\stepsize*3.5, \lvlbase + 0.5*\lvlheight) -- 
        (\stepsize*3.5, \lvlbase + -0.50*\lvlheight) --
        (\stepsize*-2.5, \lvlbase + -0.50*\lvlheight) --
        (\stepsize*-2.5, \lvlbase + 0.5*\lvlheight) ;

    \node[align=left,anchor=north west] 
        at (\stepsize * -2.5,\lvlbase + 0.5*\lvlheight) 
        {\textit{Sentence Embeddings}\\\textit{(Sentence Encoder Output)}};


    \draw[rectangle,draw=black,dotted] 
        (\stepsize*-2.5,\lvlbase + 1.7*\lvlheight) -- 
        (\stepsize*3.5, \lvlbase + 1.7*\lvlheight) -- 
        (\stepsize*3.5, \lvlbase + 0.6*\lvlheight) --
        (\stepsize*-2.5, \lvlbase + 0.6*\lvlheight) --
        (\stepsize*-2.5, \lvlbase + 1.7*\lvlheight) ;


    \node[align=left,anchor=north west] 
        at (\stepsize * -2.5,\lvlbase + 1.7*\lvlheight) 
        {\textit{(a) Left and Right Partial}\\\textit{\phantom{(a) }Contexual Sentence Embeddings}};


\end{tikzpicture}}

\caption{Schematic for the \rnnext~sentence extractor.}
\label{fig:rnnext}
\end{minipage}}
\end{figure}

%\begin{figure}[H!]
\center
\begin{tikzpicture}[
  dep/.style ={
    ->,line width=0.3mm
  },
  hid/.style 2 args={
    rectangle split,
    draw=#2,
    rectangle split parts=#1,
    fill=#2!20,
    minimum width=5mm,
    minimum height=5mm,
    outer sep=2mm},
  mlp/.style 2 args={
    rectangle split,
    rectangle split horizontal,
    draw=#2,
    rectangle split parts=#1,
    fill=#2!20,
    outer sep=2mm},
  sal/.style={
    circle, 
    minimum width=8mm,
    outer sep=2mm,
    draw=#1, 
    fill=#1!20},
]

  \def\stepsize{1.5}%
  \def\lvlbase{0}%
  \def\lvlheight{3}%
 

    % Sentence Embeddings    
  \foreach \step in {1,...,3} {
    \node[hid={3}{sentemb}] (s\step) at (\stepsize*\step, \lvlbase) {};    
    \node at (\stepsize*\step, \lvlbase) {$\sentEmb_\step$};    
   }
%    % RNN hidden states
    \foreach \step in {1,...,3} {
        \node[hid={3}{encemb}] (rrnn_\step) 
            at (\stepsize *\step-0.5, \lvlbase + \lvlheight) {};    
        \node at (\stepsize *\step-0.5, \lvlbase + \lvlheight) 
            {$\rnnextRHid_\step$}; 
        \node[hid={3}{yellow}] (lrnn_\step) 
            at (\stepsize *\step+0.5, \lvlbase + \lvlheight + 1.0) {};    
        \node at (\stepsize *\step+0.5, \lvlbase + \lvlheight+ 1.0) 
            {$\rnnextLHid_\step$}; 
        \draw[dep] (s\step.north) -- (rrnn_\step.south);
        \draw[dep] (s\step.north) -- (lrnn_\step.south);


%        \node[hid={3}{orange}] (ctx_\step) 
%            at (\stepsize *\step, \lvlbase + 2.25*\lvlheight) {};    
%        \node at (\stepsize *\step, \lvlbase + 2.25*\lvlheight) 
%            {$\rnnextHid_\step$}; 
%        \draw[dep] (rrnn_\step.north) -- (ctx_\step.south);
%        \draw[dep] (lrnn_\step.north) -- (ctx_\step.south);
%
%
%        \node[sal={blue}] (sal_\step) 
%            at (\stepsize *\step, \lvlbase + 3*\lvlheight) {};    
%        \node at (\stepsize *\step, \lvlbase + 3*\lvlheight) 
%            {$\bsal_i$}; 
%        \draw[dep] (ctx_\step.north) -- (sal_\step.south);
        %\draw[dep] (lrnn_\step.north) -- (ctx_\step.south);


    }
  \def\stepsize{1.5}%
  \def\lvlbase{0}%
  \def\lvlheight{3}%
 

    \foreach \step in {1,...,3} {
        \node[hid={3}{yellow}] (rrnn_\step) 
            at (\stepsize *\step-0.35 + 5, \lvlbase + 0*\lvlheight) {};    
        \node at (\stepsize*\step-0.35 + 5, \lvlbase + 0*\lvlheight) 
            {$\rnnextRHid_\step$}; 
        \node[hid={3}{yellow}] (lrnn_\step) 
            at (\stepsize *\step+5.35, \lvlbase + 0*\lvlheight + 0.5) {};    
        \node at (\stepsize *\step+5.35, \lvlbase + 0*\lvlheight+ 0.5) 
            {$\rnnextLHid_\step$}; 
    }

    \foreach \step in {1,...,3} {
        \node[hid={3}{yellow}] (rrnn_\step) 
            at (\stepsize *\step-0.35 + 10, \lvlbase + 0*\lvlheight) {};    
        \node at (\stepsize*\step-0.35 + 10, \lvlbase + 0*\lvlheight) 
            {$\rnnextRHid_\step$}; 
        \node[hid={3}{yellow}] (lrnn_\step) 
            at (\stepsize *\step+10.35, \lvlbase + 0*\lvlheight + 0.5) {};    
        \node at (\stepsize *\step+10.35, \lvlbase + 0*\lvlheight+ 0.5) 
            {$\rnnextLHid_\step$}; 
    }

    \foreach \step in {1,...,3} {
        \node[hid={3}{yellow}] (ctx_\step) 
            at (\stepsize *\step + 5, \lvlbase + 1*\lvlheight) {};    
        \node at (\stepsize*\step + 5, \lvlbase + 1*\lvlheight) 
            {$\srHid_\step$}; 
    }

        \node[hid={3}{yellow}] (doc) 
            at (\stepsize *2 + 10, \lvlbase + 1*\lvlheight) {};    
        \node at (\stepsize*2 + 10, \lvlbase + 1*\lvlheight) 
            {$\srDocEmb$}; 


        \node[hid={3}{yellow}] (ctx_i) 
            at (\stepsize *1 , \lvlbase + -2*\lvlheight) {};    
        \node at (\stepsize*1, \lvlbase + -2*\lvlheight) 
            {$\srHid_i$}; 

        \node[hid={1}{yellow}] (content_i) 
            at (\stepsize *1, \lvlbase + -1*\lvlheight) {};    
        \node at (\stepsize*1, \lvlbase + -1*\lvlheight) 
            {$\srContentFactor_i$}; 

        \node[hid={3}{yellow}] (ctx_i) 
            at (\stepsize *3 , \lvlbase + -2*\lvlheight) {};    
        \node at (\stepsize*3, \lvlbase + -2*\lvlheight) 
            {$\srHid_i$}; 

        \node[hid={1}{yellow}] (salience_i) 
            at (\stepsize *3.5, \lvlbase + -1*\lvlheight) {};    
        \node at (\stepsize*3.5, \lvlbase + -1*\lvlheight) 
            {$\srSalienceFactor_i$}; 
        \node[hid={3}{yellow}] (doc) 
            at (\stepsize *4, \lvlbase + -2*\lvlheight) {};    
        \node at (\stepsize*4, \lvlbase + -2*\lvlheight) 
            {$\srDocEmb$}; 

        \node[hid={3}{yellow}] (ctx_i) 
            at (\stepsize *6 , \lvlbase + -2*\lvlheight) {};    
        \node at (\stepsize*6, \lvlbase + -2*\lvlheight) 
            {$\srHid_i$}; 

        \node[hid={1}{yellow}] (finepos_i) 
            at (\stepsize *6, \lvlbase + -1*\lvlheight) {};    
        \node at (\stepsize*6, \lvlbase + -1*\lvlheight) 
            {$\srFinePositionFactor_i$}; 

        \node[hid={3}{yellow}] (ctx_i) 
            at (\stepsize *8 , \lvlbase + -2*\lvlheight) {};    
        \node at (\stepsize*8, \lvlbase + -2*\lvlheight) 
            {$\srHid_i$}; 
        \node[hid={1}{yellow}] (coarsepos_i) 
            at (\stepsize *8, \lvlbase + -1*\lvlheight) {};    
        \node at (\stepsize*8, \lvlbase + -1*\lvlheight) 
            {$\srCoarsePositionFactor_i$}; 

    %    \node[hid={3}{yellow}] (ctx_i) 
    %        at (\stepsize *1 , \lvlbase + -5*\lvlheight) {};    
    %    \node at (\stepsize*1, \lvlbase + -5*\lvlheight) 
    %        {$\srHid_i$}; 

        \node[hid={3}{green}] (summary_1) 
            at (\stepsize *1 , \lvlbase + -4*\lvlheight) {};    
        \node at (\stepsize*1, \lvlbase + -4*\lvlheight) 
            {$\srSum_1$}; 

        \node[hid={3}{yellow}] (ctx_1) 
            at (\stepsize *2 , \lvlbase + -4*\lvlheight) {};    
        \node at (\stepsize*2, \lvlbase + -4*\lvlheight) 
            {$\srHid_1$}; 

            \node[hid={1}{red}] (salience_1) 
            at (\stepsize *2 , \lvlbase + -3*\lvlheight) {};    
        \node at (\stepsize*2, \lvlbase + -3*\lvlheight) 
            {$\srSalienceFactor_1$}; 


        \draw[dep] (summary_1.north) to (salience_1.south);
        \draw[dep] (ctx_1.north) to (salience_1.south);


        \node[hid={3}{yellow}] (ctx_1) 
            at (\stepsize *3 , \lvlbase + -5*\lvlheight) {};    
        \node at (\stepsize*3, \lvlbase + -5*\lvlheight) 
            {$\srHid_1$}; 

        \node[sal={blue}] (p_1) 
            at (\stepsize *4 , \lvlbase + -5*\lvlheight) {};    
        \node at (\stepsize*4, \lvlbase + -5*\lvlheight) 
            {$\bsal_1$}; 

        \node[hid={3}{green}] (summary_2) 
            at (\stepsize *4 , \lvlbase + -4*\lvlheight) {};    
        \node at (\stepsize*4, \lvlbase + -4*\lvlheight) 
            {$\srSum_2$}; 

        \node[hid={3}{yellow}] (ctx_2) 
            at (\stepsize *5 , \lvlbase + -4*\lvlheight) {};    
        \node at (\stepsize*5, \lvlbase + -4*\lvlheight) 
            {$\srHid_2$}; 

            \node[hid={1}{red}] (salience_2) 
            at (\stepsize *5 , \lvlbase + -3*\lvlheight) {};    
        \node at (\stepsize*5, \lvlbase + -3*\lvlheight) 
            {$\srSalienceFactor_2$}; 


        \draw[dep] (summary_2.north) to (salience_2.south);
        \draw[dep] (ctx_2.north) to (salience_2.south);


        \draw[dep] (ctx_1.north) to (summary_2.south);
        \draw[dep] (p_1.north) to (summary_2.south);








        \node[hid={3}{yellow}] (ctx_1) 
            at (\stepsize *6 , \lvlbase + -5*\lvlheight) {};    
        \node at (\stepsize*6, \lvlbase + -5*\lvlheight) 
            {$\srHid_1$}; 

        \node[sal={blue}] (p_1) 
            at (\stepsize *7 , \lvlbase + -5*\lvlheight) {};    
        \node at (\stepsize*7, \lvlbase + -5*\lvlheight) 
            {$\bsal_1$}; 

        \node[hid={3}{yellow}] (ctx_2) 
            at (\stepsize *8 , \lvlbase + -5*\lvlheight) {};    
        \node at (\stepsize*8, \lvlbase + -5*\lvlheight) 
            {$\srHid_2$}; 

        \node[sal={blue}] (p_2) 
            at (\stepsize *9 , \lvlbase + -5*\lvlheight) {};    
        \node at (\stepsize*9, \lvlbase + -5*\lvlheight) 
            {$\bsal_2$}; 



        \node[hid={3}{green}] (summary_3) 
            at (\stepsize *9 , \lvlbase + -4*\lvlheight) {};    
        \node at (\stepsize*9, \lvlbase + -4*\lvlheight) 
            {$\srSum_3$}; 

        \node[hid={3}{yellow}] (ctx_3) 
            at (\stepsize *10 , \lvlbase + -4*\lvlheight) {};    
        \node at (\stepsize*10, \lvlbase + -4*\lvlheight) 
            {$\srHid_3$}; 

            \node[hid={1}{red}] (salience_3) 
            at (\stepsize *10 , \lvlbase + -3*\lvlheight) {};    
        \node at (\stepsize*10, \lvlbase + -3*\lvlheight) 
            {$\srSalienceFactor_3$}; 


        \draw[dep] (ctx_1.north) to (summary_3.south);
        \draw[dep] (p_1.north) to (summary_3.south);

        \draw[dep] (ctx_2.north) to (summary_3.south);
        \draw[dep] (p_2.north) to (summary_3.south);

        \draw[dep] (summary_3.north) to (salience_3.south);
        \draw[dep] (ctx_3.north) to (salience_3.south);













%        \node[hid={3}{yellow}] (ctx_1) 
%            at (\stepsize *4 , \lvlbase + -5*\lvlheight) {};    
%        \node at (\stepsize*4, \lvlbase + -5*\lvlheight) 
%            {$\srHid_1$}; 
%        \node[hid={3}{yellow}] (ctx_2) 
%            at (\stepsize *5 , \lvlbase + -5*\lvlheight) {};    
%        \node at (\stepsize*5, \lvlbase + -5*\lvlheight) 
%            {$\srHid_2$}; 
%
%
%

%    \foreach \start [count=\stop from 2] in {1,...,2} {
%        \draw[dep] ($ (rrnn_\start.east) - (0,0.3)$) 
%            -- ($ (rrnn_\stop.west) - (0,0.3) $);
%        \draw[dep] ($(lrnn_\stop.west) + (0,0.3)$) 
%            -- ($ (lrnn_\start.east) + (0,0.3)   $);
%    }

%    \draw[rectangle,draw=black,dotted] 
%        (\stepsize*-2.5,\lvlbase + 3.5*\lvlheight) -- 
%        (\stepsize*3.5, \lvlbase + 3.5*\lvlheight) -- 
%        (\stepsize*3.5, \lvlbase + 2.7*\lvlheight) --
%        (\stepsize*-2.5, \lvlbase + 2.7*\lvlheight) --
%        (\stepsize*-2.5, \lvlbase + 3.5*\lvlheight) ;
%
%    \node[align=left,anchor=north west] 
%        at (\stepsize * -2.5,\lvlbase + 3.5*\lvlheight) 
%        {\textit{Salience Estimates}};
%
%    \draw[rectangle,draw=black,dotted] 
%        (\stepsize*-2.5,\lvlbase + 2.6*\lvlheight) -- 
%        (\stepsize*3.5, \lvlbase + 2.6*\lvlheight) -- 
%        (\stepsize*3.5, \lvlbase + 1.8*\lvlheight) --
%        (\stepsize*-2.5, \lvlbase + 1.8*\lvlheight) --
%        (\stepsize*-2.5, \lvlbase + 2.6*\lvlheight) ;
%
%    \node[align=left,anchor=north west] 
%        at (\stepsize * -2.5,\lvlbase + 2.6*\lvlheight) 
%        {\textit{Contextual Sentence Embeddings}};
%
%    \draw[rectangle,draw=black,dotted] 
%        (\stepsize*-2.5,\lvlbase + 0.5*\lvlheight) -- 
%        (\stepsize*3.5, \lvlbase + 0.5*\lvlheight) -- 
%        (\stepsize*3.5, \lvlbase + -0.50*\lvlheight) --
%        (\stepsize*-2.5, \lvlbase + -0.50*\lvlheight) --
%        (\stepsize*-2.5, \lvlbase + 0.5*\lvlheight) ;
%
%    \node[align=left,anchor=north west] 
%        at (\stepsize * -2.5,\lvlbase + 0.5*\lvlheight) 
%        {\textit{Sentence Embeddings}\\\textit{(Sentence Encoder Output)}};
%
%
%    \draw[rectangle,draw=black,dotted] 
%        (\stepsize*-2.5,\lvlbase + 1.7*\lvlheight) -- 
%        (\stepsize*3.5, \lvlbase + 1.7*\lvlheight) -- 
%        (\stepsize*3.5, \lvlbase + 0.6*\lvlheight) --
%        (\stepsize*-2.5, \lvlbase + 0.6*\lvlheight) --
%        (\stepsize*-2.5, \lvlbase + 1.7*\lvlheight) ;
%
%
%    \node[align=left,anchor=north west] 
%        at (\stepsize * -2.5,\lvlbase + 1.7*\lvlheight) 
%        {\textit{Forward and Backward Partial}\\\textit{Contexual Sentence Embeddings}};


\end{tikzpicture}


\end{figure}

%\begin{figure}[H!]
\center
\begin{tikzpicture}[
  dep/.style ={
    ->,line width=0.3mm
  },
  hid/.style 2 args={
    rectangle split,
    draw=#2,
    rectangle split parts=#1,
    fill=#2!20,
    minimum width=5mm,
    minimum height=5mm,
    outer sep=2mm},
  mlp/.style 2 args={
    rectangle split,
    rectangle split horizontal,
    draw=#2,
    rectangle split parts=#1,
    fill=#2!20,
    outer sep=2mm},
  sal/.style={
    circle, 
    minimum width=8mm,
    outer sep=2mm,
    draw=#1, 
    fill=#1!20},
]

  \def\stepsize{2}%
  \def\lvlbase{0}%
  \def\lvlheight{3}%
 

    % Sentence Embeddings    
  \foreach \step in {1,...,3} {
    \node[hid={3}{green}] (s\step) at (\stepsize*\step, \lvlbase) {};    
    \node at (\stepsize*\step, \lvlbase) {$\sentEmb_\step$};    
   }
%    \foreach \step [count=\i from 1] in {5,...,7} {
%        \node[hid={3}{green}] (s\step) at (\stepsize*\step, \lvlbase) {};    
%        \node at (\stepsize*\step, \lvlbase) {$\sentEmb_\i$};    
%    %\draw[->] (i\step.north) -> (e\step.south);
%    }

%       \node[hid={3}{red}] (s4) at (\stepsize*4, \lvlbase) {};    
%       \node at (\stepsize*4, \lvlbase) {$\sentEmb_0$};    
%
%    % RNN hidden states
    \foreach \step in {1,...,3} {
        \node[hid={3}{yellow}] (rrnn_\step) 
            at (\stepsize *\step+0.5, \lvlbase + \lvlheight) {};    
        \node at (\stepsize *\step+0.5, \lvlbase + \lvlheight) 
            {$\stsextREncHid_\step$}; 

        \node[hid={3}{yellow}] (lrnn_\step) 
            at (\stepsize *\step-0.5, \lvlbase + \lvlheight+1.0) {};    
        \node at (\stepsize *\step-0.5, \lvlbase + \lvlheight+1.0) 
            {$\stsextLEncHid_\step$}; 


        \node[hid={3}{yellow}] (enc_ctx_\step) 
            at (\stepsize *\step, \lvlbase + 2.5*\lvlheight) {};    
        \node at (\stepsize *\step, \lvlbase + 2.5*\lvlheight) 
            {$\stsextEncHid_\step$}; 

        \draw[dep] (s\step.north) -- (lrnn_\step.south);
        \draw[dep] (s\step.north) -- (rrnn_\step.south);

        \draw[dep] (lrnn_\step.north) -- (enc_ctx_\step.south);
        \draw[dep] (rrnn_\step.north) -- (enc_ctx_\step.south);
    }
    \foreach \start [count=\stop from 2] in {1,...,2} {
        \draw[dep] ($ (rrnn_\start.east) - (0,0.3)$) 
            -- ($ (rrnn_\stop.west) - (0,0.3) $);
        \draw[dep] ($(lrnn_\stop.west) + (0,0.3)$) 
            -- ($ (lrnn_\start.east) + (0,0.3)   $);
    }




%    \foreach \step [count=\i from 0] in {4,...,7} {
%        \node[hid={3}{orange}] (rnn_\step) 
%            at (\stepsize *\step, \lvlbase + \lvlheight) {};    
%        \node at (\stepsize *\step, \lvlbase + \lvlheight) {$\xDecHid_\i$}; 
%    }
%
%    \foreach \step in {1,...,6} {
%        \draw[dep] (s\step.north) to (rnn_\step.south);
%    }
%    \foreach \start [count=\stop from 2] in {1,...,5} {
%        \draw[dep] (rnn_\start.east) to (rnn_\stop.west);
%    }
%
%
%%    \foreach \step [count=\i from 1] in {4,...,6} {
%%        \node[hid={3}{green}] (ctx_\i) 
%%            at (\stepsize *\step, \lvlbase + 2*\lvlheight) {};    
%%        \node at (\stepsize *\step, \lvlbase + 2*\lvlheight) {$\xPredHid_\i$}; 
%%        \draw[dep] (rnn_\step.north) to (ctx_\i.south);
%%        \draw[dep] (rnn_\i.north) to (ctx_\i.south west);
%%        \node[sal={blue}] (sal_\i) at (\stepsize * \step,\lvlbase + 3*\lvlheight) {};
%%        \node at (\stepsize * \step,\lvlbase + 3*\lvlheight) {$\bsal_\i$};
%%        \draw[dep] (ctx_\i.north) to (sal_\i.south);
%%    }
%%
%%    \foreach \step [count=\i from 1] in {5,...,6} {
%%        \draw[dep] (sal_\i) -- (\stepsize * \step - \stepsize / 2,
%%                            \lvlbase + 3*\lvlheight) 
%%                     -- (\stepsize * \step - \stepsize / 2,
%%                            \lvlbase + 0* \lvlheight) -> (s\step) ;
%%
%%    }
%%    \draw[dep] (s1.south) -- ($ (s1.south) + (0,-0.3)$) --($ (s5.south) + (0,-0.3)$) -- (s5.south);
%%
%%    \draw[dep] (s2.south) -- ($ (s2.south) + (0,-0.5)$) --($ (s6.south) + (0,-0.5)$) -- (s6.south);
%%
%%    \draw[rectangle,draw=black,dotted] 
%%        (\stepsize * 3.5,\lvlbase + 3.5*\lvlheight) -- 
%%        (\stepsize*6.5, \lvlbase + 3.5*\lvlheight) -- 
%%        (\stepsize*6.5, \lvlbase + 2.6*\lvlheight) --
%%        (\stepsize*3.5, \lvlbase + 2.6*\lvlheight) --
%%        (\stepsize*3.5, \lvlbase + 3.5*\lvlheight) ;
%%
%%    \node[align=left,anchor=north west] 
%%        at (\stepsize * 3.5,\lvlbase + 3.5*\lvlheight) 
%%        {\textit{Salience Estimates}};
%%
%%    \draw[rectangle,draw=black,dotted] 
%%        (\stepsize*0.5,\lvlbase + 2.5*\lvlheight) -- 
%%        (\stepsize*6.5, \lvlbase + 2.5*\lvlheight) -- 
%%        (\stepsize*6.5, \lvlbase + 1.5*\lvlheight) --
%%        (\stepsize*0.5, \lvlbase + 1.5*\lvlheight) --
%%        (\stepsize*0.5, \lvlbase + 2.5*\lvlheight) ;
%%
%%    \node[align=left,anchor=north west] 
%%        at (\stepsize * 0.5,\lvlbase + 2.5*\lvlheight) 
%%        {\textit{Contextual Sentence Embeddings}};
%%
%%    \draw[rectangle,draw=black,dotted] 
%%        (\stepsize*-0.5,\lvlbase + 0.5*\lvlheight) -- 
%%        (\stepsize*3.5, \lvlbase + 0.5*\lvlheight) -- 
%%        (\stepsize*3.5, \lvlbase + -0.9*\lvlheight) --
%%        (\stepsize*-0.5, \lvlbase + -0.9*\lvlheight) --
%%        (\stepsize*-0.5, \lvlbase + 0.5*\lvlheight) ;
%%
%%    \node[align=left,anchor=south west] 
%%        at (\stepsize * -0.5,\lvlbase + -0.9*\lvlheight) 
%%        {\textit{Sentence Embeddings}\\\textit{(Sentence Encoder Output)}};
%%
%%    \draw[rectangle,draw=black,dotted] 
%%        (\stepsize*4.3,\lvlbase + 0.5*\lvlheight) -- 
%%        (\stepsize*6.5, \lvlbase + 0.5*\lvlheight) -- 
%%        (\stepsize*6.5, \lvlbase + -0.9*\lvlheight) --
%%        (\stepsize*4.3, \lvlbase + -0.9*\lvlheight) --
%%        (\stepsize*4.3, \lvlbase + 0.5*\lvlheight) ;
%%
%%    \node[align=left,anchor=south west] 
%%        at (\stepsize * 4.3,\lvlbase + -0.9*\lvlheight) 
%%        {\textit{Salience Gated}\\\textit{Sentence Embeddings}};
%%

    



\end{tikzpicture}

\begin{tikzpicture}[
  dep/.style ={
    ->,line width=0.3mm
  },
  hid/.style 2 args={
    rectangle split,
    draw=#2,
    rectangle split parts=#1,
    fill=#2!20,
    minimum width=5mm,
    minimum height=5mm,
    outer sep=2mm},
  mlp/.style 2 args={
    rectangle split,
    rectangle split horizontal,
    draw=#2,
    rectangle split parts=#1,
    fill=#2!20,
    outer sep=2mm},
  sal/.style={
    circle, 
    minimum width=8mm,
    outer sep=2mm,
    draw=#1, 
    fill=#1!20},
]

  \def\stepsize{2}%
  \def\lvlbase{0}%
  \def\lvlheight{3}%
 

    \node[hid={3}{green}] (s0) at (\stepsize*0, \lvlbase) {};    
    \node at (\stepsize*0, \lvlbase) {$\sentEmb_0$};    


    \node[hid={3}{green}] (s4) at (\stepsize*4, \lvlbase) {};    
    \node at (\stepsize*4, \lvlbase) {$\sentEmb_0$};    

    % Sentence Embeddings    
  \foreach \step in {1,...,3} {
    \node[hid={3}{green}] (s\step) at (\stepsize*\step, \lvlbase) {};    
    \node at (\stepsize*\step, \lvlbase) {$\sentEmb_\step$};    
   }
%    \foreach \step [count=\i from 1] in {5,...,7} {
%        \node[hid={3}{green}] (s\step) at (\stepsize*\step, \lvlbase) {};    
%        \node at (\stepsize*\step, \lvlbase) {$\sentEmb_\i$};    
%    %\draw[->] (i\step.north) -> (e\step.south);
%    }

%       \node[hid={3}{red}] (s4) at (\stepsize*4, \lvlbase) {};    
%       \node at (\stepsize*4, \lvlbase) {$\sentEmb_0$};    
%
%    % RNN hidden states

        \node[hid={3}{yellow}] (rrnn_0) 
            at (\stepsize *0+0.5, \lvlbase + \lvlheight) {};    
        \node at (\stepsize *0+0.5, \lvlbase + \lvlheight) 
            {$\stsextRDecHid_0$}; 

        \node[hid={3}{yellow}] (lrnn_4) 
            at (\stepsize *4-0.5, \lvlbase + \lvlheight+1.0) {};    
        \node at (\stepsize *4-0.5, \lvlbase + \lvlheight+1.0) 
            {$\stsextLDecHid_4$}; 

        \node[hid={3}{yellow}] (rrnn_m1) 
            at (-\stepsize +0.5, \lvlbase + \lvlheight) {};    
        \node at (-\stepsize +0.5, \lvlbase + \lvlheight) 
            {$\stsextREncHid_3$}; 

        \node[hid={3}{yellow}] (lrnn_m1) 
            at (\stepsize*5 -0.5, \lvlbase + \lvlheight+1.0) {};    
        \node at (\stepsize*5 -0.5, \lvlbase + \lvlheight+1.0) 
            {$\stsextLEncHid_1$}; 



    \foreach \step in {1,...,3} {
        \node[hid={3}{yellow}] (rrnn_\step) 
            at (\stepsize *\step+0.5, \lvlbase + \lvlheight) {};    
        \node at (\stepsize *\step+0.5, \lvlbase + \lvlheight) 
            {$\stsextRDecHid_\step$}; 

        \node[hid={3}{yellow}] (lrnn_\step) 
            at (\stepsize *\step-0.5, \lvlbase + \lvlheight+1.0) {};    
        \node at (\stepsize *\step-0.5, \lvlbase + \lvlheight+1.0) 
            {$\stsextLDecHid_\step$}; 


        \node[hid={3}{yellow}] (enc_ctx_\step) 
            at (\stepsize *\step, \lvlbase + 2.5*\lvlheight) {};    
        \node at (\stepsize *\step, \lvlbase + 2.5*\lvlheight) 
            {$\stsextDecHid_\step$}; 

        \draw[dep] (s\step.north) -- (lrnn_\step.south);
        \draw[dep] (s\step.north) -- (rrnn_\step.south);

        \draw[dep] (lrnn_\step.north) -- (enc_ctx_\step.south);
        \draw[dep] (rrnn_\step.north) -- (enc_ctx_\step.south);
    }
    \foreach \start [count=\stop from 2] in {1,...,2} {
        \draw[dep] ($ (rrnn_\start.east) - (0,0.3)$) 
            -- ($ (rrnn_\stop.west) - (0,0.3) $);
        \draw[dep] ($(lrnn_\stop.west) + (0,0.3)$) 
            -- ($ (lrnn_\start.east) + (0,0.3)   $);
    }




%    \foreach \step [count=\i from 0] in {4,...,7} {
%        \node[hid={3}{orange}] (rnn_\step) 
%            at (\stepsize *\step, \lvlbase + \lvlheight) {};    
%        \node at (\stepsize *\step, \lvlbase + \lvlheight) {$\xDecHid_\i$}; 
%    }
%
%    \foreach \step in {1,...,6} {
%        \draw[dep] (s\step.north) to (rnn_\step.south);
%    }
%    \foreach \start [count=\stop from 2] in {1,...,5} {
%        \draw[dep] (rnn_\start.east) to (rnn_\stop.west);
%    }
%
%
%%    \foreach \step [count=\i from 1] in {4,...,6} {
%%        \node[hid={3}{green}] (ctx_\i) 
%%            at (\stepsize *\step, \lvlbase + 2*\lvlheight) {};    
%%        \node at (\stepsize *\step, \lvlbase + 2*\lvlheight) {$\xPredHid_\i$}; 
%%        \draw[dep] (rnn_\step.north) to (ctx_\i.south);
%%        \draw[dep] (rnn_\i.north) to (ctx_\i.south west);
%%        \node[sal={blue}] (sal_\i) at (\stepsize * \step,\lvlbase + 3*\lvlheight) {};
%%        \node at (\stepsize * \step,\lvlbase + 3*\lvlheight) {$\bsal_\i$};
%%        \draw[dep] (ctx_\i.north) to (sal_\i.south);
%%    }
%%
%%    \foreach \step [count=\i from 1] in {5,...,6} {
%%        \draw[dep] (sal_\i) -- (\stepsize * \step - \stepsize / 2,
%%                            \lvlbase + 3*\lvlheight) 
%%                     -- (\stepsize * \step - \stepsize / 2,
%%                            \lvlbase + 0* \lvlheight) -> (s\step) ;
%%
%%    }
%%    \draw[dep] (s1.south) -- ($ (s1.south) + (0,-0.3)$) --($ (s5.south) + (0,-0.3)$) -- (s5.south);
%%
%%    \draw[dep] (s2.south) -- ($ (s2.south) + (0,-0.5)$) --($ (s6.south) + (0,-0.5)$) -- (s6.south);
%%
%%    \draw[rectangle,draw=black,dotted] 
%%        (\stepsize * 3.5,\lvlbase + 3.5*\lvlheight) -- 
%%        (\stepsize*6.5, \lvlbase + 3.5*\lvlheight) -- 
%%        (\stepsize*6.5, \lvlbase + 2.6*\lvlheight) --
%%        (\stepsize*3.5, \lvlbase + 2.6*\lvlheight) --
%%        (\stepsize*3.5, \lvlbase + 3.5*\lvlheight) ;
%%
%%    \node[align=left,anchor=north west] 
%%        at (\stepsize * 3.5,\lvlbase + 3.5*\lvlheight) 
%%        {\textit{Salience Estimates}};
%%
%%    \draw[rectangle,draw=black,dotted] 
%%        (\stepsize*0.5,\lvlbase + 2.5*\lvlheight) -- 
%%        (\stepsize*6.5, \lvlbase + 2.5*\lvlheight) -- 
%%        (\stepsize*6.5, \lvlbase + 1.5*\lvlheight) --
%%        (\stepsize*0.5, \lvlbase + 1.5*\lvlheight) --
%%        (\stepsize*0.5, \lvlbase + 2.5*\lvlheight) ;
%%
%%    \node[align=left,anchor=north west] 
%%        at (\stepsize * 0.5,\lvlbase + 2.5*\lvlheight) 
%%        {\textit{Contextual Sentence Embeddings}};
%%
%%    \draw[rectangle,draw=black,dotted] 
%%        (\stepsize*-0.5,\lvlbase + 0.5*\lvlheight) -- 
%%        (\stepsize*3.5, \lvlbase + 0.5*\lvlheight) -- 
%%        (\stepsize*3.5, \lvlbase + -0.9*\lvlheight) --
%%        (\stepsize*-0.5, \lvlbase + -0.9*\lvlheight) --
%%        (\stepsize*-0.5, \lvlbase + 0.5*\lvlheight) ;
%%
%%    \node[align=left,anchor=south west] 
%%        at (\stepsize * -0.5,\lvlbase + -0.9*\lvlheight) 
%%        {\textit{Sentence Embeddings}\\\textit{(Sentence Encoder Output)}};
%%
%%    \draw[rectangle,draw=black,dotted] 
%%        (\stepsize*4.3,\lvlbase + 0.5*\lvlheight) -- 
%%        (\stepsize*6.5, \lvlbase + 0.5*\lvlheight) -- 
%%        (\stepsize*6.5, \lvlbase + -0.9*\lvlheight) --
%%        (\stepsize*4.3, \lvlbase + -0.9*\lvlheight) --
%%        (\stepsize*4.3, \lvlbase + 0.5*\lvlheight) ;
%%
%%    \node[align=left,anchor=south west] 
%%        at (\stepsize * 4.3,\lvlbase + -0.9*\lvlheight) 
%%        {\textit{Salience Gated}\\\textit{Sentence Embeddings}};
%%

    



\end{tikzpicture}



\begin{tikzpicture}[
  dep/.style ={
    ->,line width=0.3mm
  },
  hid/.style 2 args={
    rectangle split,
    draw=#2,
    rectangle split parts=#1,
    fill=#2!20,
    minimum width=5mm,
    minimum height=5mm,
    outer sep=2mm},
  mlp/.style 2 args={
    rectangle split,
    rectangle split horizontal,
    draw=#2,
    rectangle split parts=#1,
    fill=#2!20,
    outer sep=2mm},
  sal/.style={
    circle, 
    minimum width=8mm,
    outer sep=2mm,
    draw=#1, 
    fill=#1!20},
]

  \def\stepsize{3}%
  \def\lvlbase{0}%
  \def\lvlheight{3}%
 


    % Sentence Embeddings    
  \foreach \step in {1,...,3} {
    \node[hid={3}{green}] (enc\step) at (\stepsize*\step, \lvlbase) {};    
    \node at (\stepsize*\step, \lvlbase) {$\stsextEncHid_\step$};    
   }

  \foreach \step in {1,...,3} {
    \node[hid={1}{yellow}] (a\step) at (\stepsize*\step-0.5*\stepsize, \lvlbase) {};    
    \node at (\stepsize*\step-0.5*\stepsize, \lvlbase) {$\stsextAttn_{i,\step}$};    

    \draw[dep] (enc\step.west) to (a\step.east);
   }


    \node[hid={3}{orange}] (deci) at (\stepsize*4, \lvlbase) {};    
    \node at (\stepsize*4, \lvlbase) {$\stsextDecHid_i$};    

    \draw[dep] (deci.south) to [out=270,in=270] (a3.south);
    \draw[dep] (deci.south) to [out=270,in=270] (a2.south);
    \draw[dep] (deci.south) to [out=270,in=270] (a1.south);

    \node[hid={3}{yellow}] (encattn) at (\stepsize*3, \lvlbase+\lvlheight) {};    
    \node at (\stepsize*3, \lvlbase+\lvlheight) {$\stsextAttnHid_i$};    


    \node[hid={3}{yellow}] (encattn) at (\stepsize*4, \lvlbase+\lvlheight) {};    
    \node at (\stepsize*4, \lvlbase+\lvlheight) {$\stsextHid_i$};    

        \node[sal={blue}] at (\stepsize * 5,\lvlbase + \lvlheight) {};
        \node at (\stepsize * 5,\lvlbase + \lvlheight) {$\bsal_i$};
%  \foreach \step [count=\i from 1] in {4,...,6} {
%    \node[hid={3}{green}] (s\i) at (\stepsize*\step-0.5*\stepsize, \lvlbase) {};    
%    \node at (\stepsize*\step-0.5*\stepsize, \lvlbase) {$\stsextDecHid_\i$};    
%   }
%
%  \foreach \step [count=\i from 1]  in {2,...,4} {
%    \node[hid={3}{green}] (s\i) at (\stepsize*\step, \lvlbase+\lvlheight) {};    
%    \node at (\stepsize*\step, \lvlbase+\lvlheight) {$\stsextAttnHid_\i$};    
%   }
%



%    \foreach \step [count=\i from 1] in {5,...,7} {
%        \node[hid={3}{green}] (s\step) at (\stepsize*\step, \lvlbase) {};    
%        \node at (\stepsize*\step, \lvlbase) {$\sentEmb_\i$};    
%    %\draw[->] (i\step.north) -> (e\step.south);
%    }

%       \node[hid={3}{red}] (s4) at (\stepsize*4, \lvlbase) {};    
%       \node at (\stepsize*4, \lvlbase) {$\sentEmb_0$};    
%
%    % RNN hidden states

%        \node[hid={3}{yellow}] (rrnn_0) 
%            at (\stepsize *0+0.5, \lvlbase + \lvlheight) {};    
%        \node at (\stepsize *0+0.5, \lvlbase + \lvlheight) 
%            {$\stsextRDecHid_0$}; 
%
%        \node[hid={3}{yellow}] (lrnn_4) 
%            at (\stepsize *4-0.5, \lvlbase + \lvlheight+1.0) {};    
%        \node at (\stepsize *4-0.5, \lvlbase + \lvlheight+1.0) 
%            {$\stsextLDecHid_4$}; 
%
%        \node[hid={3}{yellow}] (rrnn_m1) 
%            at (-\stepsize +0.5, \lvlbase + \lvlheight) {};    
%        \node at (-\stepsize +0.5, \lvlbase + \lvlheight) 
%            {$\stsextREncHid_3$}; 
%
%        \node[hid={3}{yellow}] (lrnn_m1) 
%            at (\stepsize*5 -0.5, \lvlbase + \lvlheight+1.0) {};    
%        \node at (\stepsize*5 -0.5, \lvlbase + \lvlheight+1.0) 
%            {$\stsextLEncHid_1$}; 
%
%
%
%    \foreach \step in {1,...,3} {
%        \node[hid={3}{yellow}] (rrnn_\step) 
%            at (\stepsize *\step+0.5, \lvlbase + \lvlheight) {};    
%        \node at (\stepsize *\step+0.5, \lvlbase + \lvlheight) 
%            {$\stsextRDecHid_\step$}; 
%
%        \node[hid={3}{yellow}] (lrnn_\step) 
%            at (\stepsize *\step-0.5, \lvlbase + \lvlheight+1.0) {};    
%        \node at (\stepsize *\step-0.5, \lvlbase + \lvlheight+1.0) 
%            {$\stsextLDecHid_\step$}; 
%
%
%        \node[hid={3}{yellow}] (enc_ctx_\step) 
%            at (\stepsize *\step, \lvlbase + 2.5*\lvlheight) {};    
%        \node at (\stepsize *\step, \lvlbase + 2.5*\lvlheight) 
%            {$\stsextDecHid_\step$}; 
%
%        \draw[dep] (s\step.north) -- (lrnn_\step.south);
%        \draw[dep] (s\step.north) -- (rrnn_\step.south);
%
%        \draw[dep] (lrnn_\step.north) -- (enc_ctx_\step.south);
%        \draw[dep] (rrnn_\step.north) -- (enc_ctx_\step.south);
%    }
%    \foreach \start [count=\stop from 2] in {1,...,2} {
%        \draw[dep] ($ (rrnn_\start.east) - (0,0.3)$) 
%            -- ($ (rrnn_\stop.west) - (0,0.3) $);
%        \draw[dep] ($(lrnn_\stop.west) + (0,0.3)$) 
%            -- ($ (lrnn_\start.east) + (0,0.3)   $);
%    }
%
%
%
%
%%    \foreach \step [count=\i from 0] in {4,...,7} {
%%        \node[hid={3}{orange}] (rnn_\step) 
%%            at (\stepsize *\step, \lvlbase + \lvlheight) {};    
%%        \node at (\stepsize *\step, \lvlbase + \lvlheight) {$\xDecHid_\i$}; 
%%    }
%%
%%    \foreach \step in {1,...,6} {
%%        \draw[dep] (s\step.north) to (rnn_\step.south);
%%    }
%%    \foreach \start [count=\stop from 2] in {1,...,5} {
%%        \draw[dep] (rnn_\start.east) to (rnn_\stop.west);
%%    }
%%
%%
%%%    \foreach \step [count=\i from 1] in {4,...,6} {
%%%        \node[hid={3}{green}] (ctx_\i) 
%%%            at (\stepsize *\step, \lvlbase + 2*\lvlheight) {};    
%%%        \node at (\stepsize *\step, \lvlbase + 2*\lvlheight) {$\xPredHid_\i$}; 
%%%        \draw[dep] (rnn_\step.north) to (ctx_\i.south);
%%%        \draw[dep] (rnn_\i.north) to (ctx_\i.south west);
%%%        \node[sal={blue}] (sal_\i) at (\stepsize * \step,\lvlbase + 3*\lvlheight) {};
%%%        \node at (\stepsize * \step,\lvlbase + 3*\lvlheight) {$\bsal_\i$};
%%%        \draw[dep] (ctx_\i.north) to (sal_\i.south);
%%%    }
%%%
%%%    \foreach \step [count=\i from 1] in {5,...,6} {
%%%        \draw[dep] (sal_\i) -- (\stepsize * \step - \stepsize / 2,
%%%                            \lvlbase + 3*\lvlheight) 
%%%                     -- (\stepsize * \step - \stepsize / 2,
%%%                            \lvlbase + 0* \lvlheight) -> (s\step) ;
%%%
%%%    }
%%    \draw[dep] (s1.south) -- ($ (s1.south) + (0,-0.3)$) --($ (s5.south) + (0,-0.3)$) -- (s5.south);
%%
%%    \draw[dep] (s2.south) -- ($ (s2.south) + (0,-0.5)$) --($ (s6.south) + (0,-0.5)$) -- (s6.south);
%%
%%    \draw[rectangle,draw=black,dotted] 
%%        (\stepsize * 3.5,\lvlbase + 3.5*\lvlheight) -- 
%%        (\stepsize*6.5, \lvlbase + 3.5*\lvlheight) -- 
%%        (\stepsize*6.5, \lvlbase + 2.6*\lvlheight) --
%%        (\stepsize*3.5, \lvlbase + 2.6*\lvlheight) --
%%        (\stepsize*3.5, \lvlbase + 3.5*\lvlheight) ;
%%
%%    \node[align=left,anchor=north west] 
%%        at (\stepsize * 3.5,\lvlbase + 3.5*\lvlheight) 
%%        {\textit{Salience Estimates}};
%%
%%    \draw[rectangle,draw=black,dotted] 
%%        (\stepsize*0.5,\lvlbase + 2.5*\lvlheight) -- 
%%        (\stepsize*6.5, \lvlbase + 2.5*\lvlheight) -- 
%%        (\stepsize*6.5, \lvlbase + 1.5*\lvlheight) --
%%        (\stepsize*0.5, \lvlbase + 1.5*\lvlheight) --
%%        (\stepsize*0.5, \lvlbase + 2.5*\lvlheight) ;
%%
%%    \node[align=left,anchor=north west] 
%%        at (\stepsize * 0.5,\lvlbase + 2.5*\lvlheight) 
%%        {\textit{Contextual Sentence Embeddings}};
%%
%%    \draw[rectangle,draw=black,dotted] 
%%        (\stepsize*-0.5,\lvlbase + 0.5*\lvlheight) -- 
%%        (\stepsize*3.5, \lvlbase + 0.5*\lvlheight) -- 
%%        (\stepsize*3.5, \lvlbase + -0.9*\lvlheight) --
%%        (\stepsize*-0.5, \lvlbase + -0.9*\lvlheight) --
%%        (\stepsize*-0.5, \lvlbase + 0.5*\lvlheight) ;
%%
%%    \node[align=left,anchor=south west] 
%%        at (\stepsize * -0.5,\lvlbase + -0.9*\lvlheight) 
%%        {\textit{Sentence Embeddings}\\\textit{(Sentence Encoder Output)}};
%%
%%    \draw[rectangle,draw=black,dotted] 
%%        (\stepsize*4.3,\lvlbase + 0.5*\lvlheight) -- 
%%        (\stepsize*6.5, \lvlbase + 0.5*\lvlheight) -- 
%%        (\stepsize*6.5, \lvlbase + -0.9*\lvlheight) --
%%        (\stepsize*4.3, \lvlbase + -0.9*\lvlheight) --
%%        (\stepsize*4.3, \lvlbase + 0.5*\lvlheight) ;
%%
%%    \node[align=left,anchor=south west] 
%%        at (\stepsize * 4.3,\lvlbase + -0.9*\lvlheight) 
%%        {\textit{Salience Gated}\\\textit{Sentence Embeddings}};
%%

    



\end{tikzpicture}


\end{figure}




\begin{figure}
    \fbox{\begin{minipage}{\textwidth}
\center
\scalebox{0.75}{
\begin{tikzpicture}[
  dep/.style ={
    ->,line width=0.3mm
  },
  hid/.style 2 args={
    rectangle split,
    draw=#2,
    rectangle split parts=#1,
    fill=#2!20,
    minimum width=5mm,
    minimum height=5mm,
    outer sep=2mm},
  mlp/.style 2 args={
    rectangle split,
    rectangle split horizontal,
    draw=#2,
    rectangle split parts=#1,
    fill=#2!20,
    outer sep=2mm},
  sal/.style={
    circle, 
    minimum width=8mm,
    outer sep=2mm,
    draw=#1, 
    fill=#1!20},
]

  \def\stepsize{2}%
  \def\lvlbase{0}%
  \def\lvlheight{3}%
 

    % Sentence Embeddings    
  \foreach \step in {1,...,3} {
    \node[hid={3}{sentemb}] (s\step) at (\stepsize*\step, \lvlbase) {};    
    \node at (\stepsize*\step, \lvlbase) {$\sentEmb_\step$};    
   }
    \foreach \step [count=\i from 1] in {5,6} {
        \node[hid={3}{salsentemb}] (s\step) at (\stepsize*\step, \lvlbase) {};    
        \node at (\stepsize*\step, \lvlbase) {$\salSentEmb_\i$};    
    %\draw[->] (i\step.north) -> (e\step.south);
    }

       \node[hid={3}{red}] (s4) at (\stepsize*4, \lvlbase) {};    
       \node at (\stepsize*4, \lvlbase) {$\salSentEmb_0$};    

    % RNN hidden states
    \foreach \step in {1,...,3} {
        \node[hid={3}{encemb}] (rnn_\step) 
            at (\stepsize *\step, \lvlbase + \lvlheight) {};    
        \node at (\stepsize *\step, \lvlbase + \lvlheight) {$\xEncHid_\step$}; 
    }
    \foreach \step [count=\i from 1] in {4,...,6} {
        \node[hid={3}{decemb}] (rnn_\step) 
            at (\stepsize *\step, \lvlbase + \lvlheight) {};    
        \node at (\stepsize *\step, \lvlbase + \lvlheight) {$\xDecHid_\i$}; 
    }

    \foreach \step in {1,...,6} {
        \draw[dep] (s\step.north) to (rnn_\step.south);
    }
    \foreach \start [count=\stop from 2] in {1,...,5} {
        \draw[dep] (rnn_\start.east) to (rnn_\stop.west);
    }


    \foreach \step [count=\i from 1] in {4,...,6} {
        \node[hid={3}{ctxemb}] (ctx_\i) 
            at (\stepsize *\step, \lvlbase + 2*\lvlheight) {};    
        \node at (\stepsize *\step, \lvlbase + 2*\lvlheight) {$\xPredHid_\i$}; 
        \draw[dep] (rnn_\step.north) to (ctx_\i.south);
        \draw[dep] (rnn_\i.north) to (ctx_\i.south west);
        \node[sal={sal}] (sal_\i) at (\stepsize * \step,\lvlbase + 3*\lvlheight) {};
        \node at (\stepsize * \step,\lvlbase + 3*\lvlheight) {$\psal_\i$};
        \draw[dep] (ctx_\i.north) to (sal_\i.south);
    }

    \foreach \step [count=\i from 1] in {5,...,6} {
        \draw[dep] (sal_\i) -- (\stepsize * \step - \stepsize / 2,
                            \lvlbase + 3*\lvlheight) 
                     -- (\stepsize * \step - \stepsize / 2,
                            \lvlbase + 0* \lvlheight) -> (s\step) ;

    }
    \draw[dep] (s1.south) -- ($ (s1.south) + (0,-0.3)$) --($ (s5.south) + (0,-0.3)$) -- (s5.south);

    \draw[dep] (s2.south) -- ($ (s2.south) + (0,-0.5)$) --($ (s6.south) + (0,-0.5)$) -- (s6.south);

    \draw[rectangle,draw=black,dotted] 
        (\stepsize * 3.5,\lvlbase + 3.5*\lvlheight) -- 
        (\stepsize*6.5, \lvlbase + 3.5*\lvlheight) -- 
        (\stepsize*6.5, \lvlbase + 2.6*\lvlheight) --
        (\stepsize*3.5, \lvlbase + 2.6*\lvlheight) --
        (\stepsize*3.5, \lvlbase + 3.5*\lvlheight) ;

    \node[align=left,anchor=north west] 
        at (\stepsize * 3.5,\lvlbase + 3.5*\lvlheight) 
        {\textit{(e) Salience Estimates}};

    \draw[rectangle,draw=black,dotted] 
        (\stepsize*0.5,\lvlbase + 2.5*\lvlheight) -- 
        (\stepsize*6.5, \lvlbase + 2.5*\lvlheight) -- 
        (\stepsize*6.5, \lvlbase + 1.5*\lvlheight) --
        (\stepsize*0.5, \lvlbase + 1.5*\lvlheight) --
        (\stepsize*0.5, \lvlbase + 2.5*\lvlheight) ;

    \node[align=left,anchor=north west] 
        at (\stepsize * 0.5,\lvlbase + 2.5*\lvlheight) 
        {\textit{(d) Contextual Sentence Embeddings}};

    \draw[rectangle,draw=black,dotted] 
        (\stepsize*-1.5,\lvlbase + 0.5*\lvlheight) -- 
        (\stepsize*3.4, \lvlbase + 0.5*\lvlheight) -- 
        (\stepsize*3.4, \lvlbase + -0.9*\lvlheight) --
        (\stepsize*-1.5, \lvlbase + -0.9*\lvlheight) --
        (\stepsize*-1.5, \lvlbase + 0.5*\lvlheight) ;

    \node[align=left,anchor=south west] 
        at (\stepsize * -1.5,\lvlbase + -0.9*\lvlheight) 
        {\textit{Sentence Embeddings (Sentence Encoder Output)}};

    \draw[rectangle,draw=black,dotted] 
        (\stepsize*-1.5,\lvlbase + 1.4*\lvlheight) -- 
        (\stepsize*3.4, \lvlbase + 1.4*\lvlheight) -- 
        (\stepsize*3.4, \lvlbase + 0.6*\lvlheight) --
        (\stepsize*-1.5, \lvlbase + 0.6*\lvlheight) --
        (\stepsize*-1.5, \lvlbase + 1.4*\lvlheight) ;

    \node[align=left,anchor=north west] 
        at (\stepsize * -1.5,\lvlbase + 1.4*\lvlheight) 
        {\textit{(a) Extractor -- Encoder}};

    \draw[rectangle,draw=black,dotted] 
        (\stepsize*3.6,\lvlbase + 1.4*\lvlheight) -- 
        (\stepsize*8.5, \lvlbase + 1.4*\lvlheight) -- 
        (\stepsize*8.5, \lvlbase + 0.6*\lvlheight) --
        (\stepsize*3.6, \lvlbase + 0.6*\lvlheight) --
        (\stepsize*3.6, \lvlbase + 1.4*\lvlheight) ;

    \node[align=left,anchor=north east] 
        at (\stepsize * 8.5,\lvlbase + 1.4*\lvlheight) 
        {\textit{(c) Extractor -- Decoder}};





    \draw[rectangle,draw=black,dotted] 
        (\stepsize*4.6,\lvlbase + 0.5*\lvlheight) -- 
        (\stepsize*8.5, \lvlbase + 0.5*\lvlheight) -- 
        (\stepsize*8.5, \lvlbase + -0.9*\lvlheight) --
        (\stepsize*4.6, \lvlbase + -0.9*\lvlheight) --
        (\stepsize*4.6, \lvlbase + 0.5*\lvlheight) ;

    \node[align=left,anchor=south west] 
        at (\stepsize * 4.6,\lvlbase + -0.9*\lvlheight) 
        {\textit{(b) Salience Gated Sentence Embeddings}};


    



\end{tikzpicture}
}

\caption{Schematic for the \clext~sentence extractor.}
\label{fig:clext} \end{minipage}}
\end{figure}



\subsubsection{\clext~Extractor}
% We consider two recent state-of-the-art extractors.
%\hal{if you're hurting for space, you could probably describe these both in one paragraph, leaving off the stuff about what they use as encoders, etc., and really just making the point that they use $y<i$ to predict $yi$}

%\paragraph{Cheng \& Lapata Extractor} 
The \clext~extractor \citep{cheng2016neural} 
%is built around a 
%\sequencetosequence~model over the sentence embeddings
% The first, proposed by 
%\citet{cheng2016neural}, %, which we refer to as the Cheng \& Lapata Extractor,
is built around a somewhat idiosyncratic 
\unidirectional~\sequencetosequence~model. A schematic outlining the 
structure of the encoder and closely following the subsequent 
equations can be found in \autoref{fig:clext}.

The encoder is fairly standard. The initial state is initialized to
a zero embedding, and each sentence embedding $\sentEmb_i$ is fed into 
the encoder, to obtain the final encoder hidden state $\xEncHid_\docSize \in \reals^\xhidSize$ . That is,
\begin{align}
    \textit{(a) Extractor -- Encoder} & \nonumber \\
    \xEncHid_0 & = \zeroEmb \\
    \forall i : \;\; i \in \{1,\ldots,\docSize\}&\nonumber\\
    \xEncHid_i &= \fgru(\sentEmb_i, \xEncHid_{i-1};\clEncParams).
\end{align}

%The decoder departs from the typical \sequencetosequence~setup.
The initial decoder hidden state $\xDecHid_0 \in \reals^{\xhidSize}$ is 
initialized with the last encoder hidden state, $\xEncHid_\docSize$.
The inputs
to decoder step $i$, for $i >1$, are the salience gated $(i-1)^\textrm{th}$ sentence embeddings,
\begin{align}
    \textit{(b) Salience Gated Sentence Embeddings}&\nonumber\\
    \forall i: \;\; i \in \{2,\dots,\docSize\}& \nonumber\\
    \salSentEmb_{i-1} & = \xweight_{i-1}\sentEmb_{i-1},
\end{align}
where the salience gate is 
% p_i = p(y_i|y_1,...,y_i-1,h_1,...,h_n) %
$\xweight_i = p(\bsal_i|\bsal_1,\ldots,\bsal_{i-1},
                        \sentEmb_1,\ldots,\sentEmb_\docSize)$, is the
                        salience estimate computed for sentence $\sent_{i-1}$.
                        For the first decoder step (i.e. $i=1$), since there is no $\xweight_0$, $\salSentEmb_0$ is a special learned parameter.



The extractor decoder outputs are then computed as,
%
%The initial decoder hidden state $\xDecHid_0 \in \reals^{\xhidSize}$ is then 
%initialized with the last encoder hidden state, 
%$\xEncHid_\docSize$. The inputs to the decoder \gru~are the same 
%sentence embeddings fed into the encoder but weighted and delayed by one time step so
%that the $i^\textrm{th}$ step decoder hidden state $\xDecHid_i$ is dependent
%on the previous decoder hidden state and the $(i-1)^\textrm{th}$ sentence
%embedding $\sentEmb_{i-1}$. Formally, we have:
\begin{align}
    \textit{(c) Extractor -- Decoder} & \nonumber \\
    \xDecHid_0  &= \xEncHid_\docSize  \\
    \forall i : \;\; i \in \{1,\ldots,\docSize\}&\nonumber\\
    %\xDecHid_1 &= \fgru(\sentEmb_0, \xDecHid_{\docSize};\clDecParams) \\
   \xDecHid_i &= \fgru(\salSentEmb_{i-1}, \xDecHid_{i-1};\clDecParams). \label{eq:cl1} 
\end{align}
Note in \autoref{eq:cl1} that 
the decoder side \gru~input is the sentence embedding from the previous time
step, $\sentEmb_{i-1}$, weighted by its probability of extraction, $p_{i-1}$, 
from the 
previous step, inducing dependence of each output $\bsal_i$ on all previous 
outputs $\bsal_1,\ldots,\bsal_{i-1}$.


The contextual sentence embeddings $\xPredHid_i$ are then computed by concatenating
the encoder and decoder outputs $\xEncHid_i$ and $\xDecHid_i$ and running
them through a \feedforward~layer with $\relu$ activation,
\begin{align}
\textit{(d) Contextual Sentence Embeddings} & \nonumber\\
    \forall i : \;\; i \in \{1,\ldots,\docSize\}&\nonumber \\
\xPredHid_i &= \relu\left(\xpWeight^{(1)} \left[\begin{array}{c}\xEncHid_i \\ \xDecHid_i \end{array}\right] + \xpBias^{(1)} \right).
\end{align}



The actual salience
estimate for sentence $\sent_i$ is then computed by feeding $\xPredHid_i$
through another \feedforward~layer with logistic sigmoid activation,
%where $\sentEmb_0 \in \reals^{\sentDim}$ can be thought of as a special
%``begin decoding'' sentence embedding that is a learned parameter of the 
%extractor. The weights $\xweight_{i-1} = p(\bsal_{i-1}|\bsal_1,\ldots,\bsal_{i-2},
%\sentEmb_1,\ldots,\sentEmb_\docSize)$ are the probabilities of extracting the
%$(i-1)^\textrm{th}$ sentence under the model. The actual prediction 
%probabilities are computed by running the $i^\textrm{th}$ encoder and decoder 
%hidden states through 
%a ``prediction head,'' that is,  a two layer feed forward netork, defined
%as follows,
\begin{align}
    \textit{(e) Salience Estimates} & \nonumber\\
    \forall i : \;\; i \in \{1,\ldots,\docSize\}& \nonumber \\
 p_i = p(\bsal_i =1|\bsal_1,\ldots,\bsal_{i-1}, \sentEmb_1,\ldots,\sentEmb_\docSize) &= \sigma\left(\xpWeight^{(2)}\xPredHid_i + \xpBias^{(2)}  \right).
\end{align}

The contextual embedding and salience estimate layers have parameters are $\xpWeight^{(1)} \in \reals^{\xpHidSize \times 2 \xhidSize}$, $\xpBias^{(1)} \in \reals^{\xpHidSize}$, $\xpWeight^{(2)} \in \reals^{1 \times \xpHidSize}$, and $\xpBias^{(2)}\in \reals$.
The entire set of learned parameters for the \clext~extractor are
\[
    \chi = \left\{ 
    \clEncParams, \clDecParams,
    \salSentEmb_0,
    \xpWeight^{(1)}, \xpBias^{(1)}, \xpWeight^{(2)}, \xpBias^{(2)}\right\}.
\]
The hidden layer dimensionality of the \gru~and the contextual embedding layer
is 
$\xhidSize = 300$ and $\xpHidSize=100$, respectively.
{\color{red}Dropout with drop probability .25 is applied to the \gru~outputs
($\xEncHid_i$ and $\xDecHid_i$),   and to $\xPredHid_i$.}

%TODO
%Note that in the original paper, the Cheng \& Lapata extractor was paired 
%with
%a \textit{CNN} sentence encoder, but in this work we experiment with a variety
%of sentence encoders.


%~\\~\\~\\
%
%On the decoder side, the same sentence 
%embeddings are fed as input to the decoder and decoder outputs are used to
%predict each $y_i$. The decoder input is weighted by the previous extraction
%probability, inducing the dependence of $y_i$ on $y_{<i}$.
%
%
%\begin{align}
%\textit{(Extractor -- Decoder}) & \\
%    \xDecHid_0  &= \xEncHid_\docSize  \\
%    \xDecHid_1 &= \fgru(\sentEmb_*, \xDecHid_{\docSize};\clDecParams) \\
%%    \rdxhid_i &= \fgru(p_{i-1} \sentEmb_{i-1},  \rdxhid_{i-1};\chi_d) \label{eq:cl1}\\
%   \xDecHid_i &= \fgru(p_{i-1} \cdot \sentEmb_{i-1}, \xDecHid_{i-1};\clDecParams) \label{eq:cl1} \\
%\textit{(Extractor -- Prediction Head)} & \\
%o_i &= \relu\left(U \cdot \left[\begin{array}{c}\xhid_i \\ \rdxhid_i \end{array}\right] + u \right)\\
% p_i = p(\bsal_i&=1|\bsal_{<i}, \sentEmb_1,\ldots,\sentEmb_\docSize) = \sigma\left(V\cdot o_i + v  \right) 
%\end{align}
%w
%
%
%First, each sentence embedding\footnote{\citet{cheng2016neural} used an CNN sentence encoder with 
%this extractor architecture; in this work we pair the Cheng \& Lapata extractor
%with several different encoders.} is
%fed into an encoder side RNN, with the final encoder state passed to the
%first step of the decoder RNN. On the decoder side, the same sentence 
%embeddings are fed as input to the decoder and decoder outputs are used to
%predict each $y_i$. The decoder input is weighted by the previous extraction
%probability, inducing the dependence of $y_i$ on $y_{<i}$.
%See \autoref{fig:extractors}.c for a graphical layout of the extractor.
%%and \autoref{app:clextractor} for details.
%
%
%The basic architecture is a unidirectional
%sequence-to-sequence
%model defined as follows:
%\begin{align}
%    \xhid_0 & = 0 \\
%    \xhid_i &= \fgru(\sentEmb_i, \xhid_{i-1};\chi_e) \\
%    \rdxhid_1 &= \fgru(\sentEmb_*, \xhid_{\docSize};\chi_d) \\
%    \rdxhid_i &= \fgru(p_{i-1} \sentEmb_{i-1},  \rdxhid_{i-1};\chi_d) \label{eq:cl1}\\
%%    \decExtHidden_i &= \gru_{dec}(p_{i-1} \cdot \sentvec_{i-1}, \decExtHidden_{i-1}) \label{eq:cl1} \\
%o_i &= \relu\left(U \cdot \left[\begin{array}{c}\xhid_i \\ \rdxhid_i \end{array}\right] + u \right)\\
% p_i = p(\bsal_i&=1|\bsal_{<i}, \sentEmb_1,\ldots,\sentEmb_\docSize) = \sigma\left(V\cdot o_i + v  \right) 
%\end{align}
%where $\sentEmb_*$ is a learned ``begin decoding'' sentence embedding
%(see \autoref{fig:extractors}.c).
%Each GRU has separate learned 
%parameters; $U, V$ and $u, v$ are learned weight and bias parameters.
%Note in Equation~\ref{eq:cl1} that 
%the decoder side GRU input is the sentence embedding from the previous time
%step weighted by its probabilitiy of extraction ($p_{i-1}$) from the 
%previous step, inducing dependence of each output $y_i$ on all previous 
%outputs $y_{<i}$.
%The hidden layer size of the GRU is 300 and the MLP hidden layer
%size is 100. 
%Dropout with drop probability .25 is applied to the GRU outputs and to $a_i$.
%
%Note that in the original paper, the Cheng \& Lapata extractor was paired 
%with
%a \textit{CNN} sentence encoder, but in this work we experiment with a variety
%of sentence encoders.



%?, but are delayed by one step and 
%?weighted by their prediction probability, i.e. at decoder step $t$,
%?$p(\slabel[t-1]|\slabel[<t-1], \sentEmb[<t-1]) \cdot \sentEmb[t-1]$\hal{why did you switch from $i$ to $t$?}
%?is fed into the decoder\hal{i don't udnerstand what this means. what op is $\cdot$?}. The decoder output at step $t$ is concatenated 
%?to the encoder output step $t$ and fed through a multi-layer perceptron
%?with one hidden layer and sigmoid unit output computing the $t$-th
%?extraction probability $p(\slabel[t]|\slabel[<t], \sentEmb[<t])$. \textcolor{red}{See Figure 2.c. for a graphical view. Full model details are presented in ??}.
%?





\subsubsection{\srext~Extractor}

\citet{nallapati2017summarunner} proposed
a sentence extractor, which we refer to as the \srext~Extractor,
that factorizes the salience estimates for each sentence into contributions 
from five different sources, which we refer to as \saliencefactors.
The \saliencefactors~take into account interactions between 
contextual sentence embeddings and document embeddings or summary embeddings,
as well as sentence position embeddings. Salience estimates are made sequentially, starting with the first sentence $\sent_1$ and preceding to the last $\sent_\docSize$. When computing the salience estimate of sentence $\sent_i$, 
the previous $i-1$ salience estimates are used to update the summary
representation.

In order to construct the contextual sentence embeddings, document 
embeddings, and summary embeddings, the \srext~extractor first runs 
a \bidirectional~\gru~over the sentence 
embeddings created by the sentence encoder (visually depicted in \autoref{fig:sr1}), 

\begin{figure}[h!]
    \fbox{\begin{minipage}{\textwidth}
\center
\scalebox{0.75}{
\begin{tikzpicture}[
  dep/.style ={
    ->,line width=0.3mm
  },
  hid/.style 2 args={
    rectangle split,
    draw=#2,
    rectangle split parts=#1,
    fill=#2!20,
    minimum width=5mm,
    minimum height=5mm,
    outer sep=2mm},
  mlp/.style 2 args={
    rectangle split,
    rectangle split horizontal,
    draw=#2,
    rectangle split parts=#1,
    fill=#2!20,
    outer sep=2mm},
  sal/.style={
    circle, 
    minimum width=8mm,
    outer sep=2mm,
    draw=#1, 
    fill=#1!20},
]

  \def\stepsize{1.5}%
  \def\lvlbase{0}%
  \def\lvlheight{3}%
 

    % Sentence Embeddings    
  \foreach \step in {1,...,3} {
    \node[hid={3}{sentemb}] (s\step) at (\stepsize*\step, \lvlbase) {};    
    \node at (\stepsize*\step, \lvlbase) {$\sentEmb_\step$};    
   }
    \draw[rectangle,draw=black,dotted] 
        (\stepsize*0.0,\lvlbase + 2*\lvlheight) -- 
        (\stepsize*3.9, \lvlbase + 2*\lvlheight) -- 
        (\stepsize*3.9, \lvlbase + 0.5*\lvlheight) --
        (\stepsize*0.0, \lvlbase + 0.5*\lvlheight) --
        (\stepsize*0.0, \lvlbase + 2*\lvlheight) ;

    \node[align=left,anchor=north west] 
        at (\stepsize * 0.0,\lvlbase + 2*\lvlheight) 
        {\textit{(a) Forward and Backward \gru}\\\textit{\phantom{(a) }Outputs}};

    \draw[rectangle,draw=black,dotted] 
        (\stepsize*0.0,\lvlbase + 0.4*\lvlheight) -- 
        (\stepsize*3.9, \lvlbase + 0.4*\lvlheight) -- 
        (\stepsize*3.9, \lvlbase -0.75*\lvlheight) --
        (\stepsize*0.0, \lvlbase -0.75*\lvlheight) --
        (\stepsize*0.0, \lvlbase + 0.4*\lvlheight) ;

    \node[align=left,anchor=south west] 
        at (\stepsize * 0.0,\lvlbase + -0.75*\lvlheight) 
        {\textit{Sentence Embeddings}\\\textit{(Sentence Encoder Output)}};


    \draw[rectangle,draw=black,dotted] 
        (\stepsize*4.0,\lvlbase + 2*\lvlheight) -- 
        (\stepsize*7.9, \lvlbase + 2*\lvlheight) -- 
        (\stepsize*7.9, \lvlbase + 0.5*\lvlheight) --
        (\stepsize*4.0, \lvlbase + 0.5*\lvlheight) --
        (\stepsize*4.0, \lvlbase + 2*\lvlheight) ;

    \node[align=left,anchor=north west] 
        at (\stepsize * 4.0,\lvlbase + 2*\lvlheight) 
        {\textit{(b) Contextual Sentence}\\ \textit{\phantom{(b) } Embeddings}};

    \draw[rectangle,draw=black,dotted] 
        (\stepsize*8.0,\lvlbase + 2*\lvlheight) -- 
        (\stepsize*11.9, \lvlbase + 2*\lvlheight) -- 
        (\stepsize*11.9, \lvlbase + 0.5*\lvlheight) --
        (\stepsize*8.0, \lvlbase + 0.5*\lvlheight) --
        (\stepsize*8.0, \lvlbase + 2*\lvlheight) ;

    \node[align=left,anchor=north west] 
        at (\stepsize * 8.0,\lvlbase + 2*\lvlheight) 
        {\textit{(c) Document Embedding}};




%    % RNN hidden states
    \foreach \step in {1,...,3} {
        \node[hid={3}{rencemb}] (rrnn_\step) 
            at (\stepsize *\step-0.3, \lvlbase + \lvlheight) {};    
        \node at (\stepsize *\step-0.3, \lvlbase + \lvlheight) 
            {$\rnnextRHid_\step$}; 
        \node[hid={3}{lencemb}] (lrnn_\step) 
            at (\stepsize *\step+0.3, \lvlbase + \lvlheight + 1.0) {};    
        \node at (\stepsize *\step+0.3, \lvlbase + \lvlheight+ 1.0) 
            {$\rnnextLHid_\step$}; 
        \draw[dep] (s\step.north) -- (rrnn_\step.south);
        \draw[dep] (s\step.north) -- (lrnn_\step.south);

    }
    \foreach \start [count=\stop from 2] in {1,...,2} {
        \draw[dep] ($ (rrnn_\start.east) - (0,0.3)$) 
            -- ($ (rrnn_\stop.west) - (0,0.3) $);
        \draw[dep] ($(lrnn_\stop.west) + (0,0.3)$) 
            -- ($ (lrnn_\start.east) + (0,0.3)   $);
    }

  \def\stepsize{1.5}%
  \def\lvlbase{0}%
  \def\lvlheight{3}%
 

    \foreach \step in {1,...,3} {
        \node[hid={3}{rencemb}] (rrnn_\step) 
            at (\stepsize *\step-0.35 + 5 + 1, \lvlbase + 0*\lvlheight) {};    
        \node at (\stepsize*\step-0.35 + 5 + 1, \lvlbase + 0*\lvlheight) 
            {$\rnnextRHid_\step$}; 
        \node[hid={3}{lencemb}] (lrnn_\step) 
            at (\stepsize *\step+5.35 + 1, \lvlbase + 0*\lvlheight + 0.5) {};    
        \node at (\stepsize *\step+5.35 + 1, \lvlbase + 0*\lvlheight+ 0.5) 
            {$\rnnextLHid_\step$}; 
        \node[hid={3}{ctxemb}] (ctx_\step) 
            at (\stepsize *\step + 5 + 1, \lvlbase + 1*\lvlheight) {};    
        \node at (\stepsize*\step + 5 + 1, \lvlbase + 1*\lvlheight) 
            {$\srHid_\step$}; 

        \draw[dep] (rrnn_\step.north) to (ctx_\step.south);
        \draw[dep] (lrnn_\step.north) to (ctx_\step.south);
    }


        \node[hid={3}{doc}] (doc) 
            at (\stepsize *2 + 10+2, \lvlbase + 1*\lvlheight) {};    
        \node at (\stepsize*2 + 10+2, \lvlbase + 1*\lvlheight) 
            {$\srDocEmb$}; 



    \foreach \step in {1,...,3} {
        \node[hid={3}{rencemb}] (rrnn_\step) 
            at (\stepsize *\step-0.35 + 10 + 2, \lvlbase + 0*\lvlheight) {};    
        \node at (\stepsize*\step-0.35 + 10 + 2, \lvlbase + 0*\lvlheight) 
            {$\rnnextRHid_\step$}; 
        \node[hid={3}{lencemb}] (lrnn_\step) 
            at (\stepsize *\step+10.35+2, \lvlbase + 0*\lvlheight + 0.5) {};    
        \node at (\stepsize *\step+10.35+2, \lvlbase + 0*\lvlheight+ 0.5) 
            {$\rnnextLHid_\step$}; 

        \draw[dep] (rrnn_\step.north) to (doc.south);
        \draw[dep] (lrnn_\step.north) to (doc.south);
    }


\end{tikzpicture}}


\caption{\srext~contextual sentence embedding and document embeddings.}
\label{fig:sr1}\end{minipage}}
\end{figure}


%
\begin{wrapfigure}{L}{0.45\textwidth}
    \fbox{\begin{minipage}{0.40\textwidth}
\center
\scalebox{0.75}{
\begin{tikzpicture}[
  dep/.style ={
    ->,line width=0.3mm
  },
  hid/.style 2 args={
    rectangle split,
    draw=#2,
    rectangle split parts=#1,
    fill=#2!20,
    minimum width=5mm,
    minimum height=5mm,
    outer sep=2mm},
]

  \def\stepsize{1.5}%
  \def\lvlbase{0}%
  \def\lvlheight{3}%
 

    % Sentence Embeddings    
  \foreach \step in {1,...,3} {
    \node[hid={3}{sentemb}] (s\step) at (\stepsize*\step, \lvlbase) {};    
    \node at (\stepsize*\step, \lvlbase) {$\sentEmb_\step$};    
   }

%    % RNN hidden states
    \foreach \step in {1,...,3} {
        \node[hid={3}{rencemb}] (rrnn_\step) 
            at (\stepsize *\step-0.3, \lvlbase + \lvlheight) {};    
        \node at (\stepsize *\step-0.3, \lvlbase + \lvlheight) 
            {$\rnnextRHid_\step$}; 
        \node[hid={3}{lencemb}] (lrnn_\step) 
            at (\stepsize *\step+0.3, \lvlbase + \lvlheight + 1.0) {};    
        \node at (\stepsize *\step+0.3, \lvlbase + \lvlheight+ 1.0) 
            {$\rnnextLHid_\step$}; 
        \draw[dep] (s\step.north) -- (rrnn_\step.south);
        \draw[dep] (s\step.north) -- (lrnn_\step.south);

    }
    \foreach \start [count=\stop from 2] in {1,...,2} {
        \draw[dep] ($ (rrnn_\start.east) - (0,0.3)$) 
            -- ($ (rrnn_\stop.west) - (0,0.3) $);
        \draw[dep] ($(lrnn_\stop.west) + (0,0.3)$) 
            -- ($ (lrnn_\start.east) + (0,0.3)   $);
    }




    \draw[rectangle,draw=black,dotted] 
        (\stepsize*0.0,\lvlbase + 2*\lvlheight) -- 
        (\stepsize*3.9, \lvlbase + 2*\lvlheight) -- 
        (\stepsize*3.9, \lvlbase + 0.5*\lvlheight) --
        (\stepsize*0.0, \lvlbase + 0.5*\lvlheight) --
        (\stepsize*0.0, \lvlbase + 2*\lvlheight) ;

    \node[align=left,anchor=north west] 
        at (\stepsize * 0.0,\lvlbase + 2*\lvlheight) 
        {\textit{Forward and Backward \gru}};
    \node[align=left,anchor=north west] 
        at (\stepsize * 0.0,\lvlbase + 1.85*\lvlheight) 
        {\textit{Outputs}};

    \draw[rectangle,draw=black,dotted] 
        (\stepsize*0.0,\lvlbase + 0.4*\lvlheight) -- 
        (\stepsize*3.9, \lvlbase + 0.4*\lvlheight) -- 
        (\stepsize*3.9, \lvlbase -0.75*\lvlheight) --
        (\stepsize*0.0, \lvlbase -0.75*\lvlheight) --
        (\stepsize*0.0, \lvlbase + 0.4*\lvlheight) ;

    \node[align=left,anchor=south west] 
        at (\stepsize * 0.0,\lvlbase + -0.60*\lvlheight) 
        {\textit{Sentence Embeddings}};

    \node[align=left,anchor=south west] 
        at (\stepsize * 0.0,\lvlbase + -0.75*\lvlheight) 
        {\textit{(Sentence Encoder Output)}};



\end{tikzpicture}}


\caption{\srext~forward and backward \gru~outputs.}
\label{fig:srpartial}
\end{minipage}}\end{wrapfigure}


\noindent \textit{(Fig.~\ref{fig:sr1}.a) Forward and Backward \gru~Outputs}
\begin{align}
    \srrHid_0 &= \zeroEmb, \quad \srlHid_{\docSize + 1} = \zeroEmb, \\
    \forall i : \;\; i \in \{1,\ldots,\docSize\}&\nonumber \\
    \srrHid_i &= \fgru(\sentEmb_i, \srrHid_{i-1}; \srRRNNParams), \\
    \srlHid_i &= \fgru(\sentEmb_i, \srlHid_{i+1}; \srLRNNParams),
\end{align}
where $\srrHid_i,\srlHid_i \in \reals^{\srRNNDim}$ and $\srRRNNParams$ and $\srLRNNParams$ are the forward and backward \gru~parameters respectively.



%\footnote{\citet{nallapati2017summarunner}
%    use an RNN sentence encoder with 
%this extractor architecture; in this work we pair the \srext~extractor
%with different encoders. }% 
    The \gru~output is
concatenated and run through a feed-forward layer to obtain 
a contextual sentence embedding representation $\srHid_i \in \reals^{\srRepDim}$, 


\vspace{10pt}
\noindent \textit{(Fig.~\ref{fig:sr1}.b) Contextual Sentence Embeddings} 
\begin{align}
    \forall i : \;\; i \in \{1,\ldots,\docSize\}&\nonumber \\
\srHid_i  & = \relu\left(\srSentBias + \srSentWeight \left[ \begin{array}{c} \srrHid_i \\ \srlHid_i  \end{array} \right]  \right),
\end{align}

where $\srSentWeight \in \reals^{\srRepDim \times 2\srRNNDim}$ and $\srSentBias \in \reals^{\srRepDim}$ are learned parameters.

To construct the document embedding $\srDocEmb$, the forward and backward 
\gru~outputs are concatenated and averaged before running through a different 
\feedforward~layer,
%and another \feedforward~layer to obtain contextual sentence embeddings
%$\srHid_i \in \reals^{{\color{red}???}}$, depicted visually in the schematic in \autoref{fig:sr1}, and defined by the following equations,

\vspace{10pt}
\noindent\textit{(Fig~\ref{fig:sr1}.c) Document Embedding}
\begin{align}
\srDocEmb  & = \tanh\left(\srDocBias + \srDocWeight \left(\frac{1}{\docSize}\sum_{i=1}^{\docSize} \left[ \begin{array}{c} \srrHid_i \\ \srlHid_i  \end{array} \right] \right) \right)
%    \srHid_i & = \left[ \begin{array}{c} \srrHid \\ \srlHid  \end{array} \right] 
\end{align}
where $\srDocWeight \in \reals^{\srRepDim \times 2\srRNNDim}$ and $\srDocBias \in \reals^{\srRepDim}$ are learned parameters.

%\begin{figure}[h!]
    \fbox{\begin{minipage}{\textwidth}
\center
\scalebox{0.75}{
\begin{tikzpicture}[
  dep/.style ={
    ->,line width=0.3mm
  },
  hid/.style 2 args={
    rectangle split,
    draw=#2,
    rectangle split parts=#1,
    fill=#2!20,
    minimum width=5mm,
    minimum height=5mm,
    outer sep=2mm},
  mlp/.style 2 args={
    rectangle split,
    rectangle split horizontal,
    draw=#2,
    rectangle split parts=#1,
    fill=#2!20,
    outer sep=2mm},
  sal/.style={
    circle, 
    minimum width=8mm,
    outer sep=2mm,
    draw=#1, 
    fill=#1!20},
]

  \def\stepsize{1.5}%
  \def\lvlbase{0}%
  \def\lvlheight{3}%
 

    % Sentence Embeddings    
  \foreach \step in {1,...,3} {
    \node[hid={3}{sentemb}] (s\step) at (\stepsize*\step, \lvlbase) {};    
    \node at (\stepsize*\step, \lvlbase) {$\sentEmb_\step$};    
   }
    \draw[rectangle,draw=black,dotted] 
        (\stepsize*0.0,\lvlbase + 2*\lvlheight) -- 
        (\stepsize*3.9, \lvlbase + 2*\lvlheight) -- 
        (\stepsize*3.9, \lvlbase + 0.5*\lvlheight) --
        (\stepsize*0.0, \lvlbase + 0.5*\lvlheight) --
        (\stepsize*0.0, \lvlbase + 2*\lvlheight) ;

    \node[align=left,anchor=north west] 
        at (\stepsize * 0.0,\lvlbase + 2*\lvlheight) 
        {\textit{(a) Forward and Backward \gru}\\\textit{\phantom{(a) }Outputs}};

    \draw[rectangle,draw=black,dotted] 
        (\stepsize*0.0,\lvlbase + 0.4*\lvlheight) -- 
        (\stepsize*3.9, \lvlbase + 0.4*\lvlheight) -- 
        (\stepsize*3.9, \lvlbase -0.75*\lvlheight) --
        (\stepsize*0.0, \lvlbase -0.75*\lvlheight) --
        (\stepsize*0.0, \lvlbase + 0.4*\lvlheight) ;

    \node[align=left,anchor=south west] 
        at (\stepsize * 0.0,\lvlbase + -0.75*\lvlheight) 
        {\textit{Sentence Embeddings}\\\textit{(Sentence Encoder Output)}};


    \draw[rectangle,draw=black,dotted] 
        (\stepsize*4.0,\lvlbase + 2*\lvlheight) -- 
        (\stepsize*7.9, \lvlbase + 2*\lvlheight) -- 
        (\stepsize*7.9, \lvlbase + 0.5*\lvlheight) --
        (\stepsize*4.0, \lvlbase + 0.5*\lvlheight) --
        (\stepsize*4.0, \lvlbase + 2*\lvlheight) ;

    \node[align=left,anchor=north west] 
        at (\stepsize * 4.0,\lvlbase + 2*\lvlheight) 
        {\textit{(b) Contextual Sentence}\\ \textit{\phantom{(b) } Embeddings}};

    \draw[rectangle,draw=black,dotted] 
        (\stepsize*8.0,\lvlbase + 2*\lvlheight) -- 
        (\stepsize*11.9, \lvlbase + 2*\lvlheight) -- 
        (\stepsize*11.9, \lvlbase + 0.5*\lvlheight) --
        (\stepsize*8.0, \lvlbase + 0.5*\lvlheight) --
        (\stepsize*8.0, \lvlbase + 2*\lvlheight) ;

    \node[align=left,anchor=north west] 
        at (\stepsize * 8.0,\lvlbase + 2*\lvlheight) 
        {\textit{(c) Document Embedding}};




%    % RNN hidden states
    \foreach \step in {1,...,3} {
        \node[hid={3}{rencemb}] (rrnn_\step) 
            at (\stepsize *\step-0.3, \lvlbase + \lvlheight) {};    
        \node at (\stepsize *\step-0.3, \lvlbase + \lvlheight) 
            {$\rnnextRHid_\step$}; 
        \node[hid={3}{lencemb}] (lrnn_\step) 
            at (\stepsize *\step+0.3, \lvlbase + \lvlheight + 1.0) {};    
        \node at (\stepsize *\step+0.3, \lvlbase + \lvlheight+ 1.0) 
            {$\rnnextLHid_\step$}; 
        \draw[dep] (s\step.north) -- (rrnn_\step.south);
        \draw[dep] (s\step.north) -- (lrnn_\step.south);

    }
    \foreach \start [count=\stop from 2] in {1,...,2} {
        \draw[dep] ($ (rrnn_\start.east) - (0,0.3)$) 
            -- ($ (rrnn_\stop.west) - (0,0.3) $);
        \draw[dep] ($(lrnn_\stop.west) + (0,0.3)$) 
            -- ($ (lrnn_\start.east) + (0,0.3)   $);
    }

  \def\stepsize{1.5}%
  \def\lvlbase{0}%
  \def\lvlheight{3}%
 

    \foreach \step in {1,...,3} {
        \node[hid={3}{rencemb}] (rrnn_\step) 
            at (\stepsize *\step-0.35 + 5 + 1, \lvlbase + 0*\lvlheight) {};    
        \node at (\stepsize*\step-0.35 + 5 + 1, \lvlbase + 0*\lvlheight) 
            {$\rnnextRHid_\step$}; 
        \node[hid={3}{lencemb}] (lrnn_\step) 
            at (\stepsize *\step+5.35 + 1, \lvlbase + 0*\lvlheight + 0.5) {};    
        \node at (\stepsize *\step+5.35 + 1, \lvlbase + 0*\lvlheight+ 0.5) 
            {$\rnnextLHid_\step$}; 
        \node[hid={3}{ctxemb}] (ctx_\step) 
            at (\stepsize *\step + 5 + 1, \lvlbase + 1*\lvlheight) {};    
        \node at (\stepsize*\step + 5 + 1, \lvlbase + 1*\lvlheight) 
            {$\srHid_\step$}; 

        \draw[dep] (rrnn_\step.north) to (ctx_\step.south);
        \draw[dep] (lrnn_\step.north) to (ctx_\step.south);
    }


        \node[hid={3}{doc}] (doc) 
            at (\stepsize *2 + 10+2, \lvlbase + 1*\lvlheight) {};    
        \node at (\stepsize*2 + 10+2, \lvlbase + 1*\lvlheight) 
            {$\srDocEmb$}; 



    \foreach \step in {1,...,3} {
        \node[hid={3}{rencemb}] (rrnn_\step) 
            at (\stepsize *\step-0.35 + 10 + 2, \lvlbase + 0*\lvlheight) {};    
        \node at (\stepsize*\step-0.35 + 10 + 2, \lvlbase + 0*\lvlheight) 
            {$\rnnextRHid_\step$}; 
        \node[hid={3}{lencemb}] (lrnn_\step) 
            at (\stepsize *\step+10.35+2, \lvlbase + 0*\lvlheight + 0.5) {};    
        \node at (\stepsize *\step+10.35+2, \lvlbase + 0*\lvlheight+ 0.5) 
            {$\rnnextLHid_\step$}; 

        \draw[dep] (rrnn_\step.north) to (doc.south);
        \draw[dep] (lrnn_\step.north) to (doc.south);
    }


\end{tikzpicture}}


\caption{\srext~contextual sentence embedding and document embeddings.}
\label{fig:sr1}\end{minipage}}
\end{figure}



%%\noindent where $\srrHid, \srlHid \in \reals^{\srRNNDim}$, 
%    $\srHid \in \reals^{2\srRNNDim}$, and $\srRRNNParams, \srLRNNParams$
%are the parameters for the forward and backward \gru~respectively.



Additionally, an iterative representation of the extract summary at step $i$,
 $\srSum_i$, is constructed by summing the $i-1$ contextual sentence 
embeddings weighted by their salience estimates,

\vspace{10pt}
   \noindent \textit{(Fig.~\ref{fig:sr2}) Summary Embeddings}
\begin{align}
\srSum_1 & = \zeroEmb, \\
\srSum_i & = \tanh\left(\sum_{j=1}^{i-1} \psal_j \cdot \srHid_j\right).
\end{align}
where $\psal_j= \model\left(\bsal_j=1|\bsal_1,\ldots,\bsal_{j-1},\sentEmb_1, \ldots, \sentEmb_\docSize; \xParams\right)$ are previously computed salience estimates for
sentences $\sent_1,\ldots,\sent_{i-1}$.

\begin{figure}[t]
    \fbox{\begin{minipage}{\textwidth}
\center
\scalebox{0.75}{
\begin{tikzpicture}[
  dep/.style ={
    ->,line width=0.3mm
  },
  hid/.style 2 args={
    rectangle split,
    draw=#2,
    rectangle split parts=#1,
    fill=#2!20,
    minimum width=5mm,
    minimum height=5mm,
    outer sep=2mm},
  mlp/.style 2 args={
    rectangle split,
    rectangle split horizontal,
    draw=#2,
    rectangle split parts=#1,
    fill=#2!20,
    outer sep=2mm},
  sal/.style={
    circle, 
    minimum width=8mm,
    outer sep=2mm,
    draw=#1, 
    fill=#1!20},
]

  \def\stepsize{1.5}%
  \def\lvlbase{0}%
  \def\lvlheight{3}%
 
  \def\stepsize{1.5}%
  \def\lvlbase{0}%
  \def\lvlheight{3}%


        \node[hid={3}{ctxemb}] (ctx_1) 
            at (\stepsize *6 , \lvlbase + -5*\lvlheight) {};    
        \node at (\stepsize*6, \lvlbase + -5*\lvlheight) 
            {$\srHid_1$}; 

        \node[sal={sal}] (p_1) 
            at (\stepsize *7 , \lvlbase + -5*\lvlheight) {};    
        \node at (\stepsize*7, \lvlbase + -5*\lvlheight) 
            {$\psal_1$}; 

        \node[hid={3}{ctxemb}] (ctx_2) 
            at (\stepsize *8 , \lvlbase + -5*\lvlheight) {};    
        \node at (\stepsize*8, \lvlbase + -5*\lvlheight) 
            {$\srHid_2$}; 

        \node[sal={sal}] (p_2) 
            at (\stepsize *9 , \lvlbase + -5*\lvlheight) {};    
        \node at (\stepsize*9, \lvlbase + -5*\lvlheight) 
            {$\psal_2$}; 


        \node[hid={3}{sum}] (summary_3) 
            at (\stepsize *9 , \lvlbase + -4*\lvlheight) {};    
        \node at (\stepsize*9, \lvlbase + -4*\lvlheight) 
            {$\srSum_{i}$}; 


            \node at (\stepsize *9 , \lvlbase + -3.65*\lvlheight) {\textit{Summary Embedding}};

        \node[hid={3}{ctxemb}] (ctx_n) 
            at (\stepsize *11 , \lvlbase + -5*\lvlheight) {};    
        \node at (\stepsize*11, \lvlbase + -5*\lvlheight) 
            {$\srHid_{i-1}$}; 

        \node[sal={sal}] (p_n) 
            at (\stepsize *12 , \lvlbase + -5*\lvlheight) {};    
        \node at (\stepsize*12, \lvlbase + -5*\lvlheight) 
            {$\psal_{i-1}$}; 

        \node at (\stepsize*10, \lvlbase + -5*\lvlheight) {\Large $\cdots$};





        \draw[dep] (ctx_1.north) to (summary_3.south);
        \draw[dep] (p_1.north) to (summary_3.south);

        \draw[dep] (ctx_2.north) to (summary_3.south);
        \draw[dep] (p_2.north) to (summary_3.south);

        \draw[dep] (ctx_n.north) to (summary_3.south);
        \draw[dep] (p_n.north) to (summary_3.south);













%        \node[hid={3}{yellow}] (ctx_1) 
%            at (\stepsize *4 , \lvlbase + -5*\lvlheight) {};    
%        \node at (\stepsize*4, \lvlbase + -5*\lvlheight) 
%            {$\srHid_1$}; 
%        \node[hid={3}{yellow}] (ctx_2) 
%            at (\stepsize *5 , \lvlbase + -5*\lvlheight) {};    
%        \node at (\stepsize*5, \lvlbase + -5*\lvlheight) 
%            {$\srHid_2$}; 
%
%
%

%    \foreach \start [count=\stop from 2] in {1,...,2} {
%        \draw[dep] ($ (rrnn_\start.east) - (0,0.3)$) 
%            -- ($ (rrnn_\stop.west) - (0,0.3) $);
%        \draw[dep] ($(lrnn_\stop.west) + (0,0.3)$) 
%            -- ($ (lrnn_\start.east) + (0,0.3)   $);
%    }

%    \draw[rectangle,draw=black,dotted] 
%        (\stepsize*-2.5,\lvlbase + 3.5*\lvlheight) -- 
%        (\stepsize*3.5, \lvlbase + 3.5*\lvlheight) -- 
%        (\stepsize*3.5, \lvlbase + 2.7*\lvlheight) --
%        (\stepsize*-2.5, \lvlbase + 2.7*\lvlheight) --
%        (\stepsize*-2.5, \lvlbase + 3.5*\lvlheight) ;
%
%    \node[align=left,anchor=north west] 
%        at (\stepsize * -2.5,\lvlbase + 3.5*\lvlheight) 
%        {\textit{Salience Estimates}};
%
%    \draw[rectangle,draw=black,dotted] 
%        (\stepsize*-2.5,\lvlbase + 2.6*\lvlheight) -- 
%        (\stepsize*3.5, \lvlbase + 2.6*\lvlheight) -- 
%        (\stepsize*3.5, \lvlbase + 1.8*\lvlheight) --
%        (\stepsize*-2.5, \lvlbase + 1.8*\lvlheight) --
%        (\stepsize*-2.5, \lvlbase + 2.6*\lvlheight) ;
%
%    \node[align=left,anchor=north west] 
%        at (\stepsize * -2.5,\lvlbase + 2.6*\lvlheight) 
%        {\textit{Contextual Sentence Embeddings}};
%
%    \draw[rectangle,draw=black,dotted] 
%        (\stepsize*-2.5,\lvlbase + 0.5*\lvlheight) -- 
%        (\stepsize*3.5, \lvlbase + 0.5*\lvlheight) -- 
%        (\stepsize*3.5, \lvlbase + -0.50*\lvlheight) --
%        (\stepsize*-2.5, \lvlbase + -0.50*\lvlheight) --
%        (\stepsize*-2.5, \lvlbase + 0.5*\lvlheight) ;
%
%    \node[align=left,anchor=north west] 
%        at (\stepsize * -2.5,\lvlbase + 0.5*\lvlheight) 
%        {\textit{Sentence Embeddings}\\\textit{(Sentence Encoder Output)}};
%
%
%    \draw[rectangle,draw=black,dotted] 
%        (\stepsize*-2.5,\lvlbase + 1.7*\lvlheight) -- 
%        (\stepsize*3.5, \lvlbase + 1.7*\lvlheight) -- 
%        (\stepsize*3.5, \lvlbase + 0.6*\lvlheight) --
%        (\stepsize*-2.5, \lvlbase + 0.6*\lvlheight) --
%        (\stepsize*-2.5, \lvlbase + 1.7*\lvlheight) ;
%
%
%    \node[align=left,anchor=north west] 
%        at (\stepsize * -2.5,\lvlbase + 1.7*\lvlheight) 
%        {\textit{Forward and Backward Partial}\\\textit{Contexual Sentence Embeddings}};

\end{tikzpicture}}

        \caption{\srext~iterative summary embeddings.}
        \label{fig:sr2}

\end{minipage}}
\end{figure}



Each salience estimate $\psal_i$ is calculated as the sum of five \saliencefactors~run through a logistic sigmoid function (depicted in \autoref{fig:sr4}),

\vspace{10pt}
    \noindent \textit{(Fig.~\ref{fig:sr4}) Salience Estimates}
\begin{align}
  \psal_i =  \model(\bsal_i=1|\bsal_1,\dots,\bsal_{i-1},\sentEmb_1,\ldots,\sentEmb_\docSize;\xParams)
         & = 
        \sigma\left(\srContentFactor_i 
        + \srSalienceFactor_i + \srNoveltyFactor_i
    + \srFinePositionFactor_i + \srCoarsePositionFactor_i   \right).
\end{align}

\begin{figure}[h!]
    \fbox{\begin{minipage}{\textwidth}
            \center
\begin{tikzpicture}[
  dep/.style ={
    ->,line width=0.3mm
  },
  hid/.style 2 args={
    rectangle split,
    rectangle split horizontal,
    draw=#2,
    rectangle split parts=#1,
    fill=#2!20,
    minimum width=5mm,
    minimum height=5mm,
    outer sep=2mm},
  mlp/.style 2 args={
    rectangle split,
    rectangle split horizontal,
    draw=#2,
    rectangle split parts=#1,
    fill=#2!20,
    outer sep=2mm},
  sal/.style={
    circle, 
    minimum width=8mm,
    outer sep=2mm,
    draw=#1, 
    fill=#1!20},
]

  \def\stepsize{2}%

    \node[sal={sal}] (sal) at (7*\stepsize,1) {}; 
    \node at (7*\stepsize,1) {$\psal_i$}; 
\node[align=left,anchor=north west] at (6.1*\stepsize,2.8) {\textit{Salience Estimate}};
\node[align=left,anchor=north west] at (1*\stepsize-1,2.8) {\textit{Salience Factors}};

    \draw[rectangle,draw=black,dotted] (6.1*\stepsize,2.8)
        -- (7.85*\stepsize,2.8)
        -- (7.85*\stepsize,0.4)
        -- (6.1*\stepsize,0.4)
        -- (6.1*\stepsize,2.8);

    \draw[rectangle,draw=black,dotted] (1*\stepsize-1,2.8) --
    (5.5*\stepsize,2.8) --
    (5.5*\stepsize,0.4) --
    (1*\stepsize-1,0.4) -- (1*\stepsize-1,2.8) ;

    \foreach \factor [count=\i from 1] in {
        \srContentFactor,\srSalienceFactor,\srNoveltyFactor,
        \srFinePositionFactor,\srContentFactor} {
    \node[hid={1}{factor}] (f\i) at (\i*\stepsize,1) {}; 
    \node at (\i*\stepsize,1) {$\factor_i$}; 
        \draw[dep] (f\i) [out=25,in=160] to (sal);
    }
%    \node[hid={1}{yellow}] (f2) at (2*\stepsize,1) {}; 
%    \node at (2*\stepsize,1) {$\srSalienceFactor_i$}; 
%    \node[hid={1}{yellow}] at (3*\stepsize,1) {}; 
%    \node at (3*\stepsize,1) {$\srNoveltyFactor_i$}; 
%
%    \node[hid={1}{yellow}] at (4*\stepsize,1) {}; 
%    \node at (4*\stepsize,1) {$\srFinePositionFactor_i$}; 
%
%    \node[hid={1}{yellow}] at (5*\stepsize,1) {}; 
%    \node at (5*\stepsize,1) {$\srCoarsePositionFactor_i$}; 


\end{tikzpicture}
\caption{Schematic for the \srext~extractor's salience estimates.}
\label{fig:sr4}
\end{minipage}}
\end{figure}



%\begin{figure}[h!]
    \fbox{\begin{minipage}{\textwidth}
\center
\scalebox{0.75}{
\begin{tikzpicture}[
  dep/.style ={
    ->,line width=0.3mm
  },
  hid/.style 2 args={
    rectangle split,
    draw=#2,
    rectangle split parts=#1,
    fill=#2!20,
    minimum width=5mm,
    minimum height=5mm,
    outer sep=2mm},
  mlp/.style 2 args={
    rectangle split,
    rectangle split horizontal,
    draw=#2,
    rectangle split parts=#1,
    fill=#2!20,
    outer sep=2mm},
  sal/.style={
    circle, 
    minimum width=8mm,
    outer sep=2mm,
    draw=#1, 
    fill=#1!20},
]

  \def\stepsize{1.5}%
  \def\lvlbase{0}%
  \def\lvlheight{3}%
 

    % Sentence Embeddings    
  \foreach \step in {1,...,3} {
    \node[hid={3}{sentemb}] (s\step) at (\stepsize*\step, \lvlbase) {};    
    \node at (\stepsize*\step, \lvlbase) {$\sentEmb_\step$};    
   }
    \draw[rectangle,draw=black,dotted] 
        (\stepsize*0.0,\lvlbase + 2*\lvlheight) -- 
        (\stepsize*3.9, \lvlbase + 2*\lvlheight) -- 
        (\stepsize*3.9, \lvlbase + 0.5*\lvlheight) --
        (\stepsize*0.0, \lvlbase + 0.5*\lvlheight) --
        (\stepsize*0.0, \lvlbase + 2*\lvlheight) ;

    \node[align=left,anchor=north west] 
        at (\stepsize * 0.0,\lvlbase + 2*\lvlheight) 
        {\textit{(a) Forward and Backward \gru}\\\textit{\phantom{(a) }Outputs}};

    \draw[rectangle,draw=black,dotted] 
        (\stepsize*0.0,\lvlbase + 0.4*\lvlheight) -- 
        (\stepsize*3.9, \lvlbase + 0.4*\lvlheight) -- 
        (\stepsize*3.9, \lvlbase -0.75*\lvlheight) --
        (\stepsize*0.0, \lvlbase -0.75*\lvlheight) --
        (\stepsize*0.0, \lvlbase + 0.4*\lvlheight) ;

    \node[align=left,anchor=south west] 
        at (\stepsize * 0.0,\lvlbase + -0.75*\lvlheight) 
        {\textit{Sentence Embeddings}\\\textit{(Sentence Encoder Output)}};


    \draw[rectangle,draw=black,dotted] 
        (\stepsize*4.0,\lvlbase + 2*\lvlheight) -- 
        (\stepsize*7.9, \lvlbase + 2*\lvlheight) -- 
        (\stepsize*7.9, \lvlbase + 0.5*\lvlheight) --
        (\stepsize*4.0, \lvlbase + 0.5*\lvlheight) --
        (\stepsize*4.0, \lvlbase + 2*\lvlheight) ;

    \node[align=left,anchor=north west] 
        at (\stepsize * 4.0,\lvlbase + 2*\lvlheight) 
        {\textit{(b) Contextual Sentence}\\ \textit{\phantom{(b) } Embeddings}};

    \draw[rectangle,draw=black,dotted] 
        (\stepsize*8.0,\lvlbase + 2*\lvlheight) -- 
        (\stepsize*11.9, \lvlbase + 2*\lvlheight) -- 
        (\stepsize*11.9, \lvlbase + 0.5*\lvlheight) --
        (\stepsize*8.0, \lvlbase + 0.5*\lvlheight) --
        (\stepsize*8.0, \lvlbase + 2*\lvlheight) ;

    \node[align=left,anchor=north west] 
        at (\stepsize * 8.0,\lvlbase + 2*\lvlheight) 
        {\textit{(c) Document Embedding}};




%    % RNN hidden states
    \foreach \step in {1,...,3} {
        \node[hid={3}{rencemb}] (rrnn_\step) 
            at (\stepsize *\step-0.3, \lvlbase + \lvlheight) {};    
        \node at (\stepsize *\step-0.3, \lvlbase + \lvlheight) 
            {$\rnnextRHid_\step$}; 
        \node[hid={3}{lencemb}] (lrnn_\step) 
            at (\stepsize *\step+0.3, \lvlbase + \lvlheight + 1.0) {};    
        \node at (\stepsize *\step+0.3, \lvlbase + \lvlheight+ 1.0) 
            {$\rnnextLHid_\step$}; 
        \draw[dep] (s\step.north) -- (rrnn_\step.south);
        \draw[dep] (s\step.north) -- (lrnn_\step.south);

    }
    \foreach \start [count=\stop from 2] in {1,...,2} {
        \draw[dep] ($ (rrnn_\start.east) - (0,0.3)$) 
            -- ($ (rrnn_\stop.west) - (0,0.3) $);
        \draw[dep] ($(lrnn_\stop.west) + (0,0.3)$) 
            -- ($ (lrnn_\start.east) + (0,0.3)   $);
    }

  \def\stepsize{1.5}%
  \def\lvlbase{0}%
  \def\lvlheight{3}%
 

    \foreach \step in {1,...,3} {
        \node[hid={3}{rencemb}] (rrnn_\step) 
            at (\stepsize *\step-0.35 + 5 + 1, \lvlbase + 0*\lvlheight) {};    
        \node at (\stepsize*\step-0.35 + 5 + 1, \lvlbase + 0*\lvlheight) 
            {$\rnnextRHid_\step$}; 
        \node[hid={3}{lencemb}] (lrnn_\step) 
            at (\stepsize *\step+5.35 + 1, \lvlbase + 0*\lvlheight + 0.5) {};    
        \node at (\stepsize *\step+5.35 + 1, \lvlbase + 0*\lvlheight+ 0.5) 
            {$\rnnextLHid_\step$}; 
        \node[hid={3}{ctxemb}] (ctx_\step) 
            at (\stepsize *\step + 5 + 1, \lvlbase + 1*\lvlheight) {};    
        \node at (\stepsize*\step + 5 + 1, \lvlbase + 1*\lvlheight) 
            {$\srHid_\step$}; 

        \draw[dep] (rrnn_\step.north) to (ctx_\step.south);
        \draw[dep] (lrnn_\step.north) to (ctx_\step.south);
    }


        \node[hid={3}{doc}] (doc) 
            at (\stepsize *2 + 10+2, \lvlbase + 1*\lvlheight) {};    
        \node at (\stepsize*2 + 10+2, \lvlbase + 1*\lvlheight) 
            {$\srDocEmb$}; 



    \foreach \step in {1,...,3} {
        \node[hid={3}{rencemb}] (rrnn_\step) 
            at (\stepsize *\step-0.35 + 10 + 2, \lvlbase + 0*\lvlheight) {};    
        \node at (\stepsize*\step-0.35 + 10 + 2, \lvlbase + 0*\lvlheight) 
            {$\rnnextRHid_\step$}; 
        \node[hid={3}{lencemb}] (lrnn_\step) 
            at (\stepsize *\step+10.35+2, \lvlbase + 0*\lvlheight + 0.5) {};    
        \node at (\stepsize *\step+10.35+2, \lvlbase + 0*\lvlheight+ 0.5) 
            {$\rnnextLHid_\step$}; 

        \draw[dep] (rrnn_\step.north) to (doc.south);
        \draw[dep] (lrnn_\step.north) to (doc.south);
    }


\end{tikzpicture}}


\caption{\srext~contextual sentence embedding and document embeddings.}
\label{fig:sr1}\end{minipage}}
\end{figure}


\pagebreak

\begin{wrapfigure}{r}{0.50\textwidth}
    \fbox{\begin{minipage}{0.50\textwidth}
      \begin{center}
          \begin{tikzpicture}[
  dep/.style ={
    ->,line width=0.3mm
  },
  hid/.style 2 args={
    rectangle split,
    rectangle split horizontal,
    draw=#2,
    rectangle split parts=#1,
    fill=#2!20,
    minimum width=5mm,
    minimum height=5mm,
    outer sep=2mm},
  mlp/.style 2 args={
    rectangle split,
    rectangle split horizontal,
    draw=#2,
    rectangle split parts=#1,
    fill=#2!20,
    outer sep=2mm},
  sal/.style={
    circle, 
    minimum width=8mm,
    outer sep=2mm,
    draw=#1, 
    fill=#1!20},
]

  \def\stepsize{3}%
  \def\lvlbase{0}%
  \def\lvlheight{3}%

        \node[hid={3}{ctxemb}] (ctx_i) 
            at (\stepsize *1 , \lvlbase + -2*\lvlheight) {};    
        \node at (\stepsize*1, \lvlbase + -2*\lvlheight) 
            {$\srHid_i$}; 

        \node[hid={1}{factor}] (content_i) 
            at (\stepsize *2, \lvlbase + -2*\lvlheight) {};    
        \node at (\stepsize*2, \lvlbase + -2*\lvlheight) 
            {$\srContentFactor_i$}; 

        \draw[dep] (ctx_i) to (content_i);
    \draw[rectangle,draw=black,dotted] 
        (\stepsize*0.5,\lvlbase -1.6*\lvlheight) -- 
        (\stepsize*2.25, \lvlbase -1.6*\lvlheight) -- 
        (\stepsize*2.25, \lvlbase - 2.3*\lvlheight) --
        (\stepsize*0.5, \lvlbase -2.3*\lvlheight) --
        (\stepsize*0.5, \lvlbase -1.6*\lvlheight) ;
%

    \node[align=left,anchor=north east] 
        at (\stepsize * 2.25,\lvlbase + -1.6*\lvlheight) 
            {\textit{(a) Content Factor}};

        \node[hid={3}{ctxemb}] (ctx_i) 
            at (\stepsize *1 , \lvlbase + -2.75*\lvlheight) {};    
        \node at (\stepsize*1, \lvlbase + -2.75*\lvlheight) 
            {$\srHid_i$}; 

        \node[hid={1}{factor}] (salience_i) 
            at (\stepsize *2, \lvlbase + -3*\lvlheight) {};    
        \node at (\stepsize*2, \lvlbase + -3*\lvlheight) 
            {$\srSalienceFactor_i$}; 
        \node[hid={3}{doc}] (doc) 
            at (\stepsize *1, \lvlbase + -3.25*\lvlheight) {};    
        \node at (\stepsize*1, \lvlbase + -3.25*\lvlheight) 
            {$\srDocEmb$}; 

        \draw[dep] (ctx_i.east) to (salience_i);
        \draw[dep] (doc.east) to (salience_i);

    \draw[rectangle,draw=black,dotted] 
        (\stepsize*0.5,\lvlbase -2.4*\lvlheight) -- 
        (\stepsize*2.25, \lvlbase -2.4*\lvlheight) -- 
        (\stepsize*2.25, \lvlbase - 3.5*\lvlheight) --
        (\stepsize*0.5, \lvlbase -3.5*\lvlheight) --
        (\stepsize*0.5, \lvlbase -2.4*\lvlheight) ;

    \node[align=left,anchor=north east] 
        at (\stepsize * 2.25,\lvlbase + -2.4*\lvlheight) 
            {\textit{(b) Centrality Factor}};

        \node[hid={3}{ctxemb}] (ctx_i) 
            at (\stepsize *1 , \lvlbase + -4.0*\lvlheight) {};    
        \node at (\stepsize*1, \lvlbase + -4.0*\lvlheight) 
            {$\srHid_i$}; 

        \node[hid={1}{factor}] (salience_i) 
            at (\stepsize *2, \lvlbase + -4.25*\lvlheight) {};    
        \node at (\stepsize*2, \lvlbase + -4.25*\lvlheight) 
            {$\srNoveltyFactor_i$}; 
        \node[hid={3}{sum}] (doc) 
            at (\stepsize *1, \lvlbase + -4.5*\lvlheight) {};    
        \node at (\stepsize*1, \lvlbase + -4.5*\lvlheight) 
            {$\srSum_i$}; 

        \draw[dep] (ctx_i.east) to (salience_i);
        \draw[dep] (doc.east) to (salience_i);


    \draw[rectangle,draw=black,dotted] 
        (\stepsize*0.5,\lvlbase -3.6*\lvlheight) -- 
        (\stepsize*2.25, \lvlbase -3.6*\lvlheight) -- 
        (\stepsize*2.25, \lvlbase - 4.75*\lvlheight) --
        (\stepsize*0.5, \lvlbase -4.75*\lvlheight) --
        (\stepsize*0.5, \lvlbase -3.6*\lvlheight) ;

    \node[align=left,anchor=north east] 
        at (\stepsize * 2.25,\lvlbase + -3.6*\lvlheight) 
            {\textit{(c) Novelty Factor}};



%?        \node[hid={3}{ctxemb}] (ctx_i) 
%?           at (\stepsize *1 , \lvlbase + -5.4*\lvlheight) {};    
%?        \node at (\stepsize*1, \lvlbase + -5.4*\lvlheight) 
%?            {$\srFinePositionEmb_i$}; 
%?
%?        \node[hid={1}{yellow}] (content_i) 
%?            at (\stepsize *2, \lvlbase + -5.4*\lvlheight) {};    
%?        \node at (\stepsize*2, \lvlbase + -5.4*\lvlheight) 
%?            {$\srFinePositionFactor_i$}; 
%?
%?        \draw[dep] (ctx_i) to (content_i);
%?    \draw[rectangle,draw=black,dotted] 
%?        (\stepsize*0.5,\lvlbase -4.6*\lvlheight -0.25 *\lvlheight) -- 
%?        (\stepsize*2.25, \lvlbase -4.6*\lvlheight - 0.25*\lvlheight) -- 
%?        (\stepsize*2.25, \lvlbase - 5.3*\lvlheight-0.25*\lvlheight) --
%?        (\stepsize*0.5, \lvlbase -5.3*\lvlheight - 0.25*\lvlheight) --
%?        (\stepsize*0.5, \lvlbase -4.6*\lvlheight - 0.25*\lvlheight) ;
%?
%?    \node[align=left,anchor=north east] 
%?        at (\stepsize * 2.25,\lvlbase + -4.85*\lvlheight) 
%?            {\textit{(d) Fine-grained}\\[-7pt]\textit{\phantom{(d) }Position Factor}};
%?
%?        \node[hid={3}{ctxemb}] (ctx_i) 
%?           at (\stepsize *1 , \lvlbase + -5.4*\lvlheight-0.8*\lvlheight) {};    
%?        \node at (\stepsize*1, \lvlbase + -5.4*\lvlheight-0.8*\lvlheight) 
%?            {$\srCoarsePositionEmb_i$}; 
%?
%?        \node[hid={1}{yellow}] (content_i) 
%?            at (\stepsize *2, \lvlbase + -5.4*\lvlheight-0.8*\lvlheight) {};    
%?        \node at (\stepsize*2, \lvlbase + -5.4*\lvlheight-0.8*\lvlheight) 
%?            {$\srCoarsePositionFactor_i$}; 
%?
%?        \draw[dep] (ctx_i) to (content_i);
%?    \draw[rectangle,draw=black,dotted] 
%?        (\stepsize*0.5,\lvlbase -4.6*\lvlheight -1.05 *\lvlheight) -- 
%?        (\stepsize*2.25, \lvlbase -4.6*\lvlheight - 1.05*\lvlheight) -- 
%?        (\stepsize*2.25, \lvlbase - 5.3*\lvlheight-1.05*\lvlheight) --
%?        (\stepsize*0.5, \lvlbase -5.3*\lvlheight - 1.05*\lvlheight) --
%?        (\stepsize*0.5, \lvlbase -4.6*\lvlheight - 1.05*\lvlheight) ;
%?
%?    \node[align=left,anchor=north east] 
%?        at (\stepsize * 2.25,\lvlbase + -4.85*\lvlheight-0.8*\lvlheight) 
%?            {\textit{(e) Coarse-grained}\\[-7pt]\textit{\phantom{(d) }Position Factor}};




\end{tikzpicture}


                \end{center}
                  \caption{Schematic of \srext~factors for computing salience 
                  estimates.}
                  \label{fig:srfactors}
          \end{minipage}}
\end{wrapfigure}
Salience factors for content, centrality, and novelty are computed
%~$\phi_i^{(\cdot)}$ is computed 
via the following equations for all $i \in \{1,\ldots,\docSize\}$,

\vspace{10pt}
\noindent \textit{(Fig.~\ref{fig:srfactors}.a) Content Factor} 
\begin{align}
    \srContentFactor_i &=\srContentWeight \srHid_i, 
\end{align}
\vspace{10pt}   \noindent \textit{(Fig.~\ref{fig:srfactors}.b) Centrality\footnote{\citet{nallapati2017summarunner} refer to this as the salience factor, but we rename it here to avoid confusion with the model's final predictions which we call salience estimates.} Factor}
\begin{align}
    \srSalienceFactor_i & = \srHid_i^T\srSalienceWeight \srDocEmb, 
\end{align}
\vspace{10pt} \noindent \textit{(Fig.~\ref{fig:srfactors}.c) Novelty Factor}
\begin{align}
    \srNoveltyFactor_i &= -\srHid_i^T \srNoveltyWeight \srSum_i, \label{eq:srnov} 
\end{align}
where $\srContentWeight \in \reals^{\srRepDim}$, $\srSalienceWeight,\srNoveltyWeight \in \reals^{\srRepDim \times \srRepDim}$ are learned parameters.

Finally, there are two factors for the fine and coarse-grained position,

\vspace{10pt} 
\noindent\textit{%(\ref{fig:srfactors}.d) 
Fine-grained Position Factor}
\begin{align}
       \srFinePositionFactor& = \srFinePositionWeight \srFinePositionEmb_i, 
\end{align}
\vspace{10pt} \textit{%(\ref{fig:srfactors}.e)
Coarse-grained Position Factor}
\begin{align}
           \srCoarsePositionFactor& = \srCoarsePositionWeight \srCoarsePositionEmb_i, 
\end{align}
where $\srFinePositionEmb_i$ and $\srCoarsePositionEmb_i$ are embeddings associated with the sentence position and sentence position quartile of the $i$-th 
sentence (e.g., sentence $s_7$ in a document with 12 sentences, would have 
embeddings $\srFinePositionEmb_7$ and $\srCoarsePositionEmb_2$ corresponding
to the seventh sentence position and $2^\textrm{nd}$ sentence position
quartile respectively).
Both $\srFinePositionWeight, \srCoarsePositionWeight \in \reals^{\srPosDim}$, and $\srFinePositionEmb_1,\ldots,\srFinePositionEmb_\docSizeMax,\srCoarsePositionEmb_1,\ldots,\srCoarsePositionEmb_4 \in \reals^{\srPosDim}$ are learned parameters of the \srext~extractor, and $\docSizeMax\in\naturals$ is the maximum
document size in sentences (when handling unusually long documents, sentences
with positions greater than $\docSizeMax$ are all mapped to $\srFinePositionEmb_\docSizeMax$).

The complete set of parameters for the \srext~extractor is 
\[\xParams = \left\{\srRRNNParams,\srLRNNParams,\srSentWeight,\srSentBias,
\srDocWeight,\srDocBias, \srContentWeight, \srSalienceWeight, \srNoveltyWeight,
\srFinePositionWeight, \srCoarsePositionWeight, \srFinePositionEmb_1,\ldots,\srFinePositionEmb_{\docSizeMax}, \srCoarsePositionEmb_1,\ldots,\srCoarsePositionEmb_4,\right\}. \]
In our experiments, we set $\srRNNDim=300$, $\srRNNDim=100$, $\srPosDim=16$,
and $\docSizeMax={\color{red}???}$. Dropout with drop probability of $0.25$
is applied to the \gru~outputs $\srrHid_i$ and $\srlHid_i$, as well as 
the contextual sentence embeddings $\srHid_i$ for all $i \in \{1,\ldots,\docSize\}$.



















%%A contextual representation of each sentence, $\srHid_i$, is then obtained by 
%%concatening the forward and backward \gru~outputs and running them
%%through a feed forward layer with a $\relu$~activation,
%%\begin{align}
%%\textit{(Contextual Sentence Embedding)} & \nonumber \\
%%\srHid_i  & = \relu\left(\srSentBias + \srSentWeight \left[ \begin{array}{c} \srrHid_i \\ \srlHid_i  \end{array} \right]  \right)
%%\end{align}
%
%
%
%
%~\\~\\~\\~\\
%\pagebreak
%
%iof the previous RNN outputis weighted by their extraction
%probabilities. 
%
%
%In Equation~\ref{eq:srnov}, $g_i$ is an iterative summary representation 
%computed as the
%sum of the previous $z_{<i}$ weighted by their extraction probabilities,
%\begin{align}
%g_i & = \sum_{j=1}^{i-1} p(y_j=1|y_{<j},h) \cdot z_j.
%\end{align}
%
%
%A representation of the whole document is made by 
%averaging contextual sentence embeddings,  
%\begin{align}
%\textit{(Document Embedding)} & \nonumber \\
%\srDocEmb  & = \tanh\left(\srDocBias + \srDocWeight \left(\frac{1}{\docSize}\sum_{i=1}^{\docSize} \srHid_i \right) \right)
%\end{align}
%
%
%Extraction predictions are made using 
%the RNN output at the $i$-th step, the document representation, and 
%$i$-th version of the summary representation, along with factors for 
%sentence location in the document. The use of the iteratively constructed
%summary representation creates a dependence of $y_i$ on all $y_{<i}$.
%See \autoref{fig:extractors}.d for a graphical layout.
%%and \autoref{app:srextractor} for details.
%
%Like the
%RNN~extractor it starts with a bidrectional GRU over the sentence 
%embeddings 
%\begin{align}
%    \rxhid_0 &= 0 \\
%    \rxhid_i &= \fgru(\sentEmb_i, \rxhid_{i-1}; \overrightarrow{\chi}) \\
%   \lxhid_{\docSize + 1} &= 0 \\
%    \lxhid_i &= \fgru(\sentEmb_i, \lxhid_{i+1}; \overleftarrow{\chi})
%\end{align}
%
%It then creates a representation
%of the whole document $q$ by passing the averaged GRU output states through
%a fully connected layer: 
%\begin{align}
%q = \tanh\left(b_q + W_q\frac{1}{\docSize}\sum_{i=1}^{\docSize} [\rxhid_i; \lxhid_i] \right)
%\end{align}
%A concatentation of the GRU outputs at each step
%are passed through a separate fully connected layer to create a 
%sentence representation $z_i$, where
%\begin{align}
%    \xhid_i &= \relu\left(b_z + W_z [\rxhid_i; \lxhid_i]\right).
%\end{align}
%The extraction probability is then determined by contributions from five 
%sources:
%\begin{align}
%    \textit{content} &\quad a^{(con)}_i=W^{(con)} z_i, \\
%    \textit{salience}&\quad a^{(sal)}_i = z_i^TW^{(sal)} q, \\
%    \textit{novelty}&\quad a^{(nov)}_i = -z_i^TW^{(nov)} \tanh(g_i), \label{eq:srnov} \\
%    \textit{position}&\quad a^{(pos)}_i = W^{(pos)} l_i, \\
%    \textit{quartile}&\quad a^{(qrt)}_i = W^{(qrt)} r_i,
%\end{align}
%where $l_i$ and $r_i$ are embeddings associated with the $i$-th sentence
%position and the quarter of the document containing sentence $i$ respectively.
%In Equation~\ref{eq:srnov}, $g_i$ is an iterative summary representation 
%computed as the
%sum of the previous $z_{<i}$ weighted by their extraction probabilities,
%\begin{align}
%g_i & = \sum_{j=1}^{i-1} p(y_j=1|y_{<j},h) \cdot z_j.
%\end{align}
%Note that the presence of this term induces dependence of each 
%$\bsal_i$ to 
%all $\bsal_{<i}$ similarly to the Cheng \& Lapata extractor.
%
%The final extraction probability is the logistic sigmoid of the
%sum of these terms plus a bias,
%\begin{align}
%    p(y_i=1|y_{<i}, h) &= \sigma\left(\begin{array}{l}
%      a_i^{(con)} + a_i^{(sal)} + a_i^{(nov)} \\
%  + a_i^{(pos)}  + a_i^{(qrt)} + b \end{array}\right).
%\end{align}
%The weight matrices $W_q$, $W_z$, $W^{(con)}$, $W^{(sal)}$, $W^{(nov)}$, $W^{(pos)}$,
%$W^{(qrt)}$ and bias terms $b_q$, $b_z$, and $b$ are learned parameters;
%The GRUs have separate learned parameters.
%The hidden layer size of the GRU is 300 for each direction $z_i$, $q$, and $g_i$ have 100 dimensions. The position and quartile embeddings are 16 dimensional each.
%Dropout with drop probability .25 is applied to the GRU outputs and to $z_i$.
%%?
%%?
%%?
%
%Note that in the original paper, the SummaRunner extractor was paired 
%with
%an \textit{RNN} sentence encoder, but in this work we experiment with a variety
%of sentence encoders.
%%?
%
%
%
%%?A document representation $q$ is created by passing the 
%%?averaged RNN output through a fully connected layer.
%%?
%%?Given the RNN output $z_t$ at the step $t$, the following scores are created:
%%?\begin{enumerate}[nolistsep,noitemsep]
%%?\item a content score $W^{(con)}z_t$,
%%?\item a salience score $z_t^TW^{(sal)}q$,
%%?\item a novely score $-z_t^TW^{(nov)}\tanh(g_t)$,
%%?\end{enumerate}
%%?where $g_t = \sum_{i=1}^{t-1} p(y_i=1|y_{<i}, h_{<i}) \cdot z_i$.
%%?These scores are summed along with a bias term and a bias for sentence 
%%?position and the quarter of the document\hal{what does ``the quarter of the document'' mean? sentence position quartile?} and fed through a sigmoid activation
%%?to compute $p(y_t=1|y_{<t}, h_{<t})$.
%
%
%\paragraph{Proposed Sentence Extractors}
%We propose two sentence extractor models that 
%make a stronger conditional independence 
%assumption $p(\bsal|\sentEmb)=\prod_{i=1}^\docSize p(\bsal_i|\sentEmb)$,
%essentially making independent predictions conditioned on $\sentEmb$.
%%In theory, our models should \hal{why should they?} perform worse because of this, however, as
%%we later show, this is not the case empirically.
%
%


\subsubsection{\rnnext~Extractor}



\begin{figure}[t]
    \fbox{\begin{minipage}{\textwidth}
\center
\scalebox{0.75}{
\begin{tikzpicture}[
  dep/.style ={
    ->,line width=0.3mm
  },
  hid/.style 2 args={
    rectangle split,
    draw=#2,
    rectangle split parts=#1,
    fill=#2!20,
    minimum width=5mm,
    minimum height=5mm,
    outer sep=2mm},
  mlp/.style 2 args={
    rectangle split,
    rectangle split horizontal,
    draw=#2,
    rectangle split parts=#1,
    fill=#2!20,
    outer sep=2mm},
  sal/.style={
    circle, 
    minimum width=8mm,
    outer sep=2mm,
    draw=#1, 
    fill=#1!20},
]

  \def\stepsize{2}%
  \def\lvlbase{0}%
  \def\lvlheight{3}%
 

    % Sentence Embeddings    
  \foreach \step in {1,...,3} {
    \node[hid={3}{sentemb}] (s\step) at (\stepsize*\step, \lvlbase) {};    
    \node at (\stepsize*\step, \lvlbase) {$\sentEmb_\step$};    
   }
%    \foreach \step [count=\i from 1] in {5,6} {
%        \node[hid={3}{green}] (s\step) at (\stepsize*\step, \lvlbase) {};    
%        \node at (\stepsize*\step, \lvlbase) {$\sentEmb_\i$};    
%    %\draw[->] (i\step.north) -> (e\step.south);
%    }
%
%       \node[hid={3}{red}] (s4) at (\stepsize*4, \lvlbase) {};    
%       \node at (\stepsize*4, \lvlbase) {$\sentEmb_0$};    
%
%    % RNN hidden states
    \foreach \step in {1,...,3} {
        \node[hid={3}{rencemb}] (rrnn_\step) 
            at (\stepsize *\step-0.5, \lvlbase + \lvlheight) {};    
        \node at (\stepsize *\step-0.5, \lvlbase + \lvlheight) 
            {$\rnnextRHid_\step$}; 
        \node[hid={3}{lencemb}] (lrnn_\step) 
            at (\stepsize *\step+0.5, \lvlbase + \lvlheight + 1.0) {};    
        \node at (\stepsize *\step+0.5, \lvlbase + \lvlheight+ 1.0) 
            {$\rnnextLHid_\step$}; 
        \draw[dep] (s\step.north) -- (rrnn_\step.south);
        \draw[dep] (s\step.north) -- (lrnn_\step.south);


        \node[hid={3}{ctxemb}] (ctx_\step) 
            at (\stepsize *\step, \lvlbase + 2.25*\lvlheight) {};    
        \node at (\stepsize *\step, \lvlbase + 2.25*\lvlheight) 
            {$\rnnextHid_\step$}; 
        \draw[dep] (rrnn_\step.north) -- (ctx_\step.south);
        \draw[dep] (lrnn_\step.north) -- (ctx_\step.south);


        \node[sal={sal}] (sal_\step) 
            at (\stepsize *\step, \lvlbase + 3*\lvlheight) {};    
        \node at (\stepsize *\step, \lvlbase + 3*\lvlheight) 
            {$\psal_i$}; 
        \draw[dep] (ctx_\step.north) -- (sal_\step.south);
        %\draw[dep] (lrnn_\step.north) -- (ctx_\step.south);


    }
    \foreach \start [count=\stop from 2] in {1,...,2} {
        \draw[dep] ($ (rrnn_\start.east) - (0,0.3)$) 
            -- ($ (rrnn_\stop.west) - (0,0.3) $);
        \draw[dep] ($(lrnn_\stop.west) + (0,0.3)$) 
            -- ($ (lrnn_\start.east) + (0,0.3)   $);
    }

    \draw[rectangle,draw=black,dotted] 
        (\stepsize*-2.5,\lvlbase + 3.5*\lvlheight) -- 
        (\stepsize*3.5, \lvlbase + 3.5*\lvlheight) -- 
        (\stepsize*3.5, \lvlbase + 2.7*\lvlheight) --
        (\stepsize*-2.5, \lvlbase + 2.7*\lvlheight) --
        (\stepsize*-2.5, \lvlbase + 3.5*\lvlheight) ;

    \node[align=left,anchor=north west] 
        at (\stepsize * -2.5,\lvlbase + 3.5*\lvlheight) 
        {\textit{(c) Salience Estimates}};

    \draw[rectangle,draw=black,dotted] 
        (\stepsize*-2.5,\lvlbase + 2.6*\lvlheight) -- 
        (\stepsize*3.5, \lvlbase + 2.6*\lvlheight) -- 
        (\stepsize*3.5, \lvlbase + 1.8*\lvlheight) --
        (\stepsize*-2.5, \lvlbase + 1.8*\lvlheight) --
        (\stepsize*-2.5, \lvlbase + 2.6*\lvlheight) ;

    \node[align=left,anchor=north west] 
        at (\stepsize * -2.5,\lvlbase + 2.6*\lvlheight) 
        {\textit{(b) Contextual Sentence Embeddings}};

    \draw[rectangle,draw=black,dotted] 
        (\stepsize*-2.5,\lvlbase + 0.5*\lvlheight) -- 
        (\stepsize*3.5, \lvlbase + 0.5*\lvlheight) -- 
        (\stepsize*3.5, \lvlbase + -0.50*\lvlheight) --
        (\stepsize*-2.5, \lvlbase + -0.50*\lvlheight) --
        (\stepsize*-2.5, \lvlbase + 0.5*\lvlheight) ;

    \node[align=left,anchor=north west] 
        at (\stepsize * -2.5,\lvlbase + 0.5*\lvlheight) 
        {\textit{Sentence Embeddings}\\\textit{(Sentence Encoder Output)}};


    \draw[rectangle,draw=black,dotted] 
        (\stepsize*-2.5,\lvlbase + 1.7*\lvlheight) -- 
        (\stepsize*3.5, \lvlbase + 1.7*\lvlheight) -- 
        (\stepsize*3.5, \lvlbase + 0.6*\lvlheight) --
        (\stepsize*-2.5, \lvlbase + 0.6*\lvlheight) --
        (\stepsize*-2.5, \lvlbase + 1.7*\lvlheight) ;


    \node[align=left,anchor=north west] 
        at (\stepsize * -2.5,\lvlbase + 1.7*\lvlheight) 
        {\textit{(a) Left and Right Partial}\\\textit{\phantom{(a) }Contexual Sentence Embeddings}};


\end{tikzpicture}}

\caption{Schematic for the \rnnext~sentence extractor.}
\label{fig:rnnext}
\end{minipage}}
\end{figure}


    Our first proposed model is a very simple 
    \bidirectional~\recurrentneuralnetwork~based tagging
    model \citep{sometagfolks}, which we refer to as the \rnnext~extractor.
See \autoref{fig:rnnext} for a visual depiction of   
the extractor.
As in the \srext~extractor, the first step of the \rnnext~extractor
is to run a \bidirectional~\recurrentneuralnetwork~over the sentence 
embeddings produced by the sentence encoder layer, which produces
left and right partial contextual embeddings $\rnnextRHid_i$ and $\rnnextLHid_i$
respectively,

  \vspace{10pt} \noindent\textit{(Fig.~\ref{fig:rnnext}.a) Left and Right Partial Contexual Sentence Embeddings}
\begin{align}
    \rnnextRHid_0 = \zeroEmb, &\quad \rnnextLHid_{\docSize+1} = \zeroEmb, \\
    \forall i : \;\; i \in \{1,\ldots,\docSize\}& \nonumber \\
 \rnnextRHid_i &= \fgru\left(\sentEmb_i, \rnnextRHid_{i-1}; \rnnextRParams\right), \\
 \rnnextLHid_i &= \fgru\left(\sentEmb_i, \rnnextLHid_{i+1};\rnnextLParams\right), 
\end{align}
where $\rnnextRHid_i,\rnnextLHid_i \in \reals^{\rnnextRNNDim}$,
and $\rnnextRParams$ and $\rnnextLParams$ are the forward and backward
\gru~parameters.

%As in the \recurrentneuralnetwork~sentence encoder we use a \gru~cell
%to implement the forward and backward \recurrentneuralnetwork s.
The left and right partial contextual embeddings of each sentence 
are then passed through a \feedforward~layer to produce a contextual
sentence embeddings $\rnnextHid_i$,

\vspace{10pt}\noindent\textit{(Fig.~\ref{fig:rnnext}.b) Contextual Sentence Embeddings}
\begin{align}
    \forall i :\;\; i \in \{1,\ldots,\docSize\}& \nonumber \\
   \rnnextHid_i &= \relu\left(
    \rnnextHidWeight
    \left[ \begin{array}{c} 
        \rnnextRHid_i \\
        \rnnextLHid_i \end{array}\right] + \rnnextHidBias \right),
 %p(\bsal_i=1|\sentEmb_i,\ldots,\sentEmb_\docSize) &= \sigma\left(\rnnextPredWeight\rnnextHid_i + \rnnextPredBias  \right)
\end{align}
where $\rnnextHidWeight \in \reals^{\rnnextHidDim \times 2 \rnnextRNNDim}$
and $\rnnextHidBias \in \reals^{\rnnextHidDim}$ are learned parameters.

Another \feedforward~layer
with a logsitic sigmoid activation computes the actual salience 
estimates $\psal_1,\ldots,\psal_\docSize$ where $\psal_i = p(\bsal_i=1|\sentEmb_1,\ldots,\sentEmb_\docSize)$,

\vspace{10pt}\noindent\textit{(Fig.~\ref{fig:rnnext}.c) Salience Estimates}
\begin{align}
    \forall i : \;\; i \in \{1,\ldots,\docSize\}& \nonumber \\
    \psal_i &= p(\bsal_i=1|\sentEmb_i,\ldots,\sentEmb_\docSize) = \sigma\left(\rnnextPredWeight\rnnextHid_i + \rnnextPredBias  \right)
\end{align}
where $\rnnextPredWeight \in \reals^{1 \times \rnnextHidDim}$
and $\rnnextPredBias \in \reals$ are learned parameters.

The complete set of parameters for the extractor is 
\[ \xParams = \left\{\rnnextRParams, \rnnextLParams, \rnnextHidWeight, \rnnextHidBias, \rnnextPredWeight, \rnnextPredBias \right\}.\]
In our experiments, we set $\rnnextRNNDim=300$ and $\rnnextHidDim=100$.
Dropout with drop probability of $.25$ is applied to $\rnnextRHid_i, \rnnextLHid_i,$ and $\rnnextHid_i$ for $i \in \{1,\ldots,\docSize\}$.

\FloatBarrier




\subsubsection{\sts~Extractor} 

One shortcoming of the RNN extractor is that long range
information from one end of the document may not easily be able to affect 
extraction probabilities of sentences at the other end. 
Our second proposed model, the \sts~extractor mitigates this problem with an 
attention 
mechanism commonly
used for neural machine translation \citep{bahdanau2014neural} and 
abstractive summarization \citep{see2017get}. 
The sentence embeddings produced by the sentence encoder are first
encoded by a \bidirectional~\gru, which produces left and right
partial contextual sentence embeddings, 

\begin{figure}[t]
\fbox{\begin{minipage}{\textwidth}
\center
\scalebox{0.75}{
\begin{tikzpicture}[
  dep/.style ={
    ->,line width=0.3mm
  },
  hid/.style 2 args={
    rectangle split,
    draw=#2,
    rectangle split parts=#1,
    fill=#2!20,
    minimum width=5mm,
    minimum height=5mm,
    outer sep=2mm},
  mlp/.style 2 args={
    rectangle split,
    rectangle split horizontal,
    draw=#2,
    rectangle split parts=#1,
    fill=#2!20,
    outer sep=2mm},
  sal/.style={
    circle, 
    minimum width=8mm,
    outer sep=2mm,
    draw=#1, 
    fill=#1!20},
]

  \def\stepsize{2}%
  \def\lvlbase{0}%
  \def\lvlheight{3}%
 

    % Sentence Embeddings    
  \foreach \step in {1,...,3} {
    \node[hid={3}{sentemb}] (s\step) at (\stepsize*\step, \lvlbase) {};    
    \node at (\stepsize*\step, \lvlbase) {$\sentEmb_\step$};    
   }
%    \foreach \step [count=\i from 1] in {5,...,7} {
%        \node[hid={3}{green}] (s\step) at (\stepsize*\step, \lvlbase) {};    
%        \node at (\stepsize*\step, \lvlbase) {$\sentEmb_\i$};    
%    %\draw[->] (i\step.north) -> (e\step.south);
%    }

%       \node[hid={3}{red}] (s4) at (\stepsize*4, \lvlbase) {};    
%       \node at (\stepsize*4, \lvlbase) {$\sentEmb_0$};    
%
%    % RNN hidden states
    \foreach \step in {1,...,3} {
        \node[hid={3}{rencemb}] (rrnn_\step) 
            at (\stepsize *\step+0.5, \lvlbase + \lvlheight) {};    
        \node at (\stepsize *\step+0.5, \lvlbase + \lvlheight) 
            {$\stsextREncHid_\step$}; 

        \node[hid={3}{lencemb}] (lrnn_\step) 
            at (\stepsize *\step-0.5, \lvlbase + \lvlheight+1.0) {};    
        \node at (\stepsize *\step-0.5, \lvlbase + \lvlheight+1.0) 
            {$\stsextLEncHid_\step$}; 


        \node[hid={3}{encctxemb}] (enc_ctx_\step) 
            at (\stepsize *\step, \lvlbase + 2.5*\lvlheight) {};    
        \node at (\stepsize *\step, \lvlbase + 2.5*\lvlheight) 
            {$\stsextEncHid_\step$}; 

        \draw[dep] (s\step.north) -- (lrnn_\step.south);
        \draw[dep] (s\step.north) -- (rrnn_\step.south);

        \draw[dep] (lrnn_\step.north) -- (enc_ctx_\step.south);
        \draw[dep] (rrnn_\step.north) -- (enc_ctx_\step.south);
    }
    \foreach \start [count=\stop from 2] in {1,...,2} {
        \draw[dep] ($ (rrnn_\start.east) - (0,0.3)$) 
            -- ($ (rrnn_\stop.west) - (0,0.3) $);
        \draw[dep] ($(lrnn_\stop.west) + (0,0.3)$) 
            -- ($ (lrnn_\start.east) + (0,0.3)   $);
    }


    \draw[rectangle,draw=black,dotted] 
        (\stepsize*-2.75,\lvlbase + 0.5*\lvlheight) -- 
        (\stepsize*3.5, \lvlbase + 0.5*\lvlheight) -- 
        (\stepsize*3.5, \lvlbase + -0.50*\lvlheight) --
        (\stepsize*-2.75, \lvlbase + -0.50*\lvlheight) --
        (\stepsize*-2.75, \lvlbase + 0.5*\lvlheight) ;

    \node[align=left,anchor=north west] 
        at (\stepsize * -2.75,\lvlbase + 0.5*\lvlheight) 
        {\textit{Sentence Embeddings}\\\textit{(Sentence Encoder Output)}};

    \draw[rectangle,draw=black,dotted] 
        (\stepsize*-2.75,\lvlbase + 3.0*\lvlheight) -- 
        (\stepsize*3.5, \lvlbase + 3.0*\lvlheight) -- 
        (\stepsize*3.5, \lvlbase + 2.0*\lvlheight) --
        (\stepsize*-2.75, \lvlbase + 2.0*\lvlheight) --
        (\stepsize*-2.75, \lvlbase + 3.0*\lvlheight) ;

    \node[align=left,anchor=north west] 
        at (\stepsize * -2.75,\lvlbase + 3.0*\lvlheight) 
        {\textit{(b) Encoder Contextual Sentence Embeddings}};

    \draw[rectangle,draw=black,dotted] 
        (\stepsize*-2.75,\lvlbase + 1.8*\lvlheight) -- 
        (\stepsize*3.5, \lvlbase + 1.8*\lvlheight) -- 
        (\stepsize*3.5, \lvlbase + 0.6*\lvlheight) --
        (\stepsize*-2.75, \lvlbase + 0.6*\lvlheight) --
        (\stepsize*-2.75, \lvlbase + 1.8*\lvlheight) ;

    \node[align=left,anchor=north west] 
        at (\stepsize * -2.75,\lvlbase + 1.8*\lvlheight) 
        {\textit{(a) Encoder Left and Right Partial}\\\textit{\phantom{(a) }Contextual Sentence Embeddings}};


\end{tikzpicture}}
\caption{Schematic for the encoder contextual sentence embeddings as 
computed in the \stsext~sentence extractor.}
\label{fig:stsext1}
\end{minipage}}
\end{figure}



 \vspace{10pt}   \noindent \textit{(Fig.~\ref{fig:stsext1}.a) Encoder Left and Right Partial Contextual Sentence Embeddings}
\begin{align}
        % eta_0 = 0 
        \stsextREncHid_0  = \zeroEmb, & \quad 
        \stsextLEncHid_{\docSize + 1}  = \zeroEmb, \\
\forall i : \;\; i \in \{1,\ldots,\docSize\}& \nonumber \\
\stsextREncHid_i & = \fgru\left(
            \sentEmb_i, \stsextREncHid_{i-1}; 
            \stsextREncParams\right), \\
\stsextLEncHid_i &= \fgru\left(
            \sentEmb_i, \stsextLEncHid_{i+1}; 
            \stsextLEncParams\right),
\end{align}
where $\stsextREncHid_i,\stsextLEncHid_i \in \reals^{\stsextRNNDim}$ and 
$\stsextREncParams$ and $\stsextLEncParams$ are the forward and backward encoder \gru~parameters respectively. The encoder contextual sentence embeddings
are then formed by simply concatenating the encoder left and right 
partial contextual embeddings,

\vspace{10pt} \noindent \textit{(Fig.~\ref{fig:stsext1}.b) Encoder Contextual Sentence Embedding}
\begin{align}
        % eta_i = [eta_i,>, eta_i,<]   
\forall i : \;\; i \in \{1,\ldots,\docSize\} & \nonumber \\
        \stsextEncHid_i & = \left[\begin{array}{c}
            \stsextREncHid_i\\ 
            \stsextLEncHid_i\end{array}\right] 
\end{align}




The final output of each encoder \gru~initializes a separate decoder \gru~which is then run over the sentence embeddings a second time, 


\vspace{10pt} \noindent  \textit{(Fig.~\ref{fig:stsext2}.a) Decoder Left and Right Partial Contextual Sentence Embeddings}
\begin{align}
        % zeta_0 = eta_n
\stsextRDecHid_0 & = \fgru\left(\stsextSent_>,   \stsextREncHid_\docSize; \stsextRDecParams\right),\\ 
\stsextLDecHid_{\docSize+1} & = \fgru\left(\stsextSent_<, \stsextLEncHid_1;\stsextLDecParams\right),\\ 
    \forall i : \;\; i \in \{1,\ldots,\docSize\}& \nonumber\\
        % zeta_i = gru(h_i, zeta_i-1; \chi_zeta)  
        \stsextRDecHid_i & = \fgru(
            \sentEmb_i,  \stsextRDecHid_{i-1}; 
            \stsextRDecParams), \\
        % zeta_i = gru(h_i, zeta_i+1; chi_zeta)
        \stsextLDecHid_i & = \fgru(
            \sentEmb_i,  \stsextLDecHid_{i+1}; 
            \stsextLDecParams) 
\end{align}
where $\stsextSent_>$ and $\stsextSent_<$ are special ``begin decoding'' input
embeddings for the forward and backward decoder respectively, 
$\stsextRDecHid_i, \stsextLDecHid_i \in \reals^{\stsextRNNDim}$, 
and $\stsextRDecParams$ and $\stsextLDecParams$ are the parameters for the
forward and backward decoder \gru s respectively.
%

The decoder left and right partial contextual sentence embeddings ($\stsextRDecHid_i$ and $\stsextLDecHid_i$) are then
concatenated to form the decoder contextual sentence embeddings,


\vspace{10pt} \noindent \textit{(Fig.~\ref{fig:stsext2}.b) Decoder Contextual Sentence Embeddings} 
\begin{align}
    \forall i : \;\; i \in \{1,\ldots,\docSize\}& \nonumber \\
        \stsextDecHid_i & = \left[\begin{array}{c}
            \stsextRDecHid_i\\ 
            \stsextLDecHid_i\end{array}\right].
\end{align}



\begin{figure}[p]
\fbox{\begin{minipage}{\textwidth}
\center
\scalebox{0.75}{
\begin{tikzpicture}[
  dep/.style ={
    ->,line width=0.3mm
  },
  hid/.style 2 args={
    rectangle split,
    draw=#2,
    rectangle split parts=#1,
    fill=#2!20,
    minimum width=5mm,
    minimum height=5mm,
    outer sep=2mm},
  mlp/.style 2 args={
    rectangle split,
    rectangle split horizontal,
    draw=#2,
    rectangle split parts=#1,
    fill=#2!20,
    outer sep=2mm},
  sal/.style={
    circle, 
    minimum width=8mm,
    outer sep=2mm,
    draw=#1, 
    fill=#1!20},
]

  \def\stepsize{2}%
  \def\lvlbase{0}%
  \def\lvlheight{3}%
 

    \node[hid={3}{decsentemb}] (s0) at (\stepsize*0, \lvlbase) {};    
    \node at (\stepsize*0, \lvlbase) {$\stsextSent_>$};    


    \node[hid={3}{decsentemb}] (s4) at (\stepsize*4, \lvlbase) {};    
    \node at (\stepsize*4, \lvlbase) {$\stsextSent_{<}$};    

    % Sentence Embeddings    
  \foreach \step in {1,...,3} {
    \node[hid={3}{sentemb}] (s\step) at (\stepsize*\step, \lvlbase) {};    
    \node at (\stepsize*\step, \lvlbase) {$\sentEmb_\step$};    
   }
%    \foreach \step [count=\i from 1] in {5,...,7} {
%        \node[hid={3}{green}] (s\step) at (\stepsize*\step, \lvlbase) {};    
%        \node at (\stepsize*\step, \lvlbase) {$\sentEmb_\i$};    
%    %\draw[->] (i\step.north) -> (e\step.south);
%    }

%       \node[hid={3}{red}] (s4) at (\stepsize*4, \lvlbase) {};    
%       \node at (\stepsize*4, \lvlbase) {$\sentEmb_0$};    
%
%    % RNN hidden states

        \node[hid={3}{rdecemb}] (rrnn_0) 
            at (\stepsize *0+0.5, \lvlbase + \lvlheight) {};    
        \node at (\stepsize *0+0.5, \lvlbase + \lvlheight) 
            {$\stsextRDecHid_0$}; 

        \node[hid={3}{ldecemb}] (lrnn_4) 
            at (\stepsize *4-0.5, \lvlbase + \lvlheight+1.0) {};    
        \node at (\stepsize *4-0.5, \lvlbase + \lvlheight+1.0) 
            {$\stsextLDecHid_4$}; 

        \node[hid={3}{rencemb}] (rrnn_m1) 
            at (-\stepsize +0.5, \lvlbase + \lvlheight) {};    
        \node at (-\stepsize +0.5, \lvlbase + \lvlheight) 
            {$\stsextREncHid_3$}; 

        \node[hid={3}{lencemb}] (lrnn_m1) 
            at (\stepsize*5 -0.5, \lvlbase + \lvlheight+1.0) {};    
        \node at (\stepsize*5 -0.5, \lvlbase + \lvlheight+1.0) 
            {$\stsextLEncHid_1$}; 

            \node[align=left,anchor=north west,text width=4.3cm] (lbl)  
                at (\stepsize*5.0, \lvlbase + \lvlheight*1.65) 
                {\textit{Encoder Right Partial Contextual Sentence Embedding}};    
    %\draw[dotted] (lbl.east) to (s0);
        \draw[rectangle,draw=black,dotted] 
        (\stepsize*4.5,\lvlbase + 1.65*\lvlheight) -- 
        (\stepsize*7.5, \lvlbase + 1.65*\lvlheight) -- 
        (\stepsize*7.5, \lvlbase + 0.65*\lvlheight) --
        (\stepsize*4.5, \lvlbase + 0.65*\lvlheight) --
        (\stepsize*4.5, \lvlbase + 1.65*\lvlheight) ;


        \draw[rectangle,draw=black,dotted] 
        (\stepsize*-3.0,\lvlbase + 1.65*\lvlheight) -- 
        (\stepsize*-0.5, \lvlbase + 1.65*\lvlheight) -- 
        (\stepsize*-0.5, \lvlbase + 0.65*\lvlheight) --
        (\stepsize*-3.0, \lvlbase + 0.65*\lvlheight) --
        (\stepsize*-3.0, \lvlbase + 1.65*\lvlheight) ;

            \node[align=left,anchor=north west,text width=4.3cm] (lbl)  
                at (\stepsize*-3.0, \lvlbase + \lvlheight*1.65) 
                {Encoder Left Partial Contextual Sentence Embedding};    





    \foreach \step in {1,...,3} {
        \node[hid={3}{rdecemb}] (rrnn_\step) 
            at (\stepsize *\step+0.5, \lvlbase + \lvlheight) {};    
        \node at (\stepsize *\step+0.5, \lvlbase + \lvlheight) 
            {$\stsextRDecHid_\step$}; 

        \node[hid={3}{ldecemb}] (lrnn_\step) 
            at (\stepsize *\step-0.5, \lvlbase + \lvlheight+1.0) {};    
        \node at (\stepsize *\step-0.5, \lvlbase + \lvlheight+1.0) 
            {$\stsextLDecHid_\step$}; 


        \node[hid={3}{decctxemb}] (enc_ctx_\step) 
            at (\stepsize *\step, \lvlbase + 2.5*\lvlheight) {};    
        \node at (\stepsize *\step, \lvlbase + 2.5*\lvlheight) 
            {$\stsextDecHid_\step$}; 

        \draw[dep] (s\step.north) -- (lrnn_\step.south);
        \draw[dep] (s\step.north) -- (rrnn_\step.south);

        \draw[dep] (lrnn_\step.north) -- (enc_ctx_\step.south);
        \draw[dep] (rrnn_\step.north) -- (enc_ctx_\step.south);
    }
    \foreach \start [count=\stop from 2] in {1,...,2} {
        \draw[dep] ($ (rrnn_\start.east) - (0,0.3)$) 
            -- ($ (rrnn_\stop.west) - (0,0.3) $);
        \draw[dep] ($(lrnn_\stop.west) + (0,0.3)$) 
            -- ($ (lrnn_\start.east) + (0,0.3)   $);
    }

        \draw[dep] ($ (rrnn_m1.east) - (0,0.3)$) 
            -- ($ (rrnn_0.west) - (0,0.3) $);
        \draw[dep] ($ (rrnn_0.east) - (0,0.3)$) 
            -- ($ (rrnn_1.west) - (0,0.3) $);

            \draw[dep] (s0.north) to (rrnn_0.south);
            \draw[dep] (s4.north) to (lrnn_4.south);

        \draw[dep] ($ (lrnn_m1.west) + (0,0.3)$) 
            -- ($ (lrnn_4.east) + (0,0.3) $);

        \draw[dep] ($ (lrnn_4.west) + (0,0.3)$) 
            -- ($ (lrnn_3.east) + (0,0.3) $);

%    \foreach \step [count=\i from 0] in {4,...,7} {
%        \node[hid={3}{orange}] (rnn_\step) 
%            at (\stepsize *\step, \lvlbase + \lvlheight) {};    
%        \node at (\stepsize *\step, \lvlbase + \lvlheight) {$\xDecHid_\i$}; 
%    }
%
%    \foreach \step in {1,...,6} {
%        \draw[dep] (s\step.north) to (rnn_\step.south);
%    }
%    \foreach \start [count=\stop from 2] in {1,...,5} {
%        \draw[dep] (rnn_\start.east) to (rnn_\stop.west);
%    }
%
%
%%    \foreach \step [count=\i from 1] in {4,...,6} {
%%        \node[hid={3}{green}] (ctx_\i) 
%%            at (\stepsize *\step, \lvlbase + 2*\lvlheight) {};    
%%        \node at (\stepsize *\step, \lvlbase + 2*\lvlheight) {$\xPredHid_\i$}; 
%%        \draw[dep] (rnn_\step.north) to (ctx_\i.south);
%%        \draw[dep] (rnn_\i.north) to (ctx_\i.south west);
%%        \node[sal={blue}] (sal_\i) at (\stepsize * \step,\lvlbase + 3*\lvlheight) {};
%%        \node at (\stepsize * \step,\lvlbase + 3*\lvlheight) {$\bsal_\i$};
%%        \draw[dep] (ctx_\i.north) to (sal_\i.south);
%%    }
%%
%%    \foreach \step [count=\i from 1] in {5,...,6} {
%%        \draw[dep] (sal_\i) -- (\stepsize * \step - \stepsize / 2,
%%                            \lvlbase + 3*\lvlheight) 
%%                     -- (\stepsize * \step - \stepsize / 2,
%%                            \lvlbase + 0* \lvlheight) -> (s\step) ;
%%
%%    }
%%    \draw[dep] (s1.south) -- ($ (s1.south) + (0,-0.3)$) --($ (s5.south) + (0,-0.3)$) -- (s5.south);
%%
%%    \draw[dep] (s2.south) -- ($ (s2.south) + (0,-0.5)$) --($ (s6.south) + (0,-0.5)$) -- (s6.south);
%%
%%    \draw[rectangle,draw=black,dotted] 
%%        (\stepsize * 3.5,\lvlbase + 3.5*\lvlheight) -- 
%%        (\stepsize*6.5, \lvlbase + 3.5*\lvlheight) -- 
%%        (\stepsize*6.5, \lvlbase + 2.6*\lvlheight) --
%%        (\stepsize*3.5, \lvlbase + 2.6*\lvlheight) --
%%        (\stepsize*3.5, \lvlbase + 3.5*\lvlheight) ;
%%
%%    \node[align=left,anchor=north west] 
%%        at (\stepsize * 3.5,\lvlbase + 3.5*\lvlheight) 
%%        {\textit{Salience Estimates}};
%%
%%    \draw[rectangle,draw=black,dotted] 
%%        (\stepsize*0.5,\lvlbase + 2.5*\lvlheight) -- 
%%        (\stepsize*6.5, \lvlbase + 2.5*\lvlheight) -- 
%%        (\stepsize*6.5, \lvlbase + 1.5*\lvlheight) --
%%        (\stepsize*0.5, \lvlbase + 1.5*\lvlheight) --
%%        (\stepsize*0.5, \lvlbase + 2.5*\lvlheight) ;
%%
%%    \node[align=left,anchor=north west] 
%%        at (\stepsize * 0.5,\lvlbase + 2.5*\lvlheight) 
%%        {\textit{Contextual Sentence Embeddings}};
%%
%%    \draw[rectangle,draw=black,dotted] 
%%        (\stepsize*-0.5,\lvlbase + 0.5*\lvlheight) -- 
%%        (\stepsize*3.5, \lvlbase + 0.5*\lvlheight) -- 
%%        (\stepsize*3.5, \lvlbase + -0.9*\lvlheight) --
%%        (\stepsize*-0.5, \lvlbase + -0.9*\lvlheight) --
%%        (\stepsize*-0.5, \lvlbase + 0.5*\lvlheight) ;
%%
%%    \node[align=left,anchor=south west] 
%%        at (\stepsize * -0.5,\lvlbase + -0.9*\lvlheight) 
%%        {\textit{Sentence Embeddings}\\\textit{(Sentence Encoder Output)}};
%%
%%    \draw[rectangle,draw=black,dotted] 
%%        (\stepsize*4.3,\lvlbase + 0.5*\lvlheight) -- 
%%        (\stepsize*6.5, \lvlbase + 0.5*\lvlheight) -- 
%%        (\stepsize*6.5, \lvlbase + -0.9*\lvlheight) --
%%        (\stepsize*4.3, \lvlbase + -0.9*\lvlheight) --
%%        (\stepsize*4.3, \lvlbase + 0.5*\lvlheight) ;
%%
%%    \node[align=left,anchor=south west] 
%%        at (\stepsize * 4.3,\lvlbase + -0.9*\lvlheight) 
%%        {\textit{Salience Gated}\\\textit{Sentence Embeddings}};
%%

    

    \draw[rectangle,draw=black,dotted] 
        (\stepsize*-0.75,\lvlbase + 3.0*\lvlheight) -- 
        (\stepsize*3.5, \lvlbase + 3.0*\lvlheight) -- 
        (\stepsize*3.5, \lvlbase + 2.0*\lvlheight) --
        (\stepsize*-0.75, \lvlbase + 2.0*\lvlheight) --
        (\stepsize*-0.75, \lvlbase + 3.0*\lvlheight) ;

    \node[align=left,anchor=north west] 
        at (-0.75 *\stepsize,\lvlbase + 3.0*\lvlheight) 
        {\textit{(b) Decoder Contextual Sentence Embeddings}};

        \draw[rectangle,draw=black,dotted] 
        (\stepsize*0.5,\lvlbase + 0.5*\lvlheight) -- 
        (\stepsize*3.5, \lvlbase + 0.5*\lvlheight) -- 
        (\stepsize*3.5, \lvlbase - 0.75*\lvlheight) --
        (\stepsize*0.5, \lvlbase - 0.75*\lvlheight) --
        (\stepsize*0.5, \lvlbase + 0.5*\lvlheight) ;

    \node[align=left,anchor=south west] 
        at (0.5 *\stepsize,\lvlbase - 0.75*\lvlheight) 
        {\textit{Sentence Embeddings}\\\textit{(Sentence Encoder Outputs)}};

        \draw[rectangle,draw=black,dotted] 
        (\stepsize*-0.25,\lvlbase + 1.65*\lvlheight) -- 
        (\stepsize*4.25, \lvlbase + 1.65*\lvlheight) -- 
        (\stepsize*4.25, \lvlbase + 0.65*\lvlheight) --
        (\stepsize*-0.25, \lvlbase + 0.65*\lvlheight) --
        (\stepsize*-0.25, \lvlbase + 1.65*\lvlheight) ;

    \node[align=left,anchor=south west,text width=6.4cm] 
    (lbl) at (4.5 *\stepsize,\lvlbase + 2.25*\lvlheight) 
        {\textit{(a) Decoder Left and Right Partial \phantom{(a) }Contextual Sentence Embeddings}};

    \draw[dotted] (\stepsize*4.25, \lvlbase + 1.65*\lvlheight) to (lbl);


    \node[align=left,anchor=west,text width=4.2cm] (lbl)  at (\stepsize*5, \lvlbase) {\textit{Right Begin Decoding Embedding}};    
    \draw[dotted] (lbl.west) to (s4);

    \node[align=left,anchor=east,text width=4.5cm] (lbl)  at (\stepsize-3, \lvlbase) {\textit{Left Begin Decoding Embedding}};    
    \draw[dotted] (lbl.east) to (s0);


\end{tikzpicture}}
\caption{Schematic for the decoder contextual sentence embeddings as computed by the \stsext~sentence extractor.}
\label{fig:stsext2}

\end{minipage}}
\end{figure}



\begin{figure}[p]
\fbox{\begin{minipage}{\textwidth}
\center
\scalebox{0.75}{
\begin{tikzpicture}[
  dep/.style ={
    ->,line width=0.3mm
  },
  hid/.style 2 args={
    rectangle split,
    draw=#2,
    rectangle split parts=#1,
    fill=#2!20,
    minimum width=5mm,
    minimum height=5mm,
    outer sep=2mm},
  mlp/.style 2 args={
    rectangle split,
    rectangle split horizontal,
    draw=#2,
    rectangle split parts=#1,
    fill=#2!20,
    outer sep=2mm},
  sal/.style={
    circle, 
    minimum width=8mm,
    outer sep=2mm,
    draw=#1, 
    fill=#1!20},
]

  \def\stepsize{3}%
  \def\lvlbase{0}%
  \def\lvlheight{3}%
 


    % Sentence Embeddings    
  \foreach \step in {1,...,3} {
    \node[hid={3}{encctxemb}] (enc\step) at (\stepsize*\step, \lvlbase) {};    
    \node at (\stepsize*\step, \lvlbase) {$\stsextEncHid_\step$};    
   }

  \foreach \step in {1,...,3} {
    \node[hid={1}{yellow}] (a\step) at (\stepsize*\step-0.5*\stepsize, \lvlbase) {};    
    \node at (\stepsize*\step-0.5*\stepsize, \lvlbase) {$\stsextAttn_{i,\step}$};    

    \draw[dep] (enc\step.west) to (a\step.east);
   }


    \node[hid={3}{decctxemb}] (deci) at (\stepsize*4, \lvlbase) {};    
    \node at (\stepsize*4, \lvlbase) {$\stsextDecHid_i$};    


    \draw[dep] (deci.south) to [out=270,in=270] (a3.south);
    \draw[dep] (deci.south) to [out=270,in=270] (a2.south);
    \draw[dep] (deci.south) to [out=270,in=270] (a1.south);

    \node[hid={3}{yellow}] (encattn) at (\stepsize*2, \lvlbase+\lvlheight) {};    
    \node at (\stepsize*2, \lvlbase+\lvlheight) {$\stsextAttnHid_i$};    


    \node[hid={3}{ctxemb}] (ctxi) at (\stepsize*4, \lvlbase+\lvlheight) {};    
    \node at (\stepsize*4, \lvlbase+\lvlheight) {$\stsextHid_i$};    

        \node[sal={sal}] (sal) at (\stepsize * 5,\lvlbase + \lvlheight) {};
        \node at (\stepsize * 5,\lvlbase + \lvlheight) {$\psal_i$};

    \draw[dep] (ctxi) to (sal);
    \draw[dep] (deci.north) to (ctxi.south);
    \draw[dep] (encattn.east) to (ctxi.west);

    \node[text width=2.3cm] (info) at (0,\lvlbase + \lvlheight) {\textit{Encoder Contextual Sentence Embeddings}};
    \draw[dotted] (enc1.north west) to (info);
    \draw[dotted] (enc2.north west) to (info);
    \draw[dotted] (enc3.north west) to (info);
    \node (info) at (0,\lvlbase - 1.5*\lvlheight) {\textit{(a) Attention Weights}};
    \draw[dotted] (a1.south west) to (info);
    \draw[dotted] (a2.south west) to (info);
    \draw[dotted] (a3.south west) to (info);

    \foreach \step in {1,...,3} {
        \draw[dep] (a\step.north) to (encattn.south);
        \draw[dep] (enc\step.north) to (encattn.south);
    }


    \node[align=left,anchor=north west,text width=5.5cm] at (\stepsize*2-2.1,2.0*\lvlheight+\lvlbase)
    {\textit{(b) Attention-weighted \phantom{(b) }Encoder Sentence \phantom{(b) }Embedding}};
\draw[rectangle,draw=black,dotted] 
        (\stepsize*2.0-2.1, \lvlbase + 2.0*\lvlheight) -- 
        (\stepsize*2.0+2.1, \lvlbase + 2.0*\lvlheight) -- 
        (\stepsize*2.0+2.1, \lvlbase + 0.6*\lvlheight) --
        (\stepsize*2.0-2.1, \lvlbase + 0.6*\lvlheight) --
        (\stepsize*2.0-2.1, \lvlbase + 2.0*\lvlheight) ;




    \node[align=left,anchor=north west,text width=2.5cm] at (\stepsize*4-1.4,2.0*\lvlheight+\lvlbase)
    {\textit{(c) Contextual \phantom{(c) }Sentence \phantom{(c) }Embedding}};
\draw[rectangle,draw=black,dotted] 
        (\stepsize*4.0-1.4, \lvlbase + 2.0*\lvlheight) -- 
        (\stepsize*4.0+1.4, \lvlbase + 2.0*\lvlheight) -- 
        (\stepsize*4.0+1.4, \lvlbase + 0.6*\lvlheight) --
        (\stepsize*4.0-1.4, \lvlbase + 0.6*\lvlheight) --
        (\stepsize*4.0-1.4, \lvlbase + 2.0*\lvlheight) ;


    \node[align=left,anchor=north west,text width=2.5cm] at (\stepsize*5-1.4,2.0*\lvlheight+\lvlbase)
{\textit{(d) Salience \phantom{(d) }Estimate}};
\draw[rectangle,draw=black,dotted] 
        (\stepsize*5.0-1.4, \lvlbase + 2.0*\lvlheight) -- 
        (\stepsize*5.0+1.4, \lvlbase + 2.0*\lvlheight) -- 
        (\stepsize*5.0+1.4, \lvlbase + 0.6*\lvlheight) --
        (\stepsize*5.0-1.4, \lvlbase + 0.6*\lvlheight) --
        (\stepsize*5.0-1.4, \lvlbase + 2.0*\lvlheight) ;


    \draw[rectangle,draw=black,dotted] 
        (\stepsize*4.0-1.4, \lvlbase + 0.4*\lvlheight) -- 
        (\stepsize*5.0+1.4, \lvlbase + 0.4*\lvlheight) -- 
        (\stepsize*5.0+1.4, \lvlbase - 0.6*\lvlheight) --
        (\stepsize*4.0-1.4, \lvlbase - 0.6*\lvlheight) --
        (\stepsize*4.0-1.4, \lvlbase + 0.4*\lvlheight) ;


    \node[align=left,anchor=north east,text width=2.5cm] at (\stepsize*5+1.4,0.4*\lvlheight + \lvlbase)
    {\textit{\phantom{(a) }Decoder \phantom{(a) }Contextual \phantom{(a) }Sentence \phantom{(a) }Embedding}};


\end{tikzpicture}}
\caption{Attention layer, contextual sentence embedding, and salience
estimation layer for the \stsext~extractor.}
\label{fig:stsext3}
\end{minipage}}
\end{figure}




\FloatBarrier


Each decoder contextual sentence embedding $\stsextDecHid_i$ then attends
to the encoder contextual sentence embeddings $\stsextEncHid_1,\ldots,
\stsextEncHid_\docSize$, to produce an attention-weighted encoder sentence
embedding $\stsextAttnHid_i$,
    
\vspace{10pt} \noindent \textit{(Fig.~\ref{fig:stsext3}.a) Attention Weights}
\begin{align}
    \forall i : \;\; i \in \{1,\ldots,\docSize\}&\nonumber \\
    \stsextAttn_{i,j} & = 
        \frac{\exp \left(\stsextDecHid_i \cdot  \stsextEncHid_j \right)}{
            \sum_{j^\prime=1}^{\docSize}\exp\left(  
                \stsextDecHid_i \cdot  \stsextEncHid_{j^\prime} \right)},
\end{align}
\vspace{10pt} \noindent \textit{(Fig.~\ref{fig:stsext3}.b) Attention-weighted Encoder Sentence Embeddings} 
\begin{align}
    \forall i : \;\; i \in \{1,\ldots,\docSize\}& \nonumber \\
    \stsextAttnHid_i & = 
        \sum_{j=1}^{\docSize} \alpha_{i,j} \left[\begin{array}{c}
            \stsextREncHid_j\\ 
            \stsextLEncHid_j\end{array}\right].
\end{align}

The attention-weighted encoder sentence embeddings and the decoder 
contextual sentence embedding are then concatenated and fed through a 
\feedforward~layer to produce the $i^\textrm{th}$ contextual sentence
embedding $\stsextHid_i$, which is itself fed through a final 
\feedforward~layer to compute the $i^\textrm{th}$ salience estimate
$\psal_i = \model(\bsal_i=1|\sentEmb_1,\ldots,\sentEmb_\docSize;\xParams)$,

\vspace{10pt}   \noindent \textit{(Fig.~\ref{fig:stsext3}.c) Contextual Sentence Embedding} 
\begin{align}
    \forall i : \;\; i \in \{1,\ldots,\docSize\} & \nonumber\\
    \stsextHid_i &= \relu\left(\stsextHidWeight \left[\begin{array}{c}
        \stsextAttnHid_i \\ 
        \stsextDecHid_i \end{array}\right] 
        + \stsextHidBias \right),
\end{align}

~\\

\vspace{10pt} \noindent \textit{(Fig.~\ref{fig:stsext3}.d) Salience Estimate}
\begin{align}
    \forall i : \;\; i \in \{1,\ldots,\docSize\} & \nonumber\\
\psal_i =    \model(\bsal_i=1|\sentEmb_1, \ldots,\sentEmb_\docSize;\xParams)& = 
            \sigma\left(\stsextPredWeight \stsextHid_i + \stsextPredBias  
            \right).
\end{align}
where $\stsextHidWeight \in \reals^{\stsextHidDim \times 3\stsextRNNDim}$,
$\stsextHidBias \in \reals^{\stsextHidDim}$, 
$\stsextPredWeight \in \reals^{1 \times \stsextHidDim}$, and
$\stsextPredBias \in \reals$ are model parameters.
The complete set of \stsext~extractor parameters is 
\[\xParams = \left\{ \stsextREncParams, \stsextLEncParams,
        \stsextRDecParams, \stsextLDecParams, \stsextHidWeight, \stsextHidBias, \stsextPredWeight, \stsextPredBias, \right\}. \]
        In our experiments, we set $\stsextRNNDim=300$, $\stsextHidDim=100$.
        Dropout with drop probability of 0.25 is applied to $\stsextREncHid_i,
        \stsextLEncHid_i, \stsextRDecHid_i, \stsextLDecHid_i$, and $\stsextHid_i$ for all $i \in \{1,\ldots,\docSize\}$.

%?
%?
%?
%?sentence into a query vector which attends to the encoder output. The
%?attention weighted encoder output and the decoder $\gru$ output are concatenated
%?and fed into a multi-layer perceptron to compute the extraction probability.
%?See \autoref{fig:extractors}.b for a graphical layout.
%?%and \autoref{app:s2sextractor} for details.
%?
%?
%?



\section{Comparison of Sentence Extractors}

At first, each sentence extractor architecture can feel \latin{sui generis}
or bespoke, without one having much in common with the other. However,
at their core, they all take context-free\footnote{Here we mean that 
    the sentence embedding $\sentEmb_i$ 
    is constructed using only the words from sentence $\sent_i$ and no 
other surrounding document information.} sentence embeddings $\sentEmb_1,\ldots,
\sentEmb_\docSize$ and produce contextual sentence embeddings 
$\xHid_i$ that contain information propagated from neighboring sentence 
embeddings $\sentEmb_1,\ldots,\sentEmb_{i_1},\sentEmb_{i+1},\ldots,\sentEmb_\docSize$ and possibly previous salience estimates $\psal_1,\ldots,\psal_{i-1}$.





%The final outputs of each encoder direction are passed to the first decoder
%steps; additionally, the first step of the decoder GRUs are learned 
%``begin decoding'' vectors $\rdxhid_0$ and $\ldxhid_0$ ({\color{red} change math to reflect this}) 
%(see \autoref{fig:extractors}.b).
%Each GRU has separate learned 
%parameters; $U, V$ and $u, v$ are learned weight and bias parameters.
%The hidden layer size of the GRU is 300 for each direction and MLP hidden layer
%size is 100. Dropout with drop probability .25 is applied to the GRU outputs and to $a_i$.
%


%
%We study three architectures for the sentence encoders, namely, 
%embedding averaging, RNNs, and 
%CNNs.
%We also propose two simple models for the sentence extractor and compare
%to the previously proposed extractors of 
%\citet{cheng2016neural} and \citet{nallapati2017summarunner}.
%\hal{i think it's still confusing what's new and what's not. maybe you can somewhat mark? like things with $\star$ are new and ones without are old or something?}
%The prior works differ significantly but make the same semi-Markovian
%factorization of the extraction decisions, i.e. 
%$p(\slabel|\sentEmb)=\prod_{i=1}^\docSize p(\slabel[i]|\slabel[<i],\sentEmb)$,
%where each prediction \slabel[i] is dependent on all previous \slabel[j] for
%all $j < i$.
%By contrast, our extractors make a stronger conditional independence 
%assumption $p(\slabel|\sentEmb)=\prod_{i=1}^\docSize p(\slabel[i]|\sentEmb)$,
%essentially making independent predictions conditioned on $\sentEmb$.
%In theory, our models should perform worse because of this, however, as
%we later show, this is not the case empirically.
%
%
%
%\hal{i think you might need a subsection at the end of this section with oen or two paragraphs of compare/contrast the different models, esp if details are going to appendix}
%




%%% Local Variables:
%%% mode: latex
%%% TeX-master: "dlextsum.emnlp18"
%%% End:


