\section{The Emergence of Data Driven Extractive Summarization (2000--2014)}

While the MT community could take advantage of large (for the time) parallel
corpora,  it was not until the 2000s that there were readily available
collections of documents and their summaries for which to perform the kind of
supervised machine learning that was being successfully applied to MT.
Beginning in 2001, the Document Understanding Conferences (DUC) began steadily
producing collections of documents with reference summaries, bringing together
NLP researchers particularly around multi-document summarization
\citep{jones1999,harman2001,nenkova2005b}.
  
Many significant summarization systems from this time still relied on
unsupervised methods to create extract summaries, usually exploiting document
collection statistics to determine the salience of input text units.  The
representation of text units as TF-IDF weighted bags-of-words \citep{jones1972}
is prevalent in this era. For example, \cite{radev2000} identify salience
sentences in a document cluster by their similarity to the average TF-IDF
vector of the document cluster. In \cite{erkan2004}, sentences are treated as
vertices in a fully connected graph, with edge weights determined by the cosine
similarity between each sentence's TF-IDF weighted bag-of-words representation.
The PageRank algorithm \citep{page1999} is then used to determine the graph
centrality of each sentence; the sentences most central in the graph are
considered the most salient and extracted for the summary. 

Other methods used alternative formulations to exploit frequency for
summarization, most notably \cite{lin2000} who identified topic signatures
using a likelihood ratio test \citep{dunning1993} to identify terms that
occurred unusually frequently in a document given a large generic background
corpus. After removing stopwords, word frequency on its own has also been shown
to be an effective signal for identifying salient sentences
\citep{nenkova2005}.

Following on the initial research by \cite{kupiec1995trainable}, other work
using supervised learning for summarization begins to emerge from this time
\citep{conroy2001,osborne2002using,hirao2002extracting,sipos2012large}.  Here,
summarization is framed as a sentence classification task (i.e., which
sentences from the input document should be included in a summary). 

Most of the text-to-text summarization approaches from the 2000s are primarily
extractive systems.  While significantly constrained in their expressive
quality (i.e. only sentences found in the input can be used to construct the
output) especially compared to earlier data-to-text methods, designing
data-driven features for unsupervised and supervised statistical machine
learning proved to be much more scalable and an easier path to improved
summarization performance \citep{nenkova2011}. 
 
There are some notable works that attempted to perform limited abstractive
summarization. \cite{barzilay2005}  developed an unsupervised method of
sentence fusion (i.e., combining sentences with the same or overlapping content
using their syntactic parses as backbone structure), that has been extended and
refined by others \citep{marsi2005,filippova2008}. Heuristically driven phrase
deletions were also explored to reduce less salient information in extractive
summarization \citep{jing2000,zajic2007}.  In the supervised case, there were
several works that attempted to learn to  delete non-essential phrase
constituents.  This was formulated as either a pipeline of learned compression
and extraction models \citep{wang2013} or as a joint model of extraction and
compression \citep{martins2009,bergkirkpatrick2011}.  While sentence fusion was
capable of some abstractive rewriting, the other approaches mentioned here
predominantly focused on compression, i.e. word or phrase deletion, to generate
novel summary text, which is only one of the many ways that human abstractors
perform the summarization task \citep{jing2000b}.

Streaming or temporal summarization was first explored in the context of topic
detection and tracking \citep{khandelwal2001,allan2001} and more recently at
the Text Retrieval Conference (TREC) \citep{aslam2013}. Approaches to this
problem often perform filtering by estimated salience before relying on
multi-document summarization techniques to select text units to construct
rolling update summaries \citep{guo2013,mccreadie2014}. These pipelines while
effective, do not attempt to jointly optimize the salience and selection.
