\section{Natural Language Generation from 1980--2000}
  
Returning to the present day, computer aided production of human language
closely follows the beginning of modern computing, starting with work on
machine translation (MT) systems developed in the 1950s and '60s
\citep{ornstein1955mechanical,national1966language,hutchins2003machine}. Early
work on producing extract summaries of research articles also dates back to
this period \citep{luhn1958automatic} as well as  also notable experiments in
generating text purely from syntactic structures \citep{yngve1961random}.
  
However, NLG did not begin to coalesce as a distinct subfield until the 1980s
which saw the first workshops devoted specifically to NLG  and a convergence on
the formalisms and problems central to language generation
\citep{reiter1997building,mcdonald2010natural}.  NLG researchers of this period
were focused on at least four main research programs: (i) linguistically
motivated grammars for generation, (ii) frameworks for representing knowledge
and concepts, (iii) models of the human receiver of the generated text, and
(iv) models of discourse and control for planning the realization of utterances
\citep{mann1981text,mckeown1986}. 
    
A variety of grammatical formalism were proposed during this period to be the
syntactic backbone of language generation algorithms, including Functional
Grammar \citep{halliday2013halliday}, Transformational Grammar
\citep{chomsky1965aspects}, Generalized Phrase Structure Grammar
\citep{gazdar1985generalized}, and others. Many frameworks for generation
proliferated at this time (often incorporating one of those grammar
formalisms), including Knowledge and Modalities Planner (KAMP)
\citep{appelt1982planning}, Penman \citep{hovy1993natural}, MUMBLE
\citep{McDonald1981MUMBLEAF}, TEXT \citep{mckeown1982text} and others
\citep{mann1981text}.  While there appears to be a great diversity of
approaches, over time the community began converging on a fairly similar
pipeline of modules when it came it implementation \citep{reiter1994has}. 
  
Indeed, most NLG systems from this period could be understood as a pipeline of
modules for text planning, sentence planning, and linguistic realization
\citep{reiter1997building}.  In the text planning stage, the concepts to be
conveyed are selected, possibly discarding less essential information, and
arranged into a discourse plan or ordering. In the sentence planning stage, the
concepts from the previous stage are grouped into individual sentences, and
lexicalizing concepts and referring expression generation is performed.
Finally, in linguistic realization, the intermediate representation from the
sentence planning stage is converted into a natural language utterance, often
by linearizing and inflecting some syntactic/morphological representation.
  
Since these systems primarily generated language by starting from
non-linguistic and/or semantic representations of concepts, they are often
referred to as \textit{concept-to-text} generation or, as more commonly known
today, \textit{data-to-text} generation \citep{gatt2018survey}. These systems
were applied to a variety of data-to-text problems including weather forecast
generation \citep{goldberg1994using}, statistical report  generation (i.e.
generating a report from numerical or statistical data in a spreadsheet)
\citep{iordanskaja-etal-1992-generation}, or as a writing aid to improve the
productivity of human authors \citep{springer1991,mckeown1994,paris1995}.
Data-to-text generation came to prominence alongside expert systems
\citep{todd1992introduction}, including as a means to explain them
\citep{swartout1983xplain}.  As such, these NLG systems suffer from some of the
same drawbacks as expert systems, often requiring extensive domain knowledge
and manual rule or grammar engineering which became  difficult to maintain or
amend over time.
  
In the late 1980s and into the 1990s, as larger text corpora became available
and statistical and/or machine learning techniques spread through the
community, text-to-text generation (i.e. methods of directly mapping
unstructured text inputs to text outputs) increased in popularity.
Text-to-text generation is less defined by a specific generation task, method
or unifying theory, but on the use of large collections of example input/output
text pairs. 
  
Statistical machine translation (SMT) emerges from this period as the dominant
success story for machine learning applied to text-to-text generation problems.
The availability of digitized bilingual data made it possible to create
parallel sentence translation corpora and apply statistical word alignment and
translation techniques \citep{gale1993}.  The canonical IBM translation models
were developed in this period \citep{brown1988,brown1993}.  Statistical methods
marked a stark improvement over attempts at interlingua or semantics based
translation systems which often required significant manual effort in pre or
post editing \citep{hutchins1994}. There were even some limited experiments
using learned classifiers to predict sentence salience for summarization
\citep{kupiec1995trainable}.
