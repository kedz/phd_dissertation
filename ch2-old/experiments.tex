
\section{Experiments} \label{sec:exps}

For our main experiments, we train every possible pairing of 
sentence encoder and extractor architecture ($3\times4=12$) on each of 
 dataset $\mathcal{D} = \left\{\left(\doc^{(1)}, \bsals^{(1)}\right),\ldots, \left(\doc^{(\corpusSize)}, \bsals^{(\corpusSize)}\right) \right\}$.
We use the trained models to produce extract summaries for the test 
set, and we then evaluate summary quality with respect to the reference
abstract summaries using \rouge-2 recall.\footnote{\rouge-1 recall and \rouge-LCS trend similarity in our experiments so we omit them for space.}
We use extract summary word budgets of $\wordbudget=100$ words for news, and 
$\wordbudget=75$, $\wordbudget=290$, and $\wordbudget=200$ for Reddit, AMI, and PubMed respectively.
We also evaluate using \meteor~\citep{banerjee2005meteor},
which measures precision and recall of reference words while allowing for
more matchings on synonymy or morphology.
We use the default settings for \meteor~and use remove stopwords and no stemming options for \rouge, keeping defaults for all other parameters.

For each model configuration, we train five different versions using different
random seeds and report the mean evaluation measure.
We estimate statistical significance by first averaging each document level 
\rouge~or \meteor~score over the five random initializations. 
We then test the difference between the best system on each dataset and 
all other systems using the approximate randomization test 
 with the Bonferroni correction for multiple comparisons
\citep{riezler2005pitfalls}, testing for significance at the $0.05$ level. 

\subsection{Training}

We train all models to minimize the weighted negative log-likelihood
\[\mathcal{L(\params)} = -\sum_{(\doc,\bsals) \in \corpus } \sum_{i=1}^\docSize \omega(\bsal_i) \log \model\left(\bsal_i|\bsal_1,\ldots,\bsal_{i-1},\doc;\params\right) \]
over the training data $\corpus$
using stochastic gradient descent with the \textsc{Adam} optimizer
\citep{kingma2014adam}. Since positive salience labels (i.e. $\bsal_i = 1$)
are much rarer than negative salience labels, we reweight the negative
log likelihood above, setting
\[\omega(0)=1 \quad \textrm{and}\quad \omega(1) = \docSize_0/\docSize_1\] where $\docSize_0$ and $\docSize_1$ 
are the number of training sentences labels $0$ and $1$ respectively.
    We trained for a maximum of 50 epochs and the best
    model was selected with early stopping on the validation set according
    to ROUGE-2. Each epoch constitutes a full pass through the
    dataset. The average stopping epoch was: CNN-DailyMail, 16.2; NYT, 21.36; DUC, 37.11; Reddit, 36.59; AMI, 19.58; PubMed, 19.84.
     All experiments were repeated with five random
    %\hal{there's prolly a bunch here that could go to the appendix}
    initializations.     Unless specified, word embeddings were initialized 
    using pretrained GloVe embeddings \citep{pennington2014glove} and we did 
    not update them during training. Unknown words were mapped to a zero 
    embedding, $\zeroEmb$.

    We use a learning rate of $.0001$ and a dropout rate of $0.25$ for all dropout
    layers. We also employ gradient clipping ($-5 < \nabla_\theta < 5$).
    Weight matrix parameters are initialized using 
    Xavier initialization with the normal distribution 
    \citep{glorot2010understanding} and bias terms are set to $0$.
    We use a batch size of 32 for all datasets except AMI and PubMed, which
    are often longer and consume more memory, for
    which we use sizes two and four respectively.
%\kathy{why? Say.}

    For \clext~based models, we train for half of the maximum epochs 
    with teacher forcing, i.e. we set $\psal_i = 1$
    if $\bsal_i = 1$ in the gold data and $0$ otherwise 
    when computing the decoder input 
    $\psal_i \sentEmb_i$. We revert to the predicted model probability 
    during the second half training and during test-time inference.






\subsection{Baselines}
\paragraph{Lead} As a baseline we include the lead summary, i.e. taking the first 
$\wordbudget$ words of the document as summary, where $\wordbudget$ is the 
summary word budget for each dataset (see the 
first paragraph of \autoref{sec:exps}). While incredibly simple, this method is still a 
competitive baseline for single document summarization, especially on newswire.
\paragraph{Oracle} To measure the performance ceiling,
we show the \rouge/\meteor~scores using the 
extractive summary $\extractSummary$~which was a bi-product of our algorithm
for obtaining salience labels $\bsals$ (see \autoref{sec:labelgen} for details). Essentially, this summary represents
an approximate ceiling on \rouge~performance, as it has clairvoyant
knowledge
of the human reference summaries for each document.
%We choose ROUGE-2
%recall as our main evaluation metric since it has the strongest correlation
%to human content selection decisions.

%\kathy{Save all results for results section. I deleted your sentence.}
%\hal{i agree with kathy. put all the results together. be specific about what questions you're asking and then how you framed them as an experiment and then what the answer is. i think i'd just remove all this stuff here.}
%In most cases, the averaging encoder performance was as good or better than
%the RNN and CNN encoders, we use only the averaging encoder for the remainder
%of the experiments.

%\paragraph{Word Embedding Learning}{To futher understand how word 
%embeddings 
%can
%effect model performance we also compared extractors when embeddings 
%are updated during training. Both fixed and learned embedding variants are 
%initialized with GloVe embeddings. When learning embeddings, words occurring 
%three or fewer times in the training data are mapped to a learned unkown
%token.}


%We are also interested in the effect of lead bias. It is well known that the first few sentences of a news article, often referred to as the lead, make a good summary, and this is most commonly used as the default baseline in single document summarization. This lead bias is such a strong learning signal that the learned models almost always extract sentences from the lead despite the ground truth labeling containing a significant portion of positive labels later in the document. This begs the question, are we learning a robust model of sentence salience or simply identifying linguistic style features that are indicative of the lead?

%To better understand this phenomenon, 






%%% Local Variables:
%%% mode: latex
%%% TeX-master: "dlextsum.emnlp18"
%%% End:
