\subsubsection{\sts~Extractor} 

One shortcoming of the RNN extractor is that long range
information from one end of the document may not easily be able to affect 
extraction probabilities of sentences at the other end. 
Our second proposed model, the \sts~extractor mitigates this problem with an 
attention 
mechanism commonly
used for neural machine translation \citep{bahdanau2014neural} and 
abstractive summarization \citep{see2017get}. 
The sentence embeddings produced by the sentence encoder are first
encoded by a \bidirectional~\gru, which produces left and right
partial contextual sentence embeddings, 

\begin{figure}[t]
\fbox{\begin{minipage}{\textwidth}
\center
\scalebox{0.75}{
\begin{tikzpicture}[
  dep/.style ={
    ->,line width=0.3mm
  },
  hid/.style 2 args={
    rectangle split,
    draw=#2,
    rectangle split parts=#1,
    fill=#2!20,
    minimum width=5mm,
    minimum height=5mm,
    outer sep=2mm},
  mlp/.style 2 args={
    rectangle split,
    rectangle split horizontal,
    draw=#2,
    rectangle split parts=#1,
    fill=#2!20,
    outer sep=2mm},
  sal/.style={
    circle, 
    minimum width=8mm,
    outer sep=2mm,
    draw=#1, 
    fill=#1!20},
]

  \def\stepsize{2}%
  \def\lvlbase{0}%
  \def\lvlheight{3}%
 

    % Sentence Embeddings    
  \foreach \step in {1,...,3} {
    \node[hid={3}{sentemb}] (s\step) at (\stepsize*\step, \lvlbase) {};    
    \node at (\stepsize*\step, \lvlbase) {$\sentEmb_\step$};    
   }
%    \foreach \step [count=\i from 1] in {5,...,7} {
%        \node[hid={3}{green}] (s\step) at (\stepsize*\step, \lvlbase) {};    
%        \node at (\stepsize*\step, \lvlbase) {$\sentEmb_\i$};    
%    %\draw[->] (i\step.north) -> (e\step.south);
%    }

%       \node[hid={3}{red}] (s4) at (\stepsize*4, \lvlbase) {};    
%       \node at (\stepsize*4, \lvlbase) {$\sentEmb_0$};    
%
%    % RNN hidden states
    \foreach \step in {1,...,3} {
        \node[hid={3}{rencemb}] (rrnn_\step) 
            at (\stepsize *\step+0.5, \lvlbase + \lvlheight) {};    
        \node at (\stepsize *\step+0.5, \lvlbase + \lvlheight) 
            {$\stsextREncHid_\step$}; 

        \node[hid={3}{lencemb}] (lrnn_\step) 
            at (\stepsize *\step-0.5, \lvlbase + \lvlheight+1.0) {};    
        \node at (\stepsize *\step-0.5, \lvlbase + \lvlheight+1.0) 
            {$\stsextLEncHid_\step$}; 


        \node[hid={3}{encctxemb}] (enc_ctx_\step) 
            at (\stepsize *\step, \lvlbase + 2.5*\lvlheight) {};    
        \node at (\stepsize *\step, \lvlbase + 2.5*\lvlheight) 
            {$\stsextEncHid_\step$}; 

        \draw[dep] (s\step.north) -- (lrnn_\step.south);
        \draw[dep] (s\step.north) -- (rrnn_\step.south);

        \draw[dep] (lrnn_\step.north) -- (enc_ctx_\step.south);
        \draw[dep] (rrnn_\step.north) -- (enc_ctx_\step.south);
    }
    \foreach \start [count=\stop from 2] in {1,...,2} {
        \draw[dep] ($ (rrnn_\start.east) - (0,0.3)$) 
            -- ($ (rrnn_\stop.west) - (0,0.3) $);
        \draw[dep] ($(lrnn_\stop.west) + (0,0.3)$) 
            -- ($ (lrnn_\start.east) + (0,0.3)   $);
    }


    \draw[rectangle,draw=black,dotted] 
        (\stepsize*-2.75,\lvlbase + 0.5*\lvlheight) -- 
        (\stepsize*3.5, \lvlbase + 0.5*\lvlheight) -- 
        (\stepsize*3.5, \lvlbase + -0.50*\lvlheight) --
        (\stepsize*-2.75, \lvlbase + -0.50*\lvlheight) --
        (\stepsize*-2.75, \lvlbase + 0.5*\lvlheight) ;

    \node[align=left,anchor=north west] 
        at (\stepsize * -2.75,\lvlbase + 0.5*\lvlheight) 
        {\textit{Sentence Embeddings}\\\textit{(Sentence Encoder Output)}};

    \draw[rectangle,draw=black,dotted] 
        (\stepsize*-2.75,\lvlbase + 3.0*\lvlheight) -- 
        (\stepsize*3.5, \lvlbase + 3.0*\lvlheight) -- 
        (\stepsize*3.5, \lvlbase + 2.0*\lvlheight) --
        (\stepsize*-2.75, \lvlbase + 2.0*\lvlheight) --
        (\stepsize*-2.75, \lvlbase + 3.0*\lvlheight) ;

    \node[align=left,anchor=north west] 
        at (\stepsize * -2.75,\lvlbase + 3.0*\lvlheight) 
        {\textit{(b) Encoder Contextual Sentence Embeddings}};

    \draw[rectangle,draw=black,dotted] 
        (\stepsize*-2.75,\lvlbase + 1.8*\lvlheight) -- 
        (\stepsize*3.5, \lvlbase + 1.8*\lvlheight) -- 
        (\stepsize*3.5, \lvlbase + 0.6*\lvlheight) --
        (\stepsize*-2.75, \lvlbase + 0.6*\lvlheight) --
        (\stepsize*-2.75, \lvlbase + 1.8*\lvlheight) ;

    \node[align=left,anchor=north west] 
        at (\stepsize * -2.75,\lvlbase + 1.8*\lvlheight) 
        {\textit{(a) Encoder Left and Right Partial}\\\textit{\phantom{(a) }Contextual Sentence Embeddings}};


\end{tikzpicture}}
\caption{Schematic for the encoder contextual sentence embeddings as 
computed in the \stsext~sentence extractor.}
\label{fig:stsext1}
\end{minipage}}
\end{figure}



 \vspace{10pt}   \noindent \textit{(Fig.~\ref{fig:stsext1}.a) Encoder Left and Right Partial Contextual Sentence Embeddings}
\begin{align}
        % eta_0 = 0 
        \stsextREncHid_0  = \zeroEmb, & \quad 
        \stsextLEncHid_{\docSize + 1}  = \zeroEmb, \\
\forall i : \;\; i \in \{1,\ldots,\docSize\}& \nonumber \\
\stsextREncHid_i & = \fgru\left(
            \sentEmb_i, \stsextREncHid_{i-1}; 
            \stsextREncParams\right), \\
\stsextLEncHid_i &= \fgru\left(
            \sentEmb_i, \stsextLEncHid_{i+1}; 
            \stsextLEncParams\right),
\end{align}
where $\stsextREncHid_i,\stsextLEncHid_i \in \reals^{\stsextRNNDim}$ and 
$\stsextREncParams$ and $\stsextLEncParams$ are the forward and backward encoder \gru~parameters respectively. The encoder contextual sentence embeddings
are then formed by simply concatenating the encoder left and right 
partial contextual embeddings,

\vspace{10pt} \noindent \textit{(Fig.~\ref{fig:stsext1}.b) Encoder Contextual Sentence Embedding}
\begin{align}
        % eta_i = [eta_i,>, eta_i,<]   
\forall i : \;\; i \in \{1,\ldots,\docSize\} & \nonumber \\
        \stsextEncHid_i & = \left[\begin{array}{c}
            \stsextREncHid_i\\ 
            \stsextLEncHid_i\end{array}\right] 
\end{align}




The final output of each encoder \gru~initializes a separate decoder \gru~which is then run over the sentence embeddings a second time, 


\vspace{10pt} \noindent  \textit{(Fig.~\ref{fig:stsext2}.a) Decoder Left and Right Partial Contextual Sentence Embeddings}
\begin{align}
        % zeta_0 = eta_n
\stsextRDecHid_0 & = \fgru\left(\stsextSent_>,   \stsextREncHid_\docSize; \stsextRDecParams\right),\\ 
\stsextLDecHid_{\docSize+1} & = \fgru\left(\stsextSent_<, \stsextLEncHid_1;\stsextLDecParams\right),\\ 
    \forall i : \;\; i \in \{1,\ldots,\docSize\}& \nonumber\\
        % zeta_i = gru(h_i, zeta_i-1; \chi_zeta)  
        \stsextRDecHid_i & = \fgru(
            \sentEmb_i,  \stsextRDecHid_{i-1}; 
            \stsextRDecParams), \\
        % zeta_i = gru(h_i, zeta_i+1; chi_zeta)
        \stsextLDecHid_i & = \fgru(
            \sentEmb_i,  \stsextLDecHid_{i+1}; 
            \stsextLDecParams) 
\end{align}
where $\stsextSent_>$ and $\stsextSent_<$ are special ``begin decoding'' input
embeddings for the forward and backward decoder respectively, 
$\stsextRDecHid_i, \stsextLDecHid_i \in \reals^{\stsextRNNDim}$, 
and $\stsextRDecParams$ and $\stsextLDecParams$ are the parameters for the
forward and backward decoder \gru s respectively.
%

The decoder left and right partial contextual sentence embeddings ($\stsextRDecHid_i$ and $\stsextLDecHid_i$) are then
concatenated to form the decoder contextual sentence embeddings,


\vspace{10pt} \noindent \textit{(Fig.~\ref{fig:stsext2}.b) Decoder Contextual Sentence Embeddings} 
\begin{align}
    \forall i : \;\; i \in \{1,\ldots,\docSize\}& \nonumber \\
        \stsextDecHid_i & = \left[\begin{array}{c}
            \stsextRDecHid_i\\ 
            \stsextLDecHid_i\end{array}\right].
\end{align}



\begin{figure}[p]
\fbox{\begin{minipage}{\textwidth}
\center
\scalebox{0.75}{
\begin{tikzpicture}[
  dep/.style ={
    ->,line width=0.3mm
  },
  hid/.style 2 args={
    rectangle split,
    draw=#2,
    rectangle split parts=#1,
    fill=#2!20,
    minimum width=5mm,
    minimum height=5mm,
    outer sep=2mm},
  mlp/.style 2 args={
    rectangle split,
    rectangle split horizontal,
    draw=#2,
    rectangle split parts=#1,
    fill=#2!20,
    outer sep=2mm},
  sal/.style={
    circle, 
    minimum width=8mm,
    outer sep=2mm,
    draw=#1, 
    fill=#1!20},
]

  \def\stepsize{2}%
  \def\lvlbase{0}%
  \def\lvlheight{3}%
 

    \node[hid={3}{decsentemb}] (s0) at (\stepsize*0, \lvlbase) {};    
    \node at (\stepsize*0, \lvlbase) {$\stsextSent_>$};    


    \node[hid={3}{decsentemb}] (s4) at (\stepsize*4, \lvlbase) {};    
    \node at (\stepsize*4, \lvlbase) {$\stsextSent_{<}$};    

    % Sentence Embeddings    
  \foreach \step in {1,...,3} {
    \node[hid={3}{sentemb}] (s\step) at (\stepsize*\step, \lvlbase) {};    
    \node at (\stepsize*\step, \lvlbase) {$\sentEmb_\step$};    
   }
%    \foreach \step [count=\i from 1] in {5,...,7} {
%        \node[hid={3}{green}] (s\step) at (\stepsize*\step, \lvlbase) {};    
%        \node at (\stepsize*\step, \lvlbase) {$\sentEmb_\i$};    
%    %\draw[->] (i\step.north) -> (e\step.south);
%    }

%       \node[hid={3}{red}] (s4) at (\stepsize*4, \lvlbase) {};    
%       \node at (\stepsize*4, \lvlbase) {$\sentEmb_0$};    
%
%    % RNN hidden states

        \node[hid={3}{rdecemb}] (rrnn_0) 
            at (\stepsize *0+0.5, \lvlbase + \lvlheight) {};    
        \node at (\stepsize *0+0.5, \lvlbase + \lvlheight) 
            {$\stsextRDecHid_0$}; 

        \node[hid={3}{ldecemb}] (lrnn_4) 
            at (\stepsize *4-0.5, \lvlbase + \lvlheight+1.0) {};    
        \node at (\stepsize *4-0.5, \lvlbase + \lvlheight+1.0) 
            {$\stsextLDecHid_4$}; 

        \node[hid={3}{rencemb}] (rrnn_m1) 
            at (-\stepsize +0.5, \lvlbase + \lvlheight) {};    
        \node at (-\stepsize +0.5, \lvlbase + \lvlheight) 
            {$\stsextREncHid_3$}; 

        \node[hid={3}{lencemb}] (lrnn_m1) 
            at (\stepsize*5 -0.5, \lvlbase + \lvlheight+1.0) {};    
        \node at (\stepsize*5 -0.5, \lvlbase + \lvlheight+1.0) 
            {$\stsextLEncHid_1$}; 

            \node[align=left,anchor=north west,text width=4.3cm] (lbl)  
                at (\stepsize*5.0, \lvlbase + \lvlheight*1.65) 
                {\textit{Encoder Right Partial Contextual Sentence Embedding}};    
    %\draw[dotted] (lbl.east) to (s0);
        \draw[rectangle,draw=black,dotted] 
        (\stepsize*4.5,\lvlbase + 1.65*\lvlheight) -- 
        (\stepsize*7.5, \lvlbase + 1.65*\lvlheight) -- 
        (\stepsize*7.5, \lvlbase + 0.65*\lvlheight) --
        (\stepsize*4.5, \lvlbase + 0.65*\lvlheight) --
        (\stepsize*4.5, \lvlbase + 1.65*\lvlheight) ;


        \draw[rectangle,draw=black,dotted] 
        (\stepsize*-3.0,\lvlbase + 1.65*\lvlheight) -- 
        (\stepsize*-0.5, \lvlbase + 1.65*\lvlheight) -- 
        (\stepsize*-0.5, \lvlbase + 0.65*\lvlheight) --
        (\stepsize*-3.0, \lvlbase + 0.65*\lvlheight) --
        (\stepsize*-3.0, \lvlbase + 1.65*\lvlheight) ;

            \node[align=left,anchor=north west,text width=4.3cm] (lbl)  
                at (\stepsize*-3.0, \lvlbase + \lvlheight*1.65) 
                {Encoder Left Partial Contextual Sentence Embedding};    





    \foreach \step in {1,...,3} {
        \node[hid={3}{rdecemb}] (rrnn_\step) 
            at (\stepsize *\step+0.5, \lvlbase + \lvlheight) {};    
        \node at (\stepsize *\step+0.5, \lvlbase + \lvlheight) 
            {$\stsextRDecHid_\step$}; 

        \node[hid={3}{ldecemb}] (lrnn_\step) 
            at (\stepsize *\step-0.5, \lvlbase + \lvlheight+1.0) {};    
        \node at (\stepsize *\step-0.5, \lvlbase + \lvlheight+1.0) 
            {$\stsextLDecHid_\step$}; 


        \node[hid={3}{decctxemb}] (enc_ctx_\step) 
            at (\stepsize *\step, \lvlbase + 2.5*\lvlheight) {};    
        \node at (\stepsize *\step, \lvlbase + 2.5*\lvlheight) 
            {$\stsextDecHid_\step$}; 

        \draw[dep] (s\step.north) -- (lrnn_\step.south);
        \draw[dep] (s\step.north) -- (rrnn_\step.south);

        \draw[dep] (lrnn_\step.north) -- (enc_ctx_\step.south);
        \draw[dep] (rrnn_\step.north) -- (enc_ctx_\step.south);
    }
    \foreach \start [count=\stop from 2] in {1,...,2} {
        \draw[dep] ($ (rrnn_\start.east) - (0,0.3)$) 
            -- ($ (rrnn_\stop.west) - (0,0.3) $);
        \draw[dep] ($(lrnn_\stop.west) + (0,0.3)$) 
            -- ($ (lrnn_\start.east) + (0,0.3)   $);
    }

        \draw[dep] ($ (rrnn_m1.east) - (0,0.3)$) 
            -- ($ (rrnn_0.west) - (0,0.3) $);
        \draw[dep] ($ (rrnn_0.east) - (0,0.3)$) 
            -- ($ (rrnn_1.west) - (0,0.3) $);

            \draw[dep] (s0.north) to (rrnn_0.south);
            \draw[dep] (s4.north) to (lrnn_4.south);

        \draw[dep] ($ (lrnn_m1.west) + (0,0.3)$) 
            -- ($ (lrnn_4.east) + (0,0.3) $);

        \draw[dep] ($ (lrnn_4.west) + (0,0.3)$) 
            -- ($ (lrnn_3.east) + (0,0.3) $);

%    \foreach \step [count=\i from 0] in {4,...,7} {
%        \node[hid={3}{orange}] (rnn_\step) 
%            at (\stepsize *\step, \lvlbase + \lvlheight) {};    
%        \node at (\stepsize *\step, \lvlbase + \lvlheight) {$\xDecHid_\i$}; 
%    }
%
%    \foreach \step in {1,...,6} {
%        \draw[dep] (s\step.north) to (rnn_\step.south);
%    }
%    \foreach \start [count=\stop from 2] in {1,...,5} {
%        \draw[dep] (rnn_\start.east) to (rnn_\stop.west);
%    }
%
%
%%    \foreach \step [count=\i from 1] in {4,...,6} {
%%        \node[hid={3}{green}] (ctx_\i) 
%%            at (\stepsize *\step, \lvlbase + 2*\lvlheight) {};    
%%        \node at (\stepsize *\step, \lvlbase + 2*\lvlheight) {$\xPredHid_\i$}; 
%%        \draw[dep] (rnn_\step.north) to (ctx_\i.south);
%%        \draw[dep] (rnn_\i.north) to (ctx_\i.south west);
%%        \node[sal={blue}] (sal_\i) at (\stepsize * \step,\lvlbase + 3*\lvlheight) {};
%%        \node at (\stepsize * \step,\lvlbase + 3*\lvlheight) {$\bsal_\i$};
%%        \draw[dep] (ctx_\i.north) to (sal_\i.south);
%%    }
%%
%%    \foreach \step [count=\i from 1] in {5,...,6} {
%%        \draw[dep] (sal_\i) -- (\stepsize * \step - \stepsize / 2,
%%                            \lvlbase + 3*\lvlheight) 
%%                     -- (\stepsize * \step - \stepsize / 2,
%%                            \lvlbase + 0* \lvlheight) -> (s\step) ;
%%
%%    }
%%    \draw[dep] (s1.south) -- ($ (s1.south) + (0,-0.3)$) --($ (s5.south) + (0,-0.3)$) -- (s5.south);
%%
%%    \draw[dep] (s2.south) -- ($ (s2.south) + (0,-0.5)$) --($ (s6.south) + (0,-0.5)$) -- (s6.south);
%%
%%    \draw[rectangle,draw=black,dotted] 
%%        (\stepsize * 3.5,\lvlbase + 3.5*\lvlheight) -- 
%%        (\stepsize*6.5, \lvlbase + 3.5*\lvlheight) -- 
%%        (\stepsize*6.5, \lvlbase + 2.6*\lvlheight) --
%%        (\stepsize*3.5, \lvlbase + 2.6*\lvlheight) --
%%        (\stepsize*3.5, \lvlbase + 3.5*\lvlheight) ;
%%
%%    \node[align=left,anchor=north west] 
%%        at (\stepsize * 3.5,\lvlbase + 3.5*\lvlheight) 
%%        {\textit{Salience Estimates}};
%%
%%    \draw[rectangle,draw=black,dotted] 
%%        (\stepsize*0.5,\lvlbase + 2.5*\lvlheight) -- 
%%        (\stepsize*6.5, \lvlbase + 2.5*\lvlheight) -- 
%%        (\stepsize*6.5, \lvlbase + 1.5*\lvlheight) --
%%        (\stepsize*0.5, \lvlbase + 1.5*\lvlheight) --
%%        (\stepsize*0.5, \lvlbase + 2.5*\lvlheight) ;
%%
%%    \node[align=left,anchor=north west] 
%%        at (\stepsize * 0.5,\lvlbase + 2.5*\lvlheight) 
%%        {\textit{Contextual Sentence Embeddings}};
%%
%%    \draw[rectangle,draw=black,dotted] 
%%        (\stepsize*-0.5,\lvlbase + 0.5*\lvlheight) -- 
%%        (\stepsize*3.5, \lvlbase + 0.5*\lvlheight) -- 
%%        (\stepsize*3.5, \lvlbase + -0.9*\lvlheight) --
%%        (\stepsize*-0.5, \lvlbase + -0.9*\lvlheight) --
%%        (\stepsize*-0.5, \lvlbase + 0.5*\lvlheight) ;
%%
%%    \node[align=left,anchor=south west] 
%%        at (\stepsize * -0.5,\lvlbase + -0.9*\lvlheight) 
%%        {\textit{Sentence Embeddings}\\\textit{(Sentence Encoder Output)}};
%%
%%    \draw[rectangle,draw=black,dotted] 
%%        (\stepsize*4.3,\lvlbase + 0.5*\lvlheight) -- 
%%        (\stepsize*6.5, \lvlbase + 0.5*\lvlheight) -- 
%%        (\stepsize*6.5, \lvlbase + -0.9*\lvlheight) --
%%        (\stepsize*4.3, \lvlbase + -0.9*\lvlheight) --
%%        (\stepsize*4.3, \lvlbase + 0.5*\lvlheight) ;
%%
%%    \node[align=left,anchor=south west] 
%%        at (\stepsize * 4.3,\lvlbase + -0.9*\lvlheight) 
%%        {\textit{Salience Gated}\\\textit{Sentence Embeddings}};
%%

    

    \draw[rectangle,draw=black,dotted] 
        (\stepsize*-0.75,\lvlbase + 3.0*\lvlheight) -- 
        (\stepsize*3.5, \lvlbase + 3.0*\lvlheight) -- 
        (\stepsize*3.5, \lvlbase + 2.0*\lvlheight) --
        (\stepsize*-0.75, \lvlbase + 2.0*\lvlheight) --
        (\stepsize*-0.75, \lvlbase + 3.0*\lvlheight) ;

    \node[align=left,anchor=north west] 
        at (-0.75 *\stepsize,\lvlbase + 3.0*\lvlheight) 
        {\textit{(b) Decoder Contextual Sentence Embeddings}};

        \draw[rectangle,draw=black,dotted] 
        (\stepsize*0.5,\lvlbase + 0.5*\lvlheight) -- 
        (\stepsize*3.5, \lvlbase + 0.5*\lvlheight) -- 
        (\stepsize*3.5, \lvlbase - 0.75*\lvlheight) --
        (\stepsize*0.5, \lvlbase - 0.75*\lvlheight) --
        (\stepsize*0.5, \lvlbase + 0.5*\lvlheight) ;

    \node[align=left,anchor=south west] 
        at (0.5 *\stepsize,\lvlbase - 0.75*\lvlheight) 
        {\textit{Sentence Embeddings}\\\textit{(Sentence Encoder Outputs)}};

        \draw[rectangle,draw=black,dotted] 
        (\stepsize*-0.25,\lvlbase + 1.65*\lvlheight) -- 
        (\stepsize*4.25, \lvlbase + 1.65*\lvlheight) -- 
        (\stepsize*4.25, \lvlbase + 0.65*\lvlheight) --
        (\stepsize*-0.25, \lvlbase + 0.65*\lvlheight) --
        (\stepsize*-0.25, \lvlbase + 1.65*\lvlheight) ;

    \node[align=left,anchor=south west,text width=6.4cm] 
    (lbl) at (4.5 *\stepsize,\lvlbase + 2.25*\lvlheight) 
        {\textit{(a) Decoder Left and Right Partial \phantom{(a) }Contextual Sentence Embeddings}};

    \draw[dotted] (\stepsize*4.25, \lvlbase + 1.65*\lvlheight) to (lbl);


    \node[align=left,anchor=west,text width=4.2cm] (lbl)  at (\stepsize*5, \lvlbase) {\textit{Right Begin Decoding Embedding}};    
    \draw[dotted] (lbl.west) to (s4);

    \node[align=left,anchor=east,text width=4.5cm] (lbl)  at (\stepsize-3, \lvlbase) {\textit{Left Begin Decoding Embedding}};    
    \draw[dotted] (lbl.east) to (s0);


\end{tikzpicture}}
\caption{Schematic for the decoder contextual sentence embeddings as computed by the \stsext~sentence extractor.}
\label{fig:stsext2}

\end{minipage}}
\end{figure}



\begin{figure}[p]
\fbox{\begin{minipage}{\textwidth}
\center
\scalebox{0.75}{
\begin{tikzpicture}[
  dep/.style ={
    ->,line width=0.3mm
  },
  hid/.style 2 args={
    rectangle split,
    draw=#2,
    rectangle split parts=#1,
    fill=#2!20,
    minimum width=5mm,
    minimum height=5mm,
    outer sep=2mm},
  mlp/.style 2 args={
    rectangle split,
    rectangle split horizontal,
    draw=#2,
    rectangle split parts=#1,
    fill=#2!20,
    outer sep=2mm},
  sal/.style={
    circle, 
    minimum width=8mm,
    outer sep=2mm,
    draw=#1, 
    fill=#1!20},
]

  \def\stepsize{3}%
  \def\lvlbase{0}%
  \def\lvlheight{3}%
 


    % Sentence Embeddings    
  \foreach \step in {1,...,3} {
    \node[hid={3}{encctxemb}] (enc\step) at (\stepsize*\step, \lvlbase) {};    
    \node at (\stepsize*\step, \lvlbase) {$\stsextEncHid_\step$};    
   }

  \foreach \step in {1,...,3} {
    \node[hid={1}{yellow}] (a\step) at (\stepsize*\step-0.5*\stepsize, \lvlbase) {};    
    \node at (\stepsize*\step-0.5*\stepsize, \lvlbase) {$\stsextAttn_{i,\step}$};    

    \draw[dep] (enc\step.west) to (a\step.east);
   }


    \node[hid={3}{decctxemb}] (deci) at (\stepsize*4, \lvlbase) {};    
    \node at (\stepsize*4, \lvlbase) {$\stsextDecHid_i$};    


    \draw[dep] (deci.south) to [out=270,in=270] (a3.south);
    \draw[dep] (deci.south) to [out=270,in=270] (a2.south);
    \draw[dep] (deci.south) to [out=270,in=270] (a1.south);

    \node[hid={3}{yellow}] (encattn) at (\stepsize*2, \lvlbase+\lvlheight) {};    
    \node at (\stepsize*2, \lvlbase+\lvlheight) {$\stsextAttnHid_i$};    


    \node[hid={3}{ctxemb}] (ctxi) at (\stepsize*4, \lvlbase+\lvlheight) {};    
    \node at (\stepsize*4, \lvlbase+\lvlheight) {$\stsextHid_i$};    

        \node[sal={sal}] (sal) at (\stepsize * 5,\lvlbase + \lvlheight) {};
        \node at (\stepsize * 5,\lvlbase + \lvlheight) {$\psal_i$};

    \draw[dep] (ctxi) to (sal);
    \draw[dep] (deci.north) to (ctxi.south);
    \draw[dep] (encattn.east) to (ctxi.west);

    \node[text width=2.3cm] (info) at (0,\lvlbase + \lvlheight) {\textit{Encoder Contextual Sentence Embeddings}};
    \draw[dotted] (enc1.north west) to (info);
    \draw[dotted] (enc2.north west) to (info);
    \draw[dotted] (enc3.north west) to (info);
    \node (info) at (0,\lvlbase - 1.5*\lvlheight) {\textit{(a) Attention Weights}};
    \draw[dotted] (a1.south west) to (info);
    \draw[dotted] (a2.south west) to (info);
    \draw[dotted] (a3.south west) to (info);

    \foreach \step in {1,...,3} {
        \draw[dep] (a\step.north) to (encattn.south);
        \draw[dep] (enc\step.north) to (encattn.south);
    }


    \node[align=left,anchor=north west,text width=5.5cm] at (\stepsize*2-2.1,2.0*\lvlheight+\lvlbase)
    {\textit{(b) Attention-weighted \phantom{(b) }Encoder Sentence \phantom{(b) }Embedding}};
\draw[rectangle,draw=black,dotted] 
        (\stepsize*2.0-2.1, \lvlbase + 2.0*\lvlheight) -- 
        (\stepsize*2.0+2.1, \lvlbase + 2.0*\lvlheight) -- 
        (\stepsize*2.0+2.1, \lvlbase + 0.6*\lvlheight) --
        (\stepsize*2.0-2.1, \lvlbase + 0.6*\lvlheight) --
        (\stepsize*2.0-2.1, \lvlbase + 2.0*\lvlheight) ;




    \node[align=left,anchor=north west,text width=2.5cm] at (\stepsize*4-1.4,2.0*\lvlheight+\lvlbase)
    {\textit{(c) Contextual \phantom{(c) }Sentence \phantom{(c) }Embedding}};
\draw[rectangle,draw=black,dotted] 
        (\stepsize*4.0-1.4, \lvlbase + 2.0*\lvlheight) -- 
        (\stepsize*4.0+1.4, \lvlbase + 2.0*\lvlheight) -- 
        (\stepsize*4.0+1.4, \lvlbase + 0.6*\lvlheight) --
        (\stepsize*4.0-1.4, \lvlbase + 0.6*\lvlheight) --
        (\stepsize*4.0-1.4, \lvlbase + 2.0*\lvlheight) ;


    \node[align=left,anchor=north west,text width=2.5cm] at (\stepsize*5-1.4,2.0*\lvlheight+\lvlbase)
{\textit{(d) Salience \phantom{(d) }Estimate}};
\draw[rectangle,draw=black,dotted] 
        (\stepsize*5.0-1.4, \lvlbase + 2.0*\lvlheight) -- 
        (\stepsize*5.0+1.4, \lvlbase + 2.0*\lvlheight) -- 
        (\stepsize*5.0+1.4, \lvlbase + 0.6*\lvlheight) --
        (\stepsize*5.0-1.4, \lvlbase + 0.6*\lvlheight) --
        (\stepsize*5.0-1.4, \lvlbase + 2.0*\lvlheight) ;


    \draw[rectangle,draw=black,dotted] 
        (\stepsize*4.0-1.4, \lvlbase + 0.4*\lvlheight) -- 
        (\stepsize*5.0+1.4, \lvlbase + 0.4*\lvlheight) -- 
        (\stepsize*5.0+1.4, \lvlbase - 0.6*\lvlheight) --
        (\stepsize*4.0-1.4, \lvlbase - 0.6*\lvlheight) --
        (\stepsize*4.0-1.4, \lvlbase + 0.4*\lvlheight) ;


    \node[align=left,anchor=north east,text width=2.5cm] at (\stepsize*5+1.4,0.4*\lvlheight + \lvlbase)
    {\textit{\phantom{(a) }Decoder \phantom{(a) }Contextual \phantom{(a) }Sentence \phantom{(a) }Embedding}};


\end{tikzpicture}}
\caption{Attention layer, contextual sentence embedding, and salience
estimation layer for the \stsext~extractor.}
\label{fig:stsext3}
\end{minipage}}
\end{figure}




\FloatBarrier


Each decoder contextual sentence embedding $\stsextDecHid_i$ then attends
to the encoder contextual sentence embeddings $\stsextEncHid_1,\ldots,
\stsextEncHid_\docSize$, to produce an attention-weighted encoder sentence
embedding $\stsextAttnHid_i$,
    
\vspace{10pt} \noindent \textit{(Fig.~\ref{fig:stsext3}.a) Attention Weights}
\begin{align}
    \forall i : \;\; i \in \{1,\ldots,\docSize\}&\nonumber \\
    \stsextAttn_{i,j} & = 
        \frac{\exp \left(\stsextDecHid_i \cdot  \stsextEncHid_j \right)}{
            \sum_{j^\prime=1}^{\docSize}\exp\left(  
                \stsextDecHid_i \cdot  \stsextEncHid_{j^\prime} \right)},
\end{align}
\vspace{10pt} \noindent \textit{(Fig.~\ref{fig:stsext3}.b) Attention-weighted Encoder Sentence Embeddings} 
\begin{align}
    \forall i : \;\; i \in \{1,\ldots,\docSize\}& \nonumber \\
    \stsextAttnHid_i & = 
        \sum_{j=1}^{\docSize} \alpha_{i,j} \left[\begin{array}{c}
            \stsextREncHid_j\\ 
            \stsextLEncHid_j\end{array}\right].
\end{align}

The attention-weighted encoder sentence embeddings and the decoder 
contextual sentence embedding are then concatenated and fed through a 
\feedforward~layer to produce the $i^\textrm{th}$ contextual sentence
embedding $\stsextHid_i$, which is itself fed through a final 
\feedforward~layer to compute the $i^\textrm{th}$ salience estimate
$\psal_i = \model(\bsal_i=1|\sentEmb_1,\ldots,\sentEmb_\docSize;\xParams)$,

\vspace{10pt}   \noindent \textit{(Fig.~\ref{fig:stsext3}.c) Contextual Sentence Embedding} 
\begin{align}
    \forall i : \;\; i \in \{1,\ldots,\docSize\} & \nonumber\\
    \stsextHid_i &= \relu\left(\stsextHidWeight \left[\begin{array}{c}
        \stsextAttnHid_i \\ 
        \stsextDecHid_i \end{array}\right] 
        + \stsextHidBias \right),
\end{align}

~\\

\vspace{10pt} \noindent \textit{(Fig.~\ref{fig:stsext3}.d) Salience Estimate}
\begin{align}
    \forall i : \;\; i \in \{1,\ldots,\docSize\} & \nonumber\\
\psal_i =    \model(\bsal_i=1|\sentEmb_1, \ldots,\sentEmb_\docSize;\xParams)& = 
            \sigma\left(\stsextPredWeight \stsextHid_i + \stsextPredBias  
            \right).
\end{align}
where $\stsextHidWeight \in \reals^{\stsextHidDim \times 3\stsextRNNDim}$,
$\stsextHidBias \in \reals^{\stsextHidDim}$, 
$\stsextPredWeight \in \reals^{1 \times \stsextHidDim}$, and
$\stsextPredBias \in \reals$ are model parameters.
The complete set of \stsext~extractor parameters is 
\[\xParams = \left\{ \stsextREncParams, \stsextLEncParams,
        \stsextRDecParams, \stsextLDecParams, \stsextHidWeight, \stsextHidBias, \stsextPredWeight, \stsextPredBias, \right\}. \]
        In our experiments, we set $\stsextRNNDim=300$, $\stsextHidDim=100$.
        Dropout with drop probability of 0.25 is applied to $\stsextREncHid_i,
        \stsextLEncHid_i, \stsextRDecHid_i, \stsextLDecHid_i$, and $\stsextHid_i$ for all $i \in \{1,\ldots,\docSize\}$.

%?
%?
%?
%?sentence into a query vector which attends to the encoder output. The
%?attention weighted encoder output and the decoder $\gru$ output are concatenated
%?and fed into a multi-layer perceptron to compute the extraction probability.
%?See \autoref{fig:extractors}.b for a graphical layout.
%?%and \autoref{app:s2sextractor} for details.
%?
%?
%?
