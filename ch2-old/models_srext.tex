
\subsubsection{\srext~Extractor}

\citet{nallapati2017summarunner} proposed
a sentence extractor, which we refer to as the \srext~Extractor,
that factorizes the salience estimates for each sentence into contributions 
from five different sources, which we refer to as \saliencefactors.
The \saliencefactors~take into account interactions between 
contextual sentence embeddings and document embeddings or summary embeddings,
as well as sentence position embeddings. Salience estimates are made sequentially, starting with the first sentence $\sent_1$ and preceding to the last $\sent_\docSize$. When computing the salience estimate of sentence $\sent_i$, 
the previous $i-1$ salience estimates are used to update the summary
representation.

In order to construct the contextual sentence embeddings, document 
embeddings, and summary embeddings, the \srext~extractor first runs 
a \bidirectional~\gru~over the sentence 
embeddings created by the sentence encoder (visually depicted in \autoref{fig:sr1}), 

\begin{figure}[h!]
    \fbox{\begin{minipage}{\textwidth}
\center
\scalebox{0.75}{
\begin{tikzpicture}[
  dep/.style ={
    ->,line width=0.3mm
  },
  hid/.style 2 args={
    rectangle split,
    draw=#2,
    rectangle split parts=#1,
    fill=#2!20,
    minimum width=5mm,
    minimum height=5mm,
    outer sep=2mm},
  mlp/.style 2 args={
    rectangle split,
    rectangle split horizontal,
    draw=#2,
    rectangle split parts=#1,
    fill=#2!20,
    outer sep=2mm},
  sal/.style={
    circle, 
    minimum width=8mm,
    outer sep=2mm,
    draw=#1, 
    fill=#1!20},
]

  \def\stepsize{1.5}%
  \def\lvlbase{0}%
  \def\lvlheight{3}%
 

    % Sentence Embeddings    
  \foreach \step in {1,...,3} {
    \node[hid={3}{sentemb}] (s\step) at (\stepsize*\step, \lvlbase) {};    
    \node at (\stepsize*\step, \lvlbase) {$\sentEmb_\step$};    
   }
    \draw[rectangle,draw=black,dotted] 
        (\stepsize*0.0,\lvlbase + 2*\lvlheight) -- 
        (\stepsize*3.9, \lvlbase + 2*\lvlheight) -- 
        (\stepsize*3.9, \lvlbase + 0.5*\lvlheight) --
        (\stepsize*0.0, \lvlbase + 0.5*\lvlheight) --
        (\stepsize*0.0, \lvlbase + 2*\lvlheight) ;

    \node[align=left,anchor=north west] 
        at (\stepsize * 0.0,\lvlbase + 2*\lvlheight) 
        {\textit{(a) Forward and Backward \gru}\\\textit{\phantom{(a) }Outputs}};

    \draw[rectangle,draw=black,dotted] 
        (\stepsize*0.0,\lvlbase + 0.4*\lvlheight) -- 
        (\stepsize*3.9, \lvlbase + 0.4*\lvlheight) -- 
        (\stepsize*3.9, \lvlbase -0.75*\lvlheight) --
        (\stepsize*0.0, \lvlbase -0.75*\lvlheight) --
        (\stepsize*0.0, \lvlbase + 0.4*\lvlheight) ;

    \node[align=left,anchor=south west] 
        at (\stepsize * 0.0,\lvlbase + -0.75*\lvlheight) 
        {\textit{Sentence Embeddings}\\\textit{(Sentence Encoder Output)}};


    \draw[rectangle,draw=black,dotted] 
        (\stepsize*4.0,\lvlbase + 2*\lvlheight) -- 
        (\stepsize*7.9, \lvlbase + 2*\lvlheight) -- 
        (\stepsize*7.9, \lvlbase + 0.5*\lvlheight) --
        (\stepsize*4.0, \lvlbase + 0.5*\lvlheight) --
        (\stepsize*4.0, \lvlbase + 2*\lvlheight) ;

    \node[align=left,anchor=north west] 
        at (\stepsize * 4.0,\lvlbase + 2*\lvlheight) 
        {\textit{(b) Contextual Sentence}\\ \textit{\phantom{(b) } Embeddings}};

    \draw[rectangle,draw=black,dotted] 
        (\stepsize*8.0,\lvlbase + 2*\lvlheight) -- 
        (\stepsize*11.9, \lvlbase + 2*\lvlheight) -- 
        (\stepsize*11.9, \lvlbase + 0.5*\lvlheight) --
        (\stepsize*8.0, \lvlbase + 0.5*\lvlheight) --
        (\stepsize*8.0, \lvlbase + 2*\lvlheight) ;

    \node[align=left,anchor=north west] 
        at (\stepsize * 8.0,\lvlbase + 2*\lvlheight) 
        {\textit{(c) Document Embedding}};




%    % RNN hidden states
    \foreach \step in {1,...,3} {
        \node[hid={3}{rencemb}] (rrnn_\step) 
            at (\stepsize *\step-0.3, \lvlbase + \lvlheight) {};    
        \node at (\stepsize *\step-0.3, \lvlbase + \lvlheight) 
            {$\rnnextRHid_\step$}; 
        \node[hid={3}{lencemb}] (lrnn_\step) 
            at (\stepsize *\step+0.3, \lvlbase + \lvlheight + 1.0) {};    
        \node at (\stepsize *\step+0.3, \lvlbase + \lvlheight+ 1.0) 
            {$\rnnextLHid_\step$}; 
        \draw[dep] (s\step.north) -- (rrnn_\step.south);
        \draw[dep] (s\step.north) -- (lrnn_\step.south);

    }
    \foreach \start [count=\stop from 2] in {1,...,2} {
        \draw[dep] ($ (rrnn_\start.east) - (0,0.3)$) 
            -- ($ (rrnn_\stop.west) - (0,0.3) $);
        \draw[dep] ($(lrnn_\stop.west) + (0,0.3)$) 
            -- ($ (lrnn_\start.east) + (0,0.3)   $);
    }

  \def\stepsize{1.5}%
  \def\lvlbase{0}%
  \def\lvlheight{3}%
 

    \foreach \step in {1,...,3} {
        \node[hid={3}{rencemb}] (rrnn_\step) 
            at (\stepsize *\step-0.35 + 5 + 1, \lvlbase + 0*\lvlheight) {};    
        \node at (\stepsize*\step-0.35 + 5 + 1, \lvlbase + 0*\lvlheight) 
            {$\rnnextRHid_\step$}; 
        \node[hid={3}{lencemb}] (lrnn_\step) 
            at (\stepsize *\step+5.35 + 1, \lvlbase + 0*\lvlheight + 0.5) {};    
        \node at (\stepsize *\step+5.35 + 1, \lvlbase + 0*\lvlheight+ 0.5) 
            {$\rnnextLHid_\step$}; 
        \node[hid={3}{ctxemb}] (ctx_\step) 
            at (\stepsize *\step + 5 + 1, \lvlbase + 1*\lvlheight) {};    
        \node at (\stepsize*\step + 5 + 1, \lvlbase + 1*\lvlheight) 
            {$\srHid_\step$}; 

        \draw[dep] (rrnn_\step.north) to (ctx_\step.south);
        \draw[dep] (lrnn_\step.north) to (ctx_\step.south);
    }


        \node[hid={3}{doc}] (doc) 
            at (\stepsize *2 + 10+2, \lvlbase + 1*\lvlheight) {};    
        \node at (\stepsize*2 + 10+2, \lvlbase + 1*\lvlheight) 
            {$\srDocEmb$}; 



    \foreach \step in {1,...,3} {
        \node[hid={3}{rencemb}] (rrnn_\step) 
            at (\stepsize *\step-0.35 + 10 + 2, \lvlbase + 0*\lvlheight) {};    
        \node at (\stepsize*\step-0.35 + 10 + 2, \lvlbase + 0*\lvlheight) 
            {$\rnnextRHid_\step$}; 
        \node[hid={3}{lencemb}] (lrnn_\step) 
            at (\stepsize *\step+10.35+2, \lvlbase + 0*\lvlheight + 0.5) {};    
        \node at (\stepsize *\step+10.35+2, \lvlbase + 0*\lvlheight+ 0.5) 
            {$\rnnextLHid_\step$}; 

        \draw[dep] (rrnn_\step.north) to (doc.south);
        \draw[dep] (lrnn_\step.north) to (doc.south);
    }


\end{tikzpicture}}


\caption{\srext~contextual sentence embedding and document embeddings.}
\label{fig:sr1}\end{minipage}}
\end{figure}


%
\begin{wrapfigure}{L}{0.45\textwidth}
    \fbox{\begin{minipage}{0.40\textwidth}
\center
\scalebox{0.75}{
\begin{tikzpicture}[
  dep/.style ={
    ->,line width=0.3mm
  },
  hid/.style 2 args={
    rectangle split,
    draw=#2,
    rectangle split parts=#1,
    fill=#2!20,
    minimum width=5mm,
    minimum height=5mm,
    outer sep=2mm},
]

  \def\stepsize{1.5}%
  \def\lvlbase{0}%
  \def\lvlheight{3}%
 

    % Sentence Embeddings    
  \foreach \step in {1,...,3} {
    \node[hid={3}{sentemb}] (s\step) at (\stepsize*\step, \lvlbase) {};    
    \node at (\stepsize*\step, \lvlbase) {$\sentEmb_\step$};    
   }

%    % RNN hidden states
    \foreach \step in {1,...,3} {
        \node[hid={3}{rencemb}] (rrnn_\step) 
            at (\stepsize *\step-0.3, \lvlbase + \lvlheight) {};    
        \node at (\stepsize *\step-0.3, \lvlbase + \lvlheight) 
            {$\rnnextRHid_\step$}; 
        \node[hid={3}{lencemb}] (lrnn_\step) 
            at (\stepsize *\step+0.3, \lvlbase + \lvlheight + 1.0) {};    
        \node at (\stepsize *\step+0.3, \lvlbase + \lvlheight+ 1.0) 
            {$\rnnextLHid_\step$}; 
        \draw[dep] (s\step.north) -- (rrnn_\step.south);
        \draw[dep] (s\step.north) -- (lrnn_\step.south);

    }
    \foreach \start [count=\stop from 2] in {1,...,2} {
        \draw[dep] ($ (rrnn_\start.east) - (0,0.3)$) 
            -- ($ (rrnn_\stop.west) - (0,0.3) $);
        \draw[dep] ($(lrnn_\stop.west) + (0,0.3)$) 
            -- ($ (lrnn_\start.east) + (0,0.3)   $);
    }




    \draw[rectangle,draw=black,dotted] 
        (\stepsize*0.0,\lvlbase + 2*\lvlheight) -- 
        (\stepsize*3.9, \lvlbase + 2*\lvlheight) -- 
        (\stepsize*3.9, \lvlbase + 0.5*\lvlheight) --
        (\stepsize*0.0, \lvlbase + 0.5*\lvlheight) --
        (\stepsize*0.0, \lvlbase + 2*\lvlheight) ;

    \node[align=left,anchor=north west] 
        at (\stepsize * 0.0,\lvlbase + 2*\lvlheight) 
        {\textit{Forward and Backward \gru}};
    \node[align=left,anchor=north west] 
        at (\stepsize * 0.0,\lvlbase + 1.85*\lvlheight) 
        {\textit{Outputs}};

    \draw[rectangle,draw=black,dotted] 
        (\stepsize*0.0,\lvlbase + 0.4*\lvlheight) -- 
        (\stepsize*3.9, \lvlbase + 0.4*\lvlheight) -- 
        (\stepsize*3.9, \lvlbase -0.75*\lvlheight) --
        (\stepsize*0.0, \lvlbase -0.75*\lvlheight) --
        (\stepsize*0.0, \lvlbase + 0.4*\lvlheight) ;

    \node[align=left,anchor=south west] 
        at (\stepsize * 0.0,\lvlbase + -0.60*\lvlheight) 
        {\textit{Sentence Embeddings}};

    \node[align=left,anchor=south west] 
        at (\stepsize * 0.0,\lvlbase + -0.75*\lvlheight) 
        {\textit{(Sentence Encoder Output)}};



\end{tikzpicture}}


\caption{\srext~forward and backward \gru~outputs.}
\label{fig:srpartial}
\end{minipage}}\end{wrapfigure}


\noindent \textit{(Fig.~\ref{fig:sr1}.a) Forward and Backward \gru~Outputs}
\begin{align}
    \srrHid_0 &= \zeroEmb, \quad \srlHid_{\docSize + 1} = \zeroEmb, \\
    \forall i : \;\; i \in \{1,\ldots,\docSize\}&\nonumber \\
    \srrHid_i &= \fgru(\sentEmb_i, \srrHid_{i-1}; \srRRNNParams), \\
    \srlHid_i &= \fgru(\sentEmb_i, \srlHid_{i+1}; \srLRNNParams),
\end{align}
where $\srrHid_i,\srlHid_i \in \reals^{\srRNNDim}$ and $\srRRNNParams$ and $\srLRNNParams$ are the forward and backward \gru~parameters respectively.



%\footnote{\citet{nallapati2017summarunner}
%    use an RNN sentence encoder with 
%this extractor architecture; in this work we pair the \srext~extractor
%with different encoders. }% 
    The \gru~output is
concatenated and run through a feed-forward layer to obtain 
a contextual sentence embedding representation $\srHid_i \in \reals^{\srRepDim}$, 


\vspace{10pt}
\noindent \textit{(Fig.~\ref{fig:sr1}.b) Contextual Sentence Embeddings} 
\begin{align}
    \forall i : \;\; i \in \{1,\ldots,\docSize\}&\nonumber \\
\srHid_i  & = \relu\left(\srSentBias + \srSentWeight \left[ \begin{array}{c} \srrHid_i \\ \srlHid_i  \end{array} \right]  \right),
\end{align}

where $\srSentWeight \in \reals^{\srRepDim \times 2\srRNNDim}$ and $\srSentBias \in \reals^{\srRepDim}$ are learned parameters.

To construct the document embedding $\srDocEmb$, the forward and backward 
\gru~outputs are concatenated and averaged before running through a different 
\feedforward~layer,
%and another \feedforward~layer to obtain contextual sentence embeddings
%$\srHid_i \in \reals^{{\color{red}???}}$, depicted visually in the schematic in \autoref{fig:sr1}, and defined by the following equations,

\vspace{10pt}
\noindent\textit{(Fig~\ref{fig:sr1}.c) Document Embedding}
\begin{align}
\srDocEmb  & = \tanh\left(\srDocBias + \srDocWeight \left(\frac{1}{\docSize}\sum_{i=1}^{\docSize} \left[ \begin{array}{c} \srrHid_i \\ \srlHid_i  \end{array} \right] \right) \right)
%    \srHid_i & = \left[ \begin{array}{c} \srrHid \\ \srlHid  \end{array} \right] 
\end{align}
where $\srDocWeight \in \reals^{\srRepDim \times 2\srRNNDim}$ and $\srDocBias \in \reals^{\srRepDim}$ are learned parameters.

%\begin{figure}[h!]
    \fbox{\begin{minipage}{\textwidth}
\center
\scalebox{0.75}{
\begin{tikzpicture}[
  dep/.style ={
    ->,line width=0.3mm
  },
  hid/.style 2 args={
    rectangle split,
    draw=#2,
    rectangle split parts=#1,
    fill=#2!20,
    minimum width=5mm,
    minimum height=5mm,
    outer sep=2mm},
  mlp/.style 2 args={
    rectangle split,
    rectangle split horizontal,
    draw=#2,
    rectangle split parts=#1,
    fill=#2!20,
    outer sep=2mm},
  sal/.style={
    circle, 
    minimum width=8mm,
    outer sep=2mm,
    draw=#1, 
    fill=#1!20},
]

  \def\stepsize{1.5}%
  \def\lvlbase{0}%
  \def\lvlheight{3}%
 

    % Sentence Embeddings    
  \foreach \step in {1,...,3} {
    \node[hid={3}{sentemb}] (s\step) at (\stepsize*\step, \lvlbase) {};    
    \node at (\stepsize*\step, \lvlbase) {$\sentEmb_\step$};    
   }
    \draw[rectangle,draw=black,dotted] 
        (\stepsize*0.0,\lvlbase + 2*\lvlheight) -- 
        (\stepsize*3.9, \lvlbase + 2*\lvlheight) -- 
        (\stepsize*3.9, \lvlbase + 0.5*\lvlheight) --
        (\stepsize*0.0, \lvlbase + 0.5*\lvlheight) --
        (\stepsize*0.0, \lvlbase + 2*\lvlheight) ;

    \node[align=left,anchor=north west] 
        at (\stepsize * 0.0,\lvlbase + 2*\lvlheight) 
        {\textit{(a) Forward and Backward \gru}\\\textit{\phantom{(a) }Outputs}};

    \draw[rectangle,draw=black,dotted] 
        (\stepsize*0.0,\lvlbase + 0.4*\lvlheight) -- 
        (\stepsize*3.9, \lvlbase + 0.4*\lvlheight) -- 
        (\stepsize*3.9, \lvlbase -0.75*\lvlheight) --
        (\stepsize*0.0, \lvlbase -0.75*\lvlheight) --
        (\stepsize*0.0, \lvlbase + 0.4*\lvlheight) ;

    \node[align=left,anchor=south west] 
        at (\stepsize * 0.0,\lvlbase + -0.75*\lvlheight) 
        {\textit{Sentence Embeddings}\\\textit{(Sentence Encoder Output)}};


    \draw[rectangle,draw=black,dotted] 
        (\stepsize*4.0,\lvlbase + 2*\lvlheight) -- 
        (\stepsize*7.9, \lvlbase + 2*\lvlheight) -- 
        (\stepsize*7.9, \lvlbase + 0.5*\lvlheight) --
        (\stepsize*4.0, \lvlbase + 0.5*\lvlheight) --
        (\stepsize*4.0, \lvlbase + 2*\lvlheight) ;

    \node[align=left,anchor=north west] 
        at (\stepsize * 4.0,\lvlbase + 2*\lvlheight) 
        {\textit{(b) Contextual Sentence}\\ \textit{\phantom{(b) } Embeddings}};

    \draw[rectangle,draw=black,dotted] 
        (\stepsize*8.0,\lvlbase + 2*\lvlheight) -- 
        (\stepsize*11.9, \lvlbase + 2*\lvlheight) -- 
        (\stepsize*11.9, \lvlbase + 0.5*\lvlheight) --
        (\stepsize*8.0, \lvlbase + 0.5*\lvlheight) --
        (\stepsize*8.0, \lvlbase + 2*\lvlheight) ;

    \node[align=left,anchor=north west] 
        at (\stepsize * 8.0,\lvlbase + 2*\lvlheight) 
        {\textit{(c) Document Embedding}};




%    % RNN hidden states
    \foreach \step in {1,...,3} {
        \node[hid={3}{rencemb}] (rrnn_\step) 
            at (\stepsize *\step-0.3, \lvlbase + \lvlheight) {};    
        \node at (\stepsize *\step-0.3, \lvlbase + \lvlheight) 
            {$\rnnextRHid_\step$}; 
        \node[hid={3}{lencemb}] (lrnn_\step) 
            at (\stepsize *\step+0.3, \lvlbase + \lvlheight + 1.0) {};    
        \node at (\stepsize *\step+0.3, \lvlbase + \lvlheight+ 1.0) 
            {$\rnnextLHid_\step$}; 
        \draw[dep] (s\step.north) -- (rrnn_\step.south);
        \draw[dep] (s\step.north) -- (lrnn_\step.south);

    }
    \foreach \start [count=\stop from 2] in {1,...,2} {
        \draw[dep] ($ (rrnn_\start.east) - (0,0.3)$) 
            -- ($ (rrnn_\stop.west) - (0,0.3) $);
        \draw[dep] ($(lrnn_\stop.west) + (0,0.3)$) 
            -- ($ (lrnn_\start.east) + (0,0.3)   $);
    }

  \def\stepsize{1.5}%
  \def\lvlbase{0}%
  \def\lvlheight{3}%
 

    \foreach \step in {1,...,3} {
        \node[hid={3}{rencemb}] (rrnn_\step) 
            at (\stepsize *\step-0.35 + 5 + 1, \lvlbase + 0*\lvlheight) {};    
        \node at (\stepsize*\step-0.35 + 5 + 1, \lvlbase + 0*\lvlheight) 
            {$\rnnextRHid_\step$}; 
        \node[hid={3}{lencemb}] (lrnn_\step) 
            at (\stepsize *\step+5.35 + 1, \lvlbase + 0*\lvlheight + 0.5) {};    
        \node at (\stepsize *\step+5.35 + 1, \lvlbase + 0*\lvlheight+ 0.5) 
            {$\rnnextLHid_\step$}; 
        \node[hid={3}{ctxemb}] (ctx_\step) 
            at (\stepsize *\step + 5 + 1, \lvlbase + 1*\lvlheight) {};    
        \node at (\stepsize*\step + 5 + 1, \lvlbase + 1*\lvlheight) 
            {$\srHid_\step$}; 

        \draw[dep] (rrnn_\step.north) to (ctx_\step.south);
        \draw[dep] (lrnn_\step.north) to (ctx_\step.south);
    }


        \node[hid={3}{doc}] (doc) 
            at (\stepsize *2 + 10+2, \lvlbase + 1*\lvlheight) {};    
        \node at (\stepsize*2 + 10+2, \lvlbase + 1*\lvlheight) 
            {$\srDocEmb$}; 



    \foreach \step in {1,...,3} {
        \node[hid={3}{rencemb}] (rrnn_\step) 
            at (\stepsize *\step-0.35 + 10 + 2, \lvlbase + 0*\lvlheight) {};    
        \node at (\stepsize*\step-0.35 + 10 + 2, \lvlbase + 0*\lvlheight) 
            {$\rnnextRHid_\step$}; 
        \node[hid={3}{lencemb}] (lrnn_\step) 
            at (\stepsize *\step+10.35+2, \lvlbase + 0*\lvlheight + 0.5) {};    
        \node at (\stepsize *\step+10.35+2, \lvlbase + 0*\lvlheight+ 0.5) 
            {$\rnnextLHid_\step$}; 

        \draw[dep] (rrnn_\step.north) to (doc.south);
        \draw[dep] (lrnn_\step.north) to (doc.south);
    }


\end{tikzpicture}}


\caption{\srext~contextual sentence embedding and document embeddings.}
\label{fig:sr1}\end{minipage}}
\end{figure}



%%\noindent where $\srrHid, \srlHid \in \reals^{\srRNNDim}$, 
%    $\srHid \in \reals^{2\srRNNDim}$, and $\srRRNNParams, \srLRNNParams$
%are the parameters for the forward and backward \gru~respectively.



Additionally, an iterative representation of the extract summary at step $i$,
 $\srSum_i$, is constructed by summing the $i-1$ contextual sentence 
embeddings weighted by their salience estimates,

\vspace{10pt}
   \noindent \textit{(Fig.~\ref{fig:sr2}) Summary Embeddings}
\begin{align}
\srSum_1 & = \zeroEmb, \\
\srSum_i & = \tanh\left(\sum_{j=1}^{i-1} \psal_j \cdot \srHid_j\right).
\end{align}
where $\psal_j= \model\left(\bsal_j=1|\bsal_1,\ldots,\bsal_{j-1},\sentEmb_1, \ldots, \sentEmb_\docSize; \xParams\right)$ are previously computed salience estimates for
sentences $\sent_1,\ldots,\sent_{i-1}$.

\begin{figure}[t]
    \fbox{\begin{minipage}{\textwidth}
\center
\scalebox{0.75}{
\begin{tikzpicture}[
  dep/.style ={
    ->,line width=0.3mm
  },
  hid/.style 2 args={
    rectangle split,
    draw=#2,
    rectangle split parts=#1,
    fill=#2!20,
    minimum width=5mm,
    minimum height=5mm,
    outer sep=2mm},
  mlp/.style 2 args={
    rectangle split,
    rectangle split horizontal,
    draw=#2,
    rectangle split parts=#1,
    fill=#2!20,
    outer sep=2mm},
  sal/.style={
    circle, 
    minimum width=8mm,
    outer sep=2mm,
    draw=#1, 
    fill=#1!20},
]

  \def\stepsize{1.5}%
  \def\lvlbase{0}%
  \def\lvlheight{3}%
 
  \def\stepsize{1.5}%
  \def\lvlbase{0}%
  \def\lvlheight{3}%


        \node[hid={3}{ctxemb}] (ctx_1) 
            at (\stepsize *6 , \lvlbase + -5*\lvlheight) {};    
        \node at (\stepsize*6, \lvlbase + -5*\lvlheight) 
            {$\srHid_1$}; 

        \node[sal={sal}] (p_1) 
            at (\stepsize *7 , \lvlbase + -5*\lvlheight) {};    
        \node at (\stepsize*7, \lvlbase + -5*\lvlheight) 
            {$\psal_1$}; 

        \node[hid={3}{ctxemb}] (ctx_2) 
            at (\stepsize *8 , \lvlbase + -5*\lvlheight) {};    
        \node at (\stepsize*8, \lvlbase + -5*\lvlheight) 
            {$\srHid_2$}; 

        \node[sal={sal}] (p_2) 
            at (\stepsize *9 , \lvlbase + -5*\lvlheight) {};    
        \node at (\stepsize*9, \lvlbase + -5*\lvlheight) 
            {$\psal_2$}; 


        \node[hid={3}{sum}] (summary_3) 
            at (\stepsize *9 , \lvlbase + -4*\lvlheight) {};    
        \node at (\stepsize*9, \lvlbase + -4*\lvlheight) 
            {$\srSum_{i}$}; 


            \node at (\stepsize *9 , \lvlbase + -3.65*\lvlheight) {\textit{Summary Embedding}};

        \node[hid={3}{ctxemb}] (ctx_n) 
            at (\stepsize *11 , \lvlbase + -5*\lvlheight) {};    
        \node at (\stepsize*11, \lvlbase + -5*\lvlheight) 
            {$\srHid_{i-1}$}; 

        \node[sal={sal}] (p_n) 
            at (\stepsize *12 , \lvlbase + -5*\lvlheight) {};    
        \node at (\stepsize*12, \lvlbase + -5*\lvlheight) 
            {$\psal_{i-1}$}; 

        \node at (\stepsize*10, \lvlbase + -5*\lvlheight) {\Large $\cdots$};





        \draw[dep] (ctx_1.north) to (summary_3.south);
        \draw[dep] (p_1.north) to (summary_3.south);

        \draw[dep] (ctx_2.north) to (summary_3.south);
        \draw[dep] (p_2.north) to (summary_3.south);

        \draw[dep] (ctx_n.north) to (summary_3.south);
        \draw[dep] (p_n.north) to (summary_3.south);













%        \node[hid={3}{yellow}] (ctx_1) 
%            at (\stepsize *4 , \lvlbase + -5*\lvlheight) {};    
%        \node at (\stepsize*4, \lvlbase + -5*\lvlheight) 
%            {$\srHid_1$}; 
%        \node[hid={3}{yellow}] (ctx_2) 
%            at (\stepsize *5 , \lvlbase + -5*\lvlheight) {};    
%        \node at (\stepsize*5, \lvlbase + -5*\lvlheight) 
%            {$\srHid_2$}; 
%
%
%

%    \foreach \start [count=\stop from 2] in {1,...,2} {
%        \draw[dep] ($ (rrnn_\start.east) - (0,0.3)$) 
%            -- ($ (rrnn_\stop.west) - (0,0.3) $);
%        \draw[dep] ($(lrnn_\stop.west) + (0,0.3)$) 
%            -- ($ (lrnn_\start.east) + (0,0.3)   $);
%    }

%    \draw[rectangle,draw=black,dotted] 
%        (\stepsize*-2.5,\lvlbase + 3.5*\lvlheight) -- 
%        (\stepsize*3.5, \lvlbase + 3.5*\lvlheight) -- 
%        (\stepsize*3.5, \lvlbase + 2.7*\lvlheight) --
%        (\stepsize*-2.5, \lvlbase + 2.7*\lvlheight) --
%        (\stepsize*-2.5, \lvlbase + 3.5*\lvlheight) ;
%
%    \node[align=left,anchor=north west] 
%        at (\stepsize * -2.5,\lvlbase + 3.5*\lvlheight) 
%        {\textit{Salience Estimates}};
%
%    \draw[rectangle,draw=black,dotted] 
%        (\stepsize*-2.5,\lvlbase + 2.6*\lvlheight) -- 
%        (\stepsize*3.5, \lvlbase + 2.6*\lvlheight) -- 
%        (\stepsize*3.5, \lvlbase + 1.8*\lvlheight) --
%        (\stepsize*-2.5, \lvlbase + 1.8*\lvlheight) --
%        (\stepsize*-2.5, \lvlbase + 2.6*\lvlheight) ;
%
%    \node[align=left,anchor=north west] 
%        at (\stepsize * -2.5,\lvlbase + 2.6*\lvlheight) 
%        {\textit{Contextual Sentence Embeddings}};
%
%    \draw[rectangle,draw=black,dotted] 
%        (\stepsize*-2.5,\lvlbase + 0.5*\lvlheight) -- 
%        (\stepsize*3.5, \lvlbase + 0.5*\lvlheight) -- 
%        (\stepsize*3.5, \lvlbase + -0.50*\lvlheight) --
%        (\stepsize*-2.5, \lvlbase + -0.50*\lvlheight) --
%        (\stepsize*-2.5, \lvlbase + 0.5*\lvlheight) ;
%
%    \node[align=left,anchor=north west] 
%        at (\stepsize * -2.5,\lvlbase + 0.5*\lvlheight) 
%        {\textit{Sentence Embeddings}\\\textit{(Sentence Encoder Output)}};
%
%
%    \draw[rectangle,draw=black,dotted] 
%        (\stepsize*-2.5,\lvlbase + 1.7*\lvlheight) -- 
%        (\stepsize*3.5, \lvlbase + 1.7*\lvlheight) -- 
%        (\stepsize*3.5, \lvlbase + 0.6*\lvlheight) --
%        (\stepsize*-2.5, \lvlbase + 0.6*\lvlheight) --
%        (\stepsize*-2.5, \lvlbase + 1.7*\lvlheight) ;
%
%
%    \node[align=left,anchor=north west] 
%        at (\stepsize * -2.5,\lvlbase + 1.7*\lvlheight) 
%        {\textit{Forward and Backward Partial}\\\textit{Contexual Sentence Embeddings}};

\end{tikzpicture}}

        \caption{\srext~iterative summary embeddings.}
        \label{fig:sr2}

\end{minipage}}
\end{figure}



Each salience estimate $\psal_i$ is calculated as the sum of five \saliencefactors~run through a logistic sigmoid function (depicted in \autoref{fig:sr4}),

\vspace{10pt}
    \noindent \textit{(Fig.~\ref{fig:sr4}) Salience Estimates}
\begin{align}
  \psal_i =  \model(\bsal_i=1|\bsal_1,\dots,\bsal_{i-1},\sentEmb_1,\ldots,\sentEmb_\docSize;\xParams)
         & = 
        \sigma\left(\srContentFactor_i 
        + \srSalienceFactor_i + \srNoveltyFactor_i
    + \srFinePositionFactor_i + \srCoarsePositionFactor_i   \right).
\end{align}

\begin{figure}[h!]
    \fbox{\begin{minipage}{\textwidth}
            \center
\begin{tikzpicture}[
  dep/.style ={
    ->,line width=0.3mm
  },
  hid/.style 2 args={
    rectangle split,
    rectangle split horizontal,
    draw=#2,
    rectangle split parts=#1,
    fill=#2!20,
    minimum width=5mm,
    minimum height=5mm,
    outer sep=2mm},
  mlp/.style 2 args={
    rectangle split,
    rectangle split horizontal,
    draw=#2,
    rectangle split parts=#1,
    fill=#2!20,
    outer sep=2mm},
  sal/.style={
    circle, 
    minimum width=8mm,
    outer sep=2mm,
    draw=#1, 
    fill=#1!20},
]

  \def\stepsize{2}%

    \node[sal={sal}] (sal) at (7*\stepsize,1) {}; 
    \node at (7*\stepsize,1) {$\psal_i$}; 
\node[align=left,anchor=north west] at (6.1*\stepsize,2.8) {\textit{Salience Estimate}};
\node[align=left,anchor=north west] at (1*\stepsize-1,2.8) {\textit{Salience Factors}};

    \draw[rectangle,draw=black,dotted] (6.1*\stepsize,2.8)
        -- (7.85*\stepsize,2.8)
        -- (7.85*\stepsize,0.4)
        -- (6.1*\stepsize,0.4)
        -- (6.1*\stepsize,2.8);

    \draw[rectangle,draw=black,dotted] (1*\stepsize-1,2.8) --
    (5.5*\stepsize,2.8) --
    (5.5*\stepsize,0.4) --
    (1*\stepsize-1,0.4) -- (1*\stepsize-1,2.8) ;

    \foreach \factor [count=\i from 1] in {
        \srContentFactor,\srSalienceFactor,\srNoveltyFactor,
        \srFinePositionFactor,\srContentFactor} {
    \node[hid={1}{factor}] (f\i) at (\i*\stepsize,1) {}; 
    \node at (\i*\stepsize,1) {$\factor_i$}; 
        \draw[dep] (f\i) [out=25,in=160] to (sal);
    }
%    \node[hid={1}{yellow}] (f2) at (2*\stepsize,1) {}; 
%    \node at (2*\stepsize,1) {$\srSalienceFactor_i$}; 
%    \node[hid={1}{yellow}] at (3*\stepsize,1) {}; 
%    \node at (3*\stepsize,1) {$\srNoveltyFactor_i$}; 
%
%    \node[hid={1}{yellow}] at (4*\stepsize,1) {}; 
%    \node at (4*\stepsize,1) {$\srFinePositionFactor_i$}; 
%
%    \node[hid={1}{yellow}] at (5*\stepsize,1) {}; 
%    \node at (5*\stepsize,1) {$\srCoarsePositionFactor_i$}; 


\end{tikzpicture}
\caption{Schematic for the \srext~extractor's salience estimates.}
\label{fig:sr4}
\end{minipage}}
\end{figure}



%\begin{figure}[h!]
    \fbox{\begin{minipage}{\textwidth}
\center
\scalebox{0.75}{
\begin{tikzpicture}[
  dep/.style ={
    ->,line width=0.3mm
  },
  hid/.style 2 args={
    rectangle split,
    draw=#2,
    rectangle split parts=#1,
    fill=#2!20,
    minimum width=5mm,
    minimum height=5mm,
    outer sep=2mm},
  mlp/.style 2 args={
    rectangle split,
    rectangle split horizontal,
    draw=#2,
    rectangle split parts=#1,
    fill=#2!20,
    outer sep=2mm},
  sal/.style={
    circle, 
    minimum width=8mm,
    outer sep=2mm,
    draw=#1, 
    fill=#1!20},
]

  \def\stepsize{1.5}%
  \def\lvlbase{0}%
  \def\lvlheight{3}%
 

    % Sentence Embeddings    
  \foreach \step in {1,...,3} {
    \node[hid={3}{sentemb}] (s\step) at (\stepsize*\step, \lvlbase) {};    
    \node at (\stepsize*\step, \lvlbase) {$\sentEmb_\step$};    
   }
    \draw[rectangle,draw=black,dotted] 
        (\stepsize*0.0,\lvlbase + 2*\lvlheight) -- 
        (\stepsize*3.9, \lvlbase + 2*\lvlheight) -- 
        (\stepsize*3.9, \lvlbase + 0.5*\lvlheight) --
        (\stepsize*0.0, \lvlbase + 0.5*\lvlheight) --
        (\stepsize*0.0, \lvlbase + 2*\lvlheight) ;

    \node[align=left,anchor=north west] 
        at (\stepsize * 0.0,\lvlbase + 2*\lvlheight) 
        {\textit{(a) Forward and Backward \gru}\\\textit{\phantom{(a) }Outputs}};

    \draw[rectangle,draw=black,dotted] 
        (\stepsize*0.0,\lvlbase + 0.4*\lvlheight) -- 
        (\stepsize*3.9, \lvlbase + 0.4*\lvlheight) -- 
        (\stepsize*3.9, \lvlbase -0.75*\lvlheight) --
        (\stepsize*0.0, \lvlbase -0.75*\lvlheight) --
        (\stepsize*0.0, \lvlbase + 0.4*\lvlheight) ;

    \node[align=left,anchor=south west] 
        at (\stepsize * 0.0,\lvlbase + -0.75*\lvlheight) 
        {\textit{Sentence Embeddings}\\\textit{(Sentence Encoder Output)}};


    \draw[rectangle,draw=black,dotted] 
        (\stepsize*4.0,\lvlbase + 2*\lvlheight) -- 
        (\stepsize*7.9, \lvlbase + 2*\lvlheight) -- 
        (\stepsize*7.9, \lvlbase + 0.5*\lvlheight) --
        (\stepsize*4.0, \lvlbase + 0.5*\lvlheight) --
        (\stepsize*4.0, \lvlbase + 2*\lvlheight) ;

    \node[align=left,anchor=north west] 
        at (\stepsize * 4.0,\lvlbase + 2*\lvlheight) 
        {\textit{(b) Contextual Sentence}\\ \textit{\phantom{(b) } Embeddings}};

    \draw[rectangle,draw=black,dotted] 
        (\stepsize*8.0,\lvlbase + 2*\lvlheight) -- 
        (\stepsize*11.9, \lvlbase + 2*\lvlheight) -- 
        (\stepsize*11.9, \lvlbase + 0.5*\lvlheight) --
        (\stepsize*8.0, \lvlbase + 0.5*\lvlheight) --
        (\stepsize*8.0, \lvlbase + 2*\lvlheight) ;

    \node[align=left,anchor=north west] 
        at (\stepsize * 8.0,\lvlbase + 2*\lvlheight) 
        {\textit{(c) Document Embedding}};




%    % RNN hidden states
    \foreach \step in {1,...,3} {
        \node[hid={3}{rencemb}] (rrnn_\step) 
            at (\stepsize *\step-0.3, \lvlbase + \lvlheight) {};    
        \node at (\stepsize *\step-0.3, \lvlbase + \lvlheight) 
            {$\rnnextRHid_\step$}; 
        \node[hid={3}{lencemb}] (lrnn_\step) 
            at (\stepsize *\step+0.3, \lvlbase + \lvlheight + 1.0) {};    
        \node at (\stepsize *\step+0.3, \lvlbase + \lvlheight+ 1.0) 
            {$\rnnextLHid_\step$}; 
        \draw[dep] (s\step.north) -- (rrnn_\step.south);
        \draw[dep] (s\step.north) -- (lrnn_\step.south);

    }
    \foreach \start [count=\stop from 2] in {1,...,2} {
        \draw[dep] ($ (rrnn_\start.east) - (0,0.3)$) 
            -- ($ (rrnn_\stop.west) - (0,0.3) $);
        \draw[dep] ($(lrnn_\stop.west) + (0,0.3)$) 
            -- ($ (lrnn_\start.east) + (0,0.3)   $);
    }

  \def\stepsize{1.5}%
  \def\lvlbase{0}%
  \def\lvlheight{3}%
 

    \foreach \step in {1,...,3} {
        \node[hid={3}{rencemb}] (rrnn_\step) 
            at (\stepsize *\step-0.35 + 5 + 1, \lvlbase + 0*\lvlheight) {};    
        \node at (\stepsize*\step-0.35 + 5 + 1, \lvlbase + 0*\lvlheight) 
            {$\rnnextRHid_\step$}; 
        \node[hid={3}{lencemb}] (lrnn_\step) 
            at (\stepsize *\step+5.35 + 1, \lvlbase + 0*\lvlheight + 0.5) {};    
        \node at (\stepsize *\step+5.35 + 1, \lvlbase + 0*\lvlheight+ 0.5) 
            {$\rnnextLHid_\step$}; 
        \node[hid={3}{ctxemb}] (ctx_\step) 
            at (\stepsize *\step + 5 + 1, \lvlbase + 1*\lvlheight) {};    
        \node at (\stepsize*\step + 5 + 1, \lvlbase + 1*\lvlheight) 
            {$\srHid_\step$}; 

        \draw[dep] (rrnn_\step.north) to (ctx_\step.south);
        \draw[dep] (lrnn_\step.north) to (ctx_\step.south);
    }


        \node[hid={3}{doc}] (doc) 
            at (\stepsize *2 + 10+2, \lvlbase + 1*\lvlheight) {};    
        \node at (\stepsize*2 + 10+2, \lvlbase + 1*\lvlheight) 
            {$\srDocEmb$}; 



    \foreach \step in {1,...,3} {
        \node[hid={3}{rencemb}] (rrnn_\step) 
            at (\stepsize *\step-0.35 + 10 + 2, \lvlbase + 0*\lvlheight) {};    
        \node at (\stepsize*\step-0.35 + 10 + 2, \lvlbase + 0*\lvlheight) 
            {$\rnnextRHid_\step$}; 
        \node[hid={3}{lencemb}] (lrnn_\step) 
            at (\stepsize *\step+10.35+2, \lvlbase + 0*\lvlheight + 0.5) {};    
        \node at (\stepsize *\step+10.35+2, \lvlbase + 0*\lvlheight+ 0.5) 
            {$\rnnextLHid_\step$}; 

        \draw[dep] (rrnn_\step.north) to (doc.south);
        \draw[dep] (lrnn_\step.north) to (doc.south);
    }


\end{tikzpicture}}


\caption{\srext~contextual sentence embedding and document embeddings.}
\label{fig:sr1}\end{minipage}}
\end{figure}


\pagebreak

\begin{wrapfigure}{r}{0.50\textwidth}
    \fbox{\begin{minipage}{0.50\textwidth}
      \begin{center}
          \begin{tikzpicture}[
  dep/.style ={
    ->,line width=0.3mm
  },
  hid/.style 2 args={
    rectangle split,
    rectangle split horizontal,
    draw=#2,
    rectangle split parts=#1,
    fill=#2!20,
    minimum width=5mm,
    minimum height=5mm,
    outer sep=2mm},
  mlp/.style 2 args={
    rectangle split,
    rectangle split horizontal,
    draw=#2,
    rectangle split parts=#1,
    fill=#2!20,
    outer sep=2mm},
  sal/.style={
    circle, 
    minimum width=8mm,
    outer sep=2mm,
    draw=#1, 
    fill=#1!20},
]

  \def\stepsize{3}%
  \def\lvlbase{0}%
  \def\lvlheight{3}%

        \node[hid={3}{ctxemb}] (ctx_i) 
            at (\stepsize *1 , \lvlbase + -2*\lvlheight) {};    
        \node at (\stepsize*1, \lvlbase + -2*\lvlheight) 
            {$\srHid_i$}; 

        \node[hid={1}{factor}] (content_i) 
            at (\stepsize *2, \lvlbase + -2*\lvlheight) {};    
        \node at (\stepsize*2, \lvlbase + -2*\lvlheight) 
            {$\srContentFactor_i$}; 

        \draw[dep] (ctx_i) to (content_i);
    \draw[rectangle,draw=black,dotted] 
        (\stepsize*0.5,\lvlbase -1.6*\lvlheight) -- 
        (\stepsize*2.25, \lvlbase -1.6*\lvlheight) -- 
        (\stepsize*2.25, \lvlbase - 2.3*\lvlheight) --
        (\stepsize*0.5, \lvlbase -2.3*\lvlheight) --
        (\stepsize*0.5, \lvlbase -1.6*\lvlheight) ;
%

    \node[align=left,anchor=north east] 
        at (\stepsize * 2.25,\lvlbase + -1.6*\lvlheight) 
            {\textit{(a) Content Factor}};

        \node[hid={3}{ctxemb}] (ctx_i) 
            at (\stepsize *1 , \lvlbase + -2.75*\lvlheight) {};    
        \node at (\stepsize*1, \lvlbase + -2.75*\lvlheight) 
            {$\srHid_i$}; 

        \node[hid={1}{factor}] (salience_i) 
            at (\stepsize *2, \lvlbase + -3*\lvlheight) {};    
        \node at (\stepsize*2, \lvlbase + -3*\lvlheight) 
            {$\srSalienceFactor_i$}; 
        \node[hid={3}{doc}] (doc) 
            at (\stepsize *1, \lvlbase + -3.25*\lvlheight) {};    
        \node at (\stepsize*1, \lvlbase + -3.25*\lvlheight) 
            {$\srDocEmb$}; 

        \draw[dep] (ctx_i.east) to (salience_i);
        \draw[dep] (doc.east) to (salience_i);

    \draw[rectangle,draw=black,dotted] 
        (\stepsize*0.5,\lvlbase -2.4*\lvlheight) -- 
        (\stepsize*2.25, \lvlbase -2.4*\lvlheight) -- 
        (\stepsize*2.25, \lvlbase - 3.5*\lvlheight) --
        (\stepsize*0.5, \lvlbase -3.5*\lvlheight) --
        (\stepsize*0.5, \lvlbase -2.4*\lvlheight) ;

    \node[align=left,anchor=north east] 
        at (\stepsize * 2.25,\lvlbase + -2.4*\lvlheight) 
            {\textit{(b) Centrality Factor}};

        \node[hid={3}{ctxemb}] (ctx_i) 
            at (\stepsize *1 , \lvlbase + -4.0*\lvlheight) {};    
        \node at (\stepsize*1, \lvlbase + -4.0*\lvlheight) 
            {$\srHid_i$}; 

        \node[hid={1}{factor}] (salience_i) 
            at (\stepsize *2, \lvlbase + -4.25*\lvlheight) {};    
        \node at (\stepsize*2, \lvlbase + -4.25*\lvlheight) 
            {$\srNoveltyFactor_i$}; 
        \node[hid={3}{sum}] (doc) 
            at (\stepsize *1, \lvlbase + -4.5*\lvlheight) {};    
        \node at (\stepsize*1, \lvlbase + -4.5*\lvlheight) 
            {$\srSum_i$}; 

        \draw[dep] (ctx_i.east) to (salience_i);
        \draw[dep] (doc.east) to (salience_i);


    \draw[rectangle,draw=black,dotted] 
        (\stepsize*0.5,\lvlbase -3.6*\lvlheight) -- 
        (\stepsize*2.25, \lvlbase -3.6*\lvlheight) -- 
        (\stepsize*2.25, \lvlbase - 4.75*\lvlheight) --
        (\stepsize*0.5, \lvlbase -4.75*\lvlheight) --
        (\stepsize*0.5, \lvlbase -3.6*\lvlheight) ;

    \node[align=left,anchor=north east] 
        at (\stepsize * 2.25,\lvlbase + -3.6*\lvlheight) 
            {\textit{(c) Novelty Factor}};



%?        \node[hid={3}{ctxemb}] (ctx_i) 
%?           at (\stepsize *1 , \lvlbase + -5.4*\lvlheight) {};    
%?        \node at (\stepsize*1, \lvlbase + -5.4*\lvlheight) 
%?            {$\srFinePositionEmb_i$}; 
%?
%?        \node[hid={1}{yellow}] (content_i) 
%?            at (\stepsize *2, \lvlbase + -5.4*\lvlheight) {};    
%?        \node at (\stepsize*2, \lvlbase + -5.4*\lvlheight) 
%?            {$\srFinePositionFactor_i$}; 
%?
%?        \draw[dep] (ctx_i) to (content_i);
%?    \draw[rectangle,draw=black,dotted] 
%?        (\stepsize*0.5,\lvlbase -4.6*\lvlheight -0.25 *\lvlheight) -- 
%?        (\stepsize*2.25, \lvlbase -4.6*\lvlheight - 0.25*\lvlheight) -- 
%?        (\stepsize*2.25, \lvlbase - 5.3*\lvlheight-0.25*\lvlheight) --
%?        (\stepsize*0.5, \lvlbase -5.3*\lvlheight - 0.25*\lvlheight) --
%?        (\stepsize*0.5, \lvlbase -4.6*\lvlheight - 0.25*\lvlheight) ;
%?
%?    \node[align=left,anchor=north east] 
%?        at (\stepsize * 2.25,\lvlbase + -4.85*\lvlheight) 
%?            {\textit{(d) Fine-grained}\\[-7pt]\textit{\phantom{(d) }Position Factor}};
%?
%?        \node[hid={3}{ctxemb}] (ctx_i) 
%?           at (\stepsize *1 , \lvlbase + -5.4*\lvlheight-0.8*\lvlheight) {};    
%?        \node at (\stepsize*1, \lvlbase + -5.4*\lvlheight-0.8*\lvlheight) 
%?            {$\srCoarsePositionEmb_i$}; 
%?
%?        \node[hid={1}{yellow}] (content_i) 
%?            at (\stepsize *2, \lvlbase + -5.4*\lvlheight-0.8*\lvlheight) {};    
%?        \node at (\stepsize*2, \lvlbase + -5.4*\lvlheight-0.8*\lvlheight) 
%?            {$\srCoarsePositionFactor_i$}; 
%?
%?        \draw[dep] (ctx_i) to (content_i);
%?    \draw[rectangle,draw=black,dotted] 
%?        (\stepsize*0.5,\lvlbase -4.6*\lvlheight -1.05 *\lvlheight) -- 
%?        (\stepsize*2.25, \lvlbase -4.6*\lvlheight - 1.05*\lvlheight) -- 
%?        (\stepsize*2.25, \lvlbase - 5.3*\lvlheight-1.05*\lvlheight) --
%?        (\stepsize*0.5, \lvlbase -5.3*\lvlheight - 1.05*\lvlheight) --
%?        (\stepsize*0.5, \lvlbase -4.6*\lvlheight - 1.05*\lvlheight) ;
%?
%?    \node[align=left,anchor=north east] 
%?        at (\stepsize * 2.25,\lvlbase + -4.85*\lvlheight-0.8*\lvlheight) 
%?            {\textit{(e) Coarse-grained}\\[-7pt]\textit{\phantom{(d) }Position Factor}};




\end{tikzpicture}


                \end{center}
                  \caption{Schematic of \srext~factors for computing salience 
                  estimates.}
                  \label{fig:srfactors}
          \end{minipage}}
\end{wrapfigure}
Salience factors for content, centrality, and novelty are computed
%~$\phi_i^{(\cdot)}$ is computed 
via the following equations for all $i \in \{1,\ldots,\docSize\}$,

\vspace{10pt}
\noindent \textit{(Fig.~\ref{fig:srfactors}.a) Content Factor} 
\begin{align}
    \srContentFactor_i &=\srContentWeight \srHid_i, 
\end{align}
\vspace{10pt}   \noindent \textit{(Fig.~\ref{fig:srfactors}.b) Centrality\footnote{\citet{nallapati2017summarunner} refer to this as the salience factor, but we rename it here to avoid confusion with the model's final predictions which we call salience estimates.} Factor}
\begin{align}
    \srSalienceFactor_i & = \srHid_i^T\srSalienceWeight \srDocEmb, 
\end{align}
\vspace{10pt} \noindent \textit{(Fig.~\ref{fig:srfactors}.c) Novelty Factor}
\begin{align}
    \srNoveltyFactor_i &= -\srHid_i^T \srNoveltyWeight \srSum_i, \label{eq:srnov} 
\end{align}
where $\srContentWeight \in \reals^{\srRepDim}$, $\srSalienceWeight,\srNoveltyWeight \in \reals^{\srRepDim \times \srRepDim}$ are learned parameters.

Finally, there are two factors for the fine and coarse-grained position,

\vspace{10pt} 
\noindent\textit{%(\ref{fig:srfactors}.d) 
Fine-grained Position Factor}
\begin{align}
       \srFinePositionFactor& = \srFinePositionWeight \srFinePositionEmb_i, 
\end{align}
\vspace{10pt} \textit{%(\ref{fig:srfactors}.e)
Coarse-grained Position Factor}
\begin{align}
           \srCoarsePositionFactor& = \srCoarsePositionWeight \srCoarsePositionEmb_i, 
\end{align}
where $\srFinePositionEmb_i$ and $\srCoarsePositionEmb_i$ are embeddings associated with the sentence position and sentence position quartile of the $i$-th 
sentence (e.g., sentence $s_7$ in a document with 12 sentences, would have 
embeddings $\srFinePositionEmb_7$ and $\srCoarsePositionEmb_2$ corresponding
to the seventh sentence position and $2^\textrm{nd}$ sentence position
quartile respectively).
Both $\srFinePositionWeight, \srCoarsePositionWeight \in \reals^{\srPosDim}$, and $\srFinePositionEmb_1,\ldots,\srFinePositionEmb_\docSizeMax,\srCoarsePositionEmb_1,\ldots,\srCoarsePositionEmb_4 \in \reals^{\srPosDim}$ are learned parameters of the \srext~extractor, and $\docSizeMax\in\naturals$ is the maximum
document size in sentences (when handling unusually long documents, sentences
with positions greater than $\docSizeMax$ are all mapped to $\srFinePositionEmb_\docSizeMax$).

The complete set of parameters for the \srext~extractor is 
\[\xParams = \left\{\srRRNNParams,\srLRNNParams,\srSentWeight,\srSentBias,
\srDocWeight,\srDocBias, \srContentWeight, \srSalienceWeight, \srNoveltyWeight,
\srFinePositionWeight, \srCoarsePositionWeight, \srFinePositionEmb_1,\ldots,\srFinePositionEmb_{\docSizeMax}, \srCoarsePositionEmb_1,\ldots,\srCoarsePositionEmb_4,\right\}. \]
In our experiments, we set $\srRNNDim=300$, $\srRNNDim=100$, $\srPosDim=16$,
and $\docSizeMax={\color{red}???}$. Dropout with drop probability of $0.25$
is applied to the \gru~outputs $\srrHid_i$ and $\srlHid_i$, as well as 
the contextual sentence embeddings $\srHid_i$ for all $i \in \{1,\ldots,\docSize\}$.



















%%A contextual representation of each sentence, $\srHid_i$, is then obtained by 
%%concatening the forward and backward \gru~outputs and running them
%%through a feed forward layer with a $\relu$~activation,
%%\begin{align}
%%\textit{(Contextual Sentence Embedding)} & \nonumber \\
%%\srHid_i  & = \relu\left(\srSentBias + \srSentWeight \left[ \begin{array}{c} \srrHid_i \\ \srlHid_i  \end{array} \right]  \right)
%%\end{align}
%
%
%
%
%~\\~\\~\\~\\
%\pagebreak
%
%iof the previous RNN outputis weighted by their extraction
%probabilities. 
%
%
%In Equation~\ref{eq:srnov}, $g_i$ is an iterative summary representation 
%computed as the
%sum of the previous $z_{<i}$ weighted by their extraction probabilities,
%\begin{align}
%g_i & = \sum_{j=1}^{i-1} p(y_j=1|y_{<j},h) \cdot z_j.
%\end{align}
%
%
%A representation of the whole document is made by 
%averaging contextual sentence embeddings,  
%\begin{align}
%\textit{(Document Embedding)} & \nonumber \\
%\srDocEmb  & = \tanh\left(\srDocBias + \srDocWeight \left(\frac{1}{\docSize}\sum_{i=1}^{\docSize} \srHid_i \right) \right)
%\end{align}
%
%
%Extraction predictions are made using 
%the RNN output at the $i$-th step, the document representation, and 
%$i$-th version of the summary representation, along with factors for 
%sentence location in the document. The use of the iteratively constructed
%summary representation creates a dependence of $y_i$ on all $y_{<i}$.
%See \autoref{fig:extractors}.d for a graphical layout.
%%and \autoref{app:srextractor} for details.
%
%Like the
%RNN~extractor it starts with a bidrectional GRU over the sentence 
%embeddings 
%\begin{align}
%    \rxhid_0 &= 0 \\
%    \rxhid_i &= \fgru(\sentEmb_i, \rxhid_{i-1}; \overrightarrow{\chi}) \\
%   \lxhid_{\docSize + 1} &= 0 \\
%    \lxhid_i &= \fgru(\sentEmb_i, \lxhid_{i+1}; \overleftarrow{\chi})
%\end{align}
%
%It then creates a representation
%of the whole document $q$ by passing the averaged GRU output states through
%a fully connected layer: 
%\begin{align}
%q = \tanh\left(b_q + W_q\frac{1}{\docSize}\sum_{i=1}^{\docSize} [\rxhid_i; \lxhid_i] \right)
%\end{align}
%A concatentation of the GRU outputs at each step
%are passed through a separate fully connected layer to create a 
%sentence representation $z_i$, where
%\begin{align}
%    \xhid_i &= \relu\left(b_z + W_z [\rxhid_i; \lxhid_i]\right).
%\end{align}
%The extraction probability is then determined by contributions from five 
%sources:
%\begin{align}
%    \textit{content} &\quad a^{(con)}_i=W^{(con)} z_i, \\
%    \textit{salience}&\quad a^{(sal)}_i = z_i^TW^{(sal)} q, \\
%    \textit{novelty}&\quad a^{(nov)}_i = -z_i^TW^{(nov)} \tanh(g_i), \label{eq:srnov} \\
%    \textit{position}&\quad a^{(pos)}_i = W^{(pos)} l_i, \\
%    \textit{quartile}&\quad a^{(qrt)}_i = W^{(qrt)} r_i,
%\end{align}
%where $l_i$ and $r_i$ are embeddings associated with the $i$-th sentence
%position and the quarter of the document containing sentence $i$ respectively.
%In Equation~\ref{eq:srnov}, $g_i$ is an iterative summary representation 
%computed as the
%sum of the previous $z_{<i}$ weighted by their extraction probabilities,
%\begin{align}
%g_i & = \sum_{j=1}^{i-1} p(y_j=1|y_{<j},h) \cdot z_j.
%\end{align}
%Note that the presence of this term induces dependence of each 
%$\bsal_i$ to 
%all $\bsal_{<i}$ similarly to the Cheng \& Lapata extractor.
%
%The final extraction probability is the logistic sigmoid of the
%sum of these terms plus a bias,
%\begin{align}
%    p(y_i=1|y_{<i}, h) &= \sigma\left(\begin{array}{l}
%      a_i^{(con)} + a_i^{(sal)} + a_i^{(nov)} \\
%  + a_i^{(pos)}  + a_i^{(qrt)} + b \end{array}\right).
%\end{align}
%The weight matrices $W_q$, $W_z$, $W^{(con)}$, $W^{(sal)}$, $W^{(nov)}$, $W^{(pos)}$,
%$W^{(qrt)}$ and bias terms $b_q$, $b_z$, and $b$ are learned parameters;
%The GRUs have separate learned parameters.
%The hidden layer size of the GRU is 300 for each direction $z_i$, $q$, and $g_i$ have 100 dimensions. The position and quartile embeddings are 16 dimensional each.
%Dropout with drop probability .25 is applied to the GRU outputs and to $z_i$.
%%?
%%?
%%?
%
%Note that in the original paper, the SummaRunner extractor was paired 
%with
%an \textit{RNN} sentence encoder, but in this work we experiment with a variety
%of sentence encoders.
%%?
%
%
%
%%?A document representation $q$ is created by passing the 
%%?averaged RNN output through a fully connected layer.
%%?
%%?Given the RNN output $z_t$ at the step $t$, the following scores are created:
%%?\begin{enumerate}[nolistsep,noitemsep]
%%?\item a content score $W^{(con)}z_t$,
%%?\item a salience score $z_t^TW^{(sal)}q$,
%%?\item a novely score $-z_t^TW^{(nov)}\tanh(g_t)$,
%%?\end{enumerate}
%%?where $g_t = \sum_{i=1}^{t-1} p(y_i=1|y_{<i}, h_{<i}) \cdot z_i$.
%%?These scores are summed along with a bias term and a bias for sentence 
%%?position and the quarter of the document\hal{what does ``the quarter of the document'' mean? sentence position quartile?} and fed through a sigmoid activation
%%?to compute $p(y_t=1|y_{<t}, h_{<t})$.
%
%
%\paragraph{Proposed Sentence Extractors}
%We propose two sentence extractor models that 
%make a stronger conditional independence 
%assumption $p(\bsal|\sentEmb)=\prod_{i=1}^\docSize p(\bsal_i|\sentEmb)$,
%essentially making independent predictions conditioned on $\sentEmb$.
%%In theory, our models should \hal{why should they?} perform worse because of this, however, as
%%we later show, this is not the case empirically.
%
%
