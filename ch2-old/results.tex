
\begin{table*}[p]
    \center
    %\begin{tabular}{ccggccggccggcc}
    \begin{tabular}{cccccccccccccc}
        \toprule
        \multirow{2}{*}{Extractor} &\multirow{2}{*}{Encoder}  & \multicolumn{2}{c}{\textbf{CNN/DM}} & \multicolumn{2}{c}{\textbf{NYT}} & \multicolumn{2}{c}{\textbf{DUC 2002}} \\
         &  & M & R-2 & M & R-2 & M & R-2 \\
        \midrule
%        Random &  -- & 18.2 & 12.7 & 20.7 & 16.0 & 21.7 & 15.8 & \textbf{20.4} & \textbf{12.9} & 12.7 &  2.0 & 16.9 &  9.3\\
%        \hline
        Lead &  -- & 24.1 & 24.4 & 30.0 & 32.3 & 25.1 & 21.5 \\
%        \hline
%        Tail &  -- & 13.7 &  5.4 & 12.5 &  4.1 & 17.8 &  9.1 & \textbf{19.8} & \textbf{10.0} & 11.5 &  2.7 & 17.0 &  9.9\\
        \hline
        \multirow{3}{*}{RNN} & Avg. & \textbf{25.2} & 25.4 & 29.8 & 34.7 & \textbf{26.8} & 22.7 \\
         & RNN & 25.1 & 25.4 & 29.6 & 34.9 & \textbf{26.8} & 22.6 \\
         & CNN & 25.0 & 25.1 & 29.0 & 33.7 & \textbf{26.7} & \textbf{22.7}\\
        \hline
        \multirow{3}{*}{Seq2Seq} & Avg. & \textbf{25.2} & \textbf{25.6} & \textbf{30.5} & \textbf{35.7} & \textbf{27.0} & \textbf{22.8} \\
         & RNN & \textbf{25.1} & 25.3 & 30.2 & \textbf{35.9} & \textbf{26.7} & 22.5 \\
         & CNN & 25.0 & 25.1 & 29.9 & 35.1 & \textbf{26.7} & \textbf{22.7} \\
        \hline
    \multirow{3}{*}{\begin{tabular}{c} Cheng \\ \& \\ Lapata \end{tabular}} & Avg. & 25.0 & 25.3 & 30.4 & \textbf{35.6} & \textbf{27.1} & \textbf{23.1} \\
         & RNN & 25.0 & 25.0 & \textbf{30.3} & \textbf{35.8} & \textbf{27.0} & \textbf{23.0} \\
         & CNN & \textbf{25.2} & 25.1 & 29.9 & 35.0 & \textbf{26.9} & \textbf{23.0} \\
        \hline
    \multirow{3}{*}{\begin{tabular}{c}Summa \\ Runner \end{tabular} } & Avg. & 25.1 & 25.4 & 30.2 & 35.4 & 26.7 & 22.3 \\
         & RNN & 25.1 & 25.2 & 30.0 & 35.5 & 26.5 & 22.1 \\
         & CNN & 24.9 & 25.0 & 29.3 & 34.4 & 26.4 & 22.2 \\
        \hline
        Oracle & -- & 31.1 & 36.2 & 35.3 & 48.9 & 31.3 &  31.8 \\
        \bottomrule
    \end{tabular}

    \caption{News domain \meteor~(M) and \rouge-2 recall (R-2)  results across all 
        extractor/encoder pairs.
           Results that are statistically indistinguishable from the best 
           system are shown in bold face.}
  \label{tab:newsresults}
\end{table*}

\begin{table*}[p]
    \center
    %\begin{tabular}{ccggccggccggcc}
    \begin{tabular}{cccccccccccccc}
        \toprule
        \multirow{2}{*}{Extractor} &\multirow{2}{*}{Encoder}  &  \multicolumn{2}{c}{\textbf{Reddit}} & \multicolumn{2}{c}{\textbf{AMI}} & \multicolumn{2}{c}{\textbf{PubMed}}\\
         &  & M & R-2 & M & R-2 & M & R-2 \\
        \midrule
%        Random &  -- & 18.2 & 12.7 & 20.7 & 16.0 & 21.7 & 15.8 & \textbf{20.4} & \textbf{12.9} & 12.7 &  2.0 & 16.9 &  9.3\\
%        \hline
        Lead &  -- &  \textbf{20.1} & \textbf{10.9} & 12.3 &  2.0 & 15.9 &  9.3\\
%        \hline
%        Tail &  -- & 13.7 &  5.4 & 12.5 &  4.1 & 17.8 &  9.1 & \textbf{19.8} & \textbf{10.0} & 11.5 &  2.7 & 17.0 &  9.9\\
        \hline
        \multirow{3}{*}{RNN} & Avg. &  \textbf{20.4} & \textbf{11.4} & \textbf{17.0} & \textbf{ 5.5} & 19.8 & 17.0\\
         & RNN &  \textbf{20.2} & \textbf{11.4} & 16.2 & \textbf{ 5.2} & 19.7 & 16.6\\
         & CNN & \textbf{20.9} & \textbf{12.8} & 14.4 &  3.2 & 19.9 & 16.8\\
        \hline
        \multirow{3}{*}{Seq2Seq} & Avg. &  \textbf{20.9} & \textbf{13.6} & \textbf{17.0} & \textbf{ 5.5} & \textbf{20.1} & \textbf{17.7}\\
         & RNN &  \textbf{20.5} & \textbf{12.0} & 16.1 & \textbf{ 5.3} & 19.7 & 16.7\\
         & CNN &  \textbf{20.7} & \textbf{13.2} & 14.2 &  2.9 & 19.8 & 16.9\\
        \hline
    \multirow{3}{*}{\begin{tabular}{c} Cheng \&  Lapata \end{tabular}} & Avg. &  \textbf{20.9} & \textbf{13.6} & \textbf{16.7} & \textbf{ 6.1} & \textbf{20.1} & \textbf{17.7}\\
         & RNN &  \textbf{20.3} & \textbf{12.6} & \textbf{16.3} & \textbf{ 5.0} & 19.7 & 16.7\\
         & CNN &  \textbf{20.5} & \textbf{13.4} & 14.3 &  2.8 & 19.9 & 16.9\\
        \hline
    \multirow{3}{*}{\begin{tabular}{c}Summa Runner \end{tabular} } & Avg. & \textbf{21.0} & \textbf{13.4} & \textbf{17.0} & \textbf{ 5.6} & 19.9 & 17.2\\
         & RNN &\textbf{20.9} & \textbf{12.5} & \textbf{16.5} & \textbf{ 5.4} & 19.7 & 16.5\\
         & CNN &  \textbf{20.4} & \textbf{12.3} & 14.5 &  3.2 & 19.8 & 16.8\\
        \hline
        Oracle & -- &  24.3 &  16.2 &17.8 &  8.7  & 24.1 &25.0 \\
        \bottomrule
    \end{tabular}

    \caption{Non-news domain \meteor~(M) and \rouge-2 recall (R-2)  results across all 
        extractor/encoder pairs.
           Results that are statistically indistinguishable from the best 
           system are shown in bold face.}
  \label{tab:otherresults}
\end{table*}


\section{Results}

The results of our main experiment comparing 
the different extractors/encoders on news and non-news domains are shown in 
\autoref{tab:newsresults} and \autoref{tab:otherresults} respectively.
Overall, we find no major advantage when using the \convolutionalneuralnetwork~and \recurrentneuralnetwork~sentence
encoders over the averaging encoder. The best performing encoder/extractor pair either 
uses the averaging 
encoder (five out of six datasets) or the differences 
are not statistically significant. %When only comparing within the 
%same extractor choice,  the averaging encoder is the better choice
%in 14 of 20 cases. 
%\hal{i wonder if it would be worth adding another ``average performance metric'' column to \autoref{tab:results}.
%  i'm thinking have ``Average $\Delta$-Best'' meaning how far (on average across the datasets) is this setting from the best setting available on that dataset.
%  so since the best numbers are: 25.56, 35.85, 23.11, 13.65, 5.63
%  and the first row numbers are: 25.42, 34.67, 22.65, 11.37, 5.50
%  then the deltas are:            0.14,  1.18,  0.46,  2.28, 0.13
%  the the average delta is 0.84 (assuming my math is right)
%  there's an argument to do multiplicative, in which case
%  the multipliers for first row:  0.99,  0.97,  0.98,  0.83, 0.97
%  and the average is 0.95
%  either way this gives a quick way to make comparisons between rows. you could do the same for the other tables too.}

  %\hal{in some of the tables you list R-2 as headers even though all the numbers are R-2. just put that in the caption.}


When looking at extractors, the Seq2Seq extractor is either part of 
the best performing system (three out of six datasets) or is not 
statistically distinguishable from the best extractor. 

Overall, on the news and medical journal domains, the differences are 
quite small with the 
differences between worst and best systems on the CNN/DM dataset 
spanning only .56 of a ROUGE point. While there is more performance variability
 in the Reddit and AMI data, there is less distinction among systems: 
 no differences are significant on Reddit
and every extractor has at least one configuration that is indistinguishable
from the best system on the AMI corpus. This is probably due to the small test
size of these datasets.
%\hal{this is probably at least partially because of test set size. maybe mention this.}





%?\textcolor{red}{Overall we find that the \modelTwoBF~extractor achieves the 
%?best ROUGE scores on three out of four domains (STILL RUNNING ON AMI AND PUBMED). 
%?However, most
%?differences are not signficant. (Need to discuss stat sig and how to show it).}
%?On the larger CNN-DailyMail dataset, especially, 
%?differences are quite smail across all extractor/encoder pairs.
%?The \baselineOneBF~extractor achieves the best performance on the DUC 2002
%?dataset. It is disappointing that the \baselineOneBF~and \baselineTwoBF~based 
%?models do not gain any apparent advantage in conditioning on previous 
%?sentence selection decisions; this result suggests the need to improve
%?the representation of the summary as it is being constructed iteratively.
%?
%?\textbf{Choice of Encoder} We also find there to be no major advantage 
%?between the different sentence encoders. \textcolor{red}{In most cases,
%?there is no statistical significance between the averaging encoder and either
%?the RNN or CNN encoders.} 

%The lack of differentiation amongst the different encoders concerning; one
%would assume learning with the appropriate structure would be helpful.
%The results of next 



\begin{table*}[t]
\center
\begin{tabular}{cccm{.45cm}cm{.45cm}cm{.65cm}cm{.65cm}cm{.70cm}cm{.4cm}}
    \toprule
    \multirow{1}{*}{\textbf{Ext.}} &\multirow{1}{*}{\textbf{Emb.}}  & \multicolumn{2}{c}{\textbf{CNN/DM}} & \multicolumn{2}{c}{\textbf{NYT}} & \multicolumn{2}{c}{\textbf{DUC}} & \multicolumn{2}{c}{\textbf{Reddit}} & \multicolumn{2}{c}{\textbf{AMI}} & \multicolumn{2}{c}{\textbf{PubMed}}\\
   %  &  & R-2 & R-2 & R-2 & R-2 & R-2 & R-2\\
   % \hline
    \midrule
    \multirow{2}{*}{RNN} & Fixed & \textbf{25.4} & & \textbf{34.7} & & \textbf{22.7} & & \textbf{11.4} & & \textbf{ 5.5} & & \textbf{17.0}\\
                         & F.-T. & 25.2& \scriptsize{(0.2)} & 34.3 &\scriptsize{(0.4)} & \textbf{22.6} & \scriptsize{(0.1)} & \textbf{11.3} & \scriptsize{(0.1)} & \textbf{ 5.3} & \scriptsize{(0.2)} & 16.4& \scriptsize{(0.6)}\\
    \hline
    \multirow{2}{*}{Seq2Seq} & Fixed & \textbf{25.6} && \textbf{35.7}& & \textbf{22.8}& & \textbf{13.6} &&  5.5 && \textbf{17.7}\\
   & F.-T. & 25.3 &\scriptsize{(0.3)} & \textbf{35.7}& \scriptsize{(0.0)} & \textbf{22.9} & \scriptsize{(-0.1)} & \textbf{13.8} &\scriptsize{ (-0.2)} & \textbf{ 5.8} & \scriptsize{(-0.3)} & 16.9 & \scriptsize{(0.8)}\\
    \hline
    \multirow{2}{*}{C\&L} & Fixed & \textbf{25.3} && \textbf{35.6} && \textbf{23.1} && \textbf{13.6} && \textbf{ 6.1} && \textbf{17.7}&\\
                      & F.-T. & 24.9 &\scriptsize{(0.4)} & 35.4 & \scriptsize{(0.2)} & \textbf{23.0} &\scriptsize{ (0.1)} & \textbf{13.4} &\scriptsize{ (0.2)} & \textbf{ 6.2} &\scriptsize{ (-0.1)} & 16.4 &\scriptsize{ (1.3)} \\
    \hline
\multirow{2}{*}{\begin{tabular}{c} Summa \\ Runner \end{tabular}} & Fixed & \textbf{25.4} && \textbf{35.4} && \textbf{22.3} && \textbf{13.4} && \textbf{ 5.6} & &\textbf{17.2}&\\
                                                                  & F.-T. & 25.1 &\scriptsize{(0.3)} & 35.2 &\scriptsize{(0.2)} & \textbf{22.2} & \scriptsize{(0.1)} & 12.6 & \scriptsize{(0.8)} & \textbf{ 5.8} & \scriptsize{(-0.2)} & 16.8&\scriptsize{ (0.4) }\\
    \bottomrule
\end{tabular}

\caption{ROUGE-2 recall across sentence extractors
    when using fixed pretrained embeddings or when embeddings are fine-tuned (F.-T.) during training. In both cases embeddings
    are initialized with pretrained GloVe embeddings. All extractors use the averaging 
sentence encoder. When both fine-tuned and fixed settings are bolded,
there is no signifcant performance difference. Difference in scores shown in parenthesis.}
\label{tab:embeddings}
\end{table*}

%\begin{table*}
%\center
%\begin{tabular}{| c | c || c | c | c | c | c | c | c | c |}
%\hline
%  &   & \multicolumn{2}{|c|}{cnn-dailymail} & \multicolumn{2}{|c|}{nyt} & \multicolumn{2}{|c|}{duc-sds} & \multicolumn{2}{|c|}{reddit} \\
%system & embeddings & R1 & R2  & R1 & R2  & R1 & R2  & R1 & R2  \\
%\hline
%\multirow{2}{*}{RNN} & fixed & 55.3 & 25.4 & 51.4 & 34.7 & 44.1 & 22.6 & 45.2 & 11.4\\ \cline{2-10}
% & learned & 55.1 & 25.2 & 51.1 & 34.3 & 44.1 & 22.6 & 45.3 & 11.3\\
%\hline
%\multirow{2}{*}{Seq2Seq} & fixed & 55.6 & 25.6 & 52.5 & 35.7 & 44.4 & 22.8 & 49.1 & 13.6\\ \cline{2-10}
% & learned & 55.2 & 25.3 & 52.4 & 35.7 & 44.5 & 22.9 & 49.4 & 13.8\\
%\hline
%\multirow{2}{*}{C\&L} & fixed & 55.1 & 25.3 & 52.3 & 35.6 & 44.8 & 23.1 & 48.3 & 13.6\\ \cline{2-10}
% & learned & 54.8 & 25.0 & 52.1 & 35.4 & 44.6 & 23.0 & 48.6 & 13.5\\
%\hline
%\multirow{2}{*}{SummaRunner} & fixed & 55.3 & 25.4 & 52.1 & 35.4 & 44.0 & 22.3 & 48.8 & 13.4\\ \cline{2-10}
% & learned & 55.0 & 25.1 & 52.0 & 35.2 & 43.8 & 22.1 & 47.8 & 12.6\\
%\hline
%\end{tabular}
%\caption{ROUGE 1 and 2 recall results across different sentence extractors
%    when using learned or pretrained embeddings. In both cases embeddings
%    are initialized with pretrained GloVe embeddings. All results are 
%averaged from five random initializations. All extractors use the averaging 
%sentence encoder.}
%\label{tab:embeddings}
%\end{table*}



\subsection{Ablation Experiments}

In addition to our main evaluation above, we also perform several ablation 
experiments further understand how the various summarization models perform
when certain information is witheld from the model. In particular, we 
evaluate the effect of fine-tuning word embeddings, part-of-speech (POS) based ablations, and sentence-order shuffling.

\paragraph{Word Embedding Fine-tuning}
 Given that learning a sentence encoder (averaging has no learned parameters)
 does not yield significant improvement, it is natural to consider whether
 fine-tuning word embeddings is also necessary. 
 In \autoref{tab:embeddings} we compare the performance of different extractors
 using the averaging encoder, when the word embeddings are held fixed or 
 fine-tuned during training. In both cases, word embeddings are initialized with
 GloVe embeddings trained on a combination of Gigaword and Wikipedia.
% \hal{TRAINED ON WHAT? JUST THE DEFAULT ONES?}
 When fine-tuning embeddings, words occurring 
 fewer than three times in the training data are mapped to an unknown
 token (with learned embedding).
 
% shows ROUGE recall
%when using fixed or updated word embeddings. 
 In all but one case,
fixed embeddings are as good or better than the fine-tuned embeddings.
This is a somewhat surprising finding on the CNN/DM data since it is reasonably
large, and learning embeddings should give the models more
flexibility to identify important word features.\footnote{The AMI corpus is an exception here where learning \emph{does} lead to small
performance boosts, however, only in the Seq2Seq extractor is this diference 
significant; it is quite possible that this is an artifact of the very small
test set size.}
%\hal{why is it surprising?} \textcolor{green}{[[CK: Because learning embeddings should give the model more capacity to represent important patterns]]}
This suggests that we cannot extract much generalizable learning signal 
from the content other than what is already present from initialization. 
Even on PubMed, where the language is quite different from the news/Wikipedia
articles the GloVe embeddings were trained on, fine-tuning leads to 
significantly worse results.

%The language of this corpus is quite different from the 
%data that the GloVe embeddings were trained on and so it makes sense 
%that  there would be more benefit to learning word representations; one
%explanation for only seeing modest improvements is purely the small size
%of the test dataset which has only 20 training meetings.

%textcolor{red}{(NOTE TO CK -- expect learning to help on pubmed)}. \hal{yes, the dataset size is certainly an issue here. probably worth pointing this out. also when you learned the embeddings, did you initialize to pretrained embeddings? did you regularize toward them?}

\begin{table*}[ht]
\center
\begin{tabular}{ccccccc}
    \toprule
    \multirow{1}{*}{\textbf{Ablation}}  & \multicolumn{1}{c}{\textbf{CNN/DM}} & \multicolumn{1}{c}{\textbf{NYT}} & \multicolumn{1}{c}{\textbf{DUC}} & \multicolumn{1}{c}{\textbf{Reddit}} & \multicolumn{1}{c}{\textbf{AMI}} & \multicolumn{1}{c}{\textbf{PubMed}}\\
    \hline
    all words & \textbf{25.4}\textsuperscript{~} ~~~~~~~ & \textbf{34.7}\textsuperscript{~} ~~~~~~~~& 22.7\textsuperscript{~} ~~~~~~~~& \textbf{11.4}\textsuperscript{~} ~~~~~~~~& 5.5\textsuperscript{~} ~~~~~~~~~& \textbf{17.0}\textsuperscript{~} ~~~~~~~ \\
    -nouns & 25.3\textsuperscript{$\dagger$} \footnotesize{(0.1)}& 34.3\textsuperscript{$\dagger$} \footnotesize{(0.4)}& 22.3\textsuperscript{$\dagger$} ~\footnotesize{(0.4)}& 10.3\textsuperscript{$\dagger$} \footnotesize{(1.1)} & 3.8\textsuperscript{$\dagger$} \footnotesize{(1.7)}& 15.7\textsuperscript{$\dagger$} \footnotesize{(1.3)}\\
    -verbs & 25.3\textsuperscript{$\dagger$} \footnotesize{(0.1)}& 34.4\textsuperscript{$\dagger$} \footnotesize{(0.3)} & 22.4\textsuperscript{$\dagger$} ~\footnotesize{(0.3)}& 10.8\textsuperscript{~} ~\footnotesize{(0.6)} & 5.8\textsuperscript{~} \footnotesize{(-0.3)} & 16.6\textsuperscript{$\dagger$} \footnotesize{(0.4)}\\
    -adj/adv & 25.3\textsuperscript{$\dagger$} \footnotesize{(0.1)}& 34.4\textsuperscript{$\dagger$} \footnotesize{(0.3)} & 22.5\textsuperscript{~} ~\footnotesize{(0.2)} & ~~9.5\textsuperscript{$\dagger$} \footnotesize{(1.9)} & 5.4\textsuperscript{~} ~\footnotesize{(0.1)} & 16.8\textsuperscript{$\dagger$} \footnotesize{(0.2)}\\
    -function & 25.2\textsuperscript{$\dagger$} \footnotesize{(0.2)} & 34.5\textsuperscript{$\dagger$} \footnotesize{(0.2)} & \textbf{22.9}\textsuperscript{$\dagger$} \footnotesize{(-0.2)} & 10.3\textsuperscript{$\dagger$} \footnotesize{(1.1)}& \textbf{6.3}\textsuperscript{$\dagger$} \footnotesize{(-0.8)}& 16.6\textsuperscript{$\dagger$} \footnotesize{(0.4)}\\
    \bottomrule
\end{tabular}

\caption{ROUGE-2 recall after removing nouns, verbs, adjectives/adverbs, and 
    function words. Ablations are
    performed using the averaging sentence encoder and the RNN
extractor. 
Bold indicates best performing system. $\dagger$ indicates significant 
difference with the non-ablated system. Difference in score from \textit{all words} shown in parenthesis.}
\label{tab:ablations}
\end{table*}


\paragraph{POS Tag Ablation}
It is also not well explored what word features are being used by the encoders.
To understand which classes of words were most important we ran an ablation
study, selectively removing nouns, verbs 
(including participles and auxiliaries), adjectives \& adverbs, and 
function words (adpositions, determiners, conjunctions).
%Additionally, we ran ablation experiments
%using part-of-speech (POS) tags. \hal{this needs to be justified. why is this experiment interesting?}
All datasets were automatically tagged using
the spaCy POS
tagger\footnote{https://github.com/explosion/spaCy}.   
%\kathy{I'm still curious what would happen if you separately removed all conjunction tags and later remaining POS.}
%We experimented with selectively removing 
%\begin{itemize}
%    \item nouns (NOUN and PROPN tags), 
%    \item verbs (VERB, PART, and AUX tags), 
%    \item adjectives/adverbs (ADJ and ADV tags), 
%    \item numerical expressions (NUM and SYM tags), and 
%    \item miscellaneous words (ADP, CONJ, CCONJ, DET, INTJ, and SCONJ tags)
%\end{itemize}
%from each sentece. 
The embeddings of removed words were replaced with a zero vector,
preserving the order and position of the non-ablated words in the sentence.
Ablations were performed on training, validation, and test partitions,
using the RNN extractor with averaging encoder.
\autoref{tab:ablations} shows the results of the POS
tag ablation experiments. 
While removing any word class from the representation generally hurts 
performance (with statistical significance), on the news domains,
the absolute values of the differences are quite small 
(.18 on CNN/DM, .41 on NYT, .3 on DUC) suggesting that the model's predictions
are not overly dependent on any particular word types.
On the non-news datasets, the ablations have a larger effect 
(max differences are 1.89 on Reddit, 2.56 on AMI, and 1.3 on PubMed).
Removing nouns leads to the largest drop on AMI and PubMed.
Removing adjectives and adverbs leads to the largest drop on Reddit,
suggesting the intensifiers and descriptive words are useful for 
identifying important content in personal narratives.
Curiously, 
removing the function word POS class yields a significant improvement
on DUC 2002 and AMI.


%The newswire domain does not appear to be sensative
%to these ablations; this suggests that the models are still able to identify
%the lead section of the document with the remaining word classes \textcolor{red}{(Verify this with histogram analysis)}. 
%The Reddit domain, which is not lead biased, is significantly effected.
%Notably, removing adjectives and adverbs results in a 1.8 point drop 
%in ROUGE-2 recall. 

\begin{table*}[t]
\center

\begin{tabular}{cccm{.5cm}cm{.5cm}cm{.5cm}cm{.75cm}cm{.75cm}cm{.5cm}}
    \toprule
    \textbf{Ext.} &\textbf{Order}  & \multicolumn{2}{c}{\textbf{CNN/DM}} & \multicolumn{2}{c}{\textbf{NYT}} & \multicolumn{2}{c}{\textbf{DUC}} & \multicolumn{2}{c}{\textbf{Reddit}} & \multicolumn{2}{c}{\textbf{AMI}} & \multicolumn{2}{c}{\textbf{PubMed}}\\
    \midrule
    \multirow{2}{*}{Seq2Seq} & In-Order & \textbf{25.6} & & \textbf{35.7} && \textbf{22.8}& & \textbf{13.6} && 5.5 && \textbf{17.7} &\\
                             & Shuffled & 21.7&\scriptsize{(3.9)} & 25.6 & \scriptsize{(10.1)} & 21.2 & \scriptsize{(1.6)} &\textbf{13.5} &\scriptsize{(0.1)} &\textbf{6.0} & \scriptsize{(-0.5)}&14.9 &\scriptsize{(2.8)}\\
    \bottomrule
\end{tabular}

\caption{ROUGE-2 recall using models trained on in-order and shuffled
documents. Extractor uses the averaging sentence encoder. 
When both in-order and shuffled settings are bolded,
there is no signifcant performance difference. Difference in scores shown in parenthesis.
}
\label{tab:shuffle}
\end{table*}


\textbf{Sentence Order Shuffling} Sentence position is a well known and 
powerful feature for news summarization \cite{hong2014improving}, owing 
to the intentional lead bias in the news article writing\footnote{\url{https://en.wikipedia.org/wiki/Inverted_pyramid_(journalism)}}; it also explains the difficulty in beating
the lead baseline for single-document summarization 
\cite{nenkova2005automatic,rau:1999}.
In examining the generated summaries, we found
most of the selected sentences in the news domain came from the lead paragraph
%\hal{i feel like there must be citations to dig up here from like the 90s about lead summarization in news... it's also an intentional bias: maybe the right thing is to cite a style guide from a newsppaer that says to write this way}
of the document. This is despite the fact that there is a long tail of 
sentence extractions from later in the document in the ground truth extract 
summaries (31\%, 28.3\%, and 11.4\% of DUC, CNN/DM, and NYT training extract labels come 
from the second half of the document). 
%\hal{can you be more specific? like give some stats? what \%age come from first quarter of doc and what \%age from last half or something}. 
Because this lead bias is so strong, it is questionable whether
the models are learning to identify important content or just find the start
of the document. We conduct a sentence order experiment where 
each document's sentences are randomly shuffled during training. We then
%KM - I think below should be shuffled. I changed.
%CK - models are trained on shuffled data but evaluated on in order models.
%evaluate each model performance on the unshuffled test data, comparing to 
evaluate each model performance on the unshuffled test data, comparing to 
the model trained on unshuffled data; if the models trained on shuffled data
drop in performance, then this indicates the lead bias is the relevant factor.
%in learning content selection.

\autoref{tab:shuffle} shows the results
of the shuffling experiments. 
The news domains and PubMed suffer a significant drop in performance 
when the document order is shuffled. By comparison, there is no significant difference between the shuffled and in-order models on 
the Reddit domain, and shuffling actually improves performance on AMI.
%\hal{what about in the cross-domain setting?} 
This suggest that position 
is being learned by the models in the news/journal article domain even when 
the model has no explicit position features, and that this feature is more 
important than either content or function words.





%\input{tables/table_cross_domain.tex}

%%% Local Variables:
%%% mode: latex
%%% TeX-master: "dlextsum.emnlp18"
%%% End:
