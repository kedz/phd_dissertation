\chapter{GRU-based Sequence-to-Sequence Architecture}
\label{sec:nlggru}

\begin{figure}[t]

    \resizebox{\textwidth}{!}{
\begin{tikzpicture}[
    dhid/.style={draw,minimum height=2cm,fill=white,text width=5mm,font=\footnotesize,fill=green!20,draw=green,rounded corners},
    emb/.style={draw,minimum height=2cm,fill=white,text width=5mm,font=\small,rounded corners},
    emb.enc/.style={emb,fill=red!20,draw=red},
    emb.dech/.style={emb,fill=blue!20,draw=blue},
    emb.dec/.style={emb,fill=orange!20,draw=orange},
    emb.bridge/.style={emb,fill=violet!20,draw=violet},
    emb.attn/.style={emb,fill=cyan!20,draw=cyan},
    emb.pred/.style={emb,fill=purple!20,draw=purple},
    con/.style={line width=0.5mm}]

\node (x1) at (0,0) {$\mrtok_1$};
\node (x2) at (2,0) {$\mrtok_2$};
\node (x3) at (4,0) {$\mrtok_3$};
\node (x4) at (6,0) {$\mrtok_4$};

\node[emb.enc] (m1) at ($(x1)+(0,2)$) {$\encEmbs_{\mrtok_1}$};
\node[emb.enc] (m2) at ($(x2)+(0,2)$) {$\encEmbs_{\mrtok_2}$};
\node[emb.enc] (m3) at ($(x3)+(0,2)$) {$\encEmbs_{\mrtok_3}$};
\node[emb.enc] (m4) at ($(x4)+(0,2)$) {$\encEmbs_{\mrtok_4}$};

\draw[con,->] (x1.north) -- (m1.south);
\draw[con,->] (x2.north) -- (m2.south);
\draw[con,->] (x3.north) -- (m3.south);
\draw[con,->] (x4.north) -- (m4.south);

\node[draw=red,dotted,fill=red!20,text width=2cm,align=left] at ($(m1)+(-2.25,0)$) {Encoder\\ Embedding\\ Layer};

\node[draw=green,dotted,fill=green!20,text width=2cm,align=left] at ($(m1)+(-2.25,6.75)$) {Encoder\\ GRU\\ Layer 2};

\node[draw=green,dotted,fill=green!20,text width=2cm,align=left] at ($(m1)+(-2.25,3)$) {Encoder\\ GRU\\ Layer 1};

\node[emb.bridge] (g10) at ($(m4)+(2,3)$) {$\decHidState^{(1)}_0$};
\node[emb.bridge] (g20) at ($(m4)+(2,6.75)$) {$\decHidState^{(2)}_0$};

\node[draw=violet,dotted,fill=violet!20,text width=1.5cm,align=left] at (8,6.9) {Bridge\\ Layers};


\node (y1) at (10,0) {$\utttok_1$};
\node (y2) at (12,0) {$\utttok_2$};
\node (y3) at (14,0) {$\utttok_3$};

\node[emb.dec] (w1) at ($(y1)+(0,2)$) {$\decWordEmb_1$};
\node[emb.dec] (w2) at ($(y2)+(0,2)$) {$\decWordEmb_2$};
\node[emb.dec] (w3) at ($(y3)+(0,2)$) {$\decWordEmb_3$};

\node[draw=orange,dotted,fill=orange!20,text width=2cm,align=left] at ($(w3)+(2.25,0)$) {Decoder\\ Embedding\\ Layer};

\draw[con,->] (y1.north) -- (w1.south);
\draw[con,->] (y2.north) -- (w2.south);
\draw[con,->] (y3.north) -- (w3.south);

\node[emb.dech] (g11) at ($(w1)+(0,3)$) {$\decHidState^{(1)}_1$};
\node[emb.dech] (g12) at ($(w2)+(0,3)$) {$\decHidState^{(1)}_2$};
\node[emb.dech] (g13) at ($(w3)+(0,3)$) {$\decHidState^{(1)}_3$};

\node[emb.dech] (g21) at ($(w1)+(0,6.75)$) {$\decHidState^{(2)}_1$};
\node[emb.dech] (g22) at ($(w2)+(0,6.75)$) {$\decHidState^{(2)}_2$};
\node[emb.dech] (g23) at ($(w3)+(0,6.75)$) {$\decHidState^{(2)}_3$};


\node[dhid] (rh11) at ($(m1)+(-0.5,3+.25)$) {$\encBwdHidState^{(1)}_1$};
\node[dhid] (rh12) at ($(m2)+(-0.5,3+.25)$) {$\encBwdHidState^{(1)}_2$};
\node[dhid] (rh13) at ($(m3)+(-0.5,3+.25)$) {$\encBwdHidState^{(1)}_3$};
\node[dhid] (rh14) at ($(m4)+(-0.5,3+.25)$) {$\encBwdHidState^{(1)}_4$};
\draw[con,->] (rh14.west) -- (rh13.east);
\draw[con,->] (rh13.west) -- (rh12.east);
\draw[con,->] (rh12.west) -- (rh11.east);
\draw[con,->] (rh14.east) -- (g10.west);




\node[dhid] (fh11) at ($(m1)+(0.5,3-.25)$) {$\encFwdHidState^{(1)}_1$};
\node[dhid] (fh12) at ($(m2)+(0.5,3-.25)$) {$\encFwdHidState^{(1)}_2$};
\node[dhid] (fh13) at ($(m3)+(0.5,3-.25)$) {$\encFwdHidState^{(1)}_3$};
\node[dhid] (fh14) at ($(m4)+(0.5,3-.25)$) {$\encFwdHidState^{(1)}_4$};


\draw[con,->] (fh11.east) -- (fh12.west);
\draw[con,->] (fh12.east) -- (fh13.west);
\draw[con,->] (fh13.east) -- (fh14.west);
\draw[con,->] (fh14.east) -- (g10.west);

\node[dhid] (rh21) at ($(m1)+(-0.5,2.25*3+.25)$) {$\encBwdHidState^{(2)}_1$};
\node[dhid] (rh22) at ($(m2)+(-0.5,2.25*3+.25)$) {$\encBwdHidState^{(2)}_2$};
\node[dhid] (rh23) at ($(m3)+(-0.5,2.25*3+.25)$) {$\encBwdHidState^{(2)}_3$};
\node[dhid] (rh24) at ($(m4)+(-0.5,2.25*3+.25)$) {$\encBwdHidState^{(2)}_4$};
\draw[con,->] (rh24.west) -- (rh23.east);
\draw[con,->] (rh23.west) -- (rh22.east);
\draw[con,->] (rh22.west) -- (rh21.east);
\draw[con,->] (rh24.east) -- (g20.west);




\node[dhid] (fh21) at ($(m1)+(0.5,2.25*3-.25)$) {$\encFwdHidState^{(2)}_1$};
\node[dhid] (fh22) at ($(m2)+(0.5,2.25*3-.25)$) {$\encFwdHidState^{(2)}_2$};
\node[dhid] (fh23) at ($(m3)+(0.5,2.25*3-.25)$) {$\encFwdHidState^{(2)}_3$};
\node[dhid] (fh24) at ($(m4)+(0.5,2.25*3-.25)$) {$\encFwdHidState^{(2)}_4$};


\draw[con,->] (fh21.east) -- (fh22.west);
\draw[con,->] (fh22.east) -- (fh23.west);
\draw[con,->] (fh23.east) -- (fh24.west);
\draw[con,->] (fh24.east) -- (g20.west);



\draw[con,->] (m1.north) -- (fh11.south);
\draw[con,->] (m2.north) -- (fh12.south);
\draw[con,->] (m3.north) -- (fh13.south);
\draw[con,->] (m4.north) -- (fh14.south);
\draw[con,->] (m1.north) -- (rh11.south);
\draw[con,->] (m2.north) -- (rh12.south);
\draw[con,->] (m3.north) -- (rh13.south);
\draw[con,->] (m4.north) -- (rh14.south);

\foreach \i in {1,...,4} {
    \draw[con,->] (fh1\i.north) -- ($(fh1\i)!0.5!(rh1\i)+(0,1.75)$) -- 
        ($(fh1\i)!0.5!(rh1\i)+(0,2.05)$) -- (fh2\i.south);

    \draw[con,->] (rh1\i.north) -- ($(fh1\i)!0.5!(rh1\i)+(0,1.75)$) -- 
        ($(fh1\i)!0.5!(rh1\i)+(0,2.05)$) -- (rh2\i.south);
}


\draw[con,->] (g10) -- (g11);
\draw[con,->] (g20) -- (g21);
\draw[con,->] (w1) -- (g11);
\draw[con,->] (w2) -- (g12);
\draw[con,->] (w3) -- (g13);
\draw[con,->] (g11) -- (g21);
\draw[con,->] (g12) -- (g22);
\draw[con,->] (g13) -- (g23);
\draw[con,->] (g11) -- (g12);
\draw[con,->] (g12) -- (g13);
\draw[con,->] (g21) -- (g22);
\draw[con,->] (g22) -- (g23);

\node[emb.attn] (attn) at ($(g22)+(0,3)$) {$\astate_3$};


\draw[con,->] (rh21.north) -- ($(rh21.north)!0.5!(fh21.north)+(0,2.0)$) 
    -- (attn.west);
\draw[con,->] (fh21.north) -- ($(rh21.north)!0.5!(fh21.north)+(0,2.0)$) 
    -- (attn.west);

\draw[con,->] (rh22.north) -- ($(rh22.north)!0.5!(fh22.north)+(0,1.5)$) 
    -- (attn.west);
\draw[con,->] (fh22.north) -- ($(rh22.north)!0.5!(fh22.north)+(0,1.5)$) 
    -- (attn.west);

\draw[con,->] (rh23.north) -- ($(rh23.north)!0.5!(fh23.north)+(0,1.0)$) 
    -- (attn.west);
\draw[con,->] (fh23.north) -- ($(rh23.north)!0.5!(fh23.north)+(0,1.0)$) 
    -- (attn.west);

\draw[con,->] (rh24.north) -- ($(rh24.north)!0.5!(fh24.north)+(0,0.5)$) 
    -- (attn.west);
\draw[con,->] (fh24.north) -- ($(rh24.north)!0.5!(fh24.north)+(0,0.5)$) 
    -- (attn.west);

\draw[con,->] (g23.north) -- (attn.south);

\node[emb.pred] (z) at ($(g23)+(0,3)$) {\phantom{$z_3$}};

\node (p) at ($(z)+(0,3)$) {$\gen(\utttok_4|\utttoks_{1:3},\mrtoks)$};
\draw[con,->] (z.north) -- (p.south);
\draw[con,->] (g23.north) -- (z.south);
\draw[con,->] (attn.east) -- (z.west);

\node[draw=blue,dotted,fill=blue!20,text width=2cm,align=left] at ($(w3)+(2.25,3)$) {Decoder\\ GRU\\ Layer 1};
\node[draw=blue,dotted,fill=blue!20,text width=2cm,align=left] at ($(w3)+(2.25,6.75)$) {Decoder\\ GRU\\ Layer 2};


\node[draw=cyan,dotted,fill=cyan!20,text width=2cm,align=left] at ($(attn)+(-2,1)$) {Attention\\ Layer};
\node[draw=purple,dotted,fill=purple!20,text width=2cm,align=left] at ($(z)+(2.25,1)$) {Prediction\\ Layer};
\end{tikzpicture}
}
\caption{Schematic of the bi-directional GRU-based \sequencetosequence~model.}
\label{fig:s2s}
\end{figure}


The GRU is a form of reccurent neural network \citep{elman1990} that operates
over discrete sequences, which upon receiving a new input token, updates a
``hidden state'' or internal representation using the current input and the
previous hidden state. In the \sequencetosequence~paradigm, both the encoder
and decoder are built upon distinct GRU layers. 

The encoder consists of an embedding layer which maps the discrete input
sequence to a sequence embeddings. The encoder input embedding sequence is
then fed through one or more GRU layers. Optionally, the encoder GRUs can be
run uni-directionally (i.e., proceeding left-to-right), or bi-directionally
(i.e. distinct left-to-right and right-to-left GRUs process the input sequence
and concatenate the output). We describe both cases below. After encoding the
input, the final state of the encoder is optionally run through a bridge layer
to project it to a compatible size for the decoder. 

The decoder also has an embedding layer which it uses to map previously
generated output tokens to embeddings which are then fed into the one or more
uni-directional decoder GRUs. The decoder hidden state at each step attends to
the encoder hidden states, producing an ``attention vector,'' i.e. a weighted
sum of the encoder hidden states. The decoder state and the attention vector
are concatenated and fed through a \feedforward~layer with softmax output to
produce a probability distribution over the next token. 

See \autoref{fig:s2s} for a schematic example of the GRU-based
\sequencetosequence~model. We now describe the individual components in
detail.

%\paragraph{\redbox{Encoder Embedding Layer}}
\paragraph{{Encoder Embedding Layer}}
Let $\mrtoks = \left[ \mrtok_1,\ldots,\mrtok_\mrSize\right]$ be a linearized
\meaningrepresentation~token sequence. Before feeding $\mrtoks$ into the
encoder GRU layer, we first embed each token. Let $\encEmbs \in
\reals^{\setsize{\mrvocab} \times \embDim}$ be the encoder input embedding
matrix, where each row, $\encWordEmb_i$, is a $\embDim$-dimensional embedding
for a token in $\utttok \in \mrvocab$, i.e.,
\[ 
    \encEmbs = \left[ \begin{array}{c} \encWordEmb_1\\ \vdots \\ \encWordEmb_\setsize{\mrvocab} \end{array}\right]. 
\]
We assume each element $\mrtok \in \mrvocab$ is uniquely identified with a row
$i \in \left\{1, \ldots, \setsize{\mrvocab}\right\}$. We indicate the
embedding of $\mrtok$ as $\encEmbs_\mrtok = \encWordEmb_i$.  The input to the
encoder GRU layer then is 
\[
\left[\encHidState^{(0)}_1,\ldots,\encHidState^{(0)}_\mrSize\right] = 
    \left[\encEmbs_{\mrtok_1},\ldots,   \encEmbs_{\mrtok_\mrSize}\right]. 
\]


%\paragraph{\greenbox{Encoder Uni-directional GRU Layers}}
\paragraph{{Encoder Uni-directional GRU Layers}}
We then compute the GRU hidden states. The encoder can have an arbitray number of layers $\numLayer$.
For each layer $l \in \{1,\ldots, \numLayer\}$ we compute,
\begin{align*}
    \encHidState_0^{(l)} & = \zeroEmb \\
    \encHidState_i^{(l)} & =\fgru(\encHidState_i^{(l - 1)}, \encHidState_{i-1}^{(l)}; \gruEncParams^{(l)}) & \forall i : i \in \{1,\ldots, \mrSize\}
\end{align*}
where $\gruEncParams^{(l)}$ are the GRU encoder parameters\footnote{See \autoref{eqn:gru} for the definition of the GRU function.} for the $l$-th layer and $\encHidState^{(l)}_i \in \reals^{\hidDim}$. The encoder GRU layers output the 
sequence of hidden states, $\encHidState_1,\ldots,\encHidState_\mrSize$, used by the decoder to represent the input; 
in the uni-directional case, these are simply the last GRU layer outputs, i.e. $\encHidState_i = \encHidState_i^{(\numLayer)} \in \reals^{\encDim}$ (in the uni-directional case, $\hidDim = \encDim$).

%\paragraph{\greenbox{Encoder Bi-directional GRU Layers}}
\paragraph{{Encoder Bi-directional GRU Layers}}
The uni-directional encoder may suffer from a recency bias when creating the initial state for the 
decoder and for longer input sequences the encoder may ``forget'' information encoded in the 
early hidden states. In practice to alleviate this another GRU is run in the opposite direction and
its outputs are concatenated. For the first layer, we have,
\begin{align*}
    \textit{(Encoder Forward GRU)}\\
    \encFwdHidState_0^{(1)} & = \zeroEmb,\\ 
    \encFwdHidState_i^{(1)} & =\fgru\left(\encHidState_i^{(0)}, \encFwdHidState_{i-1}^{(1)}; \gruEncFwdParams^{(l)}\right), & \forall i : i \in \{1,\ldots, \mrSize\}\\
    \textit{(Encoder Backward GRU)}\\
     \encBwdHidState_0^{(1)} &= \zeroEmb, \\
    \encBwdHidState_i^{(1)} & =\fgru\left(\encHidState_i^{(0)}, \encBwdHidState_{i+1}^{(1)}; \gruEncBwdParams^{(l)}\right)    & \forall i : i \in \{1,\ldots, \mrSize\}\\
  \encHidState^{(1)}_i & = \left[\begin{array}{c} \encFwdHidState_i^{(1)} \\ \encBwdHidState_i^{(1)} \end{array}\right]
& \forall i : i \in \{1,\ldots, \mrSize\}\\
\end{align*}
where $\gruEncFwdParams$ and $\gruEncBwdParams$ are forward and backward GRU parameters respectively,
$\encFwdHidState^{(1)}_i,\encBwdHidState^{(1)}_i\in\reals^{\hidDim}$, and first layer hidden state,
$\encHidState^{(1)}_i\in\reals^{2\hidDim}$, is a concatenation of the forward and backward hidden
states at step $i$. 
The subsequent layers are computed similarly, but the input to the GRUs are $2\hidDim$-dimensional.
Like before, the encoder outputs are the hidden state outputs of the last layer, $\encHidState_i = \encHidState_i^{(\numLayer)} \in \reals^{\encDim}$ where $\encDim = 2\hidDim$.


%\paragraph{\orangebox{Decoder Embedding Layer}}
\paragraph{{Decoder Embedding Layer}}
We then embed the utterance token sequence, 
$\utttoks = \left[ \utttok_1,\ldots,\utttok_\uttSize\right]$,
before feeding it to the decoder GRU layers.. 
Let $\decEmbs \in \reals^{\setsize{\uttvocab} \times \embDim}$ be an
embedding matrix of the utterance tokens $\utttok \in \uttvocab$ defined
analogously to the encoder embedding matrix $\encEmbs$. 
%where each row is an embedding for a token in $\uttvocab$ (analogous to $\encEmbs$),
%\[ \decEmbs = \left[ \begin{array}{c} \decWordEmb_1\\ \vdots \\ \decWordEmb_\setsize{\uttvocab} \end{array}\right]. \]
%We assume each element in $\uttvocab$ is uniquely identified with a row in $\encEmbs$; let $\mrtok\in\mrvocab$ be identified with the $i$-th row, then we indicate it's embedding by $\encEmbs_\mrtok = \encWordEmb_i$.  
The input to the decoder then is 
\[\left[\decHidState^{(0)}_1,\ldots,\decHidState^{(0)}_{\uttSize-1}\right] = \left[\decEmbs_{\utttok_1},\ldots,
\decEmbs_{\utttok_{\uttSize-1}}\right].\footnote{The decoder input sequences have length $\uttSize-1$ since the $\uttSize^\textrm{th}$ token is always the stop token \stoptok, which is never fed into the decoder input. } \]

%\paragraph{\violetbox{Bridge Layer}}
\paragraph{{Bridge Layer}}
We initialize the decoder hidden state with the final (i.e. $\mrSize$-th) state of the encoder GRU.
In the case where $\encDim \ne \hidDim$, we need to project $\encHidState_\mrSize^{(l)}$ to $\hidDim$
dimensions,
\[ \decHidState_0^{(l)} = \begin{cases} \tanh\left(\weight{br}_l \encHidState_\mrSize^{(l)} + \bias{br}_l\right) & \encDim \ne \hidDim  \\
\encHidState_\mrSize^{(l)} & \textrm{otherwise} \end{cases},\]
where $\weight{br}_l \in \reals^{\hidDim \times \encDim}$ and $\bias{br}_l \in \reals^{\hidDim}$
for $l \in \{1,\ldots,\numLayer\}$
are the weight and bias parameters for the ``bridge layer'' between the encoder and decoder networks.

%\paragraph{\bluebox{Decoder GRU Layers}}
\paragraph{{Decoder GRU Layers}}
The decoder GRU is then computed analogously to the uni-directional encoder GRU,
\begin{align*}
    \decHidState_i^{(l)} & =\fgru\left(\decHidState_i^{(l - 1)}, \decHidState_{i-1}^{(l)}; \gruDecParams^{(l)}\right) & \forall i : i \in \{1,\ldots, \uttSize-1\},
\end{align*}
where $\gruDecParams^{(l)}$ are the decoder GRU parameters and
$ \decHidState_i^{(l)} \in \reals^{\hidDim}$ for $i \in \{1,\ldots,\uttSize-1\}$ and $l \in \{1,\ldots,\numLayer\}$.
The decoder outputs, $\decHidState_1,\ldots,\decHidState_{\uttSize-1}$, are the decoder hidden states of the last decoder layer, i.e. $\decHidState_i = \decHidState^{(\numLayer)}_i \in \reals^{\decDim}$ where $\decDim=\hidDim$.

%\paragraph{\cyanbox{Attention Layer}}
\paragraph{{Attention Layer}}

As mentioned before, one drawback of the recurrent neural network design is that information
from earlier states my not be preserved in later states. To ameliorate this, the attention
mechanism was proposed to allow an arbitrary decoder state to retrieve information from an arbitrary
encoder state \citep{bahdanau2015}. This works by taking a weighted average of the encoder states,
\begin{align*}
    \astate_i & = \sum^{\mrSize}_{j=1}\alpha_{i,j}\encHidState_j & \forall i : i \in \{1,\ldots, \uttSize-1\}
\end{align*}
where $\alpha_{i,j} \in (0,1)$ is proportional to a score function $\ascore(\decHidState_i,\encHidState_j)$ which 
measures some notion of ``relevance'' for decoder state $i$ to encoder start $j$,
\[\alpha_{i,j} = \frac{\exp \ascore(\decHidState_i, \encHidState_j) }{ \sum_{j^\prime=1}^\mrSize \exp \ascore(\decHidState_i, \encHidState_{j^\prime})} \quad \forall i,j : j \in \{1,\ldots,\mrSize\}, i \in \{1,\ldots,\uttSize-1\}.\]

There are several popular ways to implement $\ascore$ which we consider; we refer to three of them
using the names given in \citet{luong2015}. When the encoder and decoder
hidden states are of the same dimension, the simplest function is just the dot product, which 
we refer to as ``dot-style'' attention,\\
\textit{(Dot-Style Attention)}
\[ \quad \ascore(\decHidState_i,\encHidState_j) = \decHidState_i \cdot \encHidState_j. \]
If they are not the same dimension, one can insert a parameter matrix in place of the dot product,\\
\textit{(General-Style Attention)}
\[ \quad \ascore(\decHidState_i,\encHidState_j) = \decHidState_i \attnkernel \encHidState_j \]
where $\attnkernel \in \reals^{\decDim \times \encDim}$ is a learned parameter of the model.

The third method called ``concat'' by \citet{luong2015} but also commonly
referred to as ``Bahdanau,'' since it was introduced in the 
\citet{bahdanau2015}, uses a \feedforward~layer to project the pair of states
down to a scalar,\\
\textit{(Concat-Style Attention)}
\[ \ascore(\decHidState_i, \encHidState_j) = \attnff\cdot\tanh\left(\attnkernel\left[ \begin{array}{c} \decHidState_i \\ \encHidState_j \end{array} \right] \right),  \]
    where $\attnkernel \in \reals^{\hidDim \times (\decDim + \encDim)}$
    and $\attnff \in \reals^{\hidDim}$ are learned parameters.



%\paragraph{\purplebox{Prediction Layer}}
\paragraph{{Prediction Layer}}
Finally, the attention output $\astate_i$ and decoder state $\decHidState_i$
are run through a two~layer \feedforward~network to produce a distribution
over the utterance token vocabulary $\uttvocab$,
\[ \gen\left(\utttok_{i+1}|\utttoks_{1:i},\ls(\mr);\params\right) = \softmax\left(   \weight{2}\cdot \tanh\left(\weight{1} \left[ \begin{array}{c}\decHidState_i \\ \encHidState_j \end{array} \right] + \bias{1}\right) + \bias{2} \right)_{\utttok_{i+1}}\quad \forall i: i \in \{1,\ldots,\uttSize-1\} \]
    where $\weight{1} \in \reals^{\hidDim \times \left(\decDim + \encDim\right)}$,
    $\bias{1} \in \reals^{\hidDim}$, $\weight{2} \in \reals^{\setsize{\uttvocab} \times \hidDim}$, and $\bias{2}\in\reals^{\setsize{\uttvocab}}$ are learned
    parameters and we associate each utterance token $\utttok$ with a unique 
    element in the final $\softmax$ distribution (similar to how we indexed
    into the embeddings matrices $\encEmbs$ and $\decEmbs$). 



%    The complete set parameters associated with the GRU architecture 
%    with uni-directional encoder is
%    \[ \params = \left\{ \gruEncParams^{(1)},
%    \ldots, \gruEncParams^{(\numLayer)}, 
%    \weight{1}, \bias{1}, \weight{2}, \bias{2}
%    \right\}  \]
%    while the GRU
%    with bi-directional encoder parameters are
%    \[ \params = \left\{ \gruEncFwdParams^{(1)}, \gruEncBwdParams^{(1)},
%            \weight{br}_1, \bias{br}_1,
%    \ldots, \gruEncFwdParams^{(\numLayer)}, \gruEncBwdParams^{(\numLayer)},
%    \weight{br}_\numLayer, \bias{br}_\numLayer, \weight{1}, \bias{1}, \weight{2}, \bias{2}
%    \right\}  \]
%    TODO add attention params.
% 
