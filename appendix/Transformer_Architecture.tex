
\chapter{Transformer-based Sequence-to-Sequence Architecture}
\label{sec:nlgtf}

\begin{figure}[p]
\resizebox{\textwidth}{!}{
\begin{tikzpicture}[
    emb/.style={draw,minimum height=1.5cm,fill=white,text width=3.5mm,font=\tiny,rounded corners},
    emb.enc/.style={emb,fill=red!20,draw=red},
    emb.dec/.style={emb,fill=orange!20,draw=orange},
    penc/.style={emb,fill=white,draw=black},
    con/.style={line width=0.5mm},
    state/.style={draw,rounded corners,fill=white},
    ln/.style={state},
    selfattn/.style={state},
    encattn/.style={state},
    ff/.style={state},
    enclayer/.style={draw=green,text width=8.6cm,minimum height=6.2cm,
                     fill=green!5},
    declayer/.style={draw=blue,text width=8.8cm,minimum height=9.2cm,
                     fill=blue!5},
    ]

    \def\encbsz{6.25};
    \def\decbsz{9.25};

    \node[state] (H0) at (3,-1) {$\tfEncInput^{(0)}$};
    \foreach \i [count=\j from 1] in {0,...,1} {
        \node[enclayer] (enclayer\j) at (1.01,\i*\encbsz+2.5) {};
            \node[rotate=90,anchor=north,font=\large,align=center] at (enclayer\j.west) {Transformer Encoder\\ Layer \j};

        \node[draw,text width=6.85cm,minimum height=2.9cm,draw=cyan,fill=cyan!15] 
            (HA{\j}Block) at (3-1.25,\i*\encbsz+1) {};
        \node[anchor=north,rotate=90,align=center] 
            at (HA{\j}Block.west) {Self-Attention\\ Block};

        \node[draw,text width=6.85cm,minimum height=2.9cm,draw=purple,fill=purple!5] 
            (HFF{\j}Block) at (3-1.25,\i*\encbsz+4) {};
        \node[anchor=north,rotate=90,align=center] 
            at (HFF{\j}Block.west) {Feed-Forward\\ Block};

        \node[ln] (HLN\j1) at (3,\i*\encbsz+0) {Layer Normalization};
        \node[selfattn] (HA\j) at (3,\i*\encbsz+1) {Self-Attention Layer};
        \node[state] (H{\i}a) at (3,\i*\encbsz+2) 
            {$\boldsymbol{\bar{\tfEncInput}}^{(\j)}$};

        \node[ln] (HLN\j2) at (3,\i*\encbsz+3) {Layer Normalization};
        \node[ff] (HFF\j) at (3,\i*\encbsz+4) {Feed-Forward Layer};
        \node[state] (H\j) at (3,\i*\encbsz+5) 
            {$\tfEncInput^{(\j)}$};

        \draw[con,dashed,->] (H\i) to [out=180,in=270] ($(HLN\j1)!0.5!(HA\j)+(-3,0)$)
            to [out=90,in=180] (H{\i}a);
        \draw[con,dashed,->] (H{\i}a) to [out=180,in=270] ($(HLN\j2)!0.5!(HFF\j)+(-3,0)$)
            to [out=90,in=180] (H\j);
            
        \draw[con,->] (H\i) -- (HLN\j1);
        \draw[con,->] (HLN\j1) -- (HA\j);
        \draw[con,->] (HA\j) -- (H{\i}a);
        \draw[con,->] (H{\i}a) -- (HLN\j2);
        \draw[con,->] (HLN\j2) -- (HFF\j);
        \draw[con,->] (HFF\j) -- (H\j);


    }


    \node[state] (G0) at (10,-1) {$\tfDecInput^{(0)}$};
    \foreach \i [count=\j from 1] in {0,...,1} {
    
        \node[declayer] (declayer\j) at (11.75,\i*\decbsz+4) {};
        \node[rotate=-90,anchor=north,font=\large] at (declayer\j.east) {Transformer Decoder Layer \j};

        \node[draw=lime,fill=lime!15,text width=7.50cm,minimum height=3cm] 
            (GA{\j}Block) at (10+1.25,\i*\decbsz+1) {};
        \node[anchor=north,rotate=-90,align=center] 
            at (GA{\j}Block.east) {Masked \\ Self-Attention\\ Block};
 
        \node[draw=teal,text width=7.50cm,minimum height=3cm,fill=teal!15] 
            (GEA{\j}Block) at (10+1.25,\i*\decbsz+4) {};
        \node[anchor=north,rotate=-90,align=center] 
            at (GEA{\j}Block.east) {Encoder\\Attention\\ Block};
 
        \node[draw=purple,fill=purple!5,text width=7.50cm,minimum height=3cm] 
            (GFF{\j}Block) at (10+1.25,\i*\decbsz+7) {};
        \node[anchor=north,rotate=-90,align=center] 
            at (GFF{\j}Block.east) {Feed-Forward\\ Block};
        
        \node[ln] (GLN\j1) at (10,\i*\decbsz+0) {Layer Normalization};
        \node[selfattn] (GA\j) at (10,\i*\decbsz+1) {Masked Self-Attention Layer};
        \node[state] (G{\i}a) at (10,\i*\decbsz+2) 
            {$\boldsymbol{\bar{\tfDecInput}}^{(\j)}$};

        \node[ln] (GLN\j2) at (10,\i*\decbsz+3) {Layer Normalization};
        \node[encattn] (GEA\j) at (10,\i*\decbsz+4) {Encoder Attention Layer};
        \node[state] (G{\i}b) at (10,\i*\decbsz+5) 
            {$\boldsymbol{\check{\tfDecInput}}^{(\j)}$};

        \node[ln] (GLN\j3) at (10,\i*\decbsz+6) {Layer Normalization};
        \node[ff] (GFF\j) at (10,\i*\decbsz+7) {Feed-Forward Layer};
        \node[state] (G\j) at (10,\i*\decbsz+8) {$\tfDecInput^{(\j)}$};

        \draw[con,->,dashed] (G\i) to [out=0,in=270] ($(GLN\j1)!0.5!(GA\j)+(3,0)$)
            to [out=90,in=0] (G{\i}a);
        \draw[con,->,dashed] (G{\i}a) to [out=0,in=270] ($(GLN\j2)!0.5!(GEA\j)+(3,0)$)
            to [out=90,in=0] (G{\i}b);
        \draw[con,->,dashed] (G{\i}b) to [out=0,in=270] ($(GLN\j3)!0.5!(GFF\j)+(3,0)$)
            to [out=90,in=0] (G\j);
        \draw[con,->] (G\i) -- (GLN\j1);
        \draw[con,->] (GLN\j1) -- (GA\j);
        \draw[con,->] (GA\j) -- (G{\i}a);
        \draw[con,->] (G{\i}a) -- (GLN\j2);
        \draw[con,->] (GLN\j2) -- (GEA\j);
        \draw[con,->] (GEA\j) -- (G{\i}b);
        \draw[con,->] (G{\i}b) -- (GLN\j3);
        \draw[con,->] (GLN\j3) -- (GFF\j);
        \draw[con,->] (GFF\j) -- (G\j);

    }


    \node (x1) at (0,-4.5) {$\mrtok_1$};
    \node (x2) at (2,-4.5) {$\mrtok_2$};
    \node (x3) at (4,-4.5) {$\mrtok_3$};
    \node (x4) at (6,-4.5) {$\mrtok_4$};
    \node (y1) at (8,-4.5) {$\utttok_1$};
    \node (y2) at (10,-4.5) {$\utttok_2$};
    \node (y3) at (12,-4.5) {$\utttok_3$};

    \node[declayer,minimum height=2.5cm,draw=violet,fill=violet!15] (predlayer) at (11.75,1*\decbsz+10) {};
    \node[font=\large] (prob) at ($(G2)+(0,2)$) {$\gen(\utttok_4|\utttoks_{1:3}, \mrtoks)$};

    \node[anchor=north east,font=\large]  at ($(predlayer.north east)$) {Prediction Layer};
    %\node[font=\large]  at ($(G2)+(-4,2)$) {Prediction Layer};



    \draw[con,->] (G2.north) -- (prob.south);

    \node[penc] (pe1) at ($(x1)+(-0.25,2-.2)$) {$\posEmb_1$};
    \node[penc] (pe2) at ($(x2)+(-0.25,2-.2)$) {$\posEmb_2$};
    \node[penc] (pe3) at ($(x3)+(-0.25,2-.2)$) {$\posEmb_3$};
    \node[penc] (pe4) at ($(x4)+(-0.25,2-.2)$) {$\posEmb_4$};
    \node[emb.enc] (m1) at ($(x1)+(0.25,1.75-.2)$) {$\encEmbs_{\mrtok_1}$};
    \node[emb.enc] (m2) at ($(x2)+(0.25,1.75-.2)$) {$\encEmbs_{\mrtok_2}$};
    \node[emb.enc] (m3) at ($(x3)+(0.25,1.75-.2)$) {$\encEmbs_{\mrtok_3}$};
    \node[emb.enc] (m4) at ($(x4)+(0.25,1.75-.2)$) {$\encEmbs_{\mrtok_4}$};

    \draw[con,->] (m1.north) to [out=90,in=225] (H0.south west);
    \draw[con,->] (pe1.north) to [out=90,in=225] (H0.south west);

    \draw[con,->] (m2.north) to [out=90,in=255] ($(H0.south)+(-.15,0)$);
    \draw[con,->] (pe2.north) to [out=90,in=255] ($(H0.south)+(-.15,0)$);

    \draw[con,->] (m3.north) to [out=90,in=285]  ($(H0.south)+(.15,0)$);

    \draw[con,->] (pe3.north) to [out=90,in=285] ($(H0.south)+(.15,0)$);

    \draw[con,->] (m4.north) to [out=90,in=315] (H0.south east);
    \draw[con,->] (pe4.north) to [out=90,in=315] (H0.south east);

    \draw[con,->] (x1) -- (m1.south);
    \draw[con,->] (x2) -- (m2.south);
    \draw[con,->] (x3) -- (m3.south);
    \draw[con,->] (x4) -- (m4.south);

    \node[penc] (pd1) at ($(y1)+(-0.25,2-.2)$) {$\posEmb_1$};
    \node[penc] (pd2) at ($(y2)+(-0.25,2-.2)$) {$\posEmb_2$};
    \node[penc] (pd3) at ($(y3)+(-0.25,2-.2)$) {$\posEmb_3$};

    \node[emb.dec] (w1) at ($(y1)+(0.25,1.75-.2)$) {$\decEmbs_{\utttok_1}$};
    \node[emb.dec] (w2) at ($(y2)+(0.25,1.75-.2)$) {$\decEmbs_{\utttok_2}$};
    \node[emb.dec] (w3) at ($(y3)+(0.25,1.75-.2)$) {$\decEmbs_{\utttok_3}$};
    \draw[con,->] (w1.north) to [out=90,in=225] (G0.south west);
    \draw[con,->] (pd1.north) to [out=90,in=225] (G0.south west);

    \draw[con,->] (w2.north) to [out=45,in=270] (G0.south);
    \draw[con,->] (pd2.north) to [out=45,in=270] (G0.south);

    \draw[con,->] (w3.north) to [out=90,in=315]  (G0.south east);
    \draw[con,->] (pd3.north) to [out=90,in=315] (G0.south east);

    \draw[con,->] (y1) -- (w1.south);
    \draw[con,->] (y2) -- (w2.south);
    \draw[con,->] (y3) -- (w3.south);

    \node (z1) at (H2 |- GEA2) {};

    \node (z2) at ($(z1)!0.75!(GEA2.west)$) {};

    \node (z1s) at ($(z1)+(0,-0.5)$) {};
    \node (z1e) at ($(z1)+(0.5,0.0)$) {};
    
    \node (z2s) at ($(z2)+(0,-0.5)$) {};
    \node (z2w) at ($(z2)+(-0.5,0.0)$) {};
    \node (z3) at (z2 |- GEA1) {};

    \node (z3n) at ($(z3)+(0,0.5)$) {};
    \node (z3e) at ($(z3)+(0.5,0.0)$) {};

    \draw[con,->] (H2.north) -- (z1s.center) to [out=90,in=180] (z1e.center) -- (z2.center) -- (GEA2.west);
    \draw[con,->] (z1e.center) -- (z2w.center) to [out=0,in=90] (z2s.center) -- (z3n.center) to [out=270,in=180] (z3e.center) -- (GEA1.west);

    \node[rotate=90,anchor=north,align=center,font=\large] at ($(enclayer1.west)+(0,-5)$) {Encoder \\ Embedding\\ Layer};

    \node[rotate=-90,anchor=north,align=center,font=\large] at ($(declayer1.east)+(0,-7)$) {Decoder \\ Embedding\\ Layer};
\end{tikzpicture}
}
\caption{A schematic diagram of a two layer transformer-based \sequencetosequence~model. Dashed lines indicate skip-connections.}
\label{fig:tf}
\end{figure}



The transformer \sequencetosequence~model eschews the recurrence as a mechanism
for propagating information, and instead leans solely on several 
attention mechanisms to learn representations of the input sequence $\mrtoks$
as well as the decoder input prefix $\utttoks_{1:i}$. Like the GRU,
each encoder and decoder consist of $\numLayer$ distinct
layers which are applied to $\mrtoks$ and $\utttoks_{1:i}$ respectively.
Ultimately,
the decoder outputs are used to compute the next word probability,
$\gen(\utttok_{i+1}|\utttoks_{1:i},\mrtoks)$. 

A schematic diagram of the transformer-based \sequencetosequence~model
is shown in \autoref{fig:tf}. Like the GRU schematic, the model diagram is 
color-coded to correspond with the text descriptions of each component.
While the transformer looks significantly more complex than the 
GRU architecture, it is fundamentally built around only three 
different kinds of 
neural network layers, \textit{(i)} multi-head attention, 
\textit{(ii)} \feedforward layers, and \textit{(iii)} 
layer normalization. Before describing the encoder and decoder layers
in detail, we first describe these basic components and how they form 
the various ``block'' structures which are employed throughout the model.

\subsubsection{Transformer Components}

\paragraph{Multi-Head Attention} The first basic layer to be defined is the 
multi-head attention layer. We begin by describing ``single-head'' attention
from the point of view of a soft key-value store and then generalize to 
the ``multi-head'' case. In this view, we assume we want to attend to 
a sequence of $\mrSize$ items. Each item has two representations,
a key representation and a value representation, which are written collectively
as rows in a key and value matrix, $\Key \in \reals^{\mrSize \times \embDim}$
and $\Value \in \reals^{\mrSize \times \embDim}$ respectively.
We then  have $\uttSize$ query items that will each individually attend to the 
$\mrSize$ items; we similarly represent the query items as rows in a matrix
$\Query \in \reals^{\uttSize \times \embDim}$. An attention
layer, denoted $\Attn : \reals^{\uttSize \times \embDim} \times \reals^{\mrSize \times \embDim} \times \reals^{\mrSize \times \embDim} \rightarrow \reals^{\uttSize \times \embDim}$, then computes an attention weighted read of the value
matrix as,
  \[\Attn\left(\Query, \Key, \Value\right) = \softmax\left(\frac{\Query \Key^T}{\sqrt{\embDim}} \right)\Value, \]
  where \[\softmax\left(\frac{\Query \Key^T}{\sqrt{\embDim}} \right)_{i,j} = \frac{\exp\left(\embDim^{-\frac{1}{2}} \mathbf{q}_i \cdot \mathbf{k}_j\right) }{ \sum_{j^\prime=1}^\mrSize \exp\left( \embDim^{-\frac{1}{2}} \mathbf{q}_i \cdot \mathbf{k}_{j^\prime}\right)} \quad \forall i,j : i \in \{1,\ldots,\uttSize\}, j \in \{1,\ldots,\mrSize\}.\]
  Because the $\mrSize$ items have distinct key and value matrices,
  representation of similarity between a query and a key can be different
  than the value that is produced in the output, unlike the attention
  mechanisms discussed in the GRU decoder, where essentially, the key
  and values were identical representations.


  The idea behind multi-head attention is to compute $\numHeads$ distinct
  attention operations by first projecting the query, keys, and values
  down to a smaller representation. That is, given projection
  matrices, $\weight{Q_{in},k}, \weight{K_{in},k}, \weight{V_{in},k} \in \reals^{\embDim \times \frac{\embDim}{\numHeads}}$ for $k \in \{1,\ldots,\numHeads\}$, and $\weight{V_{out}} \in \reals^{\embDim \times \embDim}$,
the  multi-headed attention layer computes
\[ \MultiAttn(\Query, \Key, \Value) = \left(\begin{array}{lc}
            &\Attn\left(\Query\weight{Q_{in},1},  
                 \Key\weight{K_{in},1}, \Value\weight{V_{in},1}\right)\\
              \oplus &   
                  \Attn\left(\Query\weight{Q_{in},2}, 
              \Key\weight{K_{in},2}, \Value\weight{V_{in},2}\right)\\
              & \vdots \\
              \oplus & 
            \Attn\left(\Query\weight{Q_{in},\numHeads},  
                 \Key\weight{K_{in},\numHeads}, \Value\weight{V_{in},\numHeads}\right)\\
        \end{array}\right) \weight{V_{out}}
 \]
  where $\oplus$ indicates column-wise concatenation. Each use of a
  multi-head attention layer uses distinct projection matrices
  and are learned parameters of the model.

%~\\~\\
%
%the single-headed attention layer takes three matrices as input:
%a query matrix, $\Query \in \reals^{\uttSize \times \embDim}$,
%a key matrix, $\Key \in \reals^{\mrSize \times \embDim}$,
%and a value matrix, $\Value \in \reals^{\mrSize \times \embDim}$.
%

%The query matrix typically represents a series of $\uttSize$ 
%state vectors, for example, the decoder hidden states of the token
%prefix $\utttoks_{1:\uttSize}$. The key matrix represents a 
%sequence of $\mrSize$ state vectors that will be attended to by the
%query matrix, for example, the encoder hidden states of the input $\mrtoks$. 

Additionally, there is a masked variant of attention, $\MaskedMultiAttn$
  where the individual attention layers are computed as
  \[\Attn\left(\Query, \Key, \Value\right) = \softmax\left(\frac{\Query \Key^T \odot \Mask }{\sqrt{\embDim}} \right)\Value \]
where $\Mask \in \reals^{\uttSize \times \mrSize}$ 
 is a lower triangular matrix, i.e. values on or below the diagonal are 1
  and all other values are $-\infty$.  The masked multi-head attention
  is used for the decoder self-attention and prevents the $i$-th decoder
  step from attending to future steps, i.e. giving it clairvoyant knowledge 
  of future tokens.

  \paragraph{\FeedForward~Layer} The next bulding block is a single hidden layer \feedforward~network, $\ff : \reals^{*\times \embDim} \rightarrow \reals^{*\times \embDim}$, with $\relu$ activation in its hidden layer and
  no activation in its output.
  %applied independently 
%to each row of  
  Let the input to the layer be a sequence of $\mrSize$ vectors, represented 
  as rows in a matrix $\tfEncInput =\left[\tfEncInputRow_1,\ldots,\tfEncInputRow_\mrSize \right]\in \reals^{\mrSize \times \tfFeats}$.
%  where $\tfEncInput$ is a sequence $\mrSize$ embeddings
%with $\tfFeats$ features, 
The output of the $\ff$ layer is then computed
%%  ($\relu(\encInput) = \max\left(\zeroEmb, \encInput\right)$) \cite{nair2010}, 
%applied to each row of an $m \times n$ input matrix, i.e. a sequence of $m$ objects
%%  with $n$ features,\\
%%  
%%  
\[\ff\left(\tfEncInput;\weight{i},\weight{j},\bias{i},\bias{j}\right) = 
\relu\left(\tfEncInput\weight{i} + \bias{i}\right)\weight{j} + \bias{j}.     \]
where $\weight{i} \in \reals^{\embDim \times \hidDim}$, $\bias{i} \in \reals^{\hidDim}$,
$\weight{j} \in \reals^{\hidDim \times \embDim}$, $\bias{j} \in \reals^{\embDim}$ are learned parameters and 
 matrix-vector additions  are broadcast across  the matrix rows
 (i.e. $\mathbf{H} + \mathbf{b} = \left[\mathbf{h}_1 + \mathbf{b};\cdots \mathbf{h}_\mrSize + \mathbf{b}\right]$)

 \paragraph{Layer Normalization}
The final component is layer normalization
 \citep{ba2016}.
 Let $\tfEncInputRow = \left[\tfEncInputEl_1,\ldots,\tfEncInputEl_\tfFeats\right] \in \reals^{\tfFeats}$ be a vector, representing
 an embedding of an item with $\tfFeats$ features, and with mean and standard deviation
\[\bar{\tfEncInputEl}= \frac{1}{\tfFeats} \sum_{i=1}^\tfFeats \tfEncInputEl_i
    \quad \textrm{and} \quad
  \tfEncInputEl_\sigma = \left(
      \frac{1}{\tfFeats-1} \sum_{k=1}^\tfFeats \left( 
  \tfEncInputEl_k - \bar{\tfEncInputEl} \right)^2  + \epsilon\right)^\frac{1}{2},\]
  respectively 
  (the $\epsilon$ term is a small constant for numerical stability,
  set to $10^{-5}$).
Layer normalization, 
$\layerNorm : \reals^{\tfFeats} \rightarrow \reals^{\tfFeats}$,
normalizes the input have zero mean/unit variance before scaling each element
and adding a bias,
\[\layerNorm(\tfEncInputRow; \lnweight, \lnbias) = \lnweight \odot \left(\tfEncInputRow - \bar{\tfEncInputEl}\right) \cdot \tfEncInputEl_\sigma^{-1} + \lnbias \]
where $\lnweight, \lnbias \in \reals^{\tfFeats}$ are learned parameters
and $\odot$ is the \elementwise~product.
When applying layer normalization to a matrix $\tfEncInput =\left[\tfEncInputRow_1,\ldots,\tfEncInputRow_\mrSize \right]\in \reals^{\mrSize \times \tfFeats}$, where $\tfEncInput$ is a sequence $\mrSize$ embeddings
with $\tfFeats$ features, layer normalization is applied independently to 
each row,
\[ \layerNorm(\tfEncInput;\lnweight, \lnbias) = \layerNorm\left( \left[ \begin{array}{c}\tfEncInputRow_1 \\ \vdots \\ \tfEncInputRow_\mrSize  \end{array} \right]; \lnweight, \lnbias\right) =  \left[ \begin{array}{c} \layerNorm\left( \tfEncInputRow_1;\lnweight,\lnbias\right) \\ \vdots \\ \layerNorm\left( \tfEncInputRow_\mrSize;\lnweight,\lnbias\right)  \end{array} \right].  \]





%
%~\\~\\
%
%$\MultiAttn$. The input to the $\MultiAttn$ 
%
%~\\~\\
%Let $\Query \in \reals^{\uttSize \times \embDim}$ be the ``query matrix,''
%which consists of $\uttSize$ query vectors, each of which will be attending
%to the ``key matrix,'' $\Key \in \reals^{\mrSize \times \embDim}$ which
%consists of $\mrSize$ key vectors. For each query vector, multi-head attention
%computes 
%$\numHeads$ attentative reads of $\Key$ and concatenates them together,
%%which is defined
%%%  as\\
%\noindent $\MultiAttn(\Query, \Key; \weight{a_1}, \weight{a_2}) =$
%\[ \Bigg(\Attn\left(\Query\weight{a_1}_{1,1}, \Key\weight{a_1}_{2,1}, \Key \weight{a_1}_{3,1} \right) \oplus 
%      \ldots \oplus
%  \Attn\left(\Query\weight{a_1}_{1,\numHeads}, \Key\weight{a_1}_{2,\numHeads}, \Key \weight{a_1}_{3,\numHeads} \right)\Bigg) \weight{a_2}
%  \]
%  where $\oplus$ indicates column-wise concatenation,  $\weight{a_1} \in \reals^{3 \times \numHeads \times \embDim\times \embDim / \numHeads}$
%  and $\weight{a_2} \in \reals^{\embDim\times \embDim}$ are learned parameters,
%  and  
%$\Attn$ is defined,
%  
%  \[\Attn\left(\Query, \Key, \Value\right) = \softmax\left(\frac{\Query \Key^T}{\sqrt{\embDim}} \right)\Value. \]
%
%
%  Additionally, there is a masked variant of attention, $\MaskedMultiAttn$
%  where the attention is computed 
%  \[\Attn\left(\Query, \Key, \Value\right) = \softmax\left(\frac{\Query \Key^T \odot \Mask }{\sqrt{\embDim}} \right)\Value \]
%where $\Mask \in \reals^{\uttSize \times \mrSize}$ 
% is a lower triangular matrix, i.e. values on or below the diagonal are 1
%  and all other values are $-\infty$.  The masked multi-head attention
%  is used for the decoder self-attention and prevents the $i$-th decoder
%  step from attending to future steps.
%
%

\subsubsection{Transformer Processing Blocks}
Each encoder and decoder transformer layer consists of several
``processing blocks'' which we define now. Each processing block 
uses some of the basic components defined in the previous subsection.
%while the decoder transformer layers consists of three processing blocks.
%The components of each transformer layer rely on basic elements which we define
%first. %%  Each Transformer layer is divided into blocks which each have three
%%  parts, (i) layer norm, (ii) feed-forward/attention, and  (iii) skip-connection.
%%  We first define the components used in the transformer blocks before
%%  describing the overall S2S transformer. 
%%  Starting with layer norm \cite{ba2016}, %%  
%  
 % Each transformer layer is built using two or three ``blocks.''
  There are fourt distinct block types, a \feedforwardblock, a \selfattentionblock, a \maskedselfattentionblock, and an \encoderattentionblock. 
  Each block consists of three layers
    \begin{enumerate}
        \item Layer Normalization
        \item Processing Layer
        \item Skip-Connections
    \end{enumerate}
    where the processing layer is determined by the block type. For instance,
    let $\tfEncInput \in \reals^{\mrSize \times \embDim}$ be a
    matrix with its rows representing a sequence of $\mrSize$ vectors; the \feedforwardblock~is defined as \\
    \purplebox{\centering \begin{minipage}{0.9\textwidth}
   \begin{align*}
      \textbf{Feed-Forward Block} & \\
       \textit{(Layer Normalization)} & & 
            \boldsymbol{\check{\tfEncInput}} &
            =  \layerNorm\left(\tfEncInput \right) \\
            \textit{(Processing Layer)} &  & \boldsymbol{\bar{\tfEncInput}} &= \ff\left(\boldsymbol{\check{\tfEncInput}}\right)\\
    \textit{(Skip-Connection)} & & \boldsymbol{\hat{\tfEncInput}} &= \tfEncInput + \boldsymbol{\bar{\tfEncInput}}\end{align*}
    \begin{center}$\feedforwardblock(\tfEncInput)  =   \boldsymbol{\hat{\tfEncInput}}.$ \end{center}
\end{minipage}}

\noindent   The \selfattentionblock~and \maskedselfattentionblock s are similarly defined as,\\
    \cyanbox{\centering \begin{minipage}{0.9\textwidth}
   \begin{align*}
       \textbf{Self-Attention Block} & \\
       \textit{(Layer Normalization)} & & 
            \boldsymbol{\check{\tfEncInput}} &
            =  \layerNorm\left(\tfEncInput\right) \\
            \textit{(Processing Layer)} &  & \boldsymbol{\bar{\tfEncInput}} &=  \MultiAttn\left(\boldsymbol{\check{\tfEncInput}}, \boldsymbol{\check{\tfEncInput}}\right)  \\
        \textit{(Skip-Connection)} & & \boldsymbol{\hat{\tfEncInput}} &= \tfEncInput + \boldsymbol{\bar{\tfEncInput}}\end{align*}
       \begin{center}  {$\selfattentionblock(\tfEncInput) =  {\boldsymbol{\hat{\tfEncInput}}}$}\end{center}
\end{minipage}}

~\\

         \noindent  and \\
    \limebox{\centering \begin{minipage}{0.9\textwidth}
         \begin{align*}
       \textbf{Masked Self-Attention Block} & \\
       \textit{(Layer Normalization)} & & 
            \boldsymbol{\check{\tfEncInput}} &
            =  \layerNorm\left(\tfEncInput\right) \\
            \textit{(Processing Layer)} &  & \boldsymbol{\bar{\tfEncInput}} &=  \MaskedMultiAttn\left(\boldsymbol{\check{\tfEncInput}}, \boldsymbol{\check{\tfEncInput}}\right)  \\
        \textit{(Skip-Connection)} & & \boldsymbol{\hat{\tfEncInput}} &= \tfEncInput + \boldsymbol{\bar{\tfEncInput}}\end{align*}
       \begin{center}  {$\maskedselfattentionblock(\tfEncInput) =  {\boldsymbol{\hat{\tfEncInput}}}$.}\end{center}
\end{minipage}}

~\\

           \noindent Finally, let $\tfDecInput \in \reals^{\uttSize \times \embDim}$ be a sequence of $\uttSize$ embeddings. The \encoderattentionblock~is
           defined as,\\
    \tealbox{\centering \begin{minipage}{0.9\textwidth}
   \begin{align*}
       \textbf{Encoder Attention Block} & \\
       \textit{(Layer Normalization)} & & 
            \boldsymbol{\check{\tfDecInput}} &
            =  \layerNorm\left(\tfDecInput\right) \\
            \textit{(Processing Layer)} &  & \boldsymbol{\bar{\tfDecInput}} &=  \MultiAttn\left(\boldsymbol{\check{\tfDecInput}}, \tfEncInput\right)  \\
        \textit{(Skip-Connection)} & & \boldsymbol{\hat{\tfDecInput}} &= \tfDecInput + \boldsymbol{\bar{\tfDecInput}}\end{align*}
       \begin{center}  {$\encoderattentionblock(\tfDecInput, \tfEncInput) =  {\boldsymbol{\hat{\tfDecInput}}}$.}\end{center}
\end{minipage}}



\subsubsection{The Transformer Encoder and Decoder Layers}

We now describe the actual transformer-based \sequencetosequence~model
using the blocks defined previously. We start first with the encoder 
and decoder input layers.


\paragraph{\redbox{Encoder Embedding Layer} and \orangebox{Decoder Embedding Layer}}
           Let $\mrtoks = \left[ \mrtok_1,\ldots,\mrtok_\mrSize\right]$ and 
           $\utttoks = \left[\utttok_1,\ldots,\utttok_{\uttSize-1}\right]$ be 
           input and output sequences, with elements $\mrtok_i$ and $\utttok_i$
           drawn from vocabularies $\mrvocab$ and $\uttvocab$ respectively.
           With each vocabulary we associate an embedding matrix,
          %Let $\mrvocab$ be the \meaningrepresentation~token vocabulary, and 
          $\encEmbs \in \reals^{\setsize{\mrvocab} \times \embDim}$ 
          and 
          $\decEmbs \in \reals^{\setsize{\uttvocab} \times \embDim}$ 
          respectively. Let $\encEmbs_\mrtok \in \reals^{\embDim}$ denote the $\embDim$-dimensional  embedding for each $\mrtok \in \mrvocab$; similarly,  
let $\decEmbs_\utttok \in \reals^{\embDim}$ denote the $\embDim$-dimensional  embedding for each $\utttok \in \uttvocab$.

  Additionally let $\posEmb \in \reals^{\mrSizeMax \times \embDim}$ be a sinusoidal position embedding matrix
  defined elementwise with 
 \begin{align*}
     \posEmb_{i,j} & = \sin\left(\frac{i}{10,000^{ \frac{2(j-1)}{\embDim} }}\right) & \forall i,j : i \in \{1,\ldots,\mrSizeMax\}, j \in \{1,3,\ldots, \embDim - 1\} \\
     \posEmb_{i,j} & = \cos\left(\frac{i}{10,000^{ \frac{2(j-1)}{\embDim} }}\right) & \forall i,j : i \in \{1,\ldots, \mrSizeMax\}, j \in \{2,4,\ldots,\embDim\}. 
  \end{align*} $\posEmb$ is not updated during training and its rows
  function as an encoding of the relative position of token in the encoder/decoder inputs. The number of total positions $\mrSizeMax$ is a hyperparameter
  and represented the longest sequence the encoder/decoder can take as input.

Before being fed into the transformer encoder/decoder, each sequence is 
embedded in its respective embedding space and the corresponding
position embeddings are added,
\[\tfEncInput^{(0)} = \left[ \begin{array}{c}\encEmbs_{\mrtok_1} + \posEmb_1\\
\vdots \\ \encEmbs_{\mrtok_\mrSize} + \posEmb_\mrSize \end{array}\right] 
\quad \textrm{and} \quad 
\tfDecInput^{(0)} = \left[ \begin{array}{c}\decEmbs_{\utttok_1} + \posEmb_1\\
\vdots \\ \decEmbs_{\utttok_{\uttSize-1}} + \posEmb_{\uttSize-1} \end{array}\right].
\]

\paragraph{\greenbox{Transformer Encoder}}
A transformer encoder layer consists of a \selfattentionblock~followed
    by a \feedforwardblock.
%?  Unlike 
%?\recurrentneuralnetwork~architectures like the GRU, the representation 
%?of each encoder input $\mrtok_i$ can incorporate information to its left
%?and right, regardless of how far away it is in the sequence.
 A transformer encoder with $\numLayer$ layers
    then computes
\begin{align*} 
    {\boldsymbol{\bar{\tfEncInput}}^{(l)}} & = \selfattentionblock^{(l)}
\left(\tfEncInput^{(l-1)}\right) \\
    \tfEncInput^{(l)} & = \feedforwardblock^{(l)}\left({\boldsymbol{\bar{\tfEncInput}}^{(l)}}\right)
\end{align*} for $l \in \{1,\ldots,\numLayer\}$.
We indicate the final encoder output as $\tfEncInput =  \tfEncInput^{(\numLayer)}$.
 

%?of $\numLayer$ 
%?distinct transformer encoder
%?layers and decoder layers which are applied to $\mrtoks$ and $\utttoks_{1:i}$
%?respectively in order to compute $\gen\left(\utttok_{i+1}|\utttoks_{1:i},\mrtoks;\params\right)$. 
%?
%?Each encoder transformer layer repeatedly apply ``self-attention'' to the encoder
%?input to obtain a representation of the input. 
%?
%?
%?When predicting the next utteratnce token $\utttok_{i+1}$ 
%?the decoder performs self-attention on the previously generated tokens
%?$\utttoks_{1:i}$ while also attending to the encoder output. Again
%?the use of attention rather than a recurrence allows for the propagation
%?of information from any part of the input sequence or previously generate output sequence to influence the next word prediction. 
%?


   
\paragraph{\bluebox{Transformer Decoder}}
    A transformer deccoder layer consists of a \maskedselfattentionblock~followed
    by an \encoderattentionblock~and a \feedforwardblock. Note that the \maskedselfattentionblock~means that though we can compute all decoder states
    in parallel during training, it is equivalent to computing each decoder state sequentially (which we must do at test time). 
    A transformer deccoder with $\numLayer$ layers
    is computed as
\begin{align*} 
    {\boldsymbol{\check{\tfDecInput}}}^{(l)} & = \maskedselfattentionblock^{(l)}\left(\tfDecInput^{(l-1)}\right) \\
    {\boldsymbol{\bar{\tfDecInput}}}^{(l)} & = \encoderattentionblock^{(l)}\left(\boldsymbol{\check{\tfDecInput}}^{(l)}, \tfEncInput\right) \\
    \tfDecInput^{(l)} & = \feedforwardblock^{(l)}\left({\boldsymbol{\bar{\tfDecInput}}^{(l)}}\right)
\end{align*} for $l \in \{1,\ldots,\numLayer\}$.
We indicate the final decoder output as $\tfDecInput =  \tfDecInput^{(\numLayer)}$.

\paragraph{\violetbox{Prediction Layer}}
    Let $\tfDecInputRow_i$ be the $i$-th row of $\tfDecInput$ corresponding
to the decoder representation of the $i$-th decoder state. The probability of 
  the next word is
   \[ \gen\left(\utttok_{i+1}|\utttoks_{1:i},\mrtoks\right) 
   = \softmax\left( \weight{o}\tfDecInputRow_i + \bias{o} \right)_{\utttok_{i+1}} \quad \forall i : i \in \{1,\ldots, \uttSize-1\}\]
  where $\weight{o} \in \reals^{\setsize{\uttvocab} \times \embDim}$ 
  and $\bias{o} \in \reals^{\embDim}$ are learned parameters. 
Each block from each encoder and decoder layer has separate learned parameters.
Because each operation in the transformer is built around matrix 
multiplication, its computation can be parallelized more heavily than
\recurrentneuralnetwork~models like the GRU architecture.
 
  
  
%



%%  

          

%%  Given a linearized MR $\lin(\mr) = \inseq= \left[ \da, \attr_1, \attr_2, \ldots,
%%  \attr_{\size{\mr}}\right] \in \Attrs^{\inSize}$ where the length
%%  of the sequence is $\inSize = \size{\mr} + 1$,
%%  let $\encWordEmb_i = \encEmbs_{\inseq_i}$ for $i \in \{1, \ldots \inSize\}$.
%%  
%%  Additionally let $\posEmb \in \reals^{\inSize_{max} \times \embDim}$ be a sinusoidal position embedding matrix
%%  defined elementwise with 
%%  \begin{align*}
%%      \posEmb_{i,2j} & = \sin\left(\frac{i}{10,000^{ \frac{2j}{\embDim} }}\right) \\
%%      \posEmb_{i,2j+1} & = \cos\left(\frac{i}{10,000^{ \frac{2j}{\embDim} }}\right). 
%%  \end{align*}
%%  The encoder input sequence $\encInput^{(0)} \in \reals^{\inSize \times \embDim}$ is then defined by
%%  \[\encInput^{(0)} = \left[\begin{array}{c} 
%%              \encWordEmb_1 + \posEmb_1,\\
%%              \encWordEmb_2 + \posEmb_2,\\
%%              \vdots \\
%%          \encWordEmb_\inSize + \posEmb_\inSize
%%      \end{array}
%%                          \right] \]
%
 

%%  
%


    %%  Each encoder transformer layer computes the following, \\
%%  
%%  \noindent \textit{(Self-Attention Block)}\\
%%  \[\tfeA = \layerNorm\left(\encInput^{(i)}; \lnweightv^{(i,1)}, \lnbias^{(i,1)}\right)\]
%%  \[\tfeB = \MultiAttn\left(\tfeA, \tfeA; \weight{i,a_1},  \weight{i,a_2}\right)\]
%%  \[\tfeC = \encInput^{(i)} + \tfeB\]
%%  

 %Given these layers ($\layerNorm, \ff, \MultiAttn$, and $\MaskedMultiAttn$) we now define the transformer encoder and decoder layers.
%%  Let $\Attrs$ be the encoder input vocabulary, and  $\encEmbs \in
%%  \reals^{\size{\Attrs} \times \embDim}$ an associated word embedding matrix
%%  where $\encEmbs_\attr \in \reals^{\embDim}$ denotes the $\embDim$-dimensional
%%  embedding for each $\attr \in \Attrs$. 
%%  Given a linearized MR $\lin(\mr) = \inseq= \left[ \da, \attr_1, \attr_2, \ldots,
%%  \attr_{\size{\mr}}\right] \in \Attrs^{\inSize}$ where the length
%%  of the sequence is $\inSize = \size{\mr} + 1$,
%%  let $\encWordEmb_i = \encEmbs_{\inseq_i}$ for $i \in \{1, \ldots \inSize\}$.
%%  
%%  Additionally let $\posEmb \in \reals^{\inSize_{max} \times \embDim}$ be a sinusoidal position embedding matrix
%%  defined elementwise with 
%%  \begin{align*}
%%      \posEmb_{i,2j} & = \sin\left(\frac{i}{10,000^{ \frac{2j}{\embDim} }}\right) \\
%%      \posEmb_{i,2j+1} & = \cos\left(\frac{i}{10,000^{ \frac{2j}{\embDim} }}\right). 
%%  \end{align*}
%%  The encoder input sequence $\encInput^{(0)} \in \reals^{\inSize \times \embDim}$ is then defined by
%%  \[\encInput^{(0)} = \left[\begin{array}{c} 
%%              \encWordEmb_1 + \posEmb_1,\\
%%              \encWordEmb_2 + \posEmb_2,\\
%%              \vdots \\
%%          \encWordEmb_\inSize + \posEmb_\inSize
%%      \end{array}
%%                          \right] \]
%%  
%%   A sequence of $l$ transformer encoder layers are then applied to the encoder
%%   input, i.e. $\encInput^{(i+1)} = \operatorname{TF}^{(i)}_{enc}\left(\encInput^{(i)}\right)$.
%%  Each encoder transformer layer computes the following, \\
%%  
%%  \noindent \textit{(Self-Attention Block)}\\
%%  \[\tfeA = \layerNorm\left(\encInput^{(i)}; \lnweightv^{(i,1)}, \lnbias^{(i,1)}\right)\]
%%  \[\tfeB = \MultiAttn\left(\tfeA, \tfeA; \weight{i,a_1},  \weight{i,a_2}\right)\]
%%  \[\tfeC = \encInput^{(i)} + \tfeB\]
%%  
%%  \noindent \textit{(Feed-Forward Block)}\\
%%  \[\tfeD = \layerNorm\left(\tfeC; \lnweightv^{(i,2)}, \lnbias^{(i,2)}\right)\]
%%  \[\tfeE = \feedforward\left(\tfeD;\weight{i,1},\weight{i,2},\bias{i,1},\bias{i,2}\right)\]
%%  \[ \encInput^{(i+1)} = \tfeC + \tfeE \]
%%  
%%  
%%  We denote the final encoder output for $l$ layers as $\encInput = \encInput^{(l)}$.  
%%  
%%  Let $\uttVocab$ be the vocabulary of utterance tokens, and 
%%  $\decEmbs \in \reals^{\size{\uttVocab} \times \embDim}$
%%  an associated embedding matrix, where
%%  $\decEmbs_\utttok \in \reals^{\embDim}$ denotes a  $\embDim$-dimensional embedding for
%%  each $\utttok \in \uttVocab$.
%%  
%%  %\placeholder{TODO: Make this true for all layers}
%%  Given the decoder input sequence $\utt = \utttok_1, \utttok_2, \ldots, 
%%  \utttok_\size{\utt}$, 
%%  let $\decWordEmb_i = \decEmbs_{\utttok_i}$ for $i \in \{1, \ldots \outSize\}$.
%%  where $\outSize = \size{\utt} - 1$
%%  
%%  \[\decInput^{(0)} = \left[\begin{array}{c} 
%%              \decWordEmb_1 + \posEmb_1,\\
%%              \decWordEmb_2 + \posEmb_2,\\
%%              \vdots \\
%%          \decWordEmb_\outSize + \posEmb_\outSize
%%      \end{array}
%%                          \right]. \]
%%  
%%  
%%  A sequence of $l$ transformer decoder layers are then applied to the decoder
%%   input, i.e. $\decInput^{(i+1)} = \operatorname{TF}^{(i)}_{dec}\left(\decInput^{(i)}\right)$.
%%  Each decoder transformer layer computes the following, \\
%%  
%%  ~\\~\\ ~\\
%%  
%%  \noindent \textit{(Masked Self-Attention Block)}\\
%%  \[\tfdA = \layerNorm\left(\decInput^{(i)}; \lnweightv^{(i,1)}, \lnbias^{(i,1)}\right)\]
%%  \[\tfdB = \MultiAttn_M\left(\tfdA, \tfdA; \weight{i,a_1}, \weight{i,a_2} \right)\]
%%  \[\tfdC = \decInput^{(i)} + \tfdB\]
%%  
%%  \noindent \textit{(Encoder-Attention Block)}\\
%%  \[\tfdD = \layerNorm\left(\tfdC; \lnweightv^{(i,2)}, \lnbias^{(i,2)}\right)\]
%%  \[\tfdE = \MultiAttn\left(\tfdD, \encInput; \weight{i,a_3}, \weight{i,a_4}\right)\]
%%  \[\tfdF = \tfdC + \tfdE\]
%%  
%%  \noindent \textit{(Feed-Forward Block)}\\
%%  \[\tfdG = \layerNorm\left(\tfdF; \lnweightv^{(i,3)}, \lnbias^{(i,3)}\right)\]
%%  \[\tfdH = \feedforward\left(\tfdG;\weight{i,1},\weight{i,2},\bias{i,1},\bias{i,2}\right)\]
%%  \[ \decInput^{(i+1)} = \tfdF + \tfdH \]
%%  
%%  Let the $\decInput = \decInput^{(l)}$ denote the final decoder output,
%%  and let $\decInputi$ be the $i$-th row of $\decInput$ corresponding
%%  to the decoder representation of the $i$-th decoder state. The probability of 
%%  the next word is\\
%%  
%%  \noindent $\model\left(\utttok_{i+1}|\utttok_{\le i},\lin(\mr)\right)$
%%  \[  = \softmax\left( \weight{o}\decInputi + \bias{o} \right)_{\utttok_{i+1}} \]
%%  where $\weight{o} \in \reals^{\size{\uttVocab} \times \embDim}$ 
%%  and $\bias{o} \in \reals^{\embDim}$ are learned parameters. 
%%  
%%  
%%  %We used the Transformer S2S as implemented in 
%%  %\href{https://pytorch.org/}{PyTorch}.
%%  The input embedding dimension is $\embDim= 512$ and inner hidden layer size 
%%  is $\hidDim=2048$. The encoder and decoder have separate parameters.
%%  We used $H=8$ heads in all multi-head attention layers. 
%%   We used Adam with the learning
%%  rate schedule provided in  \citet{rush2018} (factor=1, warmup=8000).
%%  Dropout was set to 0.1 was applied to input embeddings and each skip 
%%  connection (i.e. the third line in each block definition). As a 
%%  hyperparameter, we optionally tie the decoder input and output embeddings,
%%  i.e. $\decEmbs = \weight{o}$.







